\documentclass[../thesis.tex]{subfiles}

\begin{document}
    \todo{Spør Elias om historiske kilder}
    \section{Filtrations}
        Let $\mathcal{A}$ be an abelian category. Given two objects $A$ and $B$, we denote an inclusion $B \rightarrow A$ by $B \subseteq A$. This section is devoted to filtration terminology.

        \begin{definition}[Filtration]
            A filtration on an object $A$ is a possibly infinite collection of inclusions
            \begin{align*}
                \cdots \subseteq A_{i} \subseteq A_{i+1} \subseteq A_{i+2} \subseteq \cdots \subseteq A\tt{.}
            \end{align*}
        \end{definition}

        \begin{definition}[Bounded filtration]
            We say that a filtration on $A$ is bounded below if there is an integer $s \in \mathbb{Z}$ such that
            \begin{align*}
                0 = A_s \subseteq A_{s+1} \subseteq \cdots A_i \subseteq \cdots \subseteq A\tt{.}
            \end{align*}

            We say that a filtration on $A$ is bounded above if there is an integer $n \in \mathbb{Z}$ such that
            \begin{align*}
                \cdots \subseteq A_i \subseteq \cdots \subseteq A_t = A\tt{.}
            \end{align*}

            A filtration is bounded, or finite, if it is both bounded below and above, i.e. the filtration is finite;
            \begin{align*}
                0 = A_s \subseteq \cdots \subseteq A_i \subseteq \cdots \subseteq A_n = A\tt{.}
            \end{align*}
        \end{definition}

        \begin{definition}[Exhaustive filtrations]
            A filtration on $A$ is said to be exhaustive if $\substack{\varinjlim \\ i} A_i \simeq A$,
            \begin{center}
                \begin{tikzcd}
                    \cdots \ar[hook]{r}[]{} & A_i \ar[hook]{r}[]{} & A_{i+1} \ar[hook]{r}[]{} & \cdots \ar[hook]{r}[]{} & \substack{\varinjlim \\ i} A_i \simeq A\tt{.}
                \end{tikzcd}
            \end{center}
        \end{definition}

        \begin{definition}[Hausdorff filtrations]
            A filtration on $A$ is called Hausdorff if $\varprojlim A_i \simeq 0$.
        \end{definition}

        Every bounded below filtrations are Hausdorff by definition.

        \begin{definition}[Complete filtrations]
            Let $\sfrac{A}{A_i} = \varinjlim(A_i \rightarrow A)$. A filtration on $A$ is called complete if $\substack{\varprojlim \\ i} \sfrac{A}{A_i} \simeq A$,
            \begin{center}
                \begin{tikzcd}
                    A \simeq \substack{\varprojlim \\ i} \sfrac{A}{A_i} \ar[two heads]{r}[]{} & \cdots \ar[two heads]{r}[]{} & \sfrac{A}{A_i} \ar[two heads]{r}[]{} & \sfrac{A}{A_{i+1}} \ar[two heads]{r}[]{} & \cdots
                \end{tikzcd}
            \end{center}
        \end{definition}
        
        We denote the completion of $A$ by $\substack{\varprojlim \\ i} \sfrac{A}{A_i} \simeq \widehat{A}$, and we denote the completion of each subobject by $\widehat{A}_i = \substack{\varprojlim \\ j \leq i} \sfrac{A_i}{A_j}$ There is a filtration on $\widehat{A}$ given by
        \begin{align*}
            \cdots \subseteq \widehat{A}_i \subseteq \widehat{A}_{i+1} \subseteq \cdots \subseteq \widehat{A}\tt{.}
        \end{align*}

    \section{Spectral Sequence}
    
        For this section we will let $\mathcal{A}$ be an abelian category. To be more precise, one should assume that $\mathcal{A}$ is bicomplete, that arbitrary coproducts of epis is an epi and that arbitrary products of monos is a mono. Categories such as $\tt{Mod}^R$ for a ring $R$ has these properties.

        A spectral sequence is a method in which one may calculate the homology of chain complexes. For instance, there is a spectral sequence associated to each filtered chain complex. The spectral sequence will be defined in terms of pages.

        \begin{definition}[Homology spectral sequence]
            A homology spectral sequence $E$ starting at page $a$ is
            \begin{itemize}
                \item a collection of objects $E^r_{p,q}$ for any $p,q \in \mathbb{Z}$ and $r \geq a$,
                \item morhpisms $d^r_{p,q} : E^r_{p,q} \rightarrow E^r_{p-r,q+r-1}$ such that $d^r\circ d^r = 0$
                \item and isomorphisms between page $r+1$ and the homology of page $r$,
                \begin{align*}
                    E^{r+1}_{p,q} \simeq \sfrac{\tt{Ker}d^r_{p,q}}{\tt{Im}d^r_{p+r,q-r+1}}\tt{.}
                \end{align*}
            \end{itemize}
        \end{definition}

        We refer to the collection of objects $E^r_{\bullet,\bullet}$ for the $r$'th page of the spectral sequence $E$. A homology spectral sequence starting at the second page may be illustrated as
        \begin{center}
            $E^2$: 
            \scalebox{0.5}{
                \begin{tikzcd}[cramped, ampersand replacement = \&]
                    \phantom{A} \& \vdots \& \vdots \& \vdots \& \vdots \\
                    \cdots \& \bullet \& \bullet \ar[no head]{llu}[]{} \& \bullet \ar[no head]{llu}[]{} \& \bullet \ar[no head]{llu}[]{} \& \cdots \\
                    \cdots \& \bullet \& \bullet \ar[no head]{llu}[]{} \& \bullet \ar[]{llu}[]{} \& \bullet \ar[]{llu}[]{} \& \cdots \\
                    \cdots \& \bullet \& \bullet \ar[no head]{llu}[]{} \& \bullet \ar[]{llu}[]{} \& \bullet \ar[]{llu}[]{} \& \cdots \\
                    \cdots \& \bullet \& \bullet \ar[no head]{llu}[]{} \& \bullet \ar[]{llu}[]{} \& \bullet \ar[]{llu}[]{} \& \cdots \\
                    \& \vdots \& \vdots \& \vdots \& \vdots \&
                \end{tikzcd}
            }
            $\quad \rightsquigarrow\quad E^3$: 
            \scalebox{0.5}{
                \begin{tikzcd}[cramped, ampersand replacement = \&]
                    \phantom{A} \& \vdots \& \vdots \& \vdots \& \vdots \\
                    \cdots \& \bullet \& \bullet \& \bullet \& \bullet \& \cdots \\
                    \cdots \& \bullet \& \bullet \& \bullet \ar[no head]{llluu}[]{} \& \bullet \ar[no head]{llluu}[]{} \& \cdots \\
                    \cdots \& \bullet \& \bullet \& \bullet \ar[no head]{llluu}[]{} \& \bullet \ar[]{llluu}[]{} \& \cdots \\
                    \cdots \& \bullet \& \bullet \& \bullet \ar[no head]{llluu}[]{} \& \bullet \ar[]{llluu}[]{} \& \cdots \\
                    \& \vdots \& \vdots \& \vdots \& \vdots \&
                \end{tikzcd}
            }
        \end{center}
        where we go from the second page to the third page by taking homology. At page $r$, each line along the form $(-r, r-1)$ defines a chain complex in $\tt{Ch}(\mathcal{A})$.

        \begin{definition}[Cohomology spectral sequence]
            A cohomology spectral sequence $E$ starting at page $a$ is
            \begin{itemize}
                \item a collection of objects $E_r^{p,q} \in \mathcal{A}$ for any $p,q \in \mathbb{Z}$ and $r \geq a$,
                \item morphisms $d_r^{p,q} : E_r^{p,q} \rightarrow E_r^{p+r, q-r+1}$ such that $d_r\circ d_r = 0$
                \item and isomorphisms between page $r+1$ and the homology of page $r$,
                \begin{align*}
                    E_{r+1}^{p,q} \simeq \sfrac{\tt{Ker}d_r^{p,q}}{\tt{Im}d_r^{p-r,q+r-1}}
                \end{align*}
            \end{itemize}
        \end{definition}

        We divide a spectral sequence up into diagonals. The object $E^r_{p,q}$ is said to be of degree $n$ if $n = p + q$.

        \begin{definition}[Bounded spectral sequence]
            A homology spectral sequence $E$ starting at page $a$ is said to be bounded if there are only finitely many non-zero terms of every degree $n$.
        \end{definition}

        Given a bounded spectral sequence $E$, there is a page $r_0$, such that for any $r \geq r_0$ $p$ and $q$, $E^r_{p,q} \simeq E^{r+1}_{p,q}$. This stable unchanging page will be denoted as $E^\infty = E^r$.

        \begin{definition}[Bounded convergence]
            A bounded homology spectral sequence is said to converge to $H_*$ if for each $n$, there is a finite filtration
            \begin{align*}
                0 = F_sH_n \subseteq \cdots \subseteq F_iH_n \subseteq \cdots \subseteq FtH_n = H_n\tt{,}
            \end{align*}
            such that $E^\infty_{p,q} \simeq \sfrac{F_pH_{p+q}}{F_{p-1}H_{p+q}}$. We write this as
            \begin{align*}
                E^a_{p,q} \Rightarrow H_{p+q}\tt{.}
            \end{align*}
        \end{definition}

        Suppose that we have a bounded homology spectral sequence $E$ starting at page $a$, such that it converges $E^a \Rightarrow H$. To calculate each $H_n$, one would then have to solve extension problems. For instance, there is a short exact sequence
        \begin{center}
            \begin{tikzcd}
                0 \ar[]{r}[]{} & F_{s+1}H_n \ar[tail]{r}[]{} & F_{s+2}H_n \ar[two heads]{r}[]{} & E^\infty_{s+2,n-s-2} \ar[]{r}[]{} & 0\tt{.}
            \end{tikzcd}
        \end{center}
        In this manner, given some extra information, we could calculate the homology in terms of the infinty page.

        \begin{definition}[Collapse]
            We say that a homology spectral sequence collapse at page $r \geq 2$ if there is at most one non-zero column or row in $E^r$.
        \end{definition}

        Whenever a spectral sequence collapse at page $r$, this is automatically the $\infty$-page. If a spectral sequence converges $E^a \Rightarrow H$, then $H_n$ is the unique non-zero object of degree $n$ in $E^\infty$.

        \begin{definition}[$\infty$-page]
            Let $E$ be a homology spectral sequence starting at page $a$. Define $Z^r_{p,q} = \tt{Ker}d^r_{p,q}$ and $B^r_{p,q} = \tt{Im}d^r_{p,q}$, then $E^{r+1}_{p,q} \simeq \sfrac{Z^r_{p,q}}{B^r_{p,q}}$. We define the $\infty$-page in terms
            \begin{align*}
                Z^\infty_{p,q} & = \substack{\varprojlim \\ a \leq r}Z^r_{p,q}\tt{ and} \\
                B^\infty_{p,q} & = \substack{\varinjlim \\ a \leq r}B^r_{p,q}\tt{,} \\
            \end{align*}
            such that
            \begin{align*}
                E^\infty_{p,q} = \sfrac{Z^\infty_{p,q}}{B^\infty_{p,q}}\tt{.}
            \end{align*}
        \end{definition}

        \begin{definition}[Morphism of spectral sequences]
            A morphism of homology spectral sequences $f : E \rightarrow F$ is a collection of morphisms $f^r_{p,q} : E^r_{p,q} \rightarrow F^r_{p,q}$ such that $f^r\circ d^r_E = d^r_F \circ f^r$, and $H_*f^r \simeq f^{r+1}$.
        \end{definition}

        \begin{lemma}[Mapping lemma, {\cite[Lemma 5.2.4 and Exercise 5.2.3][123]{Weibel94}}]
            Let $f : E \rightarrow F$ be a morphism of spectral sequences. If $f^r : E^r \rightarrow F^r$ is an isomorphism, then $f^{r'} : E^{r'} \rightarrow F^{r'}$ is an isomorphism for any $r' \geq r$, and $f^\infty : E^\infty \rightarrow E^\infty$ is an isomorphism as well.
        \end{lemma}

        \begin{proof}
            The first statement is immediate from functoriality of taking homology, as isomorphisms are sent to isomorphisms. 
            
            Suppose instead that for any page $r \geq a$, there is an isomorphism $f^r : E^r \rightarrow F^r$. Restricting this morphism to the kernels yields an isomorphism by the 5-lemma,
            \begin{center}
                \begin{tikzcd}
                    ZE^r_{p,q} \ar[dashed]{d}[]{Zf^r_{p,q}} \ar[hook]{r}[]{} & E^r_{p,q} \ar[]{d}[]{\simeq} \ar[]{r}[]{d^r_{p,q}} & E^r_{p-r,q+r-1} \ar[]{d}[]{\simeq} \\
                    ZF^r_{p,q} \ar[hook]{r}[]{} & F^r_{p,q} \ar[]{r}[]{d^r_{p,q}} & F^r_{p-r,q+r-1}\tt{.}
                \end{tikzcd}
            \end{center}
            Likewise, there is an isomorphism $Bf^r_{p,q} : BE^r_{p,q} \rightarrow BF^r_{p,q}$. In this manner we obtain isomorphisms of diagrams
            \begin{center}
                \begin{tikzcd}
                    \cdots \ar[hook]{r}[]{} & ZE^r \ar[hook]{r}[]{} \ar[]{d}[]{Zf^r} & \cdots \ar[hook]{r}[]{} & ZE^a \ar[]{d}[]{Zf^a} \\
                    \cdots \ar[hook]{r}[]{} & ZF^r \ar[hook]{r}[]{} & \cdots \ar[hook]{r}[]{} & ZF^a
                \end{tikzcd}\\
                \begin{tikzcd}
                    BE^a \ar[hook]{r}[]{} \ar[]{d}[]{Bf^a} & \cdots \ar[hook]{r}[]{} & BE^r \ar[hook]{r}[]{} \ar[]{d}[]{Bf^r} & \cdots \\
                    BF^a \ar[hook]{r}[]{} & \cdots \ar[hook]{r}[]{} & BF^r \ar[hook]{r}[]{} & \cdots
                \end{tikzcd}
            \end{center}
            Thus the limits $ZE^\infty$ and $ZF^\infty$, and the colimits $BE^\infty$ and $BF^\infty$ exhibit the same universal property respectively. By the 5-lemma we obtain the isomorphism on the $\infty$-page
            \begin{center}
                \begin{tikzcd}
                    BE^\infty \ar[tail]{r}[]{} \ar[]{d}[]{Bf^\infty}[left]{\simeq} & ZE^\infty \ar[two heads]{r}[]{} \ar[]{d}[]{Zf^r}[left]{\simeq} & E^\infty \ar[dashed]{d}[]{f^\infty}[left]{\simeq} \\
                    BF^\infty \ar[tail]{r}[]{} & ZF^\infty \ar[two heads]{r}[]{} & F^\infty 
                \end{tikzcd}
            \end{center}
        \end{proof}

        \begin{definition}[Bounded below spectral sequences]
            A homology spectral sequence $E$ starting at page $a$ is said to be bounded below if for each degree $n$, there is an integer $s$ such that if $p+q = n$, then $E^a_{p,q} = 0$ for any $p < s$. 
        \end{definition}

        \begin{definition}[Regular spectral sequences]
            A homology spectral sequence $E$ is said to be regular if there is an $r$ such that for any $r' \geq r$ we have that $d^r = 0$. In other words, $Z^\infty \simeq Z^r$.
        \end{definition}

        \begin{definition}[Weak convergence]
            A homology spectral sequence $E$ weakly converges to $H_*$ if each $H_n$ has a filtration
            \begin{align*}
                \cdots \subseteq F_iH_n \subseteq \cdots \subseteq H_n
            \end{align*}
            such that there are isomorphisms $E^\infty_{p,q} \simeq \sfrac{F_pH_{p+q}}{F_{p-1}H_{p+q}}$.
        \end{definition}

        A problem with weak convergence which we did not have with bounded convergence is that the spectral sequence cannot detect the elements which may be found in either $\varprojlim F_iH_n$ or $\varinjlim F_iH_n$. This problem is amended if the filtration is both exhaustive and Hausdorff, and in this case we say that the spectral sequence approaches $H_*$.

        \begin{definition}[Convergence]
            A homology spectral sequence $E$ converges to $H_*$ if it approaches $H_*$, $E$ is regular and every $H_n$ is complete, $H_n \simeq \widehat{H}_n$.
        \end{definition}

        In this definition we require regular because of practical reasons. One may observe that every bounded below spectral sequence which approaches $H*$, converges to $H*$. Completeness is assumed for the following theorem. 

        \begin{thm}[Comparison Theorem, {\cite[Theorem 5.2.12][126]{Weibel94}}]\label{thm: comp-thm}
            Let $E$ and $E'$ be homology spectral sequences converging to $H_*$ and $H_*'$ respectively. Suppose that there is a morphism $h : H_* \rightarrow H_*'$, which is compatible with a morphism of spectral sequences $f : E \rightarrow E'$. If $f^r : E^r \rightarrow F^r$ is an isomorphism, then $h$ is an isomorphism as well.
        \end{thm}

    \section{Spectral Sequence of a Filtration}

        Associated to a filtration $F$ on a chain complex $C$, there is a homology spectral sequence $E$ starting at page $0$. We define $E^0_{p,q} = \sfrac{F_pC_{p+q}}{F_{p-1}C_{p+q}}$, where the differential is induced by the associated graded. The $1$-page is then the homology along each associated graded piece, $E^1_{p,q} = H_*(E_{p,*}^0)$.

        One may observe that the spectra sequence arising from $C$, is the same as the spectral sequence arising from its completion $\widehat{C}$.

        We describe the spectral sequence in more detail. Let $\pi_p : F_pC \rightarrow \sfrac{F_pC}{F_{p-1}C}$. We let
        \begin{align*}
            A^r_p = \{c \in F_pC \mid d(c) \in F_{p-r}C\}
        \end{align*}
        be the collection of cycles modulo $F_{p-r}C$. Then we define the complexes in $E^0$
        \begin{align*}
            Z^r_{p,*} = \pi_p(A^r_p)\tt{ and}\\
            B^{r+1}_{p-r,*} = \pi_{p-r}(d(A^r_p))\tt{.}
        \end{align*}
        Every page may then be described as $E^r_p = \sfrac{Z^r_p}{B^r_p}$.

        The important takeaway is the following theorem.
        \begin{thm}[Classical convergence theorem, {\cite[Theorem 5.5.1][135]{Weibel94}}]
            Let $C$ be a chain complex.
            \begin{itemize}
                \item Suppose that the filtration on $C$ is bounded. Then the spectral sequence $E$ is bounded and $E^1_{p,q} \Rightarrow H_{p+q}(C)$.
                \item Suppose that the filtration on $C$ is bounded below and exhaustive. Then the spectral sequence $E$ is bounded below and $E^1_{p,q} \Rightarrow H_{p+q}(C)$.
            \end{itemize}
            This convergence is also natural in the sense that given any morphism of chain complexes $f : C \rightarrow D$. Then the morphism in homology $H_*f : H_*C \rightarrow H_*D$ is compatible with the morphism of spectral sequences $E^1f : EC^1 \rightarrow ED^1$.
        \end{thm}

\end{document}