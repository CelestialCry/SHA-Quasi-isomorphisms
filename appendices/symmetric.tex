\documentclass[../thesis.tex]{subfiles}

\begin{document}
    \section{Monoidal Categories}

        Here we will give a brief summary of symmetric monoidal categories. More detailed accounts may be found in Mac Lane \cite{MacLane71}, Riehl \cite{Riehl14}, or Kelly \cite{Kelly05}. 

        \begin{definition}[Monoidal category]
            
            We say that a category $\mathcal{C}$ is a monoidal category if it comes equipped with 
            \begin{itemize}
                \item a bifunctor
                \begin{align*}
                    \argument \otimes \argument : \mathcal{C} \times \mathcal{C} \rightarrow \mathcal{C}\tt{,}
                \end{align*}
                \item a natural isomorphism in three variables
                \begin{align*}
                    \alpha_{A,B,C} : A \otimes (B \otimes C) \rightarrow (A \otimes B) \otimes C\tt{,}
                \end{align*}
                \item a unit object $Z \in \mathcal{C}$
                \item and natural isomorphisms
                \begin{align*}
                    \lambda_A & : Z \otimes A \rightarrow A\tt{,} \\
                    \rho_A & : A \otimes Z \rightarrow A\tt{.}
                \end{align*}
            \end{itemize}
            Moreover, these maps should satisfy some coherence relations. The following diagrams should commute,
            \begin{center}
                \scalebox{0.75}{
                    \begin{tikzcd}[ampersand replacement = \&]
                        \&[-100pt] \&[-60pt] A \otimes (B \otimes (C \otimes D)) \ar[]{rrd}[]{A \otimes \alpha_{B,C,D}} \ar[]{lld}[]{\alpha_{A,B,C\otimes D}} \&[-60pt] \&[-100pt] \\[20pt]
                        A \otimes ((B \otimes C) \otimes D) \ar[]{dr}[]{\alpha_{A, B\otimes C, D}} \& \& \& \& (A \otimes B ) \otimes (C \otimes D) \ar[]{dl}[]{\alpha_{A\otimes B, C, D}} \\[30pt]
                        \& (A \otimes (B \otimes C)) \otimes D \ar[]{rr}[]{\alpha_{A,B,C}\otimes D} \& \& ((A \otimes B) \otimes C) \otimes D \&
                    \end{tikzcd}
                }
                \scalebox{0.75}{
                    \begin{tikzcd}[column sep = tiny, ampersand replacement = \&]
                    A \otimes (Z \otimes B) \ar[]{rr}[]{\alpha_{A,Z,B}} \ar[]{rd}[]{A \otimes \lambda_B} \& \& (A \otimes Z) \otimes B \ar[]{ld}[]{\rho_A \otimes B} \\[20pt]
                    \& A \otimes B \&
                    \end{tikzcd}
                }
            \end{center}
        \end{definition}

        The coherence diagrams allow us to think of the monoidal product $\otimes$ as an associative and unital product. If the identities give $\alpha$, $\lambda$, and $\rho$, we say that the monoidal category is strict.

        \begin{definition}[Lax monoidal functors]
            Let $(\mathcal{C}, \otimes, Z)$ and $(\mathcal{D}, \boxtimes, W)$ be monoidal categories. A functor $F : \mathcal{C} \rightarrow \mathcal{D}$ is monoidal if it comes equipped with
            \begin{itemize}
                \item a natural transformation
                \begin{align*}
                    \mu_{A,B} : F(A) \boxtimes F(B) & \rightarrow F(A \otimes B)
                \end{align*}
                \item and a morphism of units
                \begin{align*}
                    \upsilon : W \rightarrow F(Z)\tt{.}
                \end{align*}
            \end{itemize}
            Furthermore, the following diagrams should commute.
            \begin{center}
                \scalebox{0.75}{
                    \begin{tikzcd}[ampersand replacement = \&]
                        \&[-80pt] F(A) \boxtimes (F(B) \boxtimes F(C)) \ar[]{r}[]{\alpha^\mathcal{D}_{F(A),F(B),F(C)}} \ar[]{ld}[]{F(A) \boxtimes \mu_{B,C}} \&[30pt] (F(A) \boxtimes F(B)) \boxtimes F(C) \ar[]{rd}[]{\mu_{A,B} \boxtimes F(C)} \&[-80pt] \\[30pt]
                        F(A) \boxtimes (F(B \otimes C)) \ar[]{rd}[]{\mu_{A, B\otimes C}} \& \& \& F(A \otimes B) \boxtimes F(C) \ar[]{ld}[]{\mu_{A\otimes B, C}} \\[30pt]
                        \& F(A \otimes (B \otimes C)) \ar[]{r}[]{F(\alpha^\mathcal{C}_{A,B,C})} \& F((A \otimes B) \otimes C)
                    \end{tikzcd}
                } \\
                \scalebox{1}{
                    \begin{tikzcd}[ampersand replacement = \&]
                        F(A) \boxtimes W \ar[]{r}[]{\rho_{F(A)}^\mathcal{D}} \ar[]{d}[]{F(A) \boxtimes \upsilon} \& F(A) \\
                        F(A) \boxtimes F(Z) \ar[]{r}[]{\mu_{A,Z}} \& F(A \otimes Z) \ar[]{u}[]{F(\rho_A^\mathcal{C})}
                    \end{tikzcd}
                }
                \scalebox{1}{
                    \begin{tikzcd}[ampersand replacement = \&]
                        W \boxtimes F(A) \ar[]{r}[]{\lambda^\mathcal{D}_{F(A)}} \ar[]{d}[]{\upsilon \boxtimes F(A)} \& F(A) \\
                        F(Z) \boxtimes F(A) \ar[]{r}[]{\mu_{Z,A}} \& F(Z \otimes A) \ar[]{u}[]{F(\lambda^\mathcal{C}_A)}
                    \end{tikzcd}
                }
            \end{center}
        \end{definition}

        The monoidal functor is said to be strong monoidal if $\mu$ is a natural isomorphism and $\upsilon$ is an isomorphism. If the morphisms $\mu$ and $\upsilon$ are given by identities, then we say that the functor is strict monoidal.

        \begin{definition}[Monoidal natural transformation]
            Let $F, G: \mathcal{C} \rightarrow \mathcal{D}$ be lax monoidal functors between monoidal categories. We say that a natural transformation $\theta: F \Rightarrow G$ is a monoidal natural transformation if the following diagrams commute
            \begin{center}
                \scalebox{1}{
                    \begin{tikzcd}[ampersand replacement = \&]
                        F(A) \boxtimes F(B) \ar[]{r}[]{\mu_A,B^F} \ar[]{d}[]{\theta_A \boxtimes \theta_B} \& F(A \otimes B) \ar[]{d}[]{\theta_{A\otimes B}} \\
                        G(A) \boxtimes G(B) \ar[]{r}[]{\mu_{A,B}^G} \& G(A \otimes B)
                    \end{tikzcd}
                }\quad
                \scalebox{1}{
                    \begin{tikzcd}[ampersand replacement = \&]
                        \& F(Z) \ar[]{dd}[]{\theta_Z} \\[-10pt]
                        W \ar[]{ru}[]{\upsilon^F} \ar[]{rd}[]{\upsilon^G} \\[-10pt]
                        \& G(Z)
                    \end{tikzcd}
                }
            \end{center}
        \end{definition}

        \begin{definition}[Braided monoidal category]
            Let $\mathcal{C}$ be a monoidal category. We say that the category is braided if it comes equipped with natural isomorphisms
            \begin{align*}
                \beta_{A,B} : A \otimes B \rightarrow B \otimes A\tt{,}
            \end{align*}
            which has the following commutative diagrams for any $A$, $B$ and $C$. 
            \begin{center}
                \scalebox{1}{
                    \begin{tikzcd}[ampersand replacement = \&]
                        A \otimes Z \ar[]{rr}[]{\beta_{A,Z}} \ar[]{rd}[]{\rho_A} \&[-10pt] \&[-10pt] Z \otimes A \ar[]{ld}[]{\lambda_A} \\
                        \& A
                    \end{tikzcd}
                } \\
                \scalebox{0.75}{
                    \begin{tikzcd}[ampersand replacement = \&]
                        \&[-80pt] (A \otimes B) \otimes C \ar[]{ld}[]{\alpha^{-1}_{A,B,C}} \ar[]{r}[]{\beta_{A,B\otimes C}} \&[30pt] C \otimes (A \otimes B) \ar[]{rd}[]{\alpha_{C,A,B}} \&[-80pt] \\[30pt]
                        A \otimes (B \otimes C) \ar[]{rd}[]{A\otimes \beta_{B,C}} \& \& \& (C \otimes A) \otimes B \ar[]{ld}[]{\beta_{C,A}\otimes B} \\[30pt]
                        \& A \otimes (C \otimes B) \ar[]{r}[]{\alpha_{A,C,B}} \& (A \otimes C) \otimes B
                    \end{tikzcd}
                }
                \scalebox{0.75}{
                    \begin{tikzcd}[ampersand replacement = \&]
                        \&[-80pt] A \otimes (B \otimes C) \ar[]{ld}[]{\alpha_{A,B,C}} \ar[]{r}[]{\beta_{A,B\otimes C}} \&[30pt] (B \otimes C) \otimes A \ar[]{rd}[]{\alpha^{-1}_{B,C,A}} \&[-80pt] \\[30pt]
                        (A \otimes B) \otimes C \ar[]{rd}[]{\beta_{A,B}\otimes C} \& \& \& B \otimes (C \otimes A) \ar[]{ld}[]{B \otimes \beta_{C,A}} \\[30pt]
                        \& (B \otimes A) \otimes C \ar[]{r}[]{\alpha^{-1}_{B,A,C}} \& B \otimes (A \otimes C)
                    \end{tikzcd}
                }
            \end{center}
        \end{definition}
        \begin{definition}[Symmetric monoidal category]
            A braided monoidal category $\mathcal{C}$ is called symmetric if the braiding $\beta$ is chosen so that it has its own inverses, i.e., the following diagram commutes.
            \begin{center}
                \begin{tikzcd}
                    A\otimes B \ar[equal]{rr}[]{} \ar[]{rd}[]{\beta_{A,B}} &[-10pt] &[-10pt] A \otimes B \\
                    & B \otimes A \ar[]{ru}[]{\beta_{B,A}}
                \end{tikzcd}
            \end{center}
        \end{definition}

        In the case of symmetric braiding, one only has to check that either one of the braiding hexagons commutes, as the other follows from symmetry.

        \begin{definition}[Braided lax monoidal functor]
            We say that a monoidal functor $F: \mathcal{C} \rightarrow \mathcal{D}$ between braided categories is braided if it commutes with braiding in the sense of the following commutative diagram.
            \begin{center}
                \begin{tikzcd}
                    F(A) \boxtimes F(B) \ar[]{d}[]{\mu_{A,B}} \ar[]{r}[]{\beta_{F(A),F(B)}^\mathcal{D}} & F(B) \boxtimes F(A) \ar[]{d}[]{\mu_{B,A}} \\
                    F(A \otimes B) \ar[]{r}[]{F(\beta_{A,B}^\mathcal{C})} & F(B \otimes A) 
                \end{tikzcd}
            \end{center}
        \end{definition}

        \begin{definition}[Closed symmetric monoidal category]
            A symmetric monoidal category $(\mathcal{C}, \otimes, Z)$ is said to be closed if for any $C \in \mathcal{C}$, the functor $\argument \otimes C : \mathcal{C} \rightarrow \mathcal{C}$ has a right adjoint $[C, \argument] : \mathcal{C} \rightarrow \mathcal{C}$. The object $[C, D]$ is usually called the internal hom of $C$.
        \end{definition}

\end{document}