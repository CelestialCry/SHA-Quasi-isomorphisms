\documentclass[../thesis.tex]{subfiles}



\begin{document}

    This appendix is a short exposition on the theory of monads and comonads. The results we use may be found in Riehl \cite{Riehl16} or Mac Lane \cite{MacLane71}. 
    \section{Monads and Categories of Algebras}
        \begin{definition}[Monad]
            Let $\mathcal{C}$ be a category. We say that an endofunctor $T : \mathcal{C} \rightarrow \mathcal{C}$ together with
            \begin{itemize}
                \item a multiplication $\mu : M\circ M \Rightarrow M$
                \item and a unit $\eta : \tt{Id}_{\mathcal{C}} \Rightarrow M$
            \end{itemize}
            is a monad, if the following diagrams commute
            \begin{center}
                \begin{tikzcd}
                    M\circ M\circ M \ar[]{r}[]{M(\mu)} \ar[]{d}[]{\mu_M} & M\circ M \ar[]{d}[]{\mu} \\
                M\circ M \ar[]{r}[]{\mu} & M
                \end{tikzcd}
                \begin{tikzcd}
                    M \circ \tt{Id}_{\mathcal{C}} \ar[]{r}[]{M(\eta)} \ar[equal]{rd}[]{} & M\circ M \ar[]{d}[]{\mu} & \tt{Id}_{\mathcal{C}}\circ M \ar[]{l}[above]{\eta_M} \ar[equal]{ld}[]{} \\
                & M
                \end{tikzcd}
            \end{center}
        \end{definition}
        In other words, a monad is a monoid in the category of endofunctors, $(T, \mu, \eta) \in (\tt{End}\mathcal{C}, \circ, \tt{Id}_\mathcal{C})$.

        \begin{lemma}[Monads from adjunctions, {\cite[Lemma 5.1.3.][155]{Riehl14}}]
            Given an adjunction $F \dashv G : \mathcal{C} \rightarrow \mathcal{D}$ and
            \begin{itemize}
                \item a unit $\eta : \tt{Id}_\mathcal{C} \Rightarrow GF$
                \item and a counit $\varepsilon : FG \Rightarrow \tt{Id}_\mathcal{D}$,
            \end{itemize} 
            there is an associated monad $(T, \mu, \eta)$. Let $T = GF$, together with
            \begin{itemize}
                \item a multiplication given by the counit $\mu = G(\varepsilon_F) : T\circ T \Rightarrow T$
                \item and the unit $\eta : \tt{Id}_\mathcal{C} \Rightarrow T$,
            \end{itemize}
            is a monad on $\mathcal{C}$.
        \end{lemma}

        Given any monad $(T : \mathcal{C} \rightarrow \mathcal{C}, \mu, \eta)$, we say that an object $M \in \mathcal{C}$ is a $T$-algebra if there exists a morphism $m : T(M) \rightarrow M$ such that the following diagrams commute
        \begin{center}
            \begin{tikzcd}
                T\circ T(M) \ar[]{r}[]{T(m)} \ar[]{d}[]{\varepsilon_M} & T(M) \ar[]{d}[]{m} \\
                T(M) \ar[]{r}[]{m} & M
            \end{tikzcd}
            \begin{tikzcd}
                M \ar[]{r}[]{\eta_M} \ar[equal]{rd}[]{} & T(M) \ar[]{d}[]{m} \\
                & M
            \end{tikzcd}
        \end{center}

        If $M$ and $N$ are two $T$-algebras, then we say that a morphism $f : M \rightarrow N$ is a $T$-algebra morphism if the following diagram commute
        \begin{center}
            \begin{tikzcd}
                T(M) \ar[]{r}[]{T(f)} \ar[]{d}[]{m} & T(N) \ar[]{d}[]{n} \\
                M \ar[]{r}[]{f}& N
            \end{tikzcd}
        \end{center}
        \begin{definition}[Eilenberg-Moore category]
            The Eilenberg-Moore category or the category of algebras $\mathcal{C}^T$ is the category having
            \begin{itemize}
                \item objects as $M$ as $T$-algebras
                \item and morphisms $f : M \rightarrow N$ as $T$-algebra morphisms.
            \end{itemize}
        \end{definition}
        
        There is a free functor from $\mathcal{C}$ to $T$-algebras
        \begin{align*}
            F^T : \mathcal{C} & \rightarrow \mathcal{C}^T\tt{,} \\
            M & \mapsto (T(M), \mu_M)\tt{.}
        \end{align*}
        By forgetting the $T$-algebra structure we obtain a forgetful functor
        \begin{align*}
            U^T : \mathcal{C}^T & \rightarrow \mathcal{C}\tt{,} \\
            (M, m) & \mapsto M\tt{.}
        \end{align*}
        The next lemma justifies calling these functors for free and forgetful.        
        \begin{lemma}[Adjunctions from monads, {\cite[Lemma 5.2.8][162]{Riehl14}}]
            Given any monad $(T, \mu, \eta) : \mathcal{C} \rightarrow \mathcal{C}$, then the pair of functors $F^T$ and $U^T$ defines an adjunction
            \begin{align*}
                F^T \dashv U^T : \mathcal{C} \rightarrow \mathcal{C}^T\tt{.}
            \end{align*}
        \end{lemma}

        \begin{definition}[Free $T$-algebra]
            $(M,m)$ is a free $T$-algebra if there is an object $N\in \mathcal{C}$ and an isomorphism $(M,m) \simeq F^T(N)$.
        \end{definition}
        In the category of algebras $\mathcal{C}^T$, we may approximate every $T$-algebra $M$ by free $T$-algebras. This means that we may construct a canonical free resolution of any $T$-algebra $M$.
        \begin{proposition}[Free resolutions, {\cite[Proposition 5.4.3][169]{Riehl14}}]
            Given any $T$-algebra $M$, then
            \begin{center}
                \begin{tikzcd}
                    ((T\circ T)(M), \mu_{TM}) \ar[yshift = 0.5ex]{r}[]{Tm} \ar[yshift = -0.5ex]{r}[below]{\mu_M} & (TM, \mu_M) \ar[]{r}[]{m} & (M,m)
                \end{tikzcd}
            \end{center}
            is a colimit diagram in $\mathcal{C}^T$.
        \end{proposition} 

        It is useful to recognize when a category is a category of some algebra. Then every object is generated by every free objects, which may arise from a simpler category.
        \begin{definition}[Monadicity]
            Suppose that there is an adjunction $F \vdash G : \mathcal{C} \rightarrow \mathcal{D}$ and that $T = GF$. We say that the adjunction, or $G : \mathcal{D} \rightarrow \mathcal{C}$, is monadic if there exists an equivalence of categories $K : \mathcal{D} \rightarrow \mathcal{C}^T$ such that there are natural isomorphisms $G \simeq U^T\circ K$ and $F^T \simeq K \circ F$.
            \begin{center}
                \begin{tikzcd}
                    \mathcal{D} \ar[xshift = -1ex]{dr}[below]{G} \ar[]{rr}[]{K}[below]{\simeq} & & \mathcal{C}^T \ar[xshift = -1ex]{dl}[above]{U^T} \\
                    & \mathcal{C} \ar[xshift = 1ex]{lu}[above]{F} \ar[xshift = 1ex]{ru}[below]{F^T}
                \end{tikzcd}
            \end{center}
        \end{definition}

        Many of the categories which we consider are monadic.
        \begin{example}[Ab is monadic over Set, {\cite[Corollary 5.5.3][174]{Riehl14}}]
            Consider the adjoint pair of functors $\mathbb{Z}\argument \dashv \tt{forget}: \tt{Set} \rightarrow \tt{Ab}$, where we define
            \begin{align*}
                \mathbb{Z}\argument : \tt{Set} & \rightarrow \tt{Ab}\tt{,} \\
                M & \mapsto \mathbb{Z}M\tt{.}
            \end{align*} 
            The binary operation on the group is given by formal linear combinations. This adjoint pair is monadic.
        \end{example}

        \begin{example}[$\tt{Mod}^R$ is monadic over $\tt{Mod}^\mathbb{K}$]
            The adjoint pair of functors $\argument\otimes_\mathbb{K}R \dashv \tt{forget} : \tt{Mod}^\mathbb{K} \rightarrow \tt{Mod}^R$ is monadic.
        \end{example}

        \begin{example}[$\tt{Alg}_{\mathbb{K},+}$ is monadic over $\tt{Mod}^\mathbb{K}$]
            The adjoint pair $T\argument \dashv \tt{forget} : \tt{Mod}^\mathbb{K} \rightarrow \tt{Alg}_{\mathbb{K},+}$, where $T$ is the tensor algebra, is monadic.
        \end{example}

        One very good property about categories of algebras, is that their small limits is very well-behaved. These are exactly the same limits as in $\mathcal{C}$. We have the following result:
        \begin{thm}[{\cite[Theorem 5.6.5][181]{Riehl14}}]
            A monadic functor $G : \mathcal{D} \rightarrow \mathcal{C}$
            \begin{itemize}
                \item creates any limits which $\mathcal{C}$ has,
                \item and creates any colimits $\mathcal{C}$ has and which are preserved by the monad $T$ and its square $T\circ T$.
            \end{itemize}
        \end{thm}

    \section{Comonads and Categories of Coalgebras}
        In this section we will dualize the definitions and results from the last section. One could think of the dual themselves, but we do this for clearity.

    \begin{definition}[Comonad]
        Let $\mathcal{C}$ be a category. We say that an endofunctor $W : \mathcal{C} \rightarrow \mathcal{C}$ together with
        \begin{itemize}
            \item a comulitplication $\nu : W \Rightarrow W\circ W$
            \item and a counit $\varepsilon : W \Rightarrow \tt{Id}_\mathcal{C}$
        \end{itemize}
        is a comonad, if the following diagrams commute
        \begin{center}
            \begin{tikzcd}
                W\circ W\circ W & W\circ W \ar[]{l}[above]{W(\nu)} \\
                W\circ W \ar[]{u}[]{\nu_W} & W \ar[]{l}[]{\nu} \ar[]{u}[]{\nu}
            \end{tikzcd}
            \begin{tikzcd}
                W\circ \tt{Id}_\mathcal{C} & W\circ W \ar[]{l}[above]{W(\varepsilon)} \ar[]{r}[]{\varepsilon_W} & \tt{Id}_\mathcal{C}\circ W \\
                & W \ar[]{u}[]{\nu} \ar[equal]{lu}[]{} \ar[equal]{ru}[]{}
            \end{tikzcd}
        \end{center}
    \end{definition}

    \begin{lemma}[Comonads from adjunctions]
        Given an adjunction $F \dashv G : \mathcal{C} \rightarrow \mathcal{D}$ with
        \begin{itemize}
            \item unit $\eta : \tt{Id}_\mathcal{C} \Rightarrow GF$
            \item and a counit $\varepsilon : FG \Rightarrow \tt{Id}_\mathcal{D}$,
        \end{itemize}
        there is an associated comonad $(W,\nu, \varepsilon)$. Let $W = FG$, together with
        \begin{itemize}
            \item a comulitplication given by the unit $\nu = F(\eta_G) : W \Rightarrow W\circ W$
            \item and the counit $\varepsilon : W \Rightarrow \tt{Id}_\mathcal{D}$
        \end{itemize}
        is a comonad on $\mathcal{D}$.
    \end{lemma}

    Given any comonad $(W : \mathcal{D} \rightarrow \mathcal{D}, \nu, \varepsilon)$, we say that $M$ is a $W$-coalgebra if there exists a morphism $w : M \rightarrow W(M)$ such that the following diagrams commute
    \begin{center}
        \begin{tikzcd}
            W\circ W(M) & W(M) \ar[]{l}[above]{W(w)} \\
            W(M) \ar[]{u}[]{\nu_M} & M \ar[]{l}[above]{w} \ar[]{u}[]{w}
        \end{tikzcd}
        \begin{tikzcd}
            M & W(M) \ar[]{l}[above]{\varepsilon_M} \\
            & M \ar[equal]{lu}[]{} \ar[]{u}[]{w}
        \end{tikzcd}
    \end{center}

    Given two $W$-coalgebras $M$ and $N$ we say that a morphism $f : M \rightarrow N$ is a $W$-coalgebra morphism if the following diagram commutes
    \begin{center}
        \begin{tikzcd}
            W(M) \ar[]{r}[]{W(f)} & W(N) \\
            M \ar[]{u}[]{w} \ar[]{r}[]{f} & N \ar[]{u}[]{u}
        \end{tikzcd}
    \end{center}

    \begin{definition}[Category of coalgebras]
        The category of coalgebras $\mathcal{C}_W$ is the category having
        \begin{itemize}
            \item objects $M$ as $W$-coalgebras
            \item and morphisms $f : M \rightarrow N$ as $W$-coalgebra morphisms.
        \end{itemize}
    \end{definition}

    There is a cofree functor from $\mathcal{D}$ to $W$-coalgebras
    \begin{align*}
        F_W : \mathcal{D} & \rightarrow \mathcal{D}_W\tt{,} \\
        M & \mapsto (W(M),\nu_M)\tt{.}
    \end{align*}
    By forgetting the $W$-coalgebra structure we obtain a forgetful functor
    \begin{align*}
        U_W : \mathcal{D}_W & \rightarrow \mathcal{D}\tt{,} \\
        (M, w) & \mapsto M\tt{.}
    \end{align*}
    \begin{lemma}[Adjunctions from comonads]
        Given any comonad $(W, \nu, \varepsilon) : \mathcal{D} \rightarrow \mathcal{D}$, the the pair of functors $U_W$ and $F_W$ defines an adjunction
        \begin{align*}
            U_W \dashv F_W : \mathcal{D}_W \rightarrow \mathcal{D}\tt{.}
        \end{align*} 
    \end{lemma}

    In the category of coalgebras $\mathcal{D}_W$ every object may be cogenerated from cofree $W$-coalgebras.
    \begin{definition}[Cofree $W$-coalgebras]
        $(M,w)$ is a cofree $W$-coalgebra if there is an object $N \in \mathcal{D}$ and an isomorphism $(M,w) \simeq F_W(N)$.
    \end{definition}
    \begin{proposition}[Cofree resolutions]
        Given any $W$-coalgebra $M$, then
        \begin{center}
            \begin{tikzcd}
                (M,m) \ar[]{r}[]{w} & (W(M),\nu_M) \ar[yshift = 0.5ex]{r}[above]{W(w)} \ar[yshift = -0.5ex]{r}[below]{\nu_M} & (W\circ W(M), \nu_{W(M)})
            \end{tikzcd}
        \end{center}
        is a limit diagram in $\mathcal{D}_W$.
    \end{proposition}

    \begin{definition}[Comonadicity]
        Suppose that there is an adjunction $F \dashv G : \mathcal{C} \rightarrow \mathcal{D}$ such that $W = FG$. We say that the adjunction, or the $F : \mathcal{C} \rightarrow \mathcal{D}$, is comonadic if there exists an equivalence of categories $K : \mathcal{D}_W \rightarrow \mathcal{C}$ such that there are natural isomorphisms $F\circ K \simeq F_W$ and $K\circ U_W \simeq G$.
    \end{definition}

    As we would except, we have the comonadic categories.
    \begin{example}[$\tt{coMod}^C$ is comonadic over $\tt{Mod}^\mathbb{K}$]
        The adjoint pair of functors $\tt{forget} \dashv \argument\otimes_\mathbb{K} C : \tt{coMod}^C \rightarrow \tt{Mod}^\mathbb{K}$ is comonadic.
    \end{example}

    \begin{example}[$\tt{coAlg}_{\mathbb{K},\tt{conil}}$ is comonadic over $\tt{Mod}^\mathbb{K}$]
        The adjoint pair of functors $\tt{forget} \dashv T^c\argument : \tt{coAlg}_{\mathbb{K},\tt{conil}} \rightarrow \tt{Mod}^\mathbb{K}$.
    \end{example}

    \begin{thm}\label{thm: limit-creation}
        A comonadic functor $F : \mathcal{C} \rightarrow \mathcal{D}$
        \begin{itemize}
            \item creates any colimits which $\mathcal{D}$ has
            \item and creates and limits $\mathcal{D}$ has and which are preserved by the comonad $W$ and its square $W\circ W$.
        \end{itemize}
    \end{thm}

    \section{Canonical Resolutions}
        As described by MacLane \cite[180]{MacLane71}: "Monads and their duals, the comonads, play via $\Delta$ a central role in homological algebra,...". We will here look at a method to construct resolutions associated to comonads.

        Let $(W, \nu, \varepsilon)$ be a comonad over an abelian category $\mathcal{D}$, then this is a comonoid in the category of endofunctors $(\tt{End}\mathcal{D}, \circ, \tt{Id}_\mathcal{D})$. By Proposition \ref{prop: universal-monoid}, there is then a strong monoidal functor, which we denote by $W^{\circ ?}$, $W^{\circ ?} : \Delta_+^{\tt{op}} \rightarrow \tt{End}\mathcal{D}$. Using the standard representation of simplicial objects we see that the face and degeneracy maps are given as
        \begin{center}
            \begin{tikzcd}
                \tt{Id}_\mathcal{D} & W \ar[]{r}[]{\nu} & W^{\circ 2} \ar[yshift = 0.5ex]{r}[]{W(\nu)} \ar[yshift = -0.5ex]{r}[below]{\nu_W} & W^{\circ 3} \ar[yshift = 1ex]{r}[]{} \ar[]{r}[]{} \ar[yshift = -1ex]{r}[]{} & \cdots 
            \end{tikzcd} \\
            \begin{tikzcd}
                \tt{Id}_\mathcal{D} & W \ar[]{l}[]{\varepsilon} & W^{\circ 2} \ar[yshift = 0.5ex]{l}[above]{W(\varepsilon)} \ar[yshift = -0.5ex]{l}[]{\varepsilon_W} & W^{\circ 3} \ar[yshift = 1ex]{l}[]{} \ar[]{l}[]{} \ar[yshift = -1ex]{l}[]{} & \cdots \ar[yshift = 1.5ex]{l}[]{} \ar[yshift=0.5ex]{l}[]{} \ar[yshift = -0.5ex]{l}[]{} \ar[yshift = -1.5ex]{l}[]{}
            \end{tikzcd}
        \end{center}
        Let $M$ be an object of $\mathcal{D}$. Evaluating $W^{\circ ?}$ at $M$ gives us a functor $W^{\circ ?}(M) : \Delta^{\tt{op}}_+ \rightarrow \mathcal{D}$. This may be made into a cochain complex by Example \ref{ex: ass-complex},
        \begin{center}
            \begin{tikzcd}
                \cdots \ar[]{r}[]{} & W^{\circ 3}(M) \ar[]{r}[]{} & W^{\circ 2}(M) \ar[]{r}[]{} & W(M) \ar[]{r}[]{\varepsilon_M} & M \ar[]{r}[]{} & 0 \ar[]{r}[]{} & \cdots \tt{.}
            \end{tikzcd}
        \end{center}

        \begin{definition}[Canonical $W$-resoultion]
            The cochain complex as defined above is the canonical $W$-resolution at $M$.
        \end{definition}

        These canonical resolution is more of a recipe to see how a comonad on an abelian category induces a resolution.

        \begin{example}[Free resolution]
            Let $R$ be a $\mathbb{K}$-algebra. Then there is an adjunction $\argument \otimes_\mathbb{K}R \dashv \tt{forget} : \tt{Mod}^\mathbb{K} \rightarrow \tt{Mod}^R$. The comonad $\argument \otimes_\mathbb{K} R : \tt{Mod}^R \rightarrow \tt{Mod}^R$ induces free $R$-resolutions on every right $R$-module $M$. 
            \begin{center}
                \begin{tikzcd}
                    \cdots \ar[]{r}[]{} & M \otimes_\mathbb{K} R^{\otimes 3} \ar[]{r}[]{} & M \otimes_\mathbb{K} R^{\otimes 2} \ar[]{r}[]{} & M \otimes_\mathbb{K} R \ar[]{r}[]{} & M \ar[]{r}[]{} & 0 \ar[]{r}[]{} & \cdots
                \end{tikzcd}
            \end{center}
        \end{example}
\end{document}