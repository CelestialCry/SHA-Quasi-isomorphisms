\documentclass[../../../thesis.tex]{subfiles}

\begin{document}    
    
    \section{Localization}
    
        We will briefly explain the localization of a category at a set. We will look at local objects and localizations by adjoint functors. This section is based on \cite{Krause21}.

        \subsection{Definition and Local Objects}

            The localization of a category at some set of morphisms is a universal category where these morphisms have been inverted. In this manner, we will think of the localization as adding formal inverses to the morphisms in this set to the category.

            \begin{definition}
                Let $\mathcal{C}$ be a category and $S \subseteq Mor(\mathcal{C})$ a set of morphisms in $\mathcal{C}$. We define the localization of $\mathcal{C}$ at $S$ as a category $\mathcal{C}[S^{-1}]$ together with a functor $Q : \mathcal{C} \rightarrow \mathcal{C}[S^{-1}]$ satisfying the properties:
                \begin{enumerate}
                    \item For any morphism $s\in S$, $Qs$ is an invertible morphism.
                    \item Any functor $F : \mathcal{C} \rightarrow \mathcal{D}$ satisfying $Fs$ is invertible if $s\in S$ factors through $Q$. Thus there exists a unique functor $\bar{F} : \mathcal{C}[S]^{-1} \rightarrow \mathcal{D}$ and a natural isomorphism $\alpha : F \implies \bar{F} \circ Q$ relating the functors.
                \end{enumerate}

                \begin{center}
                    \begin{tikzcd}
                        \mathcal{C} \ar{rd}{Q} \ar{rr}{F} & & \mathcal{D} \\
                        & \mathcal{C}[S^{-1}] \ar[dashed]{ur}{\bar{F}}
                    \end{tikzcd}
                \end{center}
            \end{definition}

            Informally, the category $\mathcal{C}[S^{-1}]$ may be constructed as the category with the same objects as $\mathcal{C}$. The morphisms are constructed with paths of morphisms in $\mathcal{C}$, where we also allow formal inverse paths $s^{-1}$ for any morphism in $S$. The constant paths would be the new identity morphisms, while every path from an object $X$ to $Y$ represents a morphism.

            \begin{remark}
                Beware whenever localizing a locally small category. Adding more morphisms to this category may make some set of morphisms between objects big.
            \end{remark}

            By the universality of the functor $Q$ we have the following lemma.

            \begin{lemma}
                Let $\mathcal{D}$ be a category, then the pre-composition functor is fully faithful.
                \begin{align*}
                    \_ \circ Q : Fun(\mathcal{C}[S^{-1}],\mathcal{D}) & \rightarrow Fun(\mathcal{C}, \mathcal{D}) \\
                    F & \mapsto F \circ Q
                \end{align*}
                Moreover, we may identify $Fun(\mathcal{C}[S^{-1}],\mathcal{D})$ as the full subcategory of functors in $Fun(\mathcal{C}, \mathcal{D})$ which sends every morphism in $S$ to an isomorphism.
            \end{lemma}

            We proceed to define $S$-local objects, and informally, these are all the objects which does not change after the localization process.

            \begin{definition}
                An object $Y\in \mathcal{C}$ is called $S$-local ($S$-closed or $S$-orthogonal) if for any $s\in S$, the map $\mathcal{C}(s, Y)$ is a bijection. Define $S\perp$ to be the full subcategory of $S$-local objects.
            \end{definition}

            The following lemma gives an equivalent definition for "localness".

            \begin{lemma}
                An object $Y\in \mathcal{C}$ is $S$-local if and only if the canonical map $q$ is a natural isomorphism.
                \begin{align*}
                    q_{X,Y} : \mathcal{C}(X,Y) \rightarrow \mathcal{C}[S^{-1}](X,Y)
                \end{align*}
            \end{lemma}

        \subsection{Localizing at Adjoint Functors}

            Given two functors $F : \mathcal{C} \rightarrow \mathcal{D}$ and $G : \mathcal{D} \rightarrow \mathcal{C}$ such that they form an adjoint pair $F \dashv G$, we want to see how they react with localizations.

            \begin{proposition}
                Let $F \dashv G$ be an adjoint pair as above, and define $S = $\startset{$s\in Mor\mathcal{C}$ $\mid$ $Fs$ is invertible}. Draw out the following diagram, with the relations $F = \bar{F}\circ Q$.
                \begin{center}
                    \begin{tikzcd}
                        & \mathcal{C} \ar{ld}[description]{Q} \ar[bend right = 20]{dd}[description]{F} \\
                        \mathcal{C}[S^{-1}] \ar{rd}[description]{\bar{F}} \\
                        & \mathcal{D} \ar[bend right = 20]{uu}[description]{G}
                    \end{tikzcd}
                \end{center}
                The following are equivalent:
                \begin{enumerate}
                    \item $G$ is fully faithful.
                    \item The counit $\varepsilon : FG \implies Id_\mathcal{D}$ is invertible for any object $X\in \mathcal{D}$.
                    \item The functor $F$ induces an equivalence $\bar{F}:\mathcal{C}[S^{-1}] \rightarrow \mathcal{D}$.
                \end{enumerate}
            \end{proposition}
        
            In the light of the last proposition, we may define "short exact sequences" of pre-additive categories.

            \begin{definition}
                Suppose there is a diagram of additive functors like below.
                \begin{center}
                    \begin{tikzcd}[column sep = large]
                        \mathcal{A} \ar[bend left = 30, tail]{r}{E} \ar[phantom]{r}{\bot} & \mathcal{B} \ar[bend left = 30, two heads]{r}{F} \ar[bend left = 30, two heads]{l}{E_\rho} \ar[phantom]{r}{\bot} & \mathcal{C} \ar[bend left = 30, tail]{l}{F_\rho}    
                    \end{tikzcd}
                \end{center}
                It is called a localization sequence whenever the following holds:
                \begin{enumerate}
                    \item $(E,E_\rho)$ and $(F,F_\rho)$ are adjoint pairs.
                    \item $E$ and $F_\rho$ are fully faithful.
                    \item $ImE \simeq KerF$.
                \end{enumerate}
            \end{definition}

\end{document}