\documentclass[../thesis.tex]{subfiles}

\begin{document}    

    Quillen envisioned a more general approach to homotopy theory, which he dubbed homotopical algebra. A homotopy theory was then enclosed by the structure of a model category, then a closed model category. Many of the results from classical homotopy theory was then recovered in this new setting of model categories. The theorem which we are concerned about is Whiteheads theorem:

    \begin{thm}[Whiteheads Theorem]
        Let $X$ and $Y$ be two CW-complexes. If $f:X\rightarrow Y$ is a weak equivalence, then it is also a homotopy equivalence. I.e. there exists a morphism $g: Y\rightarrow X$ such that $gf\sim id_X$ and $fg\sim id_Y$.
    \end{thm}

    If we employ Quillens model category onto the category Top, we get that a space $X$ is bifibrant if and only if it is a CW-complex. The natural generalization is then to not ask of $X$ to be a CW-complex, but a bifibrant object.

    \begin{thm}[Generalized Whiteheads Theorem]
        Let $\mathcal{C}$ be a model category. Suppose that $X$ and $Y$ are bifibrant objects of $\mathcal{C}$, and that there is a weak-equivalence $f:X\rightarrow Y$. Then $f$ is also a homotopy equivalence, i.e. there exists a morphism $g: Y\rightarrow X$ such that $gf\sim id_X$ and $fg\sim id_Y$.
    \end{thm}

    The category of differential graded (co)algebras employs such a model category. Here we let the weak-equivalences be quasi-isomorphisms. Moreover, in this case the bar and cobar construction is a Quillen equivalence between the model structures. As we will see, a dg-coalgebra will be bifibrant exactly when it is an $A_\infty$-algebra. Thus, by Whiteheads theorem, quasi-isomorphisms lift to homotopy equivalences. In this case the derived category of $A_\infty$-algebras is equivalent to the homotopy category of $A_\infty$-algebras.

    We will conclude this section by looking at the category of algebras as a subcategory of $A_\infty$-algebras. The derived category may then be expressed as the homotopy category $A_\infty$-algebras, restricted to algebras.

    \section{Model categories}

        In this section we will define Quillens model category. As one may see is that in practice there are a plethora of semantically different definitions of model categories, however they are all made to be equivalent. The difference comes down to preference. In this thesis we will use the definitions as they are developed in Mark Hoveys book \cite{Hovey99}. We will then go on to prove the basic results known about model categories, its associated homotopy category and Quillen functors between model categories.

        Before we state the definition of a model category we need some preliminary definitions. For this section, let $\mathcal{C}$ be a category.

        \begin{definition}[Retract]
            A morphism $f:A\rightarrow B$ in $\mathcal{C}$ is a retract of a morphism $g: c\rightarrow D$ if it fits in a commutative diagram:
            \begin{center}
                \begin{tikzcd}
                    A \ar[]{d}[]{f} \ar[]{r}[]{} \ar[bend left]{rr}[]{id_A}& C \ar[]{d}[]{g} \ar[]{r}[]{} & A \ar[]{d}[]{f} \\
                    B \ar[]{r}[]{} \ar[bend right]{rr}[below]{id_B} & D \ar[]{r}[]{} & B
                \end{tikzcd}
            \end{center}
        \end{definition}

        \begin{definition}[Functorial factorization]
            A pair of functors $\alpha, \beta: \mathcal{C}^\rightarrow\rightarrow\mathcal{C}^\rightarrow$ is called a functorial factorization if for any morphism $f = \beta(f)\circ\alpha(f)$. We will denote the morphisms in the factorization as $f_\alpha$ and $f_\beta$. The functorial factorization may be depicted by the following commutative diagram:
            \begin{center}
                \begin{tikzcd}
                    A \ar[]{rr}[]{f} \ar[]{rd}[below]{f_\alpha} & & B \\
                    & C \ar[]{ru}[below]{f_\beta}
                \end{tikzcd}
            \end{center}
        \end{definition}

        \begin{definition}[Lifting properties]
            Suppose that the morphisms $i: A \rightarrow B$ and $p: C \rightarrow D$ fits inside a commutative square. $i$ is said to have the left lifting property with resptect to $p$, or $p$ has the right lifting property with respect to $i$, if there is an $h : B \rightarrow C$ such that the two triangles commute.
            \begin{center}
                \begin{tikzcd}
                    A \ar[]{r}[]{} \ar[]{d}[]{i} & C \ar[]{d}[]{p} \\
                    B \ar[]{r}[]{} \ar[dashed]{ru}[]{h} & D
                \end{tikzcd}
            \end{center}
        \end{definition}

        \begin{remark}
            We will call the left lifting property for LLP and the right lifting property for RLP.
        \end{remark}

        \subsection{Model categories}

            \begin{definition}[Model category]
                Let $\mathcal{C}$ be a category with all finite limits and colimits. $\mathcal{C}$ admits a model structure if there are three wide subcategories each defining a class of morphisms:
                \begin{itemize}
                    \item $Ac\subset Mor(\mathcal{C})$ are called weak equivalences
                    \item $Cof\subset Mor(\mathcal{C})$ are called cofibrations
                    \item $Fib\subset Mor(\mathcal{C})$ are called fibrations
                \end{itemize}
                In addition we call morphisms in $Cof\cap Ac$ for acyclic cofibrations and $Fib\cap Ac$ for acyclic fibrations. Moreover, $\mathcal{C}$ has two functorial factorizations $(\alpha, \beta)$ and $(\gamma, \delta)$. The following axioms should be satisfied:
                \begin{itemize}
                    \item[\textbf{MC1}] The class of weak equivalences satisfy the $2$-out-of-$3$ property, i.e. if $f$ and $g$ are composable morphisms such that $2$ out of $f$, $g$ and $gf$ are weak equivalences, then so is the third.
                    \item[\textbf{MC2}] The three classes $Ac$, $Cof$ and $Fib$ are retraction closed, i.e. if $f$ is a retraction of $g$, and $g$ is either a weak-equivalence, cofibration or fibration, then so is $f$.
                    \item[\textbf{MC3}] The class of cofibrations have the left lifting property with respect to acyclic fibrations, and fibrations have the right lifting property with respect to acyclic cofibrations.
                    \item[\textbf{MC4}] Given any morphism $f$, $f_\alpha$ is a cofibration, $f_\beta$ is an acyclic fibration, $f_\gamma$ is an acyclic cofibration and $f_\delta$ is a fibration.      
                \end{itemize}
            \end{definition}

            \begin{remark}
                The class $Ac$ has every isomorphism. This is because every isomorphism is a retract of some identity morphism.
            \end{remark}

            \begin{remark}
                The type of category which has been introduced above was first called a closed model category by Quillen \cite{Quillen67}. In his sense, a model category does not require to have either finite limits or finite colimits. In our case, we will explicitly state whenever a model category is non-closed, i.e. does not have every limit or colimit.
            \end{remark}

            A model category $\mathcal{C}$ is now defined to be a category equipped with a particular model structure. Notice that a category may admit several model structures. We will postpone examples until sufficient theory have been developed. For more topological examples, we refer to Dwayer-Spalinski \cite{Dwyer95} and Hovey \cite{Hovey99}.

            An interesting and a not so non-trivial property of model categories is that giving all three classes $Ac$, $Cof$ and $Fib$ is redundant. Given the class of weak equivalences and either cofibrations or fibrations, the model structure is determined. Thus the classes of fibrations are determined by acyclic cofibrations and cofibrations are determined by fibrations. The next two results will show this.

            \begin{lemma}[The retract argument]\label{lem: retract-argument}
                Let $\mathcal{C}$ be a category. Suppose there is a factorization $f = pi$ and that $f$ has LLP with respect to $p$, then $f$ is a retract of $i$. Dually, if $f$ har RLP to $i$, then it is a retract of $p$.
            \end{lemma}

            \begin{proof}
                We assume that $f : A \rightarrow C$ has LLP with respect to $p : B \rightarrow C$. Then we may find a lift $r : C \rightarrow B$, which realize $f$ as a retract of $i$.
                
                \begin{center}
                    \begin{tikzcd}
                        A \ar[]{d}[]{f} \ar[]{r}[]{i} & B \ar[]{d}[]{p} \\
                        C \ar[equal]{r}[]{} \ar[dashed]{ru}[]{r} & C
                    \end{tikzcd} $\implies$
                    \begin{tikzcd}
                        A \ar[equal]{r}[]{} \ar[]{d}[]{f} & A \ar[equal]{r}[]{} \ar[]{d}[]{i} & A \ar[]{d}[]{f} \\
                        C \ar[]{r}[]{r} & B \ar[]{r}[]{p} & C
                    \end{tikzcd}
                \end{center}
            \end{proof}

            \begin{proposition}
                Let $\mathcal{C}$ be a model category. A morphism $f$ is a cofibration (acyclic cofibration) if and only if $f$ has LLP with respect acyclic fibrations (fibrations). Dually, $f$ is a fibration (acyclic fibration) if and only if it has RLP with respect to acyclic cofibrations (cofibrations).
            \end{proposition}

            \begin{proof}
            Assume that $f$ is a cofibration. By MC3, we know that $f$ has LLP with respect to acyclic fibrations. Assume instead that $f$ has LLP with respect to ever acyclic fibration. By MC4 we factor $f = f_\alpha\circ f_\beta$, where $f_\alpha$ is a cofibration and $f_\beta$ is an acyclic fibration. Since we assume $f$ to have LLP with respect to $f_\beta$, by lemma \ref{lem: retract-argument} we know that $f$ is a retract of $f_\alpha$. Thus by MC2, we know that $f$ is a cofibration. 
            \end{proof}

            \begin{corollary}\label{cor: stable-cofib-base-change}
                Let $\mathcal{C}$ be a model category. (Acyclic) Cofibrations are stable under pushouts, i.e. if $f$ is an (acyclic) cofibration, then $f'$ is an (acyclic) cofibration.
                \begin{center}
                    \begin{tikzcd}
                        A \ar[]{r}[]{} \ar[]{d}[]{f} \ar[phantom]{rd}[near end]{\lrcorner} & C \ar[]{d}[]{f'} \\
                        B \ar[]{r}[]{} & D
                    \end{tikzcd}
                \end{center}
                Dually, fibrations are stable under pullbacks.
            \end{corollary}

            \begin{proof}
                This is clear by the universal property of pushouts.
            \end{proof}

            Since we assume that every model category $\mathcal{C}$ is admits finite limits and colimits, we know that it has both an initial and a terminal object. We let $\emptyset$ denote the initial object and $*$ denote the terminal object. 

            \begin{definition}[Cofibrant, fibrant and bifibrant objects]
                Let $\mathcal{C}$ be a model category. An object $X$ is called cofibrant if the unique morphism $\emptyset \rightarrow X$ is a cofibration. Dually, $X$ is called fibrant if the unique morphism $X \rightarrow *$ is fibrant. If $X$ is both cofibrant and fibrant, we call it bifibrant.
            \end{definition}

            There is no reason for every object to be either cofibrant or fibrant. However, we may see that every object is weakly equivalent to an object which is either fibrant or cofibrant. We will make it precise what it means for two objects to be weakly equivalent later.

            \begin{construction}
                Let $X$ be an object of a model category $\mathcal{C}$. The morphism $i:\emptyset\rightarrow X$ has a functorial factorization $i=i_\beta\circ i_\alpha$, where $i_\alpha: \emptyset\rightarrow QX$ is a cofibration and $i_\beta: QX\rightarrow X$ is an acyclic fibration. By definition $QX$ is cofibrant and weakly equivalent to $X$.

                $Q: \mathcal{C}\rightarrow \mathcal{C}$ defines a functor called the cofibrant replacement. To see this we first look at the slice category $\sfrac{\emptyset}{\mathcal{C}}$. The objects are morphisms $f:\emptyset \rightarrow X$ for any object $X$ in $\mathcal{C}$, while morphisms are commutative triangles. We first observe that $\sfrac{\emptyset}{\mathcal{C}}\subset\mathcal{C}^\rightarrow$ is a subcategory of the arrow category. Thus $(\alpha, \beta)$ may be interpreted as functors on the slice category to the arrow category. Moreover, since every arrow $f:\emptyset \rightarrow X$ is unique, we observe that this category is equivalent to $\mathcal{C}$. Thus $(\alpha, \beta)$ may be interpreted as functors on $\mathcal{C}$ into arrows. We define $Q$ as the composition $Q = tar \circ \alpha$.

                Dually, we get a fibrant replacement $R$ by dualizing the above argument.
            \end{construction}

            We collect the following properties

            \begin{lemma}\label{lem: Q-preserves-weak}
                The cofibrant replacement $Q$ and fibrant replacement $R$ preserves weak equivalences. 
            \end{lemma}

            \begin{proof}
                Clear from the $2$-out-of-$3$ property.
            \end{proof}

            \begin{lemma}[Ken Brown's lemma]\label{lem: Ken-Brown}
                Let $\mathcal{C}$ be a model category and $\mathcal{D}$ be a category with weak equivalences satisfying the $2$-out-of-$3$ property. If $F:\mathcal{C} \rightarrow \mathcal{D}$ is a functor sending acyclic cofibrations between cofibrant objects to weak equivalences, then $F$ takes all weak equivalences between cofibrant objects to weak equivalences. Dually, if $F$ takes all acyclic fibrations between fibrant objects to weak equivalences, then $F$ takes all weak equivalences between fibrant objects to weak equivalences.
            \end{lemma}

            \begin{proof}
                Suppose that $A$ and $B$ are cofibrant objects and that $f:A\rightarrow B$ is a weak equivalence. Using the universal property of the coproduct we define the map $(f, id_B) = p: A\coprod B \rightarrow B$. $p$ has a functorial factorization into a cofibration and acyclic fibration, $p = p_\beta\circ p_\alpha$. We recollect the maps in the following pushout diagram:
                \begin{center}
                    \begin{tikzcd}
                        \emptyset \ar[]{d}[]{} \ar[]{r}[]{} \ar[phantom]{rd}[near end]{\lrcorner} & B \ar[]{d}[]{i_2} \ar[bend left]{rrddd}[]{id_B} \\
                        A \ar[]{r}[]{i_1} \ar[bend right]{rrrdd}[]{f} & A\coprod B \ar[]{rd}[]{p_\alpha} \\
                        & & C \ar[]{rd}[]{p_\beta} \\
                        & & & D
                    \end{tikzcd}
                \end{center}
                By lemma \ref{cor: stable-cofib-base-change} both $i_1$ and $i_2$ are cofibrations. Since $f$, $id_B$ and $p_\beta$ are weak equivalences, so are $p_\alpha\circ i_1$ and $p_\alpha\circ i_2$ by MC2. Moreover, they are acyclic cofibrations.

                Assume that $F:\mathcal{C}\rightarrow\mathcal{D}$ is a functor as described above. Then by assumption, $F(p_\alpha\circ i_1)$ and $F(p_\alpha\circ i_2)$ are weak equivalences. Since a functor sends identity to identity, we also know that $F(id_B)$ is a weak equivelnce. Thus by the $2$-out-of-$3$ property $F(p_\beta)$ is a weak equivalence, as $F(p_\beta)\circ F(p_\alpha\circ i_2) = id_{F(B)}$. Again, by $2$-out-of-$3$ property $F(f)$ is a weak equivelnce, as $F(f) = F(p_\beta)\circ F(p_\alpha\circ i_1)$.
            \end{proof}

        \subsection{Homotopy category}

            Homotopy theory at it's most abstract is the study of categories and functors up to weak equivalences. Here, a weak equivalence may be anything, but most commonly it is a weak equivalence in topological homotopy or a quasi-isomorphism in homological algebra. The biggest concern when dealing with such concepts is to make a functor well-defined up to these chosen weak-equivalences. To this end, there is a construction to amend these problems, known as derived functors. We define a homotopical category in the sense of Riehl \cite{Riehl16}.

            \begin{definition}[Homotopical Category]
                Let $\mathcal{C}$ be a category. $\mathcal{C}$ is Homotopical if there is a wide subcategory constituting a class of morphisms known as weak equivalences, $Ac\subset Mor\mathcal{C}$. The weak equivalences should satisfy the \textbf{$2$-out-of-$6$ property}, i.e. given three composable morphisms $f$, $g$ and $g$, if $gf$ and $hg$ are weak equivalences, then so are $f$, $g$, $h$ and $hgf$.

                \begin{center}
                    \begin{tikzcd}[row sep = large]
                        A \ar[]{r}[]{f} \ar[]{rd}[]{gf} \ar[dotted]{rrd}[]{} & B \ar[]{d}[]{g} \ar[]{rd}[]{hg} \\
                        & C \ar[]{r}[]{h} & D
                    \end{tikzcd}
                \end{center}
            \end{definition}

            \begin{remark}
                Notice that the $2$-out-of-$6$ property is stronger than the $2$-out-of-$3$ property. To see this, let either $f$, $g$ or $h$ be the identity, and then conclude with the $2$-out-of-$3$ property.
            \end{remark}

            \begin{remark}
                The collection of weak equivalences contains every isomorphism. To see this pick an isomorphism $f$ and $f^{-1}$, then the compositions are the identity on the domain and codomain, which are assumed to be in $Ac$.
            \end{remark}
            
            Given such a homotopical category $\mathcal{C}$, we want to invert every weak equivalence and create the homotopy category of $\mathcal{C}$. This construction is developed in Gabriel and Zisman \cite{Zisman67} called calculus of fractions. This method essentially tries to mimic localization for commutative rings in a category theoretic fashion.

            \begin{definition}
                Let $\mathcal{C}$ be a homotopical category. It's homotopy category $Ho\mathcal{C} = \mathcal{C}[Ac^{-1}]$, together with a localization functor $q:\mathcal{C}\rightarrow Ho\mathcal{C}$. Recall that the localization are determined by the following universal property: If $F:\mathcal{C}\rightarrow \mathcal{D}$ is a functor sending weak equivalences to isomorphisms, then it uniquely factors through the homotopy category up to a unique natural isomorphism $\eta$.

                \begin{center}
                    \begin{tikzcd}
                        \mathcal{C} \ar[""{name = U, below}]{rr}[]{F} \ar[]{rd}[]{q} & & \mathcal{D} \\
                        & Ho\mathcal{C} \ar[]{ru}[]{F'} \ar[Rightarrow, from=U]{}[]{\eta}
                    \end{tikzcd}
                \end{center}
            \end{definition}

            \begin{definition}
                Suppose that $\mathcal{C}$ is a homotopical category. Two objects of $\mathcal{C}$ are said to be weakly equivalent if they are isomorphic in $Ho\mathcal{C}$. I.e. there is some zig-zag relation between the objects, consisting only of weak equivalences.
            \end{definition}

            \begin{remark}
                A renown problem with localizations is that even if $\mathcal{C}$ is a locally small category, any localization $\mathcal{C}[S^{-1}]$ does not need to be. Thus, without a good theory of classes or higher universes, we cannot in general ensure that the localization still exists as a locally small category.
            \end{remark}

            We see from the definition of the homotopy category that a functor $F$ admits a lift $F'$ to the homotopy category whenever weak equivalences are sent to isomorphisms. Moreover, if we have a functor $F$ between homotopical categories which preserves weak equivalences, it then induces a functor between the homotopy categories.
            
            \begin{definition}[Homotopical functors]
                A functor $F:\mathcal{C}\rightarrow \mathcal{D}$ between homotopical categories is homotopical if it preserves weak equivalences. Moreover, there is a lift of functors as in the following diagram.

                \begin{center}
                    \begin{tikzcd}
                        \mathcal{C} \ar[""{name=U, below}]{r}[]{F} \ar[]{d}[]{q_\mathcal{C}} & \mathcal{D} \ar[]{d}[]{q_\mathcal{D}} \ar[Rightarrow]{dl}[]{\eta} \\
                        Ho\mathcal{C} \ar[dashed, ""{name=V, above}]{r}[]{F'} & Ho\mathcal{D}
                    \end{tikzcd}
                \end{center}
            \end{definition}

            Derived functors come into play whenever this is not the case. These lifts are however the closest approximation which we can make functorial. The general exposition of derived functors is beyond the scope of this thesis, but an account of it may be found in \cite{Riehl16}. As we will see, model categories are a nice environment to work with these concepts. Firstly we will amend the problem with localizations, where the homotopy category may not exists. Secondly, we will obtain a simple description of some important derived functors.

            \begin{proposition}
                Any model category $\mathcal{C}$ is a homotopical category.
            \end{proposition}

            \begin{proof}
                To show that a model category is homotopical it suffices to show that $Ac$ satisfies the $2$-out-of-$6$ property. Assume there are $3$ composable morphisms $f,g,h$ such that $gf,hg\in Ac$. By the $2$-out-of-$3$ property for $Ac$ it is enough to show that at least one of $f,g,h,fgh$ is a weak equivalence to deduce that every other morphism is a weak equivalence.
                \begin{center}
                    \begin{tikzcd}[row sep = large]
                        A \ar[]{r}[]{f} \ar[]{rd}[]{gf} \ar[dotted]{rrd}[]{} & B \ar[]{d}[]{g} \ar[]{rd}[]{hg} \\
                        & C \ar[]{r}[]{h} & D
                    \end{tikzcd}
                \end{center}

                To be able to use the model structure, we will first show that we may assume $f,g$ to be cofibrant and $g,h$ to be fibrant. We know by MC4 that $f,g,gf$ may be factored into a cofibration composed with an acyclic fibration, e.g. $f = f_\beta f_\alpha$. Since $gf$ is a weak equivalence, so is $(gf)_\alpha$ by the $2$-out-of-$3$ property.
                \begin{center}
                    \begin{tikzcd}
                        A \ar[]{rr}[]{f} \ar[]{rd}[below]{f_\alpha} && B \\
                        & B' \ar[]{ru}[below]{f_\beta}
                    \end{tikzcd}
                    \begin{tikzcd}
                        B \ar[]{rr}[]{g} \ar[]{rd}[below]{g_\alpha} && C \\
                        & C' \ar[]{ru}[below]{g_\beta}
                    \end{tikzcd}
                    \begin{tikzcd}
                        A \ar[]{rr}[]{gf} \ar[]{rd}[below]{(gf)_\alpha}&& C \\
                        & C'' \ar[]{ru}[below]{(gf)_\beta}
                    \end{tikzcd}
                \end{center}

                Notice that the "cofibrant approximation" of the map from $A$ to $C$ either goes through $C'$ or $C''$. We conjoin these by taking the pullback. Since acyclic fibrations are stable over pullbacks, we get a pullback square where every morphism is an acyclic fibration. Thus the map $A\rightarrow \widetilde{C}$ is a weak equivalence by $2$-out-of-$3$.
                \begin{center}
                    \begin{tikzcd}
                        A \ar[bend right]{rdd}[]{(gf)_\alpha} \ar[bend left]{rrd}[]{g_\alpha} \ar[dashed]{rd}[]{} \\
                        & \widetilde{C} \ar[]{d}[]{} \ar[]{r}[]{} \ar[phantom]{rd}[very near start]{\ulcorner} & C' \ar[]{d}[]{g_\beta}\\
                        & C'' \ar[]{r}[]{(gf)_\beta}& C
                    \end{tikzcd}
                \end{center}

                To replace $f$ with $f_\alpha$ we must lift the composition into our "new" $C$, which is $\widetilde{C}$. This is done by using MC3, as $f_\alpha$ is a cofibration and the pullback square above consists entirely of acyclic fibrations.
                \begin{center}
                    \begin{tikzcd}
                        A \ar[]{r}[]{} \ar[]{d}[]{f_\alpha} & \widetilde{C} \ar[]{d}[]{} \\
                        B' \ar[]{r}[]{} \ar[dashed]{ru}[]{} & C
                    \end{tikzcd}
                \end{center}

                To summarize we have the following diagram, where every squiggly arrow is a weak equivalence.
                \begin{center}
                    \begin{tikzcd}
                        A \ar[]{r}[]{f_\alpha} \ar[squiggly]{rd}[]{t} & B' \ar[squiggly]{r}[]{} \ar[]{d}[]{s} & B \ar[]{d}[]{} \ar[squiggly]{rd}[]{} \\
                        & \widetilde{C} \ar[squiggly]{r}[]{} & C \ar[]{r}[]{} & D
                    \end{tikzcd}
                \end{center}

                We now wish to promote the arrow $s:B'\rightarrow \widetilde{C}$ into a cofibration. We do this by factoring both $s$ and $t$ with $MC4$. Notice that $s_\beta$, $t_\beta$ and $t_\alpha$ are weak equivalences.
                \begin{center}
                    \begin{tikzcd}
                        B' \ar[]{rr}[]{s} \ar[]{rd}[]{s_\alpha} && \widetilde{C}\\
                        & \widetilde{C}' \ar[]{ru}[]{s_\beta}
                    \end{tikzcd}
                    \begin{tikzcd}
                        A \ar[]{rr}[]{t} \ar[]{rd}[]{t_\alpha} && \widetilde{C} \\
                        & \widetilde{C}'' \ar[]{ru}[]{t_\beta}
                    \end{tikzcd}
                \end{center}

                To obtain our final factorization we use RLP of $s_\beta$ on $t_\alpha$.
                \begin{center}
                    \begin{tikzcd}
                        & B' \ar[]{d}[]{s_\alpha}\\
                        A \ar[]{r}[]{} \ar[]{ru}[]{f_\alpha} \ar[]{d}[]{t_\alpha} & \widetilde{C}' \ar[]{d}[]{s_\beta} \\
                        \widetilde{C}'' \ar[]{r}[]{t_\beta} \ar[dashed]{ru}[]{} & \widetilde{C}
                    \end{tikzcd}
                \end{center}

                We have now obtained a factorization of $gf$ which are two cofibrations followed by an acyclic fibration, in such a manner that it is compatible with the original composition. The dual to this claim is that we may also factor $hg$ into two fibrations which is preceeded by an acyclic cofibration. In other words, we may assume without loss of generality that $f$ and $g$ are cofibrations, and that $g$ and $h$ are fibrations.
                
                In this case we will show that $g$ is an isomorphism. Consider the diagram below with lifts $i$ and $j$, these exists since we assume $gf$ and $hg$ to be weak equivalences.
                \begin{center}
                    \begin{tikzcd}
                        A \ar[]{r}[]{f} \ar[]{d}[]{gf} & B\ar[]{r}[]{id_B} \ar[]{d}[]{g} & B \ar[]{d}[]{hg} \\
                        C \ar[]{r}[]{id_C} \ar[dashed]{ru}[]{i} & C \ar[]{r}[]{h} \ar[dashed]{ru}[]{j} & D
                    \end{tikzcd}
                \end{center}
                Since the diagram is commutative, we get that $i = j$, and that $g$ is both split mono and split epi, with $i$ as its splitting.
            \end{proof}

            Since every model category is homotopical, it also has an associated homotopy category $Ho\mathcal{C}$. Let $\mathcal{C}_c$, $\mathcal{C}_f$ and $\mathcal{C}_{cf}$ denote the full subcategories consisting of cofibrant, fibrant and bifibrant objects respectively.

            \begin{proposition}
                Let $\mathcal{C}$ be a model category. The following categories are equivalent:
                \begin{itemize}
                    \item $Ho\mathcal{C}$
                    \item $Ho\mathcal{C}_c$
                    \item $Ho\mathcal{C}_f$
                    \item $Ho\mathcal{C}_{cf}$
                \end{itemize}
            \end{proposition}

            \begin{proof}
                We show that $Ho\mathcal{C} \simeq Ho\mathcal{C}_c$. The inclusion $i:\mathcal{C}_c\rightarrow \mathcal{C}$ clearly preserves weak equivalences, thus $i$ is homotopical and admits a lift. Moreover, since the cofibrant replacement is also homotopical, it also has a lift.

                \begin{center}
                    \begin{tikzcd}
                        \mathcal{C}_c \ar[]{r}[]{i} \ar[]{d}[]{} & \mathcal{C} \ar[]{d}[]{} \\
                        Ho\mathcal{C}_c \ar[dashed, bend right]{r}[below]{Ho\ i} & \mathcal{C} \ar[dashed, bend right]{l}[above]{Q}
                    \end{tikzcd}
                \end{center}

                It is clear that $Q$ is the quasi-inverse of $i$.

            \end{proof}

            As of now we still don't see how model categories will fix the size issues. To do this we will develop homotopy equivalence $\sim$. We will see that on the subcategory of bifibrant objects $\mathcal{C}_{cf}$, this homotopy equivalence will in fact be a congruence relation. Moreover, there is an equivalence of categories $Ho\mathcal{C}_{cf}\simeq \sfrac{\mathcal{C}_{cf}}{\sim}$.

            \begin{definition}[Cylinder and path objects]
                Let $\mathcal{C}$ be a model category. Given an object $X$, a cylinder object $X\wedge I$ is a factorization of the fold map $i:X\coprod X \rightarrow X$, such that $p_0$ is a cofibration and that $p_1$ is a weak equivalence. 
                
                \begin{center}
                    \begin{tikzcd}
                        X\coprod X \ar[]{rr}[]{i} \ar[]{rd}[]{p_0} & & X \\
                        & X\wedge I \ar[]{ru}[]{p_1}
                    \end{tikzcd}
                \end{center}

                Dually, a path object  $X^{I}$ is a factorization of the diagonal map $i: X \rightarrow X\prod X$, such that $p_0$ is a weak equivalence and that $p_1$ is a fibration.
                
                \begin{center}
                    \begin{tikzcd}
                        X \ar[]{rr}[]{i} \ar[]{rd}[]{p_0} & & X\prod X\\
                        & X^I \ar[]{ru}[]{p_1}
                    \end{tikzcd}
                \end{center}
            \end{definition}

            \begin{remark}
                Even though we have written $X\wedge I$ suggestively to be a functor, it is not. There may be many choices for a cylinder object. However, by using the functorial factorization from MC4, we get a canonical choice of a cylinder object, as it factors every map into a cofibration and an acyclic fibration. By letting the cylinder object be this object, we do indeed obtain a functor.
            \end{remark}

            \begin{proposition}
                Let $\mathcal{C}$ be a model category and $X$ an object of $\mathcal{C}$. Given two cylinder objects $X\wedge I$ and $X\wedge I'$, then they are weakly equivalent. 
            \end{proposition}

            \begin{proof}
                It is enough to show that there is a weak equivalence from any cylinder object into one specified cylinder object. This is in fact true for the functorial cylinder object $X\wedge I$, as the final morphism $p_1$ is an acyclic fibration, which enables a lift which is a weak equivalence by the $2$-out-of-$3$ property.

                \begin{center}
                    \begin{tikzcd}
                        X\coprod X \ar[]{r}[]{p_0} \ar[]{d}[]{p_0'} & X\wedge I \ar[]{d}[]{p_1} \\
                        X\wedge I' \ar[]{r}[]{p_1'} \ar[dashed]{ru}[]{} & X
                    \end{tikzcd}
                \end{center}
            \end{proof}

            \begin{definition}[Homotopy equivalence]
                Let $f,g: X\rightarrow Y$. A left homotopy between $f$ and $g$ is a morphism $H:X\wedge I \rightarrow Y$ such that $Hi_0 = f$ and $Hi_1 = g$. $f$ and $g$ are left homotopic if a left homotopy exists, and it is denoted $f \overset{l}{\sim} g$.

                \begin{center}
                    \begin{tikzcd}
                        X \ar[]{d}[]{} \ar[dotted]{rd}[]{i_0} \ar[bend left]{rrd}[]{f} \\
                        X\coprod X \ar[]{r}[]{p_0} & X\wedge I \ar[]{r}[]{H} & Y \\
                        X \ar[]{u}[]{} \ar[dotted]{ru}[]{i_1} \ar[bend right]{rru}[]{g}
                    \end{tikzcd}
                \end{center}
                
                A right homotopy between $f$ and $g$ is a morphism $H: X \rightarrow Y^I$ such that $i_0H = f$ and $i_1H = g$. $f$ and $g$ are right homotopic if a right homotopy exists, and it is denoted $f \overset{r}{\sim} g$.

                \begin{center}
                    \begin{tikzcd}
                        & & Y \\
                        X \ar[]{r}[]{H} \ar[bend left]{rru}[]{f} \ar[bend right]{rrd}[]{g} & Y^I \ar[]{r}[]{p_1} \ar[dotted]{ru}[]{i_0} \ar[dotted]{rd}[]{i_1} & Y\prod Y \ar[]{u}[]{} \ar[]{d}[]{} \\
                        & & Y
                    \end{tikzcd}
                \end{center}

                $f$ and $g$ are said to be homotopic if they are both left and right homotopic, it is denoted $f \sim g$. $f$ is said to be a homotopy equivalence, if it has a homotopy inverse $h: Y \rightarrow X$, such that $hf \sim id_X$ and $fh \sim id_Y$. 
            \end{definition}

            It is important to know that this is not a priori an equivalence relation. This is amended by taking both fibrant and cofibrant replacements. We see this in the following proposition.

            \begin{proposition}\label{prop: basic-homotopy}
                Let $\mathcal{C}$ be a model category, and $f,g:X\rightarrow Y$ be morphisms. We have the following:
                \begin{enumerate}
                    \item If $f \overset{l}{\sim}$ and $h: Y \rightarrow Z$, then $hf \overset{l}{\sim} hg$.
                    \item If $Y$ is fibrant, $f \overset{l}{\sim} g$ and $h: W \rightarrow X$, then $fh \overset{l}{\sim} gh$.
                    \item If $X$ is cofibrant, then left homotopy is an equivalence relation on $\mathcal{C}(X,Y)$.
                    \item If $X$ is cofibrant and $f \overset{l}{\sim} g$, then $f \overset{r}{\sim} g$.
                \end{enumerate}
            \end{proposition}

            \begin{proof}
                $(1.)$ Assume that $f \overset{l}{\sim} g$ and $h:Y\rightarrow Z$. Let $H:X\wedge I \rightarrow Y$ denote the left homotopy between $f$ and $g$. The left homotopy between $hf$ and $hg$ is given as $hH$.

                $(2.)$ Assume that $Y$ is fibrant, $f \overset{l}{\sim} g$ and that $h:W\rightarrow X$. Let $H:X\wedge I\rightarrow Y$ be a left homotopy. We construct a new cylinder object for the homotopy. Factor $p_1:X\wedge I \rightarrow X$ as $q_1\circ q_0$ where $q_0: X\wedge I \rightarrow X\wedge I'$ is an acyclic cofibration and $q_1:X\wedge I'\rightarrow X$ is a fibration. By the $2$-out-of-$3$ property $q_1$ is an acyclic fibration, as $p_1$ and $q_0$ are weak equivalences. $X\wedge I'$ is a cylinder object as $q_0\circ p_0$ is a cofibration and $q_1$ is a weak equivalence. Since we assume $Y$ to be fibrant we lift the left homotopy $H:X\wedge I\rightarrow Y$ to the left homotopy $H':X\wedge I'\rightarrow Y$ with the following diagram:
                \begin{center}
                    \begin{tikzcd}
                        X\wedge I \ar[]{r}[]{H} \ar[]{d}[]{q_0} & Y \ar[]{d}[]{} \\
                        X\wedge I' \ar[]{r}[]{} \ar[dashed]{ru}[]{H'} & *
                    \end{tikzcd}                    
                \end{center}
                We can find the appropriate homotopy needed with lift given by the following diagram:
                \begin{center}
                    \begin{tikzcd}[column sep = large]
                        W\coprod W \ar[]{r}[]{q_0p_0(h\coprod h)} \ar[]{d}[]{p_0'} & X\wedge I' \ar[]{d}[]{q_1} \\
                        W\wedge I \ar[]{r}[]{hp_1'} \ar[dashed]{ru}[]{k} & X
                    \end{tikzcd}
                \end{center}
                Then the morphism $H'k$ is the desired left homotopy witnessing $fh \overset{l}{\sim} gh$.

                $(3.)$ Assume that $X$ is cofibrant. First observe that a left homotopy is reflexive and symmetric. We must show that in this case it is also transitative. Thus, assume that $f,g,h:X\rightarrow Y$ and that $H:X\wedge I\rightarrow Y$ is a left homotopy witnessing $f \overset{l}{\sim} g$ and that $H':X\wedge I'\rightarrow Y$ is a left homotopy witnessing $g\overset{l}{\sim} h$. We first observe that $i_0: X\rightarrow X\wedge I$ is a weak equivalence, as $id_X = p_1i_0$ where $id_X$ and $p_1$ are weak equivalences. Since $X$ is assumed to be cofibrant, we see that $X\coprod X$ is cofibrant by the following pushout:
                \begin{center}
                    \begin{tikzcd}
                        * \ar[]{r}[]{} \ar[]{d}[]{} \ar[phantom]{rd}[near end]{\lrcorner} & X \ar[]{d}[]{inr}\\
                        X \ar[]{r}[]{inl} & X\coprod X
                    \end{tikzcd}
                \end{center}
                Moreover, both $inl$ and $inr$ are cofibrations. This shows that $i_0$ is a cofibration as $i_0 = p_0\circ inr$ is a composition of two cofibrations. $i_0$ is thus an acyclic cofibration. We define an almost cylinder object $C$ by the pushout of $i_1$ and $i_0'$. We define the maps $t$ and $H''$ by using the universal property in the following manner:
                \begin{center}
                    \begin{tikzcd}
                        X \ar[]{r}[]{i_1} \ar[]{d}[]{i_0'} & X\wedge I \ar[]{d}[]{} \ar[bend left]{rdd}[]{p_1} \\
                        X\wedge I' \ar[]{r}[]{} \ar[bend right]{rrd}[]{p_1'} & C \ar[dashed]{rd}[]{t} \\
                        & & X
                    \end{tikzcd}\qquad
                    \begin{tikzcd}
                        X \ar[]{r}[]{i_1} \ar[]{d}[]{i_0'} & X\wedge I \ar[]{d}[]{} \ar[bend left]{rdd}[]{H} \\
                        X\wedge I' \ar[]{r}[]{} \ar[bend right]{rrd}[]{H'} & C \ar[dashed]{rd}[]{H''} \\
                        & & X
                    \end{tikzcd}
                \end{center}
                Observe that there is a factorization of the fold map $X\coprod X \overset{s}{\rightarrow} C \overset{t}{\rightarrow} X$. However, $s$ may not be a cofibration, so we replace $C$ with the cylinder object $X\wedge I''$ such that we have the factorization $X\coprod X \overset{s_\alpha}{\rightarrow} X\wedge I'' \overset{ts_\beta}{\rightarrow} X$. The morphism $Hs_\beta$ is then our required homotopy for $f\overset{l}{\sim}g$.

                $(4.)$ Suppose that $X$ is cofibrant and that $H:X\wedge I\rightarrow Y$ is a left homotopy for $f \overset{l}{\sim} g$. Pick a path object for $Y$, such that we have the factorization $Y\overset{q_0}{\rightarrow}Y^I\overset{q_1}{\rightarrow}Y\prod Y$ where $q_0$ is a weak equivalence and $q_1$ is a fibration. Again, as $X$ is cofibrant we get that $i_0$ is an acyclic cofibration, so we have the following lift of the homotopy:
                \begin{center}
                    \begin{tikzcd}
                        X \ar[]{r}[]{q_0f} \ar[]{d}[]{i_0} & Y^I \ar[]{d}[]{q_1} \\
                        X\wedge I \ar[]{r}[]{(fp_1, H)} \ar[dashed]{ru}[]{J} & Y\prod Y
                    \end{tikzcd}
                \end{center}
                The right homotopy is given by injecting away from $f$, i.e. $H' = Ji_1$.
            \end{proof}

            \begin{corollary}
                We collect the dual results of the above proposition, and thus have the following.
                \begin{enumerate}
                    \item If $f \overset{r}{\sim}$ and $h: W \rightarrow X$, then $fh \overset{r}{\sim} gh$.
                    \item If $X$ is cofibrant, $f \overset{r}{\sim} g$ and $h: Y \rightarrow Z$, then $hf \overset{r}{\sim} hg$.
                    \item If $Y$ is fibrant, then left homotopy is an equivalence relation on $\mathcal{C}(X,Y)$.
                    \item If $Y$ is fibrant and $f \overset{r}{\sim} g$, then $f \overset{l}{\sim} g$.
                \end{enumerate}
            \end{corollary}

            \begin{corollary}\label{cor: homotopy-is-eq-rel}
                Homotopy is a congruence relation on $\mathcal{C}_{cf}$. In this manner, the category $\mathcal{C}_{cf}/\sim$ is well-defined, exists and inverts every homotopy equivalence.
            \end{corollary}

            \begin{lemma}[Weird Whitehead]\label{lem: Weird-Whitehead}
                Let $\mathcal{C}$ be a model category. Suppose that $C$ is cofibrant and $h: X \rightarrow Y$ is an acyclic fibration or a weak equivalence between fibrant objects, then $h$ induces an isomorphism:
                \begin{center}
                    \begin{tikzcd}
                        \sfrac{\mathcal{C}(C,X)}{\overset{l}{\sim}} \ar[]{r}[]{\overset{h_*}{\simeq}} & \sfrac{\mathcal{C}(C, Y)}{\overset{l}{\sim}}
                    \end{tikzcd}
                \end{center}

                Dually, if $X$ is fibrant and $h : C \rightarrow D$ is an acyclic cofibration or a weak equivalence between cofibrant objects, then $h$ induces an isomorphism:
                \begin{center}
                    \begin{tikzcd}
                        \sfrac{\mathcal{C}(D, X)}{\overset{r}{\sim}} \ar[]{r}[]{\overset{h^*}{\simeq}} & \sfrac{\mathcal{C}(C, X)}{\overset{r}{\sim}}
                    \end{tikzcd}
                \end{center}
            \end{lemma}

            \begin{proof}
                We assume $\mathcal{C}$ to be cofibrant and $h:X\rightarrow Y$ to ba an acyclic fibration. We first prove that $h$ is surjective. Let $f:C\rightarrow Y$. By RLP of $h$ there is a morphism $f':C\rightarrow X$ such that $f = hf'$.
                \begin{center}
                    \begin{tikzcd}
                        \emptyset \ar[]{r}[]{} \ar[]{d}[]{} & X \ar[]{d}[]{h} \\
                        C \ar[]{r}[]{f} \ar[dashed]{ru}[]{f'} & Y
                    \end{tikzcd}
                \end{center}
                
                To show injectivity we assume $f,g:C\rightarrow X$ such that $hf\overset{l}{\sim} hg$, in particular there is a left homotopy $H:C\wedge I \rightarrow Y$. Remember that since $C$ is cofibrant, the map $p_0$ is a cofibration. We find a left homotopy $H:C\wedge I \rightarrow X$ witnessing $f\overset{l}{\sim} g$ by the following lift.
                \begin{center}
                    \begin{tikzcd}
                        C\coprod C \ar[]{r}[]{f+g} \ar[]{d}[]{p_0} & X \ar[]{d}[]{h} \\
                        C\wedge I \ar[dashed]{ru}[]{H'} \ar[]{r}[]{H} & Y
                    \end{tikzcd}
                \end{center}

                Moreover, if we assume both $X$ and $Y$ to be fibrant, the functor $\sfrac{\mathcal{C}(C,\_)}{\overset{l}{\sim}}$ sends acyclic fibrations to isomorphisms, i.e. to weak equivalences. By Ken Brown's lemma, lemma \ref{lem: Ken-Brown}, the afformentioned functor sends weak equivalences between fibrant objects to isomorphisms.
            \end{proof}

            \begin{thm}[Generalized Whiteheads theorem]\label{thm: Whitehead}
                Let $\mathcal{C}$ be a model category. Suppose that $f : X \rightarrow Y$ is a morphism of bifibrant objects, then $f$ is a weak equivalence if and only if $f$ is a homotopy equivalence.
            \end{thm}

            \begin{proof}
                Suppose first that $f$ is a weak equivalence. Pick a bifibrant object $A$, then by lemma \ref{lem: Weird-Whitehead} $f_*:\sfrac{\mathcal{C}(A,X)}{\sim}\rightarrow \sfrac{\mathcal{C}(A,Y)}{\sim}$ is an isomorphism. Letting $A = Y$ we know that there is a morphism $g:Y\rightarrow X$, such that $f_*g = fg \sim id_Y$. Furthermore, by proposition \ref{prop: basic-homotopy}, since $X$ is bifibrant composing on the right preserves homotopy equivalence, e.g. $fgf \sim f$. By letting $A = X$, we get that $f_*gf = fgf \sim f = f_*id_X$, thus $gf \sim id_X$.

                For the opposite direction, assume that $f$ is a homotopy equivalence. We factor $f$ into an acyclic cofibration $f_\gamma$ and a fibration $f_\delta$, i.e. $X \overset{f_\gamma}{\rightarrow} Z \overset{f_\delta}{\rightarrow} Y$. Observe that $Z$ is bifibrant as $X$ and $Y$ is, in particular, $f_\gamma$ is a weak equivalence of bifibrant objects, so it is a homotopy equivalence. 

                It is enough to show that $f_\delta$ is a weak equivalence. Let $g$ be the homotopy inverse of $f$, and $H:Y\wedge I \rightarrow Y$ is a left homotopy witnessing $fg \sim id_Y$. Since $Y$ is bifibrant, the following square has a lift.
                \begin{center}
                    \begin{tikzcd}
                        Y \ar[]{r}[]{f_\gamma g} \ar[]{d}[]{i_0} & Z \ar[]{d}[]{f_\delta} \\
                        Y\wedge I \ar[]{r}[]{H} \ar[dashed]{ru}[]{H'} & Y
                    \end{tikzcd}
                \end{center}
                Let $h = H'i_1$, then by definition we know that $f_\delta H'i_1 = id_Y$. Moreover, $H$ is a left homotopy witnessing $f_\gamma g \sim h$. Let $g': Z\rightarrow X$ be the homotopy inverse of $f_\gamma$. We have the following relations $f_\delta \sim f_\delta f_\gamma g' \sim fg'$, and $hf_\delta \sim (f_\gamma g)(fg') \sim f_\gamma g' \sim id_Z$. Let $H'':Z\wedge I\rightarrow Z$ be a left homotopy witnessing this homotopy. Since $Z$ is bifibrant, $i_0$ and $i_1$ are weak equivalences. By the $2$-out-of-$3$ property $H''$ and $hf_\delta$ are weak equivalences. Since $f_\delta h = id_Y$, it follows that $f_\delta$ is a retract of $f_\delta h$, and is thus a weak equivalence.
            \end{proof}

            \begin{corollary}
                The category $\sfrac{\mathcal{C}_{cf}}{\sim}$ satisfy the universal property of the localization of $\mathcal{C}_{cf}$ by the weak equivalences. I.e. there is a categorical equivalence $Ho\mathcal{C}_{cf} \simeq \sfrac{\mathcal{C}_{cf}}{\sim}$.
            \end{corollary}

            \begin{proof}
                By generalized Whiteheads theorem, theorem \ref{thm: Whitehead} weak equivalences and homotopy equivalences coincide. The corollary follows steadily from both the universal property of the localization category and the quotient category. 
            \end{proof}

            We collect the results from above in the following theorem.

            \begin{thm}[Fundamental theorem of model categories]\label{thm: Fundamental-thm-model}
                Let $\mathcal{C}$ be a model category and denote $q: \mathcal{C} \rightarrow Ho\mathcal{C}$ the localization functor. Let $X$ and $Y$ be objects of $\mathcal{C}$.
                \begin{enumerate}
                    \item There is an equivalence of categories $Ho\mathcal{C}\simeq \sfrac{\mathcal{C}_{cf}}{\sim}$.
                    \item There are natural isomorphisms $\sfrac{\mathcal{C}_{cf}}{\sim}(QRX,QRY)\simeq Ho\mathcal{C}(X, Y) \simeq \sfrac{\mathcal{C}_{cf}}{\sim}(RQX, RQY)$. Additionally, $Ho\mathcal{C}(X, Y)\simeq \sfrac{\mathcal{C}_{cf}}{\sim}(QX, RY)$.
                    \item The localization $q$ identifies left or right homotopic morphisms.
                    \item A morphism $f: X \rightarrow Y$ is a weak equivalence if and only if $qf$ is an isomorphism.
                \end{enumerate}
            \end{thm}

            \begin{proof}
                This is clear by the results above.
            \end{proof}

        \subsection{Quillen adjoints}

            We now want to study morphisms, or certain functors, between model categories. Like in the case of homotopical functors we want these morphisms to induce a functor between the homotopy categories. However, we also want them to respect the cofibration and fibration structure, not just weak equivalences. In this way we will instead look towards derived functors to be able to define this extension to the homotopy category. We recall the definition of a total (left/right) derived functor. In the case of model categories, we get a simple description for some of these derived functors which are of special interest.

            \begin{definition}[Total derived functors]
                Let $\mathcal{C}$ and $\mathcal{D}$ be homotopical categories, and $F:\mathcal{C}\rightarrow\mathcal{D}$ a functor. Whenever it exists, a total left derived functor of $F$, is a functor $\mathbb{L}F:Ho\mathcal{C}\rightarrow Ho\mathcal{D}$ with a natural transformation $\varepsilon :\mathbb{L}F\circ q \Rightarrow q\circ F$ satisfying the universal property: If $G:Ho\mathcal{C}\rightarrow Ho\mathcal{D}$ is a functor and there is a natural transformation $\alpha :G\circ q \Rightarrow q\circ F$, then it factors uniquely up to unique isomorphism through $\varepsilon$.
                \begin{center}
                    \begin{tikzcd}
                        \mathcal{C} \ar[]{r}[]{F} \ar[]{d}[]{q} & \mathcal{D} \ar[]{d}[]{q} \\
                        Ho\mathcal{C} \ar[dashed]{r}[]{\mathbb{L}F} \ar[Rightarrow]{ru}[]{\varepsilon} & Ho\mathcal{D}
                    \end{tikzcd}\qquad
                    \begin{tikzcd}
                        \mathcal{C} \ar[]{r}[]{F} \ar[]{d}[]{q} & \mathcal{D} \ar[phantom, ""{name = Y}]{}[]{} \ar[]{d}[]{q} \\
                        Ho\mathcal{C} \ar[bend left, ""{name = V, below}, ""{name=X, above}]{r}[description]{\mathbb{L}F} \ar[bend right, ""{name = U, above}]{r}[description]{G} \ar[Rightarrow, from = U, to = V]{}[]{\exists !} \ar[Rightarrow, from = X, to = Y, end anchor = {[yshift = -0.5ex, xshift = -1ex]south west}]{}[]{\varepsilon} & Ho\mathcal{D}
                    \end{tikzcd}
                \end{center}

                Dually, whenever it exists, a total right derived functor of $F$, is a functor $\mathbb{R}F:Ho\mathcal{C}\rightarrow Ho\mathcal{D}$ with a natural transformation $\eta :q\circ F \Rightarrow \mathbb{R}F \circ q$ having the opposite universal property.
                \begin{center}
                    \begin{tikzcd}
                        \mathcal{C} \ar[]{r}[]{F} \ar[]{d}[]{q} & \mathcal{D} \ar[]{d}[]{q} \ar[Rightarrow]{ld}[]{\eta} \\
                        Ho\mathcal{C} \ar[dashed]{r}[]{\mathbb{R}F} & Ho\mathcal{D}
                    \end{tikzcd}\qquad
                    \begin{tikzcd}
                        \mathcal{C} \ar[]{r}[]{F} \ar[]{d}[]{q} & \mathcal{D} \ar[phantom, ""{name = Y}]{}[]{} \ar[]{d}[]{q} \\
                        Ho\mathcal{C} \ar[bend left, ""{name = V, below}, ""{name=X, above}]{r}[description]{\mathbb{R}F} \ar[bend right, ""{name = U, above}]{r}[description]{G} \ar[Rightarrow, from = V, to = U]{}[left]{\exists !} \ar[Rightarrow, from = Y, to = X, start anchor = {[yshift = -0.5ex, xshift = -1ex]south west}]{}[left, yshift = 0.5ex]{\eta} & Ho\mathcal{D}
                    \end{tikzcd}
                \end{center}
            \end{definition}

            \begin{definition}[Deformation]
                A left (right) deformation on a homotopical category $\mathcal{C}$ is an endofunctor $Q$ together with a natural weak equivalence $q:Q \Rightarrow Id_\mathcal{C}$ ($q:Id_\mathcal{C}\Rightarrow Q$).

                A left (right) deformation on a functor $F:\mathcal{C}\rightarrow\mathcal{D}$ between homotopical categories, is a left (right) deformation $Q$ on $\mathcal{C}$ such that weak equivalences in the image of $Q$ is preserved by $F$.
            \end{definition}

            \begin{remark}[Cofibrant and fibrant replacement]
                If $\mathcal{C}$ is a model category, then we have a left and a right deformation. The cofibrant replacement $Q$ defines a left deformation, and the fibrant replacement defines a right deformation. Notice that this is only due to the fact that the factorization system is functorial.
            \end{remark}

            \begin{proposition}
                Let $F:\mathcal{C}\rightarrow\mathcal{D}$ be a functor between homotopical categories. If $F$ has a left deformation $Q$, then the total left derived functor $\mathbb{L}F$ exists. Moreover, the functor $FQ$ is homotopical, and $\mathbb{L}F$ is the unique extension of $FQ$.
            \end{proposition}

            \begin{proof}
                Since we already have a candidate for the derived functor, the proof must just check that it has the universal property. A proof may be found in Reihl \cite{Riehl16} under proposition 6.4.11.
            \end{proof}

            \begin{remark}
                There is a somewhat weaker statement by Dwayer and Spalinski \cite{Dwyer95}. If we instead ask for functors $F$ which have the cofibrant replacement $Q$ (fibrant replacement $R$) as a left (right) deformation we may make this proof more explicit. This is theorem 9.3.
            \end{remark}

            Equipped with the above proposition and remark, it makes sense to define Quillen functors as left and right Quillen functors. A left Quillen functor should be left deformable by the cofibrant replacement. Moreover, for the composition of two left Quillen functors to make sense, we also need weak equivalences between cofibrant objects to be mapped to weak equivalences between cofibrant objects. We make the following definition.

            \begin{definition}[Quillen adjunction]
                Let $\mathcal{C}$ and $\mathcal{D}$ be model categories. \begin{enumerate}
                    \item A left Quillen functor is a functor $F:\mathcal{C}\rightarrow\mathcal{D}$ such that it preserves cofibrations and acyclic cofibrations.
                    \item A right Quillen functor is a fucntor $F:\mathcal{C}\rightarrow\mathcal{D}$ such that it preserves fibrations and acyclic fibrations.
                    \item Suppose that $(F,U)$ is an adjunction where $F:\mathcal{C}\rightarrow\mathcal{D}$ is left adjoint to $U$. $(F,U)$ is called a Quillen adjunction if $F$ is a left Quillen functor and $U$ is a right Quillen functor.
                \end{enumerate}
            \end{definition}

            \begin{remark}
                By Ken Browns lemma, lemma \ref{lem: Ken-Brown}, we see that a left Quillen functor $F$ is left deformable to the cofibrant replacement functor $Q$. Thus the total left derived functor exists and is given by $\mathbb{L}F = Ho FQ$.
            \end{remark}

            In order to eliminate the choice of left or right derivedness, we will think of a morphism of model categories as a Quillen adjunction. The direction of the arrow can be chosen to be along either the left or right adjoints, we make the convention of following the left adjoint functors. We summarize the following properties.

            \begin{lemma}\label{lem: Quill-adj}
                Let $\mathcal{C}$ and $\mathcal{D}$ be model categories, and suppose there is an adjunction $F:\mathcal{C}\rightleftharpoons\mathcal{D}:U$. The following are equivalent:
                \begin{enumerate}
                    \item $(F,U)$ is a Quillen adjunction.
                    \item $F$ is a left Quillen functor.
                    \item $U$ is a right Quillen functor.
                \end{enumerate}
            \end{lemma}

            \begin{proof}
                This follows from naturality of the adjunction. I.e. any square in $\mathcal{C}$, with the right side from $\mathcal{D}$ is commutative if and only if any square in $\mathcal{D}$ with the left side from $\mathcal{C}$ is commutative. Now, $f$ has LLP with respect to $Ug$ if and only if $Ff$ has LLP with respect to $g$.
                \begin{center}
                    \begin{tikzcd}
                        A \ar[]{r}[]{k} \ar[]{d}[left]{f} & UX \ar[]{d}[]{Ug} \\
                        B \ar[]{r}[]{l} \ar[dotted]{ru}[]{h} & UY
                    \end{tikzcd} $\rightsquigarrow$
                    \begin{tikzcd}
                        FA \ar[]{r}[]{k^T} \ar[]{d}[left]{Ff} & X \ar[]{d}[]{g} \\
                        FB \ar[]{r}[]{l^T} \ar[dotted]{ru}[]{h^T} & Y
                    \end{tikzcd}
                \end{center} 
            \end{proof}

            \begin{proposition}
                Suppose that $(F,U):\mathcal{C}\rightarrow\mathcal{D}$ is a Quillen adjunction. The functors $\mathbb{L}F:Ho\mathcal{C}\rightarrow Ho\mathcal{D}$ and $\mathbb{R}U: Ho\mathcal{D}\rightarrow Ho\mathcal{C}$ forms an adjoint pair.
            \end{proposition}

            \begin{proof}
                We must show that $Ho\mathcal{D}(\mathbb{L}FX, Y) \simeq Ho\mathcal{D}(X, \mathbb{R}UY)$. By using the fundamental theorem of model categories, theorem \ref{thm: Fundamental-thm-model}, we have the following isomorphisms: $Ho\mathcal{D}(\mathbb{L}FX,Y)\simeq \sfrac{\mathcal{C}(FQX,RY)}{\sim}$ and $Ho\mathcal{D}(X,\mathbb{R}UY)\simeq\sfrac{\mathcal{D}(QX,URY)}{\sim}$. In other words, if we assume $X$ to be cofibrant, and $Y$ to be fibrant, we must show that the adjunction preserves homotopy equivalences.

                We show it for one direction. Suppose that the morphisms $f,g:FA\rightarrow B$ are homotopic, witnessed by a right homotopy $H:FA\rightarrow B^I$. Since we assume $U$ to preserve products, fibrations and weak equivalences between fibrant objects, $U(B^I)$ is a path object for $UB$. Thus the transpose $H^T:A\rightarrow U(B^I)$ is the desired homotopy witnessing $f^T \sim g^T$
            \end{proof}

            \begin{definition}[Quillen equivalence]
                Let $\mathcal{C}$ and $\mathcal{D}$ be model categories, and $(F,U):\mathcal{C}\rightarrow\mathcal{D}$ be a Quillen adjunction. $(F,U)$ is called a Quillen equivalence if for any cofibrant $X$ in $\mathcal{C}$, fibrant $Y$ in $\mathcal{D}$ and any morphism $f:FX\rightarrow Y$ is a weak equivalence if and only if its transpose $f^T:X\rightarrow UY$ is a weak equivalence.
            \end{definition}

            \begin{proposition}\label{prop: Quill-Eq}
                Suppose that $(F,U):\mathcal{C}\rightarrow\mathcal{D}$ is a Quillen adjunction. The following are equivalent:
                \begin{enumerate}
                    \item $(F,U)$ is a Quillen equivalence.
                    \item Let $\eta :Id_\mathcal{C}\Rightarrow UF$ denote the unit, and $\varepsilon :FU\Rightarrow Id_\mathcal{D}$ denote the counit. The composite $Ur_{F}\eta : Id_{\mathcal{C}_c} \Rightarrow URF|_{\mathcal{C}_c}$, and $\varepsilon_{FQU}Fq_{U}:FQU|_{\mathcal{D}_f} \Rightarrow Id_{\mathcal{D}_f}$ are natural weak equivalences.
                    \item The derived adjunction $(\mathbb{L}F, \mathbb{R}U)$ is an equivalence of categories.
                \end{enumerate}
            \end{proposition}

            \begin{proof}
                Firstly observe that $2.\implies3.$ by definition. Secondly observe that equivalences both preserves and reflects isomorphisms, from this we get $3. \implies 1.$. We now show $1.\implies 2.$. Pick $X$ in $\mathcal{C}$ such that $X$ is cofibrant. Since $(F,U)$ is assumed to be a Quillen adjunction we know that $FX$ is still cofibrant. The fibrant replacement $r_{FX}:FX\rightarrow RFX$ gives us a weak equivalence. Furthermore, since $(F,U)$ is assumed to be a Quillen equivalence, its transpose $r_{FX}^T: X \rightarrow URFX$ is a weak equivalence. Unwinding the definition of the transpose we get that $r_{FX}^T = Ur_{rFX}\eta_X$.

            \end{proof}

            We have the following refinement.

            \begin{corollary}\label{cor: Quill-Eq}
                Suppose that $(F,U):\mathcal{C}\rightarrow\mathcal{D}$ is a Quillen adjunction. The following are equivalent:
                \begin{enumerate}
                    \item $(F,U)$ is a Quillen equivalence.
                    \item $F$ reflects weak equivalences between cofibrant objects, and $\varepsilon_{FQU}F_{qU} : FQU|_{\mathcal{D}_f} \Rightarrow Id_{\mathcal{D}_f}$ is a natural weak equivalence.
                    \item $U$ reflects weak equivalences between fibrant objects, and $U_{rF}\eta : Id_{\mathcal{C}_c} \Rightarrow URF|{\mathcal{C}_c}$ is a natural weak equivalence.
                \end{enumerate}
            \end{corollary}

            \begin{proof}
                We start by showing $1. \implies 2.$ and $3.$. We already know that the derived unit and counit are isomorphism in homotopy, so we only need to show that $F$ ($U$) reflects weak equivalences between cofibrant (fibrant) objects. Suppose that $Ff:FX\rightarrow FY$ is a weak equivalence between cofibrant objects. Since $F$ preserves weak equivalences between cofibrant objects, we get that $FQf$ is a weak equivalence, or that $\mathbb{L}Ff$ is an isomorphism. By assumption, $\mathbb{L}F$ is an equivalence of categories, so $f$ is a weak equivalence as needed.

                We will show $2.\implies 1.$, the case $3.\implies 1.$ is dual. We assume that the counit map is an isomorphism in homotopy. By assumption, the derived unit $\mathbb{L}\eta$ is split-mono on the image of $\mathbb{L}F$. Moreover, the derived counit $\mathbb{R}\varepsilon$ is assumed to be an isomorphism, in particular the derived unit $\mathbb{L}F\mathbb{L}\eta$ is an isomorphism. Unpacking this, we have a morphism, call it $\eta_X' : FQX \rightarrow FQURFQX$, which is a weak equivalence. Since $F$ and $Q$ reflects weak equivalences, we get that $\eta_X: X \rightarrow URFQX$ is a weak equivalence.
            \end{proof}

    \section{Model structures on Algebraic Categories}

            In order to understand $\infty$-quasi-isomorphism of strongly homotopy associative algebras we will study different homotopy theories of various categories. Munkholm \cite{Munkholm78} successfully showed that the derived category of augmented algebras is equivalent to the derived category of augmented algebras equipped with $\infty$-morphisms. Well, to be more precise, he showed that certain subcategories of augmented algebras had this property. Lefevre-Hasagawas phd thesis \cite{LefevreHasegawa03} builds upon this identification, but with help of further devolpment within the field. We will follow the approach of Lefevre-Hasagawa, by comparing the model structure for algebras and coalgebras,

        \subsection{DG-Algebras as a Model Category}

            Bousefield and Gugenheim \cite{Bousfield76} proved that the category of commutative dg-algebras had a model structure whenever the base field was a field of characteristic $0$. In a joint project, Jardines paper from 1997 \cite{Jardine97} shows that this construction may be extended to dg-algebras over any commutative ring. On the other hand, Munkholm expanded on the ideas from Bousfield and Gugenheim to get an identification of derived categories. Also, Hinichs paper from 1997 \cite{Hinich97} details another method to obtain the model category which we want. We will follow the approach of Hinich, as it will be usefull later on. Notice that where Hinich use theory of algebraic operads to show that the category of algebras is a model category, we will however give a more explicit formulation.

            Let $\mathbb{K}$ be a field, and $\mathcal{C}$ be a category such that there is an adjunction $F:\mathcal{C}\rightleftharpoons Ch(\mathbb{K}):\#$, where $F$ is left adjoint to $\#$. Furthermore, suppose that $\mathcal{C}$ satsifies the $2$ conditions:
            \begin{itemize}
                \item[(H0)] $\mathcal{C}$ admits finite limits and every small colimit. The functor $\#$ commutes with filtered colimits.
                \item[(H1)] Let $M$ be the complex below, concentrated in $0$ and $1$.
                \begin{center}
                    \begin{tikzcd}
                        ... \ar[]{r}[]{} & 0 \ar[]{r}[]{} & \mathbb{K} \ar[]{r}[]{id} & \mathbb{K} \ar[]{r}[]{} & 0 \ar[]{r}[]{} & ...
                    \end{tikzcd}
                \end{center}
                For any $d\in \mathbb{Z}$ and for any $A\in\mathcal{C}$ the injection $A \rightarrow A \coprod F(M[d])$ induces a quasi-isomorphism $A^\# \rightarrow (A\coprod F(M[d]))^\#$.
            \end{itemize}
                
            With this adjunction in mind, we define weak equivalences, fibrations and cofibrations as follows:
            Let $f\in \mathcal{C}$ be a morphism
            \begin{itemize}
                \item $f\in Ac$ if $f^\#$ is a quasi-isomorphism.
                \item $f\in Fib$ if $f^\#$ is surjective on each component.
                \item $f\in Cof$ if $f$ has LLP to acyclic fibrations.
            \end{itemize}

            \begin{thm}
                The category $\mathcal{C}$ equipped with the weak equivalences, fibrations and cofibrations as defined above is a model category.
            \end{thm}

            Before we show this theorem we need to understand the cofibrations better. Let $A\in\mathcal{C}$, $M\in Ch(\mathbb{K})$ and $\alpha : M \rightarrow A^\#$ a morphism in $Ch(\mathbb{K})$. We define a functor 
            \begin{align*}
                h_{A,\alpha}(B) = \startset{(f,t)\mid f\in \mathcal{C}(A,B), t\in Hom^{-1}_\mathbb{K}(M, B^\#)\ s.t.\ \partial t = f^\#\circ\alpha}.
            \end{align*}
            Note that $t$ is not a morphism of chain maps. This is a homogenous morphism of degree $-1$. The differential then promotes this morphism to a chain map, and $t$ is thus a homotopy for the comoposite $f^\#\circ\alpha$.

            This functor is represented by an object of $\mathcal{C}$. We define this representing object $A\langle M, \alpha\rangle$ as the pushout:
            \begin{center}
                \begin{tikzcd}
                    F(A^\#) \ar[]{r}[]{\varepsilon_A} \ar[]{d}[]{} \ar[phantom]{rd}[near end]{\lrcorner} & A \ar[]{d}[]{a} \\
                    F(cone(\alpha)) \ar[]{r}[]{e} & A\langle M,\alpha\rangle
                \end{tikzcd}
            \end{center}
            Let $i: M[1] \rightarrow cone(\alpha)$ be a homogenous morphism which is the injection when considered as graded modules. Notice that we have a pair of morphisms $(a, e^Ti)\in h_{A,\alpha}(A\langle M,\alpha\rangle)$.
                
            \begin{proposition}\label{prop: universal-h}
                The functor $h_{A,\alpha}$ is represented by $A\langle M,\alpha\rangle$, i.e. $h_{A,\alpha}\simeq \mathcal{C}(A\langle M,\alpha\rangle,\_)$ is a natural isomorphism. Moreover, the pair $(a,e^Ti)$ is the universal element of the functor $h_{A,\alpha}$, i.e. the natural isomorphism is induced by this element under Yoneda's lemma.
            \end{proposition}

            \begin{proof}
                Let $(f,t)\in h_{A,\alpha}(B)$ for some $B\in\mathcal{C}$. The condition that $\partial t = f^\#\alpha$ is equivalent to say that $f^\#$ extends to a morphism $f' : cone(\alpha) \rightarrow B^\#$ along $t$, i.e. $f' = \begin{pmatrix}f^\# & t\end{pmatrix}$. This concludes the isomorphism part, as being an element $(f,t)$ is equivalent to the existence of the diagram below, where $\widetilde{f}$ is uniquely determined.
                \begin{center}
                    \begin{tikzcd}
                        F(A^\#) \ar[]{r}[]{\varepsilon_A} \ar[]{d}[]{} & A \ar[]{d}[]{a} \ar[bend left]{ddr}[]{f} \\
                        F(cone(\alpha)) \ar[bend right]{rrd}[]{f'} \ar[]{r}[]{e} & A\langle M,\alpha\rangle \ar[dashed]{rd}[]{\widetilde{f}} \\
                        & & B
                    \end{tikzcd}
                \end{center}
  
                To obtain naturality, we use the adjunction to observe that the element $(a, e^Ti)$ is in fact universal.
            \end{proof}

            We are now in a position to explicitly find some important cofibrations. We collect these morphisms into the "standard" cofibrations.

            \begin{definition}
                Let $f:A\rightarrow B$ be a morphism in $\mathcal{C}$. Suppose that $f$ factors as a transfinite composition of morphisms on the form $A_i \rightarrow A_i\langle M,\alpha\rangle$, i.e. $f$ factors into the diagram below, where $A_{i+1} = A_i\langle M,\alpha\rangle$.
                \begin{center}
                    \begin{tikzcd}
                        A \ar[]{r}[]{} & A_1 \ar[]{r}[]{} & A_2 \ar[]{r}[]{} & ... \ar[]{r}[]{} & B
                    \end{tikzcd}
                \end{center}
                \begin{itemize}
                    \item If every such $M$ is a complex consisting of free $\mathbb{K}$-modules and has a $0$-differential, we call $f$ a standard cofibration.
                    \item If every such $M$ is a contractible complex (and $\alpha = 0$), we call $f$ a standard acyclic cofibration.
                \end{itemize}
            \end{definition}

            \begin{proposition}
                Every standard cofibration is a cofibration, and every standard acyclic cofibration has LLP with respect to fibrations. Moreover, if $\alpha = 0$, then every standard acyclic cofibration is also a weak equivalence. We will see that these morphisms in some sense generate every (acyclic) cofibration.
            \end{proposition}

            \begin{proof}
                First observe that every standard cofibration may be made iteratively from the chain complexes $\mathbb{K}[n]$, and likewise, every standard acyclic cofibration may be made iteratively from $M$ as in $H1$.

                We first prove that if $M \simeq \mathbb{K}[n]$, and $\alpha : M \rightarrow A^\#$ is any map, then the map $A \rightarrow A\langle M,\alpha\rangle$ is a cofibration. This amounts to show that it has LLP to every acyclic fibration. Suppose that $h: B \rightarrow C$ is an acyclic fibration and that there is a commutative square as below.
                \begin{center}
                    \begin{tikzcd}
                        A \ar[]{r}[]{f} \ar[]{d}[]{a} & B \ar[]{d}[]{h} \\
                        A\langle M,\alpha\rangle \ar[]{r}[]{g} & C
                    \end{tikzcd}
                \end{center}
                
                By the universal property of $h_{A,\alpha}$ \ref{prop: universal-h} it suffices to find a pair $(f',t')$ such that $f: A \rightarrow B$, $t' : M \rightarrow B^\#$ is homogenous of degree $-1$, $\partial t = f^\#\alpha$ and that $h$ induces a morphism $h : (f',t') \rightarrow g$. We see that we are forced to choose $f' = f$ as $hf = ga$. We know there exists a $t : M \rightarrow C^\#$ such that $\partial t = g^\#a^\#\alpha = h^\#f^\#\alpha$. Since $h$ is an acyclic fibration $h^\#$ is a surjective quasi-isomorphism. Since $M \simeq \mathbb{K}[n]$, the morphism $t$ is really an element of $C^{\#^{n-1}}$. By surjectivity of $h^\#$ there is an element $u$ of $B^{\#^{n-1}}$ such that $h^\#(u) = t$. Moreover, the difference $h^\#(\partial u - f^\#\alpha) = 0$, so $\partial u - f^\#\alpha$ factors through the kernel $Ker h^\#$, which is acyclic. This element is furthermore a cycle, so by acyclicity there is another element $u'$ such that $\partial u' = \partial u - f^\#\alpha$. We may now see that $(f, u - u')$ is our desired factorization.
                
                Secondly we prove that if $M$ is in as $H1$ and $\alpha = 0$, then the map $A \rightarrow A\langle M,\alpha\rangle$ is an acyclic cofibration.This amounts to show that it has LLP to every fibration. Suppose that $h: B \rightarrow C$ is a fibration and that there is a commutative square as below.
                \begin{center}
                    \begin{tikzcd}
                        A \ar[]{r}[]{f} \ar[]{d}[]{a} & B \ar[]{d}[]{h} \\
                        A\langle M,\alpha\rangle \ar[]{r}[]{g} & C
                    \end{tikzcd}
                \end{center}

                We will again use \ref{prop: universal-h}, so it suffices to find a $(t')$ such that $\partial t' = f^\#\alpha$. Let $t : M \rightarrow C^\#$ such that $\partial t = h^\# f^\# \alpha$. 

                Finally we will assume that $\alpha = 0$. In this case the cone is a direct sum $cone(\alpha) = A^\# \oplus M$. Since $F$ is left adjoint to $\#$, we know that $Fcone(\alpha) \simeq F(A^\#)\coprod FM$. By $H1$ we get that the map $A \rightarrow A\langle M,\alpha\rangle \simeq A\coprod FM$ is a weak equivalence.

            \end{proof}

            In light of the above proposition we would like to make some more convenient notation. If $M\simeq \mathbb{K}[n]$ and $\alpha: M \rightarrow Z^n(A^\#)$, s.t. $\alpha(1) = a$, we write $A\langle M,\alpha\rangle$ as $A\langle T; dT = a\rangle$ instead. Hinich calls this "adding a variable to kill a cycle". If $M$ is the acyclic complex as below and $\alpha = 0$, we write $A\langle T, S; dT = S\rangle$. This could be thought of "adding a variable and cycle to kill itself".

            \begin{center}
                \begin{tikzcd}
                    ... \ar[]{r}[]{} & 0 \ar[]{r}[]{} & \mathbb{K} \ar[]{r}[]{id} & \mathbb{K} \ar[]{r}[]{} & 0 \ar[]{r}[]{} & ...
                \end{tikzcd}
            \end{center}

            \begin{proof}[proof of theorem]
                \textbf{MC1} and \textbf{MC2} are satisfied. By definition we also have the first part of \textbf{MC3}. We start by checking \textbf{MC4}.

                Let $f:A\rightarrow B$ be a morphism in $\mathcal{C}$. Given any $b\in B^\#$, let $C_b = A\langle T_b,S_b; dT_b = S_b\rangle$. We define $g_b: C_b \rightarrow B$ by the conditions that it acts on $A$ as $f$, $g_b^\#(T_b) = b$ and $g_b^\#(S_b)=db$. Iterating this construction for every $b\in B$, we obtain an object $C$, such that the injection $A \rightarrow C$ is an acyclic standard cofibration, and the map $g : C \rightarrow B$ is a fibration. This gives us a factorization $f = f_\delta\circ f_\gamma$, where $f_\gamma$ is the injection and $f_\delta = g$.

                To obtain the other factorization we start with our previous factorization. We let $C_0 = C$. From $C_0$ there is a morphism $g_0 : C_0 \rightarrow B$, which is surjective and surjective on kernels. This may fail to be a quasi-isomorphism. Pick a pair of elements $(c,b)$, such that $c\in ZC_0^\#$ and $g_0^\#(c) = db$. We add a variable to $C_0$ to kill this cycle, i.e. let $C_1 = C_0\langle T, dT = c\rangle$. We iterate this procedure for any boundary in $B^\#$, to obtain $C$ as the colimit. Then the injection $A \rightarrow C$ is a colimit of standard cofibrations and the morphism $g' : C \rightarrow B$ is an acyclic fibration.
                
                It remains to check the last part of \textbf{CM3}. Suppose that $f:A\rightarrow B$ is an acyclic cofibration. By \textbf{CM4}, we know that it factors as $f = f_\delta \circ f_\gamma$, where $f_\delta$ is an acyclic fibration and $f_\gamma$ is a standard acyclic fibration. We thus obtain that $f$ is a retract of $f_\gamma$ by the commutative diagram below.
                \begin{center}
                    \begin{tikzcd}
                        A \ar[]{r}[]{f_\gamma} \ar[]{d}[]{f} & C \ar[]{d}[]{f_\delta} \\
                        B \ar[equal]{r}[]{} \ar[dashed]{ru}[]{} & B
                    \end{tikzcd}
                \end{center}
            \end{proof}

            The following corollary will concretize what it means that the standard cofibrations generate every cofibration. This corollary is really a step used within in the proof.

            \begin{corollary}
                Any (acyclic) cofibration is a retract of a standard (acyclic) cofibration.
            \end{corollary}

            We may immedietly apply this theorem to some familiar examples.

            \begin{corollary}
                Let $A$ be a dg-algebra over the field $\mathbb{K}$. The category $mod_A$ of left modules is a model category.
            \end{corollary}

            \begin{proof}[sketch of proof]
                We establish the adjunction by letting $F = A\otimes_\mathbb{K}\_$. H0 is satisfied as this category is bicomplete, and filtered colimits may be thought of as unions of sets. Moreover, since $mod_A$ is an Abelian category, the forgetful functor $\#$ commutes with coproducts, or direct sums, which makes H1 trivially satisfied.
            \end{proof}

            \begin{corollary}
                The categories $Alg^\bullet_\mathbb{K}$ ($AugAlg^\bullet_\mathbb{K}$) are model categories.
            \end{corollary}

            \begin{proof}
                We establish the adjunction by letting $F = \bar{T} (T)$, the reduced tensor algebra of a cochain complex. For the same reasons as above, H0 is trivially satisfied. To show H1 we must find the coproduct and what $\bar{T}(M)$ is. ...
            \end{proof}

            We summarize the last result:

            The category of augmented dg-algebras $AugAlg^\bullet_\mathbb{K}$ is a model category. Let $f:X\rightarrow Y$ be a homomorphism of augmented algebras. 
            \begin{itemize}
                \item $f\in Ac$ if $f^\#$ is a quasi-isomorphism.
                \item $f\in Fib$ if $f^\#$ is an epimorphism (surjective onto every component).
                \item $f\in Cof$ if $f$ has LLP wrt. to every acyclic fibration.
            \end{itemize}
            
            The category of augmented dg-algebras has an initial and a terminal object. The initial object is the stalk $\bar{\mathbb{K}}$ and the terminal object is the $0$-ring. We see that every object is fibrant, as $0$ is preserved by the forgetful functor and every map into $0$ is surjective. Every dg-algebra which is isomorphic to a tensor algebra when considered as a graded algebra is cofibrant.
            
    \subsection{A Model Structure on DG-Coalgebras}

            We now want to equip the category of dg-coalgebras with a suitable model structure. This model structure should be suitable in the sense that it give rise to the same homotopy theory of dg-algebras. The bar-cobar construction will be crucial in this construction, as it is in fact a Quillen adjunction. To this end we will follow the setup as presented by Lefevre-Hasegawa \cite{LefevreHasegawa03}. His method is a modification of Hinichs paper \cite{Hinich01} which describes a model structure on dg-coalgebras, but in relation to dg-lie algebras.

            Let $f: C \rightarrow D$, the category of dg-coalgerbas will be equipped with the three following classes of morphisms:
            \begin{itemize}
                \item $f\in Ac$ if $\Omega f$ is a quasi-isomorphism.
                \item $f\in Fib$ if $f$ has RLP wrt. to every acyclic cofibration.
                \item $f\in Cof$ if $f^\#$ is a monomorphism (injective in every component).
            \end{itemize}

            Before we show that this defines a model structure on dg-coalgebras we need some preliminary results. First recall that $f : C \rightarrow D$ a morphism between dg-coalgebras is a graded quasi-isomorphism if $grf$ is a quasi-isomorphism. Lemma \ref{lem: graded-qif-are-w} says that whenever such $f$ are graded quasi-isomorphisms, $f$ is a weak equivalence. This is quite convenient, since this gives us a method to check that a morphism is a weak equivalence without evaluating it on the cobar construction. Moreover, both the unit and the counit of the bar-cobar adjunction are weak equivalences by proposition \ref{prop: unit-counit-qif}. With the following lemma we see that the bar-cobar adjunction is a Quillen equivalence if the category of dg-coalgebras is a model category.

            \begin{lemma}\label{lem: bar-cobar-Quill-adj}
                Let $f: C\rightarrow D$ be a morphism of dg-coalgebras, then:
                \begin{itemize}
                    \item if $f$ is a cofibration, then $\Omega f$ is a standard cofibration.
                    \item if $f$ is a weak equivalence, then $\Omega f$ is as well.
                \end{itemize}

                Almost dually, let $f: A\rightarrow B$ be a morphism of dg-algebras, then:
                \begin{itemize}
                    \item if $f$ is a fibration, then $B f$ is a fibration.
                    \item if $f$ is a weak equivalence, then $B f$ is as well.
                \end{itemize}
            \end{lemma}

            \begin{proof}
                First suppose that $f : C\rightarrow D$ is a cofibration. We define a filtration on $D$ as the sum of the image of $f$ and the coradical filtration on D: $D_i = Imf + Fr_iD$. $f$ being a cofibration ensures us that $D_0 \simeq C$. Since $D$ is conilpotent we know that $D \simeq \varinjlim D_i$, and that $\Omega$ commutes with colimits, there is a sequence of algebras $\Omega C \rightarrow \Omega D_1 \rightarrow ... \rightarrow \Omega D$. It is enough to show that each morphism $\Omega D_i \rightarrow \Omega D_{i+1}$ is a standard cofibration. The quotient coalgebra $\sfrac{D_{i+1}}{D_i}$ only has a trivial comultiplication, thus every element is primitive. This means that as a cochain complex $D_{i+1}$ is constructed from $D_i$ by attaching possibly very many copies of $\mathbb{K}$. We treat the case when there is only one such $\mathbb{K}$, here $D_{i+1} \simeq D_i \oplus \mathbb{K}\startset{x}$ where $dx = y$ for some $y\in D_i$. We observe that this is exactly the condition for that the morphism $\Omega D_i \rightarrow \Omega D_{i+1}$ is a standard cofibration.

                If $f$ is a weak equivalence, then $\Omega f$ is a quasi-isomorphism by definition.

                By lemma \ref{lem: Quill-adj}, or adjointness more specifically, $B$ preserving fibrations is a consequence of $\Omega$ preserving cofibrations.

                It remains to show that if $f: A\rightarrow B$ is a quasi-isomorphism, then $Bf$ is a weak equivalence. Now, $Bf$ is a weak equivalence if and only if $\Omega Bf$ is a quasi-isomorphism. By \ref{prop: unit-counit-qif}, the counit $A \rightarrow \Omega BA$ is a quasi-isomorphism, so $Bf$ is a weak equivalence by 2-out-of-3 property.

                \begin{center}
                    \begin{tikzcd}
                        A \ar[]{r}[]{f} & B \\
                        \Omega BA \ar[]{u}[]{\varepsilon_A} \ar[]{r}[]{\Omega Bf} & \Omega BB \ar[]{u}[]{\varepsilon_B}
                    \end{tikzcd}
                \end{center}
            \end{proof}

            We will need one more technical lemma.

            \begin{lemma}\label{lem: tech-fac}
                Let $A$ be a dg-algebra, $D$ a dg-coalgebra and $p: A \rightarrow \Omega D$ a fibration of algebras. The projection morphism $\pi : BA\prod_{B\Omega D}D \rightarrow BA$ is an acyclic cofibration.
                \begin{center}
                    \begin{tikzcd}
                        BA\prod_{B\Omega D}D \ar[]{d}[]{\pi} \ar[]{r}[]{} & D \ar[]{d}[]{\eta_D} \\
                        BA \ar[]{r}[]{Bp} & B\Omega D
                    \end{tikzcd}
                \end{center}
            \end{lemma}

            \begin{proof}
                $\pi$ being a cofibration is immediate by corollary \ref{cor: stable-cofib-base-change}. To see that $\pi$ is a quasi-isomorphism it is enough to understand that it is a quasi-isomorphism as chain complexes. This is checked by Lefevre-Hasegawa \cite{LefevreHasegawa03}.
            \end{proof}

            \begin{thm}
                The category $ConilCoalg^\bullet_{\mathbb{K}}$ is a model category with the classes $Ac$, $Fib$ and $Cof$ as defined above.
            \end{thm}

            \begin{proof}
                The axioms \textbf{MC1} and \textbf{MC2} are immediet. Also, fibrations having RLP wrt. acyclic cofibrations is by definition.

                We show \textbf{MC4} first. Let $f : C\rightarrow D$ be a morphism of coalgebras. There is a factorization $\Omega f = pi$ of morphisms between algebras, where $i$ is a cofibration, $p$ is a fibration and at least one of $i$ and $p$ are quasi-isomorphisms. Applying bar we get a factorization $B\Omega f = BiBp$, where $Bp$ is a fibration and at least one of $Bi$ and $Bp$ are weak equivalences.
                \begin{center}
                    \begin{tikzcd}
                        \Omega C \ar[]{rr}[]{\Omega f} \ar[]{rd}[]{i} && \Omega D \\
                        & A \ar[]{ru}[]{p}
                    \end{tikzcd} $\rightsquigarrow$
                    \begin{tikzcd}
                        B\Omega C \ar[]{rr}[]{B\Omega F} \ar[]{rd}[]{Bi} && B\Omega D \\
                        & BA \ar[]{ru}[]{Bp}
                    \end{tikzcd}
                \end{center}

                We construct a pullback with $Bp$ and $\eta_D$. By \ref{lem: tech-fac} the morphism $\pi$ is an acyclic cofibration. We collect our morphisms in a big diagram. The dashed arrow exists since the rightmost square is a pullback.
                \begin{center}
                    \begin{tikzcd}
                        & BA\prod_{B\Omega D}D \ar[]{ddd}[]{} \ar[]{rd}[]{q} \ar[phantom]{ddddr}[very near start]{\ulcorner} \\
                        C \ar[dashed]{ru}[]{j} \ar[crossing over]{rr}[near start]{f} \ar[]{ddd}[]{\eta_C} && D \ar[]{ddd}[]{\eta_D} \\
                        \\
                        & BA \ar[]{rd}[]{Bp} \\
                        B\Omega C \ar[]{rr}[]{B\Omega f} \ar[]{ru}[]{Bi} && B\Omega D
                    \end{tikzcd}
                \end{center}
                
                First notice that $q$ is a fibration, since fibrations are stable under pullbacks. $j$ is a cofibration, or a monomorphism, as the composition $Bi\eta_C$ is a monomorphism. Thus it remains to see that if $Bi$ ($Bp$) is a weak equivalence, then $j$ ($q$) is as well. This is evident from the $2$-out-of-$3$ property, as $\eta$ is a natural weak equivalence, $\pi$ is a weak equivalence and $Bi$ ($Bp$) is a weak equivalence.
                
                We now show \textbf{CM3}. Suppose that there is a square as below, where $i$ is a cofibration and $t$ is an acyclic cofibration.
                \begin{center}
                    \begin{tikzcd}
                        E \ar[]{r}[]{} \ar[]{d}[]{i} & C \ar[]{d}[]{t} \\
                        F \ar[]{r}[]{} & D
                    \end{tikzcd}
                \end{center}
                
                We can factor $t$ as $t = qj$ by \textbf{CM4}. Notice that $t$ is a retract of $q$, i.e. there is a commutative diagram as below.
                \begin{center}
                    \begin{tikzcd}
                        C \ar[equal]{r}[]{} \ar[]{d}[]{j} & C \ar[]{d}[]{t} \\
                        BA\prod_{B\Omega A}D \ar[]{r}[]{} \ar[]{r}[]{q} \ar[]{ru}[]{} & D
                    \end{tikzcd}
                \end{center}
                
                So in order to find a lift to $C$, we may instead find a lift to $BA\prod_{B\Omega D}D$. Since $p$ is an acyclic fibration by construction and $\Omega i$ is a cofibration by \ref{lem: bar-cobar-Quill-adj}, there is a lift $h: \Omega E \rightarrow A$ of algebras. We obtain our desired lift from the bar-cobar adjunction and the universal property of the pullback.
                \begin{center}
                    \begin{tikzcd}
                        E \ar[]{d}[]{i} \ar[]{r}[]{} & BA\prod_{B\Omega D}D \ar[crossing over]{d}[]{q} \ar[]{r}[]{\pi} \ar[phantom]{rd}[very near start]{\ulcorner} & BA \ar[]{d}[]{Bp} \\
                        F \ar[]{r}[]{} \ar[dotted]{rru}[near end, below]{h^T} \ar[dashed]{ru}[]{} & D \ar[]{r}[]{\eta_D} & B\Omega D
                    \end{tikzcd} $\leftrightsquigarrow$
                    \begin{tikzcd}
                        \Omega E \ar[]{d}[]{\Omega i} \ar[]{r}[]{} & A \ar[]{d}[]{p} \\
                        \Omega F \ar[]{r}[]{} \ar[dotted]{ru}[]{h} & \Omega D
                \end{tikzcd}
            \end{center}
        \end{proof}

        We restate the corollary of the adjunction.
        \begin{corollary}\label{cor: cobar-bar-quill-eq}
            The bar-cobar construction $B: AugAlg^\bullet_\mathbb{K} \rightleftharpoons ConilCoalg^\bullet_\mathbb{K} : \Omega$ as a Quillen equivalence.
        \end{corollary}

        \begin{proof}
            We first observe that $(B, \Omega)$ is a Quillen adjunction by lemma \ref{lem: bar-cobar-Quill-adj}. Moreover, since the unit and counit are weak equivalences by proposition \ref{prop: unit-counit-qif}, it follows by either proposition \ref{prop: Quill-Eq} or its corollary \ref{cor: Quill-Eq} that $(B, \Omega)$ is a Quillen equivalence.
        \end{proof}

    \subsection{Homotopy theory of $A_{\infty}$-algebras}

        This section aims to finalize the discussion of the homotopy theory of $A_{\infty}$-algebras. We will look at the homotopy invertability of every strongly homotopy associative quasi-isomorphism, and the relation to associative algebras. This discussion will end with mentioning different results which gives a clearer description of fibrations, cofibrations and homotopy equivalences. This section follows Lefevre-Hasegawa \cite{LefevreHasegawa03}. Before we get to the main theorem, we start by discussing a non-closed model structure on the category of $Alg_\infty$.

        Let $f: A \rightsquigarrow B$ be a morphism between $A_\infty$-algebras, the category of $A_\infty$-algebras will be equipped with the three following classes of morphisms:
        \begin{itemize}
            \item $f\in Ac$ if $f$ is an $\infty$-quasi-isomorphism, i.e. $f_1$ is a quasi-isomorphism.
            \item $f\in Fib$ if $f_1$ is an epimorphism.
            \item $f\in Cof$ if $f_1$ is a monomorphism.
        \end{itemize}

        This category does not make a model category in the sense of a closed model category, as we are lacking many finite limits. It does however come quite close to be such a category.

        \begin{thm}\label{thm: model-A-inf}
            The category $Alg_\infty$ equipped with the three classes as defined above satisfies:
            \begin{itemize} 
                \item[a] The axioms \textbf{MC1} through \textbf{MC4}.
                \item[b] Given a diagram as below, where $p$ is a fibration, then its limit exists.
                \begin{center}
                    \begin{tikzcd}
                        & A \ar[]{d}[]{p} \\
                        B \ar[]{r}[]{} & C
                    \end{tikzcd}
                \end{center}
            \end{itemize}
        \end{thm}

        Before we are able to prove this, we need some lemmata.

        \begin{lemma}\label{lem: inf-creator}
            let $A$ be an $A_\infty$-algebra, and $K$ a contractible complex considered as an $A_\infty$-algebra. If $g: (A,m_1^A) \rightarrow (K, m_1^K)$ is a cochain map, then it extends to an $\infty$-morphism $f: A \rightsquigarrow K$.
        \end{lemma}

        \begin{proof}
            We construct each $f_i$ inductively. The case $i=1$ is degenerate as we have assumed $f_1 = g$.

            Assume that we have already constructed $f_1$ through $f_n$. We observe that the sum below is a cycle of $Hom^*_\mathbb{K}(A,K)$.
            \begin{align*}
                \sum_{\substack{p + 1 + r = k \\ p + q + r = n}}(-1)^{pq+r}f_k\circ_{p+1}m^A_q - \sum_{\substack{k\geq 2 \\ i_1 + ... + i_k = n}}(-1)^{e}m^B_k \circ (f_{i_1}\otimes f_{i_2}\otimes ... \otimes f_{i_k})
            \end{align*}
            Thus since $K$ is contractible, $Hom^*_\mathbb{K}(A,K)$ is acyclic and there exists some morphism $f_{n+1}$ such that $\partial (f{n+1})$ is the sum above. This says that this extension does in fact satisfy $(rel_{n+1})$.
        \end{proof}

        \begin{lemma}\label{lem: strict-replacement}
            Let $j: A \rightsquigarrow D$ be a cofibration of $A_\infty$-algberas, then there is an isomorphism $k: D\rightsquigarrow D'$ such that the composition $k\circ j : A \rightsquigarrow D'$ is a strict morphism of $A_\infty$-algebras.

            Dually, if $j: A \rightsquigarrow D$ is a fibration, then there is an isomorphism $l : A' \rightsquigarrow A$ such that the composition $j\circ l : A' \rightsquigarrow D$ is a strict morphism of $A_\infty$-algebras.
        \end{lemma}

        \begin{proof}
            A proof is given as lemma 1.3.3.3 in \cite{LefevreHasegawa03}.
        \end{proof}

        \begin{proof}[proof of \ref{thm: model-A-inf}]
            We start by showing b. Suppose that we have a diagram of $A_\infty$-algebras, such that $g_1$ is an epimorphism.
            \begin{center}
                \begin{tikzcd}
                    & A \ar[two heads]{d}[]{g} \\
                    A' \ar[]{r}[]{f} & A''
                \end{tikzcd}
            \end{center}
            First notice that as dg-coalgebras, this pullback exists and defines a new dg-coalgebra $BA \prod_{BA''}BA'$.

            Since $g_1$ is an epimorphism, $A[1]$ as a graded vector space splits into $A''[1] \oplus K$, where $K = Kerg_1$. The pullback is then naturally identified with $BA \prod_{BA''}BA' \simeq \bar{T}^c(K)\prod \bar{T}^c(A'[1])$ as graded vector spaces. Since the cofree coalgebra is right adjoint to forget, it commutes with products and we get, $\bar{T}^c(A'[1])\prod \bar{T}^c(K) \simeq \bar{T}^c(A'[1]\oplus K)$. Thus the pullback is isomorphic to a cofree coalgebra as a graded coalgebra, i.e. it is an $A_\infty$-algebra.

            We now prove a. MC1 and MC2 are immediate, so we will not prove them.

            We start by proving MC3. Suppose that there is a square of $A_\infty$-algebras as below, where $j$ is a cofibration and $q$ is a fibration.
            \begin{center}
                \begin{tikzcd}
                    A \ar[]{r}[]{f} \ar[]{d}[]{j} & B \ar[]{d}[]{q} \\
                    C \ar[]{r}[]{g} & D
                \end{tikzcd}
            \end{center}

            By lemma \ref{lem: strict-replacement}, we may assume assume that both $j$ and $q$ are strict morphisms. We now assume that $q$ is an $\infty$-quasi-isomorphism, the proof will be analogous if $j$ is an $\infty$-quasi-isomorphism instead.

            Our goal is to construct a lifting in this diagram inductively. Having a lift means finding an $\infty$-morphism $a : C \rightsquigarrow B$, such that the following hold for any $n\geq 1$:
            \begin{itemize}
                \item $a$ satisfy $(rel_n)$.
                \item $a_n \circ j_1 = f_n$.
                \item $q_1\circ a_n = g_n$.
            \end{itemize}

            We start by showing there is such an $a_1$. Consider the diagram below of chain complexes over $\mathbb{K}$.
            \begin{center}
                \begin{tikzcd}
                    A \ar[tail]{d}[]{j_1} \ar[]{r}[]{f_1} & B \ar[two heads]{d}[]{q_1} \\
                    C \ar[]{r}[]{g_1} \ar[dashed]{ru}[]{a_1} & D
                \end{tikzcd}
            \end{center}
            The lift exists since the category $Ch\mathbb{K}$ is a model category. Here $j_1$ is a cofibration, while $q_1$ is an acyclic fibration, so the lift $a_1$ exists.

            We now wish to extend this. Suppose that we have been able to create morphisms $a_1$ up to $a_n$, all satisfying the above points. A naive solution to make $a_{n+1}$ is $b = f_{n+1}r^{\otimes n+1} + sg_{n+1} - s q_1f_{n+1}r^{\otimes n+1}$. Notice that this satisfy the two last points by definition. We will augment $b$ to get an $a_{n+1}$ which also satisfies $(rel_{n+1})$.

            For our own convenience, let $-c(f_1, ..., f_n)$ denote the right hand side of $(rel_{n+1})$ formula. Since both $j$ and $q$ are strict $\infty$-morphisms we get the following identites:
            \begin{align*}
                & (\partial b + c(a_1, ..., a_n)) \circ j_1 = \partial (b\circ j_1) + c(a_1 \circ j_1, ..., a_n \circ j_1) = \partial f_{n+1} + c(f_1, ..., f_n) = 0 \\
                & q_1 \circ (\partial b + c(a_1, ..., a_n)) = \partial (q_1\circ b) + c(q_1\circ a_1, ..., q_1\circ a_n) = \partial {g_{n+1}} + c(g_1, ..., g_n) = 0
            \end{align*}

            We thus obtain that the cycle $\partial b + c(a_1, ..., a_n)$ factors thorugh the cokernel of $j$ and the kernel of $q$. Let us say that it factors like the diagram below:
            \begin{center}
                \begin{tikzcd}
                    C \ar[]{r}[]{p} & Cok j \ar[]{r}[]{c'} & Ker q \ar[]{r}[]{i} & D
                \end{tikzcd}
            \end{center}

            Now, $c'$ is a morphism between two $A_\infty$-algebras. Since $q$ is assumed to be an $\infty$-quasi-isomorphism, it follows that $Kerq$ is a acyclic. By lemma \ref{lem: inf-creator}, we get that $c'$ may be extended to an $\infty$-morphism \todo{I dont understand this :(}, call it $h$. We let $a_{n+1} = b - i\circ h\circ p$. This morphism satisfies all three properties.

            We will now show MC4. Since the two properties have a similar proof, we will only show one direction. Let $f: A \rightsquigarrow B$ be an $\infty$-morphism. Let $C = cone(id_(B[-1]))$. The complex $C$ may be considered as an $A_\infty$-algebra. Let $j : A \rightsquigarrow A\prod C$ be the morphism induced by $id_A$ and $0:A \rightarrow C$. The canonical projection $q_1 : A\oplus C \rightarrow B$ gives a lift of the following diagram.

            \begin{center}
                \begin{tikzcd}
                    A \ar[]{r}[]{f_1} \ar[]{d}[]{j_1} & B \ar[]{d}[]{} \\
                    A\oplus C \ar[]{ru}[]{q_1} \ar[]{r}[]{} & 0
                \end{tikzcd}
            \end{center}

            Since we have a morphism of chain complexes, lodged between an acyclic cofibration and a fibration we use the same technique as above to construce an $\infty$-morphism $q : A\prod C \rightarrow B$. $q$ is a fibration by construction. The morphism $f$ may be factored as $f = q\circ j$, where $j$ is an acyclic cofibration and $q$ is a fibration.
        \end{proof}

        With this model structure we are finally able to characterize the fibrant and cofibrant conilpotent dg-coalgebras.

        \begin{proposition}
            Let $C$ be a conilpotent dg-coalgebra. Then $C$ is cofibrant, and $C$ is fibrant if and only if there is a cochain complex $V$, such that $C \simeq T^c(V)$ as complexes.
        \end{proposition}

        \begin{proof}
            To see that $C$ is cofibrant is the same as to verify that the map $\mathbb{K}\rightarrow C$ is a monomorphism, but this is clear.

            We start by assuming that $C$ is fibrant. Then there is a lift in the square below, making the unit split-mono.
            \begin{center}
                \begin{tikzcd}
                    C \ar[equal]{r}[]{} \ar[]{d}[]{\eta_C} & C \ar[]{d}[]{\varepsilon_C} \\
                    B\Omega C \ar[]{r}[]{\varepsilon_{B\Omega C}} \ar[dashed]{ru}[]{r} & \mathbb{K}
                \end{tikzcd}
            \end{center}
            Consider the morphism $p_1^C : C \rightarrow Fr_1C$ which is defined as $p_1^C = Fr_1r\circ p_1\circ \eta_C$, where $p_1 : B\Omega C \rightarrow Fr_1 B\Omega C$ is the canonical projection on the filtration induced by the coradical filtration on $C$. Clearly, $r$ makes $p_1$ into a universal arrow in the category of conilpotent coalgebras, so $C \simeq T^c(Fr_1C)$.

            Now, assume that $C$ is isomorphic to $T^c(V)$ as coalgebras for some cochain complex $V$. Note that, by definition, $C$ is an $A_\infty$-algebra. By definition, we have a commutative square of $A_\infty$-algebras. Since every $A_\infty$-algebra is bifibrant, we know that this diagram has a lift, exhibiting $C$ as a retract of $B\Omega C$.
            \begin{center}
                \begin{tikzcd}
                    C \ar[equal]{r}[]{} \ar[]{d}[]{} & C \ar[]{d}[]{} \\
                    B\Omega C \ar[]{r}[]{} \ar[dashed]{ru}[]{} & \mathbb{K}
                \end{tikzcd}
            \end{center}

            We know that $\Omega C$ is fibrant, since the map $\Omega C \rightarrow 0$ is epi. By lemma \ref{lem: bar-cobar-Quill-adj}, we know that the bar construction preserves fibrations, so $B\Omega C$ is fibrant. Thus $C$ is fibrant as well.
        \end{proof}

        The model structure of $A_\infty$-algebras is compatible with the model structure of conilpotent dg-coalgebras in the following sense. If $f : A \rightsquigarrow B$ is an $\infty$-morphism, we denote its dg-coalgebra counterpart as $Bf : BA \rightarrow BB$. Remember that the bar construction is extended such that it is an equivalence of categories on its image. We use this to realize $Alg_\infty$ as a subcategory of $ConilCoalg_\mathbb{K}$ to essentially obtain 2 different model structure on this category. The following proposition tells us that these structures do not differ.

        \begin{proposition}
            Let $f : A \rightsquigarrow B$ be an $\infty$-morphism. Then we have the following:
            \begin{itemize}
                \item $f$ is an $\infty$-quasi-isomorphism if and only if $Bf$ is a weak equivalence.
                \item $f_1$ is an epimorphism if and only if $Bf$ is a fibration.
                \item $f_1$ is a monomorphism if and only if $Bf$ is a monomorphism.
            \end{itemize}
        \end{proposition}

        \begin{proof}
            This is proposition 1.3.3.5 in \cite{LefevreHasegawa03}.
        \end{proof}

    \section{The Homotopy Category of $Alg_\infty$}

    With the results we have establish, we are now ready to talk about homotopies in $Alg_\infty$.

    \begin{thm}
        In the category $Alg_\infty$ we have the following:
        \begin{itemize}
            \item Homotopy equivalence is an equivalence relation.
            \item A morphism is an $\infty$-quasi-isomorphism if and only if it is a homotopy equivalence.
            \item Let $dash \subseteq Alg_\infty$ be the full subcategory consisting of dg-algebras considered as $A_\infty$-algebras. $dash$ has an induced homotopy equivalence from $Alg_\infty$, and the inclusion $Alg \rightarrow dash$ induces an equivalence in homotopy $Alg[Qis^{-1}] \simeq \sfrac{dash}{\sim}$.
        \end{itemize}
    \end{thm}

    \begin{proof}
        The first point is obsreved from corollary \ref{cor: homotopy-is-eq-rel}, and the second point is Whiteheads theorem, theorem \ref{thm: Whitehead}.

        To see the final point, observe that the inclusion functor is given by the bar construction $B$. By corollary \ref{cor: cobar-bar-quill-eq}, we know that the bar construction induces an equivalence on the homotopy categories, i.e. $HoAlg \simeq HoCoalg$. Moreover, we know that by theorem \ref{thm: Fundamental-thm-model} that $HoCoalg \simeq \sfrac{Alg_\infty}{\sim}$. Notice that the image of $B$ is $dash$, so in homotopy, we get that the image $\sfrac{dash}{\sim}$ is equivalent to the essentiall image $HoAlg_\infty$.
    \end{proof}

    Homotopy equivalence in the algebraic sense and the model categorical sense differ in how they are defined. In homological algebra, two morphisms are called homotopic if their difference is a boundary of some homotopy, i.e. $f-g = \partial h$. In the model categorical sense, a homotopy is a morphism through either a cylinder or a path object. The following proposition tells us how they differ in the category of conilpotent dg-coalgebras.

    \begin{proposition}
        Let $C$ and $D$ be two conilpotent dg-coalgebras, where $f,g : C \rightarrow D$ are two morphisms. Then:
        \begin{itemize}
            \item If $f-g$ is null homotopic, then they are left homotopic.
            \item If $D$ is fibrant, then $f-g$ is null homotopic if and only if $f$ and $g$ are left homotopic.
        \end{itemize}
    \end{proposition}
    
    This tells us that in general, these concepts are usually not the same. However, we know that the subcategory of bifibrant objects in this category is exactly the subcategory $Alg_\infty$. Thus for $A_\infty$-algebras homological homotopy is the same as model categorical homotopy. In this sense we obtain that the homotopy category in the homological sense is equivalent to the derived category.

\end{document}