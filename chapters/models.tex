\documentclass[../thesis.tex]{subfiles}

\begin{document}    

    Quillen envisioned a more general approach to homotopy theory, which he dubbed homotopical algebra. A homotopy theory was then enclosed by the structure of a model category, then a closed model category. Many of the results from classical homotopy theory was then recovered in this new setting of model categories. The theorem which we are concerned about is Whiteheads theorem:

    \begin{thm}[Whiteheads Theorem]
        Let $X$ and $Y$ be two CW-complexes. If $f:X\rightarrow Y$ is a weak equivalence, then it is also a homotopy equivalence. I.e. there exists a morphism $g: Y\rightarrow X$ such that $gf\sim id_X$ and $fg\sim id_Y$.
    \end{thm}

    If we employ Quillens model category onto the category Top, we get that a space $X$ is bifibrant if and only if it is a CW-complex. The natural generalization is then to not ask of $X$ to be a CW-complex, but a bifibrant object.

    \begin{thm}[Generalized Whiteheads Theorem]
        Let $\mathcal{C}$ be a model category. Suppose that $X$ and $Y$ are bifibrant objects of $\mathcal{C}$, and that there is a weak-equivalence $f:X\rightarrow Y$. Then $f$ is also a homotopy equivalence, i.e. there exists a morphism $g: Y\rightarrow X$ such that $gf\sim id_X$ and $fg\sim id_Y$.
    \end{thm}

    The category of differential graded (co)algebras employs such a model category. Here we let the weak-equivalences be quasi-isomorphisms. Moreover, in this case the bar and cobar construction is a Quillen equivalence between the model structures. As we will see, a dg-coalgebra will be bifibrant exactly when it is an $A_\infty$-algebra. Thus, by Whiteheads theorem, quasi-isomorphisms lift to homotopy equivalences. In this case the derived category of $A_\infty$-algebras is equivalent to the homotopy category of $A_\infty$-algebras.

    We will conclude this section by looking at the category of algebras as a subcategory of $A_\infty$-algebras. The derived category may then be expressed as the homotopy category $A_\infty$-algebras, restricted to algebras.

    \section{Model categories}

        In this section we will define Quillens model category. As one may see is that in practice there are a plethora of semantically different definitions of model categories, however they are all made to be equivalent. The difference comes down to preference. In this thesis we will use the definitions as they are developed in Mark Hoveys book \todo{reference here}. We will then go on to prove the basic results known about model categories, its associated homotopy category and Quillen functors between model categories.

        Before we state the definition of a model category we need some preliminary definitions. For this section, let $\mathcal{C}$ be a category.

        \begin{definition}[Retract]
            A morphism $f:A\rightarrow B$ in $\mathcal{C}$ is a retract of a morphism $g: c\rightarrow D$ if it fits in a commutative diagram:
            \begin{center}
                \begin{tikzcd}
                    A \ar[]{d}[]{f} \ar[]{r}[]{} \ar[bend left]{rr}[]{id_A}& C \ar[]{d}[]{g} \ar[]{r}[]{} & A \ar[]{d}[]{f} \\
                    B \ar[]{r}[]{} \ar[bend right]{rr}[below]{id_B} & D \ar[]{r}[]{} & B
                \end{tikzcd}
            \end{center}
        \end{definition}

        \begin{definition}[Functorial factorization]
            A pair of functors $\alpha, \beta: \mathcal{C}^\rightarrow\rightarrow\mathcal{C}^\rightarrow$ is called a functorial factorization if for any morphism $f = \beta(f)\circ\alpha(f)$. We will denote the morphisms in the factorization as $f_\alpha$ and $f_\beta$. The functorial factorization may be depicted by the following commutative diagram:
            \begin{center}
                \begin{tikzcd}
                    A \ar[]{rr}[]{f} \ar[]{rd}[below]{f_\alpha} & & B \\
                    & C \ar[]{ru}[below]{f_\beta}
                \end{tikzcd}
            \end{center}
        \end{definition}

        \begin{definition}[Lifting properties]
            Suppose that the morphisms $i: A \rightarrow B$ and $p: C \rightarrow D$ fits inside a commutative square. $i$ is said to have the left lifting property with resptect to $p$, or $p$ has the right lifting property with respect to $i$, if there is an $h : B \rightarrow C$ such that the two triangles commute.
            \begin{center}
                \begin{tikzcd}
                    A \ar[]{r}[]{} \ar[]{d}[]{i} & C \ar[]{d}[]{p} \\
                    B \ar[]{r}[]{} \ar[dashed]{ru}[]{h} & D
                \end{tikzcd}
            \end{center}
        \end{definition}

        \begin{remark}
            We will call the left lifting property for LLP and the right lifting property for RLP.
        \end{remark}

        \subsection{Model categories}

            \begin{definition}[Model category]
                Let $\mathcal{C}$ be a bicomplete category, i.e. has every small limit and colimit. $\mathcal{C}$ admits a model structure if there are three wide subcategories each defining a class of morphisms:
                \begin{itemize}
                    \item $Ac\subset Mor(\mathcal{C})$ are called weak equivalences
                    \item $Cof\subset Mor(\mathcal{C})$ are called cofibrations
                    \item $Fib\subset Mor(\mathcal{C})$ are called fibrations
                \end{itemize}
                In addition we call morphisms in $Cof\cap Ac$ for acyclic cofibrations and $Fib\cap Ac$ for acyclic fibrations. Moreover, $\mathcal{C}$ has two functorial factorizations $(\alpha, \beta)$ and $(\gamma, \delta)$. The following axioms should be satisfied:
                \begin{itemize}
                    \item[\textbf{MC1}] The class of weak equivalences satisfy the $2$-out-of-$3$ property, i.e. if $f$ and $g$ are composable morphisms such that $2$ out of $f$, $g$ and $gf$ are weak equivalences, then so is the third.
                    \item[\textbf{MC2}] The three classes $Ac$, $Cof$ and $Fib$ are retraction closed, i.e. if $f$ is a retraction of $g$, and $g$ is either a weak-equivalence, cofibration or fibration, then so is $f$.
                    \item[\textbf{MC3}] The class of cofibrations have the left lifting property with respect to acyclic fibrations, and fibrations have the right lifting property with respect to acyclic cofibrations.
                    \item[\textbf{MC4}] Given any morphism $f$, $f_\alpha$ is a cofibration, $f_\beta$ is an acyclic fibration, $f_\gamma$ is an acyclic cofibration and $f_\delta$ is a fibration.      
                \end{itemize}
            \end{definition}

            A model category $\mathcal{C}$ is now defined to be a category equipped with a particular model structure. Notice that a category may admit several model structures. We will postpone examples until sufficient theory have been developed. For more topological examples, we refer to Dwayer-Spalinski and Hovey.

            An interesting and a not so non-trivial property of model categories is that giving all three classes $Ac$, $Cof$ and $Fib$ is redundant. Given the class of weak equivalences and either cofibrations or fibrations, the model structure is determined. Thus the classes of fibrations are determined by acyclic cofibrations and cofibrations are determined by fibrations. The next two results will show this.

            \begin{lemma}[The retract argument]\label{lem: retract-argument}
                Let $\mathcal{C}$ be a category. Suppose there is a factorization $f = pi$ and that $f$ has LLP with respect to $p$, then $f$ is a retract of $i$. Dually, if $f$ har RLP to $i$, then it is a retract of $p$.
            \end{lemma}

            \begin{proof}
                We assume that $f : A \rightarrow C$ has LLP with respect to $p : B \rightarrow C$. Then we may find a lift $r : C \rightarrow B$, which realize $f$ as a retract of $i$.
                
                \begin{center}
                    \begin{tikzcd}
                        A \ar[]{d}[]{f} \ar[]{r}[]{i} & B \ar[]{d}[]{p} \\
                        C \ar[equal]{r}[]{} \ar[dashed]{ru}[]{r} & C
                    \end{tikzcd} $\implies$
                    \begin{tikzcd}
                        A \ar[equal]{r}[]{} \ar[]{d}[]{f} & A \ar[equal]{r}[]{} \ar[]{d}[]{i} & A \ar[]{d}[]{f} \\
                        C \ar[]{r}[]{r} & B \ar[]{r}[]{p} & C
                    \end{tikzcd}
                \end{center}
            \end{proof}

            \begin{proposition}
                Let $\mathcal{C}$ be a model category. A morphism $f$ is a cofibration (acyclic cofibration) if and only if $f$ has LLP with respect acyclic fibrations (fibrations). Dually, $f$ is a fibration (acyclic fibration) if and only if it has RLP with respect to acyclic cofibrations (cofibrations).
            \end{proposition}

            \begin{proof}
            Assume that $f$ is a cofibration. By MC3, we know that $f$ has LLP with respect to acyclic fibrations. Assume instead that $f$ has LLP with respect to ever acyclic fibration. By MC4 we factor $f = f_\alpha\circ f_\beta$, where $f_\alpha$ is a cofibration and $f_\beta$ is an acyclic fibration. Since we assume $f$ to have LLP with respect to $f_\beta$, by lemma \ref{lem: retract-argument} we know that $f$ is a retract of $f_\alpha$. Thus by MC2, we know that $f$ is a cofibration. 
            \end{proof}

            \begin{corollary}\label{cor: stable-cofib-base-change}
                Let $\mathcal{C}$ be a model category. (Acyclic) Cofibrations are stable under pushouts, i.e. if $f$ is an (acyclic) cofibration, then $f'$ is an (acyclic) cofibration.
                \begin{center}
                    \begin{tikzcd}
                        A \ar[]{r}[]{} \ar[]{d}[]{f} \ar[phantom]{rd}[near end]{\lrcorner} & C \ar[]{d}[]{f'} \\
                        B \ar[]{r}[]{} & D
                    \end{tikzcd}
                \end{center}
                Dually, fibrations are stable under pullbacks.
            \end{corollary}

            \begin{proof}
                Follows from the universal property of pushouts. \emoji{woman-tipping-hand-medium-light-skin-tone} \emoji{nail-polish-medium-light-skin-tone} \emoji{crown}
            \end{proof}

            Since we assume that every model category $\mathcal{C}$ is bicomplete, we know that it has both an initial and a terminal object. We let $\emptyset$ denote the initial object and $*$ denote the terminal object. 

            \begin{definition}[Cofibrant, fibrant and bifibrant objects]
                Let $\mathcal{C}$ be a model category. An object $X$ is called cofibrant if the unique morphism $\emptyset \rightarrow X$ is a cofibration. Dually, $X$ is called fibrant if the unique morphism $X \rightarrow *$ is fibrant. If $X$ is both cofibrant and fibrant, we call it bifibrant.
            \end{definition}

            There is no reason for every object to be either cofibrant or fibrant. However, we may see that every object is weakly equivalent to an object which is either fibrant or cofibrant. We will make it precise what it means for two objects to be weakly equivalent later.

            \begin{construction}
                Let $X$ be an object of a model category $\mathcal{C}$. The morphism $i:\emptyset\rightarrow X$ has a functorial factorization $i=i_\beta\circ i_\alpha$, where $i_\alpha: \emptyset\rightarrow QX$ is a cofibration and $i_\beta: QX\rightarrow X$ is an acyclic fibration. By definition $QX$ is cofibrant and weakly equivalent to $X$.

                $Q: \mathcal{C}\rightarrow \mathcal{C}$ defines a functor called the cofibrant replacement. To see this we first look at the slice category $\sfrac{\emptyset}{\mathcal{C}}$. The objects are morphisms $f:\emptyset \rightarrow X$ for any object $X$ in $\mathcal{C}$, while morphisms are commutative triangles. We first observe that $\sfrac{\emptyset}{\mathcal{C}}\subset\mathcal{C}^\rightarrow$ is a subcategory of the arrow category. Thus $(\alpha, \beta)$ may be interpreted as functors on the slice category to the arrow category. Moreover, since every arrow $f:\emptyset \rightarrow X$ is unique, we observe that this category is equivalent to $\mathcal{C}$. Thus $(\alpha, \beta)$ may be interpreted as functors on $\mathcal{C}$ into arrows. We define $Q$ as the composition $Q = tar \circ \alpha$.

                Dually, we get a fibrant replacement $R$ by dualizing the above argument.
            \end{construction}

            We collect the following properties

            \begin{lemma}\label{lem: Q-preserves-weak}
                The cofibrant replacement $Q$ and fibrant replacement $R$ preserves weak equivalences. 
            \end{lemma}

            \begin{proof}
                Clear from the $2$-out-of-$3$ property.
            \end{proof}

            \begin{lemma}[Ken Brown's lemma]\label{lem: Ken-Brown}
                Let $\mathcal{C}$ be a model category and $\mathcal{D}$ be a category with weak equivalences satisfying the $2$-out-of-$3$ property. If $F:\mathcal{C} \rightarrow \mathcal{D}$ is a functor sending acyclic cofibrations between cofibrant objects to weak equivalences, then $F$ takes all weak equivalences between cofibrant objects to weak equivalences. Dually, if $F$ takes all acyclic fibrations between fibrant objects to weak equivalences, then $F$ takes all weak equivalences between fibrant objects to weak equivalences.
            \end{lemma}

            \begin{proof}
                Suppose that $A$ and $B$ are cofibrant objects and that $f:A\rightarrow B$ is a weak equivalence. Using the universal property of the coproduct we define the map $(f, id_B) = p: A\coprod B \rightarrow B$. $p$ has a functorial factorization into a cofibration and acyclic fibration, $p = p_\beta\circ p_\alpha$. We recollect the maps in the following pushout diagram:
                \begin{center}
                    \begin{tikzcd}
                        \emptyset \ar[]{d}[]{} \ar[]{r}[]{} \ar[phantom]{rd}[near end]{\lrcorner} & B \ar[]{d}[]{i_2} \ar[bend left]{rrddd}[]{id_B} \\
                        A \ar[]{r}[]{i_1} \ar[bend right]{rrrdd}[]{f} & A\coprod B \ar[]{rd}[]{p_\alpha} \\
                        & & C \ar[]{rd}[]{p_\beta} \\
                        & & & D
                    \end{tikzcd}
                \end{center}
                By lemma \ref{cor: stable-cofib-base-change} both $i_1$ and $i_2$ are cofibrations. Since $f$, $id_B$ and $p_\beta$ are weak equivalences, so are $p_\alpha\circ i_1$ and $p_\alpha\circ i_2$ by MC2. Moreover, they are acyclic cofibrations.

                Assume that $F:\mathcal{C}\rightarrow\mathcal{D}$ is a functor as described above. Then by assumption, $F(p_\alpha\circ i_1)$ and $F(p_\alpha\circ i_2)$ are weak equivalences. Since a functor sends identity to identity, we also know that $F(id_B)$ is a weak equivelnce. Thus by the $2$-out-of-$3$ property $F(p_\beta)$ is a weak equivalence, as $F(p_\beta)\circ F(p_\alpha\circ i_2) = id_{F(B)}$. Again, by $2$-out-of-$3$ property $F(f)$ is a weak equivelnce, as $F(f) = F(p_\beta)\circ F(p_\alpha\circ i_1)$.
            \end{proof}

        \subsection{Homotopy category}

            Homotopy theory at it's most abstract is the study of categories and functors up to weak equivalences. Here, a weak equivalence may be anything, but most commonly it is a weak equivalence in topological homotopy or a quasi-isomorphism in homological algebra. The biggest concern when dealing with such concepts is to make a functor well-defined up to these chosen weak-equivalences. To this end, there is a construction to amend these problems, known as derived functors. We define a homotopical category in the sense of Riehl.

            \begin{definition}[Homotopical Category]
                Let $\mathcal{C}$ be a category. $\mathcal{C}$ is Homotopical if there is a wide subcategory constituting a class of morphisms known as weak equivalences, $Ac\subset Mor\mathcal{C}$. The weak equivalences should satisfy the \textbf{$2$-out-of-$6$ property}, i.e. given three composable morphisms $f$, $g$ and $g$, if $gf$ and $hg$ are weak equivalences, then so are $f$, $g$, $h$ and $hgf$.

                \begin{center}
                    \begin{tikzcd}[row sep = large]
                        A \ar[]{r}[]{f} \ar[]{rd}[]{gf} \ar[dotted]{rrd}[]{} & B \ar[]{d}[]{g} \ar[]{rd}[]{hg} \\
                        & C \ar[]{r}[]{h} & D
                    \end{tikzcd}
                \end{center}
            \end{definition}

            \begin{remark}
                Notice that the $2$-out-of-$6$ property is stronger than the $2$-out-of-$3$ property. To see this, let either $f$, $g$ or $h$ be the identity, and then conclude with the $2$-out-of-$3$ property.
            \end{remark}
            
            Given such a homotopical category $\mathcal{C}$, we want to invert every weak equivalence and create the homotopy category of $\mathcal{C}$. This concept is due to Gabriel and Zisman \todo{citation needed}.

            \begin{definition}
                Let $\mathcal{C}$ be a homotopical category. It's homotopy category $Ho\mathcal{C} = \mathcal{C}[Ac^{-1}]$, together with a localization functor $q:\mathcal{C}\rightarrow Ho\mathcal{C}$. Recall that the localization are determined by the following universal property: If $F:\mathcal{C}\rightarrow \mathcal{D}$ is a functor sending weak equivalences to isomorphisms, then it uniquely factors through the homotopy category up to a unique natural isomorphism $\eta$.

                \begin{center}
                    \begin{tikzcd}
                        \mathcal{C} \ar[""{name = U, below}]{rr}[]{F} \ar[]{rd}[]{q} & & \mathcal{D} \\
                        & Ho\mathcal{C} \ar[]{ru}[]{F'} \ar[Rightarrow, from=U]{}[]{\eta}
                    \end{tikzcd}
                \end{center}
            \end{definition}

            \begin{definition}
                Suppose that $\mathcal{C}$ is a homotopical category. Two objects of $\mathcal{C}$ are said to be weakly equivalent if they are isomorphic in $Ho\mathcal{C}$. I.e. there is some zig-zag relation between the objects, consisting only of weak equivalences.
            \end{definition}

            \begin{remark}
                A renown problem with localizations is that even if $\mathcal{C}$ is a locally small category, any localization $\mathcal{C}[S^{-1}]$ does not need to be. Thus, without a good theory of classes or higher universes, we cannot in general ensure that the localization still exists as a locally small category.
            \end{remark}

            We see from the definition of the homotopy category that a functor $F$ admits a lift $F'$ to the homotopy category whenever weak equivalences are sent to isomorphisms. Moreover, if we have a functor $F$ between homotopical categories which preserves weak equivalences, it then induces a functor between the homotopy categories.
            
            \begin{definition}[Homotopical functors]
                A functor $F:\mathcal{C}\rightarrow \mathcal{D}$ between homotopical categories is homotopical if it preserves weak equivalences. Moreover, there is a lift of functors as in the following diagram.

                \begin{center}
                    \begin{tikzcd}
                        \mathcal{C} \ar[""{name=U, below}]{r}[]{F} \ar[]{d}[]{q_\mathcal{C}} & \mathcal{D} \ar[]{d}[]{q_\mathcal{D}} \ar[Rightarrow]{dl}[]{\eta} \\
                        Ho\mathcal{C} \ar[dashed, ""{name=V, above}]{r}[]{F'} & Ho\mathcal{D}
                    \end{tikzcd}
                \end{center}
            \end{definition}

            Derived functors come into play whenever this is not the case. These lifts are however the closest approximation which we can make functorial. The general exposition of derived functors is beyond the scope of this thesis, but a general account of it may be found in \todo{citation needed}. However, model categories serve as a usefull tool to simplify this discussion. Firstly we will amend the problem with localizations, where the homotopy category may not exists. Secondly, we will obtain a simple description of derived functors.

            \begin{proposition}
                Any model category $\mathcal{C}$ is a homotopical category.
            \end{proposition}

            \begin{proof}
                Idea for proof. We want to do use thm 3.1. on nlab \url{http://nlab-pages.s3.us-east-2.amazonaws.com/nlab/show/two-out-of-six%20property#BlumbergMandell}. Reference to the lemma which we will use, may be found on webpage.
            \end{proof}

            Since every model category is homotopical, it also has an associated homotopy category $Ho\mathcal{C}$. Let $\mathcal{C}_c$, $\mathcal{C}_f$ and $\mathcal{C}_{cf}$ denote the full subcategories consisting of cofibrant, fibrant and bifibrant objects respectively.

            \begin{proposition}
                Let $\mathcal{C}$ be a model category. The following categories are equivalent:
                \begin{itemize}
                    \item $Ho\mathcal{C}$
                    \item $Ho\mathcal{C}_c$
                    \item $Ho\mathcal{C}_f$
                    \item $Ho\mathcal{C}_{cf}$
                \end{itemize}
            \end{proposition}

            \begin{proof}
                We show that $Ho\mathcal{C} \simeq Ho\mathcal{C}_c$. The inclusion $i:\mathcal{C}_c\rightarrow \mathcal{C}$ clearly preserves weak equivalences, thus $i$ is homotopical and admits a lift. Moreover, since the cofibrant replacement is also homotopical, it also has a lift.

                \begin{center}
                    \begin{tikzcd}
                        \mathcal{C}_c \ar[]{r}[]{i} \ar[]{d}[]{} & \mathcal{C} \ar[]{d}[]{} \\
                        Ho\mathcal{C}_c \ar[dashed, bend right]{r}[below]{Ho\ i} & \mathcal{C} \ar[dashed, bend right]{l}[above]{Q}
                    \end{tikzcd}
                \end{center}

                It is clear that $Q$ is the quasi-inverse of $i$.

            \end{proof}

            As of now we still don't see how model categories will fix the size issues. To do this we will develop homotopy equivalence $\sim$. We will see that on the subcategory of bifibrant objects $\mathcal{C}_{cf}$, this homotopy equivalence will in fact be a congruence relation. Moreover, there is an equivalence of categories $Ho\mathcal{C}_{cf}\simeq \sfrac{\mathcal{C}_{cf}}{\sim}$.

            \begin{definition}[Cylinder and path objects]
                Let $\mathcal{C}$ be a model category. Given an object $X$, a cylinder object $X\wedge I$ is a factorization of the fold map $i:X\coprod X \rightarrow X$, such that $p_0$ is a cofibration and that $p_1$ is a weak equivalence. 
                
                \begin{center}
                    \begin{tikzcd}
                        X\coprod X \ar[]{rr}[]{i} \ar[]{rd}[]{p_0} & & X \\
                        & X\wedge I \ar[]{ru}[]{p_1}
                    \end{tikzcd}
                \end{center}

                Dually, a path object  $X^{I}$ is a factorization of the diagonal map $i: X \rightarrow X\prod X$, such that $p_0$ is a weak equivalence and that $p_1$ is a fibration.
                
                \begin{center}
                    \begin{tikzcd}
                        X \ar[]{rr}[]{i} \ar[]{rd}[]{p_0} & & X\prod X\\
                        & X^I \ar[]{ru}[]{p_1}
                    \end{tikzcd}
                \end{center}
            \end{definition}

            \begin{remark}
                Even though we have written $X\wedge I$ suggestively to be a functor, it is not. There may be many choices for a cylinder object. However, by using the functorial factorization from MC4, we get a canonical choice of a cylinder object, as it factors every map into a cofibration and an acyclic fibration. By letting the cylinder object be this object, we do indeed obtain a functor.
            \end{remark}

            \begin{proposition}
                Let $\mathcal{C}$ be a model category and $X$ an object of $\mathcal{C}$. Given two cylinder objects $X\wedge I$ and $X\wedge I'$, then they are weakly equivalent. 
            \end{proposition}

            \begin{proof}
                It is enough to show that there is a weak equivalence from any cylinder object into one specified cylinder object. This is in fact true for the functorial cylinder object $X\wedge I$, as the final morphism $p_1$ is an acyclic fibration, which enables a lift which is a weak equivalence by the $2$-out-of-$3$ property.

                \begin{center}
                    \begin{tikzcd}
                        X\coprod X \ar[]{r}[]{p_0} \ar[]{d}[]{p_0'} & X\wedge I \ar[]{d}[]{p_1} \\
                        X\wedge I' \ar[]{r}[]{p_1'} \ar[dashed]{ru}[]{} & X
                    \end{tikzcd}
                \end{center}
            \end{proof}

            \begin{definition}[Homotopy equivalence]
                Let $f,g: X\rightarrow Y$. A left homotopy between $f$ and $g$ is a morphism $H:X\wedge I \rightarrow Y$ such that $Hi_0 = f$ and $Hi_1 = g$. $f$ and $g$ are left homotopic if a left homotopy exists, and it is denoted $f \overset{l}{\sim} g$.

                \begin{center}
                    \begin{tikzcd}
                        X \ar[]{d}[]{} \ar[dotted]{rd}[]{i_0} \ar[bend left]{rrd}[]{f} \\
                        X\coprod X \ar[]{r}[]{p_0} & X\wedge I \ar[]{r}[]{H} & Y \\
                        X \ar[]{u}[]{} \ar[dotted]{ru}[]{i_1} \ar[bend right]{rru}[]{g}
                    \end{tikzcd}
                \end{center}
                
                A right homotopy between $f$ and $g$ is a morphism $H: X \rightarrow Y^I$ such that $i_0H = f$ and $i_1H = g$. $f$ and $g$ are right homotopic if a right homotopy exists, and it is denoted $f \overset{r}{\sim} g$.

                \begin{center}
                    \begin{tikzcd}
                        & & Y \\
                        X \ar[]{r}[]{H} \ar[bend left]{rru}[]{f} \ar[bend right]{rrd}[]{g} & Y^I \ar[]{r}[]{p_1} \ar[dotted]{ru}[]{i_0} \ar[dotted]{rd}[]{i_1} & Y\prod Y \ar[]{u}[]{} \ar[]{d}[]{} \\
                        & & Y
                    \end{tikzcd}
                \end{center}

                $f$ and $g$ are said to be homotopic if they are both left and right homotopic, it is denoted $f \sim g$. $f$ is said to be a homotopy equivalence, if it has a homotopy inverse $h: Y \rightarrow X$, such that $hf \sim id_X$ and $fh \sim id_Y$. 
            \end{definition}

            It is important to know that this is not a priori an equivalence relation. This is amended by taking both fibrant and cofibrant replacements. We see this in the following proposition.

            \begin{proposition}\label{prop: basic-homotopy}
                Let $\mathcal{C}$ be a model category, and $f,g:X\rightarrow Y$ be morphisms. We have the following:
                \begin{enumerate}
                    \item If $f \overset{l}{\sim}$ and $h: Y \rightarrow Z$, then $hf \overset{l}{\sim} hg$.
                    \item If $Y$ is fibrant, $f \overset{l}{\sim} g$ and $h: W \rightarrow X$, then $fh \overset{l}{\sim} gh$.
                    \item If $X$ is cofibrant, then left homotopy is an equivalence relation on $\mathcal{C}(X,Y)$.
                    \item If $X$ is cofibrant and $f \overset{l}{\sim} g$, then $f \overset{r}{\sim} g$.
                \end{enumerate}
            \end{proposition}

            \begin{proof}
                This proof is due to Mark Hovey \todo{citation needed}.

                $(1.)$ Assume that $f \overset{l}{\sim} g$ and $h:Y\rightarrow Z$. Let $H:X\wedge I \rightarrow Y$ denote the left homotopy between $f$ and $g$. The left homotopy between $hf$ and $hg$ is given as $hH$.

                $(2.)$ Assume that $Y$ is fibrant, $f \overset{l}{\sim} g$ and that $h:W\rightarrow X$. Let $H:X\wedge I\rightarrow Y$ be a left homotopy. We construct a new cylinder object for the homotopy. Factor $p_1:X\wedge I \rightarrow X$ as $q_1\circ q_0$ where $q_0: X\wedge I \rightarrow X\wedge I'$ is an acyclic cofibration and $q_1:X\wedge I'\rightarrow X$ is a fibration. By the $2$-out-of-$3$ property $q_1$ is an acyclic fibration, as $p_1$ and $q_0$ are weak equivalences. $X\wedge I'$ is a cylinder object as $q_0\circ p_0$ is a cofibration and $q_1$ is a weak equivalence. Since we assume $Y$ to be fibrant we lift the left homotopy $H:X\wedge I\rightarrow Y$ to the left homotopy $H':X\wedge I'\rightarrow Y$ with the following diagram:
                \begin{center}
                    \begin{tikzcd}
                        X\wedge I \ar[]{r}[]{H} \ar[]{d}[]{q_0} & Y \ar[]{d}[]{} \\
                        X\wedge I' \ar[]{r}[]{} \ar[dashed]{ru}[]{H'} & *
                    \end{tikzcd}                    
                \end{center}
                We can find the appropriate homotopy needed with lift given by the following diagram:
                \begin{center}
                    \begin{tikzcd}[column sep = large]
                        W\coprod W \ar[]{r}[]{q_0p_0(h\coprod h)} \ar[]{d}[]{p_0'} & X\wedge I' \ar[]{d}[]{q_1} \\
                        W\wedge I \ar[]{r}[]{hp_1'} \ar[dashed]{ru}[]{k} & X
                    \end{tikzcd}
                \end{center}
                Then the morphism $H'k$ is the desired left homotopy witnessing $fh \overset{l}{\sim} gh$.

                $(3.)$ Assume that $X$ is cofibrant. First observe that a left homotopy is reflexive and symmetric. We must show that in this case it is also transitative. Thus, assume that $f,g,h:X\rightarrow Y$ and that $H:X\wedge I\rightarrow Y$ is a left homotopy witnessing $f \overset{l}{\sim} g$ and that $H':X\wedge I'\rightarrow Y$ is a left homotopy witnessing $g\overset{l}{\sim} h$. We first observe that $i_0: X\rightarrow X\wedge I$ is a weak equivalence, as $id_X = p_1i_0$ where $id_X$ and $p_1$ are weak equivalences. Since $X$ is assumed to be cofibrant, we see that $X\coprod X$ is cofibrant by the following pushout:
                \begin{center}
                    \begin{tikzcd}
                        * \ar[]{r}[]{} \ar[]{d}[]{} \ar[phantom]{rd}[near end]{\lrcorner} & X \ar[]{d}[]{inr}\\
                        X \ar[]{r}[]{inl} & X\coprod X
                    \end{tikzcd}
                \end{center}
                Moreover, both $inl$ and $inr$ are cofibrations. This shows that $i_0$ is a cofibration as $i_0 = p_0\circ inr$ is a composition of two cofibrations. $i_0$ is thus an acyclic cofibration. We define an almost cylinder object $C$ by the pushout of $i_1$ and $i_0'$. We define the maps $t$ and $H''$ by using the universal property in the following manner:
                \begin{center}
                    \begin{tikzcd}
                        X \ar[]{r}[]{i_1} \ar[]{d}[]{i_0'} & X\wedge I \ar[]{d}[]{} \ar[bend left]{rdd}[]{p_1} \\
                        X\wedge I' \ar[]{r}[]{} \ar[bend right]{rrd}[]{p_1'} & C \ar[dashed]{rd}[]{t} \\
                        & & X
                    \end{tikzcd}\qquad
                    \begin{tikzcd}
                        X \ar[]{r}[]{i_1} \ar[]{d}[]{i_0'} & X\wedge I \ar[]{d}[]{} \ar[bend left]{rdd}[]{H} \\
                        X\wedge I' \ar[]{r}[]{} \ar[bend right]{rrd}[]{H'} & C \ar[dashed]{rd}[]{H''} \\
                        & & X
                    \end{tikzcd}
                \end{center}
                Observe that there is a factorization of the fold map $X\coprod X \overset{s}{\rightarrow} C \overset{t}{\rightarrow} X$. However, $s$ may not be a cofibration, so we replace $C$ with the cylinder object $X\wedge I''$ such that we have the factorization $X\coprod X \overset{s_\alpha}{\rightarrow} X\wedge I'' \overset{ts_\beta}{\rightarrow} X$. The morphism $Hs_\beta$ is then our required homotopy for $f\overset{l}{\sim}g$.

                $(4.)$ Suppose that $X$ is cofibrant and that $H:X\wedge I\rightarrow Y$ is a left homotopy for $f \overset{l}{\sim} g$. Pick a path object for $Y$, such that we have the factorization $Y\overset{q_0}{\rightarrow}Y^I\overset{q_1}{\rightarrow}Y\prod Y$ where $q_0$ is a weak equivalence and $q_1$ is a fibration. Again, as $X$ is cofibrant we get that $i_0$ is an acyclic cofibration, so we have the following lift of the homotopy:
                \begin{center}
                    \begin{tikzcd}
                        X \ar[]{r}[]{q_0f} \ar[]{d}[]{i_0} & Y^I \ar[]{d}[]{q_1} \\
                        X\wedge I \ar[]{r}[]{(fp_1, H)} \ar[dashed]{ru}[]{J} & Y\prod Y
                    \end{tikzcd}
                \end{center}
                The right homotopy is given by injecting away from $f$, i.e. $H' = Ji_1$.
            \end{proof}

            \begin{corollary}
                We collect the dual results of the above proposition, and thus have the following.
                \begin{enumerate}
                    \item If $f \overset{r}{\sim}$ and $h: W \rightarrow X$, then $fh \overset{r}{\sim} gh$.
                    \item If $X$ is cofibrant, $f \overset{r}{\sim} g$ and $h: Y \rightarrow Z$, then $hf \overset{r}{\sim} hg$.
                    \item If $Y$ is fibrant, then left homotopy is an equivalence relation on $\mathcal{C}(X,Y)$.
                    \item If $Y$ is fibrant and $f \overset{r}{\sim} g$, then $f \overset{l}{\sim} g$.
                \end{enumerate}
            \end{corollary}

            \begin{corollary}
                Homotopy is a congruence relation on $\mathcal{C}_{cf}$. In this manner, the category $\mathcal{C}_{cf}/\sim$ is well-defined, exists and inverts every homotopy equivalence.
            \end{corollary}

            \begin{lemma}[Weird Whitehead]\label{lem: Weird-Whitehead}
                Let $\mathcal{C}$ be a model category. Suppose that $C$ is cofibrant and $h: X \rightarrow Y$ is an acyclic fibration or a weak equivalence between fibrant objects, then $h$ induces an isomorphism:
                \begin{center}
                    \begin{tikzcd}
                        \sfrac{\mathcal{C}(C,X)}{\overset{l}{\sim}} \ar[]{r}[]{\overset{h_*}{\simeq}} & \sfrac{\mathcal{C}(C, Y)}{\overset{l}{\sim}}
                    \end{tikzcd}
                \end{center}

                Dually, if $X$ is fibrant and $h : C \rightarrow D$ is an acyclic cofibration or a weak equivalence between cofibrant objects, then $h$ induces an isomorphism:
                \begin{center}
                    \begin{tikzcd}
                        \sfrac{\mathcal{C}(D, X)}{\overset{r}{\sim}} \ar[]{r}[]{\overset{h^*}{\simeq}} & \sfrac{\mathcal{C}(C, X)}{\overset{r}{\sim}}
                    \end{tikzcd}
                \end{center}
            \end{lemma}

            \begin{proof}
                We assume $\mathcal{C}$ to be cofibrant and $h:X\rightarrow Y$ to ba an acyclic fibration. We first prove that $h$ is surjective. Let $f:C\rightarrow Y$. By RLP of $h$ there is a morphism $f':C\rightarrow X$ such that $f = hf'$.
                \begin{center}
                    \begin{tikzcd}
                        \emptyset \ar[]{r}[]{} \ar[]{d}[]{} & X \ar[]{d}[]{h} \\
                        C \ar[]{r}[]{f} \ar[dashed]{ru}[]{f'} & Y
                    \end{tikzcd}
                \end{center}
                
                To show injectivity we assume $f,g:C\rightarrow X$ such that $hf\overset{l}{\sim} hg$, in particular there is a left homotopy $H:C\wedge I \rightarrow Y$. Remember that since $C$ is cofibrant, the map $p_0$ is a cofibration. We find a left homotopy $H:C\wedge I \rightarrow X$ witnessing $f\overset{l}{\sim} g$ by the following lift.
                \begin{center}
                    \begin{tikzcd}
                        C\coprod C \ar[]{r}[]{f+g} \ar[]{d}[]{p_0} & X \ar[]{d}[]{h} \\
                        C\wedge I \ar[dashed]{ru}[]{H'} \ar[]{r}[]{H} & Y
                    \end{tikzcd}
                \end{center}

                Moreover, if we assume both $X$ and $Y$ to be fibrant, the functor $\sfrac{\mathcal{C}(C,\_)}{\overset{l}{\sim}}$ sends acyclic fibrations to isomorphisms, i.e. to weak equivalences. By Ken Brown's lemma, lemma \ref{lem: Ken-Brown}, the afformentioned functor sends weak equivalences between fibrant objects to isomorphisms.
            \end{proof}

            \begin{thm}[Generalized Whiteheads theorem]\label{thm: Whitehead}
                Let $\mathcal{C}$ be a model category. Suppose that $f : X \rightarrow Y$ is a morphism of bifibrant objects, then $f$ is a weak equivalence if and only if $f$ is a homotopy equivalence.
            \end{thm}

            \begin{proof}
                Suppose first that $f$ is a weak equivalence. Pick a bifibrant object $A$, then by lemma \ref{lem: Weird-Whitehead} $f_*:\sfrac{\mathcal{C}(A,X)}{\sim}\rightarrow \sfrac{\mathcal{C}(A,Y)}{\sim}$ is an isomorphism. Letting $A = Y$ we know that there is a morphism $g:Y\rightarrow X$, such that $f_*g = fg \sim id_Y$. Furthermore, by proposition \ref{prop: basic-homotopy}, since $X$ is bifibrant composing on the right preserves homotopy equivalence, e.g. $fgf \sim f$. By letting $A = X$, we get that $f_*gf = fgf \sim f = f_*id_X$, thus $gf \sim id_X$.

                For the opposite direction, assume that $f$ is a homotopy equivalence. We factor $f$ into an acyclic cofibration $f_\gamma$ and a fibration $f_\delta$, i.e. $X \overset{f_\gamma}{\rightarrow} Z \overset{f_\delta}{\rightarrow} Y$. Observe that $Z$ is bifibrant as $X$ and $Y$ is, in particular, $f_\gamma$ is a weak equivalence of bifibrant objects, so it is a homotopy equivalence. 

                It is enough to show that $f_\delta$ is a weak equivalence. Let $g$ be the homotopy inverse of $f$, and $H:Y\wedge I \rightarrow Y$ is a left homotopy witnessing $fg \sim id_Y$. Since $Y$ is bifibrant, the following square has a lift.
                \begin{center}
                    \begin{tikzcd}
                        Y \ar[]{r}[]{f_\gamma g} \ar[]{d}[]{i_0} & Z \ar[]{d}[]{f_\delta} \\
                        Y\wedge I \ar[]{r}[]{H} \ar[dashed]{ru}[]{H'} & Y
                    \end{tikzcd}
                \end{center}
                Let $h = H'i_1$, then by definition we know that $f_\delta H'i_1 = id_Y$. Moreover, $H$ is a left homotopy witnessing $f_\gamma g \sim h$. Let $g': Z\rightarrow X$ be the homotopy inverse of $f_\gamma$. We have the following relations $f_\delta \sim f_\delta f_\gamma g' \sim fg'$, and $hf_\delta \sim (f_\gamma g)(fg') \sim f_\gamma g' \sim id_Z$. Let $H'':Z\wedge I\rightarrow Z$ be a left homotopy witnessing this homotopy. Since $Z$ is bifibrant, $i_0$ and $i_1$ are weak equivalences. By the $2$-out-of-$3$ property $H''$ and $hf_\delta$ are weak equivalences. Since $f_\delta h = id_Y$, it follows that $f_\delta$ is a retract of $f_\delta h$, and is thus a weak equivalence.
            \end{proof}

            \begin{corollary}
                The category $\sfrac{\mathcal{C}_{cf}}{\sim}$ satisfy the universal property of the localization of $\mathcal{C}_{cf}$ by the weak equivalences. I.e. there is a categorical equivalence $Ho\mathcal{C}_{cf} \simeq \sfrac{\mathcal{C}_{cf}}{\sim}$.
            \end{corollary}

            \begin{proof}
                By generalized Whiteheads theorem, theorem \ref{thm: Whitehead} weak equivalences and homotopy equivalences coincide. The corollary follows steadily from both the universal property of the localization category and the quotient category. 
            \end{proof}

            We collect the results from above in the following theorem.

            \begin{thm}[Fundamental theorem of model categories]
                Let $\mathcal{C}$ be a model category and denote $q: \mathcal{C} \rightarrow Ho\mathcal{C}$ the localization functor. Let $X$ and $Y$ be objects of $\mathcal{C}$.
                \begin{enumerate}
                    \item There is an equivalence of categories $Ho\mathcal{C}\simeq \sfrac{\mathcal{C}_{cf}}{\sim}$.
                    \item There are natural isomorphisms $\sfrac{\mathcal{C}_{cf}}{\sim}(QRX,QRY)\simeq Ho\mathcal{C}(X, Y) \simeq \sfrac{\mathcal{C}_{cf}}{\sim}(RQX, RQY)$. Additionally, $Ho\mathcal{C}(X, Y)\simeq \sfrac{\mathcal{C}_{cf}}{\sim}(QX, RY)$.
                    \item The localization $q$ identifies left or right homotopic morphisms.
                    \item A morphism $f: X \rightarrow Y$ is a weak equivalence if and only if $qf$ is an isomorphism.
                \end{enumerate}
            \end{thm}

            \begin{proof}
                This is clear by the results above.
            \end{proof}

        \subsection{Quillen Functors}

    \section{A Model structure on DG-Algebras}

    \section{The Adjoint Lifted Model Structure on DG-Coalgebras and SHA-Algebras}
\end{document}