\documentclass[../thesis.tex]{subfiles}

\begin{document}

        Quillen envisioned a more general approach to homotopy theory, which he dubbed homotopical algebra. The structure of a model category first enclosed a homotopy theory, and now we mainly consider closed model categories. Many of the results from classical homotopy theory were recovered in the theory of model categories. The theorem which we are most concerned about is Whitehead's theorem:

        \begin{thm}[Whitehead's Theorem]
            Let $X$ and $Y$ be two CW-complexes. If $f:X \rightarrow Y$ is a weak equivalence, it is also a homotopy equivalence. I.e. there exists a morphism $g: Y\rightarrow X$ such that $gf\sim id_X$ and $fg\sim id_Y$.
        \end{thm}

        If we endow a Quillen model category onto the category Top, we get that a space $X$ is bifibrant if and only if it is a CW-complex. The natural generalization is not to ask $X$ to be a CW-complex but a bifibrant object.

        \begin{thm}[Generalized Whiteheads Theorem, {\cite[Proposition 1.2.8][11]{Hovey99}}]
            Let $\mathcal{C}$ be a model category. Suppose that $X$ and $Y$ are bifibrant objects of $\mathcal{C}$ and that there is a weak equivalence $f: X\rightarrow Y$. Then $f$ is also a homotopy equivalence, i.e., there exists a morphism $g: Y\rightarrow X$ such that $gf\sim id_X$ and $fg\sim id_Y$.
        \end{thm}

        The category of differential graded algebras employs such a model category, and here we let the weak equivalences be quasi-isomorphisms. On the other hand, the category of differential graded coalgebras has a model structure where the weak equivalences are the maps sent to quasi-isomorphism by the cobar construction. Moreover, the bar and cobar construction defines a Quillen equivalence between these model structures. As we will see, a dg-coalgebra will be bifibrant exactly when it is an $A_\infty$-algebra. Thus, by Whitehead's theorem, quasi-isomorphisms lift to homotopy equivalences. In this case, the derived category of $A_\infty$-algebras is equivalent to the homotopy category of $A_\infty$-algebras.

        We will conclude this chapter by looking at the category of algebras as a subcategory of $A_\infty$-algebras. The derived category may then be expressed as the homotopy category of $A_\infty$-algebras, restricted to algebras.

    \section{Model categories}

        As one may see in literature, many semantically different definitions of model categories exist, but they are all made to be equivalent under good conditions. The difference mainly comes down to preference. This thesis will use the definitions from Mark Hovey's book "Model Categories" \cite{Hovey99}. In this section, we will define Quillen's model category. We will then prove the fundamental results about model categories, their associated homotopy category, and Quillen functors between model categories.

        Before we state the definition of a model category, we need some preliminary definitions. For this section, let $\mathcal{C}$ be a category.

        \begin{definition}[Retract]
            A morphism $f:A\rightarrow B$ in $\mathcal{C}$ is a retract of a morphism $g: C\rightarrow D$ if it fits in a commutative diagram on the form
            \begin{center}
                \begin{tikzcd}
                    A \ar[]{d}[]{f} \ar[]{r}[]{} \ar[bend left]{rr}[]{id_A}& C \ar[]{d}[]{g} \ar[]{r}[]{} & A \ar[]{d}[]{f} \\
                    B \ar[]{r}[]{} \ar[bend right]{rr}[below]{id_B} & D \ar[]{r}[]{} & B
                \end{tikzcd}
            \end{center}
        \end{definition}

        \begin{definition}[Functorial factorization]
            A pair of functors $\alpha, \beta: \mathcal{C}^\rightarrow\rightarrow\mathcal{C}^\rightarrow$ is called a functorial factorization if for any morphism $f\in \tt{Mor}(\mathcal{C})$, there is a factorization $f = \beta(f)\circ\alpha(f)$. We will use the notation $f_\alpha = \alpha(f)$ and $f_\beta = \beta(f)$. The following commutative diagram depict the functorial factorization:
            \begin{center}
                \begin{tikzcd}
                    A \ar[]{rr}[]{f} \ar[]{rd}[below]{f_\alpha} & & B \\
                    & C \ar[]{ru}[below]{f_\beta}
                \end{tikzcd}
            \end{center}
        \end{definition}

        \begin{definition}[Lifting properties]
            Suppose that the morphisms $i: A \rightarrow B$ and $p: C \rightarrow D$ fit inside a commutative square. $i$ is said to have the left lifting property with respect to $p$, or $p$ has the right lifting property with respect to $i$ if there is an $h: B \rightarrow C$ such that the two triangles commute.
            \begin{center}
                \begin{tikzcd}
                    A \ar[]{r}[]{} \ar[]{d}[]{i} & C \ar[]{d}[]{p} \\
                    B \ar[]{r}[]{} \ar[dashed]{ru}[]{h} & D
                \end{tikzcd}
            \end{center}
        \end{definition}

        \begin{remark}
            We will call the left lifting property LLP and the right lifting property RLP.
        \end{remark}

        \begin{definition}[Wide subcategory]
            We call a subcategory $\mathcal{W} \subset \mathcal{C}$ wide if $\mathcal{W}$ has every object $\mathcal{C}$. In particular, $\mathcal{W}$ is a subcategory having every identity morphism.
        \end{definition}

        \subsection{Model categories}

            \begin{definition}[Model category]
                Let $\mathcal{C}$ be a category with all finite limits and colimits. $\mathcal{C}$ admits a model structure if there are three wide subcategories, each defining a class of morphisms:
                \begin{itemize}
                    \item $\tt{Ac} \subset \tt{Mor}(\mathcal{C})$ are called weak equivalences
                    \item $\tt{Cof}\subset \tt{Mor}(\mathcal{C})$ are called cofibrations
                    \item $\tt{Fib}\subset \tt{Mor}(\mathcal{C})$ are called fibrations
                \end{itemize}
                In addition, we call morphisms in $\tt{Cof}\cap \tt{Ac}$ for acyclic cofibrations and $\tt{Fib}\cap \tt{Ac}$ for acyclic fibrations. Moreover, $\mathcal{C}$ has two functorial factorizations $(\alpha, \beta)$ and $(\gamma, \delta)$. The following axioms should be satisfied:
                \begin{itemize}
                    \item[\textbf{MC1}] The class of weak equivalences satisfy the $2$-out-of-$3$ property, i.e. if $f$ and $g$ are composable morphisms such that $2$ out of $f$, $g$ and $gf$ are weak equivalences, then so is the third.
                    \item[\textbf{MC2}] The three classes $\tt{Ac}$, $\tt{Cof}$ and $\tt{Fib}$ are retraction closed, i.e., if $f$ is a retraction of $g$, and $g$ is either a weak equivalence, cofibration or fibration, then so is $f$.
                    \item[\textbf{MC3}] The class of cofibrations have the left lifting property with respect to acyclic fibrations, and fibrations have the right lifting property with respect to acyclic cofibrations.
                    \item[\textbf{MC4}] Given any morphism $f$, $f_\alpha$ is a cofibration, $f_\beta$ is an acyclic fibration, $f_\gamma$ is an acyclic cofibration and $f_\delta$ is a fibration.      
                \end{itemize}
            \end{definition}

            \begin{remark}
                The class $\tt{Ac}$ has every isomorphism, and this is because every isomorphism is a retract of some identity morphism.
            \end{remark}

            \begin{remark}
                The type of category above was first called a closed model category by Quillen \cite{Quillen67}. In his sense, a model category does not require finite limits or finite colimits. In our case, we will explicitly state whenever a model category is non-closed, i.e., it does not have every finite limit or colimit.
            \end{remark}

            A model category $\mathcal{C}$ is now defined to be a category equipped with a particular model structure. Notice that a category may admit several model structures. For more topological examples, we refer to Dwyer--Spalinski \cite{Dwyer95} and Hovey \cite{Hovey99}.

            An interesting and a not so non-trivial property of model categories is that giving all three classes $\tt{Ac}$, $\tt{Cof}$, and $\tt{Fib}$ is redundant. The model structure is determined by the class of weak equivalences and either cofibrations or fibrations. Thus the classes of fibrations are determined by acyclic cofibrations, and fibrations determine cofibrations. The following two results will show this.

            \begin{lemma}[The retract argument]\label{lem: retract-argument}
                Let $\mathcal{C}$ be a category. Suppose there is a factorization $f = pi$ and that $f$ has LLP with respect to $p$; then $f$ is a retract of $i$. Dually, if $f$ has RLP to $i$, then it is a retract of $p$.
            \end{lemma}

            \begin{proof}
                We assume that $f: A \rightarrow C$ has LLP with respect to $p: B \rightarrow C$. Then we may find a lift $r: C \rightarrow B$, which realizes $f$ as a retract of $i$.
                
                \begin{center}
                    \begin{tikzcd}
                        A \ar[]{d}[]{f} \ar[]{r}[]{i} & B \ar[]{d}[]{p} \\
                        C \ar[equal]{r}[]{} \ar[dashed]{ru}[]{r} & C
                    \end{tikzcd} $\implies$
                    \begin{tikzcd}
                        A \ar[equal]{r}[]{} \ar[]{d}[]{f} & A \ar[equal]{r}[]{} \ar[]{d}[]{i} & A \ar[]{d}[]{f} \\
                        C \ar[]{r}[]{r} & B \ar[]{r}[]{p} & C
                    \end{tikzcd}
                \end{center}
            \end{proof}

            \begin{proposition}\label{prop: sat-class}
                Let $\mathcal{C}$ be a model category. A morphism $f$ is a cofibration (acyclic cofibration) if and only if $f$ has LLP with respect to acyclic fibrations (fibrations). Dually, $f$ is a fibration (acyclic fibration) if and only if it has RLP with respect to acyclic cofibrations (cofibrations).
            \end{proposition}

            \begin{proof}
            Assume that $f$ is a cofibration. By \textbf{MC3}, we know that $f$ has LLP with respect to acyclic fibrations. Assume instead that $f$ has LLP with respect to every acyclic fibration. By \textbf{MC4}, we factor $f = f_\alpha\circ f_\beta$, where $f_\alpha$ is a cofibration, and $f_\beta$ is an acyclic fibration. Since we assume $f$ to have LLP with respect to $f_\beta$, by Lemma \ref{lem: retract-argument}, we know that $f$ is a retract of $f_\alpha$. Thus by \textbf{MC2}, we know that $f$ is a cofibration. 
            \end{proof}

            \begin{corollary}\label{cor: stable-cofib-base-change}
                Let $\mathcal{C}$ be a model category. (Acyclic) Cofibrations are stable under pushouts, i.e., if $f$ is an (acyclic) cofibration, then $f'$ is an (acyclic) cofibration.
                \begin{center}
                    \begin{tikzcd}
                        A \ar[]{r}[]{} \ar[]{d}[]{f} \ar[phantom]{rd}[near end]{\lrcorner} & C \ar[]{d}[]{f'} \\
                        B \ar[]{r}[]{} & D
                    \end{tikzcd}
                \end{center}
                Dually, fibrations are stable under pullbacks.
            \end{corollary}

            \begin{proof}
                Consider the diagram
                \begin{center}
                    \begin{tikzcd}
                        A \ar[]{r} \ar[]{d}[]{f} \ar[phantom]{dr}[near end]{\lrcorner} & C \ar[]{r} \ar[]{d}[]{f'} & E \ar[]{d}[]{g} \\
                        B \ar[]{r} & D \ar[]{r} & F
                    \end{tikzcd}
                \end{center}
                where the left-hand square is a pushout. Then $f$ has LLP to $g$ if and only if $f'$ has LLP to $g$ by the universal property of the pushout. It follows by Proposition~\ref{prop: sat-class} that $f'$ is a cofibration.
            \end{proof}

            Since we assume that every model category $\mathcal{C}$ admits finite limits and colimits, we know that it has both an initial and a terminal object. We let $\emptyset$ denote the initial object, and $*$ denote the terminal object. 

            \begin{definition}[Cofibrant, fibrant and bifibrant objects]
                Let $\mathcal{C}$ be a model category. An object $X$ is called cofibrant if the unique morphism $\emptyset \rightarrow X$ is a cofibration. Dually, $X$ is called fibrant if the unique morphism $X \rightarrow *$ is fibrant. If $X$ is both cofibrant and fibrant, we call it bifibrant.
            \end{definition}

            There is no reason for every object to be either cofibrant or fibrant. However, we may see that every object is weakly equivalent to an object which is either fibrant or cofibrant. In this case, we can think of $X$ and $Y$ being weakly equivalent if there is a weak equivalence $f: X \rightarrow Y$. We will make precise what it means for two objects to be weakly equivalent later.

            \begin{construction}
                Let $X$ be an object of a model category $\mathcal{C}$. The morphism $i:\emptyset\rightarrow X$ has a functorial factorization $i=i_\beta\circ i_\alpha$, where $i_\alpha: \emptyset\rightarrow QX$ is a cofibration and $i_\beta: QX\rightarrow X$ is an acyclic fibration. By definition, $QX$ is cofibrant and weakly equivalent to $X$.

                $Q: \mathcal{C}\rightarrow \mathcal{C}$ defines a functor called the cofibrant replacement. To see this, we first look at the slice category $\sfrac{\emptyset}{\mathcal{C}}$. The objects are morphisms $f:\emptyset \rightarrow X$ for any object $X$ in $\mathcal{C}$, while morphisms are commutative triangles. We first observe that $\sfrac{\emptyset}{\mathcal{C}}\subset\mathcal{C}^\rightarrow$ is a subcategory of the arrow category. Thus $(\alpha, \beta)$ may be interpreted as functors on the slice category to the arrow category. Moreover, since every arrow $f:\emptyset \rightarrow X$ is unique, we observe that this category is equivalent to $\mathcal{C}$. Thus $(\alpha, \beta)$ may be interpreted as functors on $\mathcal{C}$ into arrows. We define $Q$ as the composition $Q = \tt{cod} \circ \alpha$, where $\tt{cod} : \mathcal{C}^{\rightarrow} \rightarrow \mathcal{C}$ is the codomain functor.

                Dually, we get a fibrant replacement functor $R: \mathcal{C} \rightarrow \mathcal{C}$. By the functorial factorizations, we have natural transformations $q : Q \Rightarrow \tt{Id}_\mathcal{C}$ and $r : \tt{Id}_\mathcal{C} \Rightarrow R$.
            \end{construction}

            We collect the following properties

            \begin{lemma}\label{lem: Q-preserves-weak}
                The cofibrant replacement $Q$ and fibrant replacement $R$ preserve weak equivalences. 
            \end{lemma}

            \begin{proof}
                Suppose there is a weak equivalence $f: X \rightarrow Y$. Then there is a commutative square
                \begin{center}
                    \begin{tikzcd}
                        QX \ar[]{r}[]{Qf} \ar[]{d}[]{\sim} & QY \ar[]{d}[]{\sim} \\
                        X \ar[]{r}[]{f}[below]{\sim} & Y
                    \end{tikzcd}
                \end{center}
                where every morphism is a weak equivalence by the $2$-out-of-$3$ property.
            \end{proof}

            \begin{lemma}[Ken Brown's lemma]\label{lem: Ken-Brown}
                Let $\mathcal{C}$ be a model category and $\mathcal{D}$ be a category with weak equivalences satisfying the $2$-out-of-$3$ property. If $F:\mathcal{C} \rightarrow \mathcal{D}$ is a functor sending acyclic cofibrations between cofibrant objects to weak equivalences, then $F$ takes all weak equivalences between cofibrant objects to weak equivalences. Dually, if $F$ takes all acyclic fibrations between fibrant objects to weak equivalences, then $F$ takes all weak equivalences between fibrant objects to weak equivalences.
            \end{lemma}

            \begin{proof}
                Suppose that $A$ and $B$ are cofibrant objects and that $f: A\rightarrow B$ is a weak equivalence. Using the universal property of the coproduct, we define the map $(f, id_B) = p: A\coprod B \rightarrow B$. $p$ has a functorial factorization into a cofibration and acyclic fibration, $p = p_\beta\circ p_\alpha$. We recollect the maps in the following pushout diagram:
                \begin{center}
                    \begin{tikzcd}
                        \emptyset \ar[]{d}[]{} \ar[]{r}[]{} \ar[phantom]{rd}[near end]{\lrcorner} & B \ar[]{d}[]{i_2} \ar[bend left]{rrddd}[]{id_B} \\
                        A \ar[]{r}[]{i_1} \ar[bend right]{rrrdd}[]{f} & A\coprod B \ar[]{rd}[]{p_\alpha} \\
                        & & C \ar[]{rd}[]{p_\beta} \\
                        & & & B
                    \end{tikzcd}
                \end{center}
                By Corollary \ref{cor: stable-cofib-base-change}, both $i_1$ and $i_2$ are cofibrations. Since $f$, $id_B$ and $p_\beta$ are weak equivalences, so are $p_\alpha\circ i_1$ and $p_\alpha\circ i_2$ by \textbf{MC2}. Moreover, they are acyclic cofibrations.

                Assume that $F:\mathcal{C}\rightarrow\mathcal{D}$ is a functor as described above. Then by assumption, $F(p_\alpha\circ i_1)$ and $F(p_\alpha\circ i_2)$ are weak equivalences. Since a functor sends identity to identity, we also know that $F(id_B)$ is a weak equivalence. Thus by the $2$-out-of-$3$ property $F(p_\beta)$ is a weak equivalence, as $F(p_\beta)\circ F(p_\alpha\circ i_2) = id_{F(B)}$. Again, by $2$-out-of-$3$ property $F(f)$ is a weak equivalence, as $F(f) = F(p_\beta)\circ F(p_\alpha\circ i_1)$.
            \end{proof}

        \subsection{Homotopy category}

            At its most abstract, homotopy theory is the study of categories and functions up to weak equivalences. Here, a weak equivalence may be anything, but most commonly, it is a weak equivalence in topological homotopy or a quasi-isomorphism in homological algebra. The biggest concern when dealing with such concepts is to make a functor well-defined when these chosen weak equivalences are inverted. To this end, there is a construction to amend these problems, known as derived functors. We define a homotopical category in the sense of Riehl \cite{Riehl16}.

            \begin{definition}[Homotopical Category]
                Let $\mathcal{C}$ be a category. $\mathcal{C}$ is homotopical if there is a wide subcategory constituting a class of morphisms known as weak equivalences, $\tt{Ac}\subset \tt{Mor}\mathcal{C}$. The weak equivalences should satisfy the $2$-out-of-$6$ property, i.e. given three composable morphisms $f$, $g$ and $h$, if $gf$ and $hg$ are weak equivalences, then so are $f$, $g$, $h$ and $hgf$.

                \begin{center}
                    \begin{tikzcd}[row sep = large]
                        A \ar[]{r}[]{f} \ar[]{rd}[]{gf} \ar[dotted]{rrd}[]{} & B \ar[]{d}[]{g} \ar[]{rd}[]{hg} \\
                        & C \ar[]{r}[]{h} & D
                    \end{tikzcd}
                \end{center}
            \end{definition}

            \begin{remark}
                Notice that the $2$-out-of-$6$ property is stronger than the $2$-out-of-$3$ property. To see this, let either $f$, $g$, or $h$ be the identity, and then conclude with the $2$-out-of-$3$ property.
            \end{remark}

            \begin{remark}
                The collection of weak equivalences contains every isomorphism. To see this pick an isomorphism $f$ and $f^{-1}$, then the compositions are the identity on the domain and codomain, which are assumed to be in $\tt{Ac}$.
            \end{remark}
            
            Given such a homotopical category $\mathcal{C}$, we want to invert every weak equivalence and create the homotopy category of $\mathcal{C}$. This construction is developed in Gabriel and Zisman \cite{Zisman67} called the calculus of fractions. This method tries to mimic localization for commutative rings in a category-theoretic fashion. We will not give an account of the existence or construction of localizations.

            \begin{definition}
                Let $\mathcal{C}$ be a homotopical category. Its homotopy category is $\tt{Ho}\mathcal{C} = \mathcal{C}[\tt{Ac}^{-1}]$, together with a localization functor $L:\mathcal{C}\rightarrow \tt{Ho}\mathcal{C}$. The following universal property determines the localization: If $F:\mathcal{C}\rightarrow \mathcal{D}$ is a functor sending weak equivalences to isomorphisms, then it uniquely factors through the homotopy category up to a unique natural isomorphism $\eta$.

                \begin{center}
                    \begin{tikzcd}
                        \mathcal{C} \ar[""{name = U, below}]{rr}[]{F} \ar[]{rd}[]{L} & & \mathcal{D} \\
                        & \tt{Ho}\mathcal{C} \ar[dashed]{ru}[]{F'} \ar[Rightarrow, from=U]{}[]{\eta}
                    \end{tikzcd}
                \end{center}
            \end{definition}

            \begin{definition}
                Suppose that $\mathcal{C}$ is a homotopical category. Two objects of $\mathcal{C}$ are said to be weakly equivalent if they are isomorphic in $\tt{Ho}\mathcal{C}$. I.e., $X$ and $Y$ are weakly equivalent if there is some zig-zag relation between the objects, consisting only of weak equivalences.
                \begin{center}
                    \begin{tikzcd}
                        X \ar[]{rd}[]{\sim} & & \ar[]{dl}[above]{\sim} \cdots \ar[]{rd}[]{\sim} & & Y \ar[]{ld}[above]{\sim} \\
                        & X' & & Y'
                    \end{tikzcd}
                \end{center}
            \end{definition}

            \begin{remark}
                A renowned problem with localizations is that even if $\mathcal{C}$ is a locally small category, localizations $\mathcal{C}[S^{-1}]$ do not need to be. Thus, without a good theory of classes or higher universes, we cannot generally ensure that localization still exists as a locally small category.
            \end{remark}

            From the definition of the homotopy category, a functor $F$ admits a lift $F'$ from the homotopy category whenever weak equivalences are mapped to isomorphisms. Moreover, if we have a functor $F$ between homotopical categories, which preserves weak equivalences, it then induces a functor between the homotopy categories.
            
            \begin{definition}[Homotopical functors]
                A functor $F:\mathcal{C}\rightarrow \mathcal{D}$ between homotopical categories is homotopical if it preserves weak equivalences. Moreover, there is a lift of functors, as in the following diagram, where $\eta$ is a natural isomorphism.

                \begin{center}
                    \begin{tikzcd}
                        \mathcal{C} \ar[""{name=U, below}]{r}[]{F} \ar[]{d}[]{L_\mathcal{C}} & \mathcal{D} \ar[]{d}[]{L_\mathcal{D}} \ar[Rightarrow]{dl}[]{\eta} \\
                        \tt{Ho}\mathcal{C} \ar[dashed, ""{name=V, above}]{r}[]{F'} & \tt{Ho}\mathcal{D}
                    \end{tikzcd}
                \end{center}
            \end{definition}

            Derived functors becomes relevant whenever we want to make a lift of non-homotopical functors. These lifts will be the closest approximation that we can make functorial. 
            % The general exposition of derived functors is beyond the scope of this thesis, but an account of it may be found in \cite{Riehl16}. 
            We will see that a model category is a congenial environment to work with these concepts. Firstly the problem with localizations where the homotopy category may not exist will be amended. Secondly, we will obtain a simple description of some derived functors.
            
            \begin{proposition}
                Any model category $\mathcal{C}$ is a homotopical category.
            \end{proposition}

            \begin{proof}
                To show that a model category is homotopical, it suffices to show that $\tt{Ac}$ satisfies the $2$-out-of-$6$ property. Assume there are $3$ composable morphisms $f,g,h$ such that $gf,hg\in \tt{Ac}$. By the $2$-out-of-$3$ property for $\tt{Ac}$, it is enough to show that at least one of $f,g,h,fgh$ is a weak equivalence to deduce that every other morphism is a weak equivalence.
                \begin{center}
                    \begin{tikzcd}[row sep = large]
                        A \ar[]{r}[]{f} \ar[]{rd}[]{gf} \ar[dotted]{rrd}[]{} & B \ar[]{d}[]{g} \ar[]{rd}[]{hg} \\
                        & C \ar[]{r}[]{h} & D
                    \end{tikzcd}
                \end{center}

                To use the model structure, we will first show that we may assume $f,g$ to be cofibrant and $g,h$ to be fibrant. We know by \textbf{MC4} that $f,g,gf$ may be factored into a cofibration composed with an acyclic fibration, e.g., $f = f_\beta f_\alpha$. Since $gf$ is a weak equivalence, so is $(gf)_\alpha$ by the $2$-out-of-$3$ property.
                \begin{center}
                    \begin{tikzcd}
                        A \ar[]{rr}[]{f} \ar[]{rd}[below]{f_\alpha} && B \\
                        & B' \ar[]{ru}[below]{f_\beta}
                    \end{tikzcd}
                    \begin{tikzcd}
                        B \ar[]{rr}[]{g} \ar[]{rd}[below]{g_\alpha} && C \\
                        & C' \ar[]{ru}[below]{g_\beta}
                    \end{tikzcd}
                    \begin{tikzcd}
                        A \ar[]{rr}[]{gf} \ar[]{rd}[below, xshift = -1ex]{(gf)_\alpha}&& C \\
                        & C'' \ar[]{ru}[below, xshift = 1ex]{(gf)_\beta}
                    \end{tikzcd}
                \end{center}

                Notice that the "cofibrant approximation" of the map from $A$ to $C$ either goes through $C'$ or $C''$. We conjoin these by taking the pullback. Since acyclic fibrations are stable under pullbacks, we get a pullback square where every morphism is an acyclic fibration. Thus the map $A\rightarrow \widetilde{C}$ is a weak equivalence by $2$-out-of-$3$.
                \begin{center}
                    \begin{tikzcd}
                        A \ar[bend right]{rdd}[]{(gf)_\alpha} \ar[bend left]{rrd}[]{g_\alpha f} \ar[dashed]{rd}[]{t} \\
                        & \widetilde{C} \ar[]{d}[]{} \ar[]{r}[]{} \ar[phantom]{rd}[very near start]{\ulcorner} & C' \ar[]{d}[]{g_\beta}\\
                        & C'' \ar[]{r}[]{(gf)_\beta}& C
                    \end{tikzcd}
                \end{center}

                To replace $f$ with $f_\alpha$, we must lift the composition into our "new" $C$, which is $\widetilde{C}$. We do this using \textbf{MC3}, as $f_\alpha$ is a cofibration and the pullback square above consists entirely of acyclic fibrations.
                \begin{center}
                    \begin{tikzcd}
                        A \ar[]{r}[]{} \ar[]{d}[]{f_\alpha} & \widetilde{C} \ar[]{d}[]{} \\
                        B' \ar[]{r}[]{} \ar[dashed]{ru}[]{s} & C
                    \end{tikzcd}
                \end{center}

                To summarize, we have the following diagram, where every squiggly arrow is a weak equivalence.
                \begin{center}
                    \begin{tikzcd}
                        A \ar[]{r}[]{f_\alpha} \ar[]{rd}[]{t}[below]{\sim} & B' \ar[]{r}[below]{\sim} \ar[]{d}[]{s} & B \ar[]{d}[]{} \ar[]{rd}[below]{\sim} \\
                        & \widetilde{C} \ar[]{r}[below]{\sim} & C \ar[]{r}[]{} & D
                    \end{tikzcd}
                \end{center}

                We now wish to promote the arrow $s: B'\rightarrow \widetilde{C}$ into a cofibration. We do this by factoring $s$ and $t$ with \textbf{MC4}. Notice that $s_\beta$, $t_\beta$ and $t_\alpha$ are weak equivalences.
                \begin{center}
                    \begin{tikzcd}
                        B' \ar[]{rr}[]{s} \ar[]{rd}[]{s_\alpha} && \widetilde{C}\\
                        & \widetilde{C}' \ar[]{ru}[]{s_\beta}
                    \end{tikzcd}
                    \begin{tikzcd}
                        A \ar[]{rr}[]{t} \ar[]{rd}[]{t_\alpha} && \widetilde{C} \\
                        & \widetilde{C}'' \ar[]{ru}[]{t_\beta}
                    \end{tikzcd}
                \end{center}

                To obtain our final factorization, we use RLP of $s_\beta$ on $t_\alpha$.
                \begin{center}
                    \begin{tikzcd}
                        & B' \ar[]{d}[]{s_\alpha} \\
                        A \ar[]{r}[]{} \ar[]{ru}[]{f_\alpha} \ar[]{d}[]{t_\alpha} & \widetilde{C}' \ar[]{d}[]{s_\beta} \\
                        \widetilde{C}'' \ar[]{r}[]{t_\beta} \ar[dashed]{ru}[]{u} & \widetilde{C}
                    \end{tikzcd}
                \end{center}

                Since the bottom square only consists of weak equivalences, $u$ has to be a weak equivalence by the $2$-out-of-$3$ property. In this manner, we may transform our diagram into the following diagram
                \begin{center}
                    \begin{tikzcd}
                        A \ar[]{r}[]{f_\alpha} \ar[]{rd}[]{ut_\alpha}[below]{\sim} & B' \ar[]{r}[below]{\sim} \ar[]{d}[]{s_\alpha} & B \ar[]{d}[]{} \ar[]{rd}[below]{\sim} \\
                        & \widetilde{C}' \ar[]{r}[below]{\sim} & C \ar[]{r}[]{} & D
                    \end{tikzcd}
                \end{center}
                We now have a factorization of $gf$ into two cofibrations, followed by an acyclic fibration, in such a manner that it is compatible with the original diagram. The dual to this claim is that we may also factor $hg$ into two fibrations preceded by an acyclic cofibration. In other words, we may assume without loss of generality that $f$ and $g$ are cofibrations and that $g$ and $h$ are fibrations.
                
                In this case, it is enough to show the $2$-out-of-$6$ property to show that $g$ is an isomorphism. Consider the diagram below with lifts $i$ and $j$, and these exist since we assume $gf$ and $hg$ to be weak equivalences.
                \begin{center}
                    \begin{tikzcd}
                        A \ar[]{r}[]{f} \ar[]{d}[]{gf} & B\ar[]{r}[]{id_B} \ar[]{d}[]{g} & B \ar[]{d}[]{hg} \\
                        C \ar[]{r}[]{id_C} \ar[dashed]{ru}[]{i} & C \ar[]{r}[]{h} \ar[dashed]{ru}[]{j} & D
                    \end{tikzcd}
                \end{center}
                Since the diagram is commutative, we get that $i = j$, and that $g$ is both split-mono and split-epi, with $i$ as its splitting.
            \end{proof}

            Since every model category is homotopical, it also has an associated homotopy category $\tt{Ho}\mathcal{C}$. Let $\mathcal{C}_c$, $\mathcal{C}_f$, and $\mathcal{C}_{cf}$ denote the full subcategories consisting of cofibrant, fibrant and bifibrant objects, respectively.

            \begin{proposition}
                Let $\mathcal{C}$ be a model category. The following categories are equivalent:
                \begin{itemize}
                    \item $\tt{Ho}\mathcal{C}$,
                    \item $\tt{Ho}\mathcal{C}_c$,
                    \item $\tt{Ho}\mathcal{C}_f$,
                    \item $\tt{Ho}\mathcal{C}_{cf}$.
                \end{itemize}
            \end{proposition}

            \begin{proof}
                We only show that $\tt{Ho}\mathcal{C} \simeq \tt{Ho}\mathcal{C}_c$, the other arguments are similar. The inclusion $i:\mathcal{C}_c\rightarrow \mathcal{C}$ preserves weak equivalences; $i$ is homotopical and admits a lift. Moreover, since the cofibrant replacement is homotopical, it also has a lift.

                \begin{center}
                    \begin{tikzcd}
                        \mathcal{C}_c \ar[]{r}[]{i} \ar[]{d}[]{} & \mathcal{C} \ar[]{d}[]{} \\
                        \tt{Ho}\mathcal{C}_c \ar[dashed, bend right]{r}[below]{\tt{Ho}\ i} & \tt{Ho}\mathcal{C} \ar[dashed, bend right]{l}[above]{\tt{Ho}\ Q}
                    \end{tikzcd}
                \end{center}

                It is clear that $\tt{Ho}\ Q$ is the quasi-inverse of $\tt{Ho}\ i$.
            \end{proof}

            We still don't see how model categories will fix the size issues. To do this, we will develop the notion of homotopy equivalence, $\sim$. This homotopy equivalence will be a congruence relation on the subcategory of bifibrant objects $\mathcal{C}_{cf}$. We solve the size issues with this, together with the fact that there is an equivalence of categories $\tt{Ho}\mathcal{C}_{cf}\simeq \sfrac{\mathcal{C}_{cf}}{\sim}$.

            \begin{definition}[Cylinder and path objects]
                Let $\mathcal{C}$ be a model category. Given an object $X$, a cylinder object $X\wedge I$ is a factorization of the codiagonal map $i: X\coprod X \rightarrow X$, such that $p_0$ is a cofibration and that $p_1$ is a weak equivalence. 
                
                \begin{center}
                    \begin{tikzcd}
                        X\coprod X \ar[]{rr}[]{i} \ar[]{rd}[]{p_0} & & X \\
                        & X\wedge I \ar[]{ru}[]{p_1}
                    \end{tikzcd}
                \end{center}

                Dually, a path object  $X^{I}$ is a factorization of the diagonal map $i: X \rightarrow X\prod X$, such that $p_0$ is a weak equivalence and that $p_1$ is a fibration.
                
                \begin{center}
                    \begin{tikzcd}
                        X \ar[]{rr}[]{i} \ar[]{rd}[]{p_0} & & X\prod X\\
                        & X^I \ar[]{ru}[]{p_1}
                    \end{tikzcd}
                \end{center}
            \end{definition}

            \begin{remark}
                Even though we have written $X\wedge I$ suggestively to be a functor, it is not. There may be many choices for a cylinder object. However, by using the functorial factorization from \textbf{MC4}, we get a canonical choice of a cylinder object, as it factors every map into a cofibration and an acyclic fibration. If we let the cylinder object denote this functorial choice, we can define it as a functor.
            \end{remark}

            \begin{proposition}
                Let $\mathcal{C}$ be a model category and $X$ an object of $\mathcal{C}$. Given two cylinder objects $X\wedge I$ and $X\wedge I'$, they are weakly equivalent. 
            \end{proposition}

            \begin{proof}
                It is enough to show that there exists a weak equivalence from any cylinder object into one specified cylinder object. There is such a map for the functorial cylinder object $X\wedge I$, as the morphism $p_1$ is an acyclic fibration, which enables a lift that is a weak equivalence by the $2$-out-of-$3$ property.

                \begin{center}
                    \begin{tikzcd}
                        X\coprod X \ar[]{r}[]{p_0} \ar[]{d}[]{p_0'} & X\wedge I \ar[]{d}[]{p_1} \\
                        X\wedge I' \ar[]{r}[]{p_1'} \ar[dashed]{ru}[]{} & X
                    \end{tikzcd}
                \end{center}
            \end{proof}

            \begin{definition}[Homotopy equivalence]
                Let $f,g: X\rightarrow Y$. A left homotopy between $f$ and $g$ is a morphism $H:X\wedge I \rightarrow Y$ such that $Hi_0 = f$ and $Hi_1 = g$. We say that $f$ and $g$ are left homotopic if a left homotopy exists, and it is denoted $f \overset{l}{\sim} g$.

                \begin{center}
                    \begin{tikzcd}
                        X \ar[]{d}[]{} \ar[dotted]{rd}[]{i_0} \ar[bend left]{rrd}[]{f} \\
                        X\coprod X \ar[]{r}[]{p_0} & X\wedge I \ar[]{r}[]{H} & Y \\
                        X \ar[]{u}[]{} \ar[dotted]{ru}[]{i_1} \ar[bend right]{rru}[]{g}
                    \end{tikzcd}
                \end{center}
                
                A right homotopy between $f$ and $g$ is a morphism $H: X \rightarrow Y^I$ such that $i_0H = f$ and $i_1H = g$. We say that $f$ and $g$ are right homotopic if a right homotopy exists, and it is denoted $f \overset{r}{\sim} g$.

                \begin{center}
                    \begin{tikzcd}
                        & & Y \\
                        X \ar[]{r}[]{H} \ar[bend left]{rru}[]{f} \ar[bend right]{rrd}[]{g} & Y^I \ar[]{r}[]{p_1} \ar[dotted]{ru}[]{i_0} \ar[dotted]{rd}[]{i_1} & Y\prod Y \ar[]{u}[]{} \ar[]{d}[]{} \\
                        & & Y
                    \end{tikzcd}
                \end{center}

                $f$ and $g$ are said to be homotopic if they are both left and right homotopic, denoted $f \sim g$. $f$ is a homotopy equivalence if it has a homotopy inverse $h: Y \rightarrow X$, such that $hf \sim id_X$ and $fh \sim id_Y$. 
            \end{definition}

            It is important to note that homotopy equivalence is not a priori an equivalence relation. With the following two propositions, we can amend this by taking both fibrant and cofibrant replacements.

            \begin{proposition}\label{prop: basic-homotopy}
                Let $\mathcal{C}$ be a model category, and $f,g: X\rightarrow Y$ be morphisms. We have the following:
                \begin{enumerate}
                    \item If $f \overset{l}{\sim}g$ and $h: Y \rightarrow Z$, then $hf \overset{l}{\sim} hg$.
                    \item If $Y$ is fibrant, $f \overset{l}{\sim} g$ and $h: W \rightarrow X$, then $fh \overset{l}{\sim} gh$.
                    \item If $X$ is cofibrant, then left homotopy is an equivalence relation on $\mathcal{C}(X, Y)$.
                    \item If $X$ is cofibrant and $f \overset{l}{\sim} g$, then $f \overset{r}{\sim} g$.
                \end{enumerate}
            \end{proposition}

            \begin{proof}
                $(1.)$ Assume that $f \overset{l}{\sim} g$ and $h:Y\rightarrow Z$. Let $H: X\wedge I \rightarrow Y$ denote the left homotopy between $f$ and $g$. The left homotopy between $hf$ and $hg$ is $hH$.

                $(2.)$ Assume that $Y$ is fibrant, $f \overset{l}{\sim} g$ and that $h:W\rightarrow X$. Let $H: X\wedge I\rightarrow Y$ be a left homotopy. We construct a new cylinder object for the homotopy. Factor $p_1:X\wedge I \rightarrow X$ as $q_1\circ q_0$ where $q_0: X\wedge I \rightarrow X\wedge I'$ is an acyclic cofibration and $q_1:X\wedge I'\rightarrow X$ is a fibration. By the $2$-out-of-$3$ property, $q_1$ is an acyclic fibration, as $p_1$ and $q_0$ are weak equivalences. $X\wedge I'$ is a cylinder object as $q_0\circ p_0$ is a cofibration and $q_1$ is a weak equivalence. Since we assume $Y$ to be fibrant we lift the left homotopy $H:X\wedge I\rightarrow Y$ to the left homotopy $H':X\wedge I'\rightarrow Y$ with the following diagram:
                \begin{center}
                    \begin{tikzcd}
                        X\wedge I \ar[]{r}[]{H} \ar[]{d}[]{q_0} & Y \ar[]{d}[]{} \\
                        X\wedge I' \ar[]{r}[]{} \ar[dashed]{ru}[]{H'} & *
                    \end{tikzcd}                    
                \end{center}
                We let $W \smash I$ be a cylinder object for $W$, where $p_0': W \sqcup W \rightarrow Q \smash I$ is a cofibration. We can find an appropriate homotopy needed with LLP of $q_1$ against $p_0'$, as done in the diagram below.
                \begin{center}
                    \begin{tikzcd}[column sep = large]
                        W\coprod W \ar[]{r}[]{q_0p_0(h\coprod h)} \ar[]{d}[]{p_0'} & X\wedge I' \ar[]{d}[]{q_1} \\
                        W\wedge I \ar[]{r}[]{hp_1'} \ar[dashed]{ru}[]{k} & X
                    \end{tikzcd}
                \end{center}
                The morphism $H'k$ is the desired left homotopy witnessing $fh \overset{l}{\sim} gh$.

                $(3.)$ Assume that $X$ is cofibrant. First, observe that a left homotopy is reflexive and symmetric. We must show that it is also transitive. Thus, assume that $f,g,h:X\rightarrow Y$ and that $H:X\wedge I\rightarrow Y$ is a left homotopy witnessing $f \overset{l}{\sim} g$ and that $H':X\wedge I'\rightarrow Y$ is a left homotopy witnessing $g\overset{l}{\sim} h$. We first observe that $i_0: X\rightarrow X\wedge I$ is a weak equivalence, as $id_X = p_1i_0$ where $id_X$ and $p_1$ are weak equivalences. Since $X$ is assumed to be cofibrant, we see that $X\coprod X$ is cofibrant by the following pushout:
                \begin{center}
                    \begin{tikzcd}
                        \emptyset \ar[]{r}[]{} \ar[]{d}[]{} \ar[phantom]{rd}[near end]{\lrcorner} & X \ar[]{d}[]{inr}\\
                        X \ar[]{r}[]{inl} & X\coprod X
                    \end{tikzcd}
                \end{center}
                Moreover, both $inl$ and $inr$ are cofibrations. It follows that $i_0$ is a cofibration as $i_0 = p_0\circ inr$ is a composition of two cofibrations. $i_0$ is thus an acyclic cofibration. We define an almost cylinder object $C$ by the pushout of $i_1$ and $i_0'$. We define the maps $t$ and $H''$ by using the universal property in the following manner:
                \begin{center}
                    \begin{tikzcd}
                        X \ar[]{r}[]{i_1} \ar[]{d}[]{i_0'} & X\wedge I \ar[]{d}[]{} \ar[bend left]{rdd}[]{p_1} \\
                        X\wedge I' \ar[]{r}[]{} \ar[bend right]{rrd}[]{p_1'} & C \ar[dashed]{rd}[]{t} \\
                        & & X
                    \end{tikzcd}\qquad
                    \begin{tikzcd}
                        X \ar[]{r}[]{i_1} \ar[]{d}[]{i_0'} & X\wedge I \ar[]{d}[]{} \ar[bend left]{rdd}[]{H} \\
                        X\wedge I' \ar[]{r}[]{} \ar[bend right]{rrd}[]{H'} & C \ar[dashed]{rd}[]{H''} \\
                        & & Y
                    \end{tikzcd}
                \end{center}
                Observe that there is a factorization of the codiagonal map $X\coprod X \overset{s}{\rightarrow} C \overset{t}{\rightarrow} X$. However, $s$ may not be a cofibration, so we replace $C$ with the cylinder object $X\wedge I''$ such that we have the factorization $X\coprod X \overset{s_\alpha}{\rightarrow} X\wedge I'' \overset{ts_\beta}{\rightarrow} X$. The morphism $H''s_\beta$ is then our required homotopy for $f\overset{l}{\sim}g$.

                $(4.)$ Suppose that $X$ is cofibrant and that $H:X\wedge I\rightarrow Y$ is a left homotopy for $f \overset{l}{\sim} g$. Pick a path object for $Y$, such that we have the factorization $Y\overset{q_0}{\rightarrow}Y^I\overset{q_1}{\rightarrow}Y\prod Y$ where $q_0$ is a weak equivalence and $q_1$ is a fibration. Again, as $X$ is cofibrant, we get that $i_0$ is an acyclic cofibration, so we have the following lift of the homotopy:
                \begin{center}
                    \begin{tikzcd}
                        X \ar[]{r}[]{q_0f} \ar[]{d}[]{i_0} & Y^I \ar[]{d}[]{q_1} \\
                        X\wedge I \ar[]{r}[]{(fp_1, H)} \ar[dashed]{ru}[]{J} & Y\prod Y
                    \end{tikzcd}
                \end{center}
                The right homotopy is given by injecting away from $f$, i.e., $H' = Ji_1$.
            \end{proof}

            \begin{corollary}\label{cor: basic-homotopy-op}
                We collect the dual results of the above proposition and thus have the following.
                \begin{enumerate}
                    \item If $f \overset{r}{\sim}g$ and $h: W \rightarrow X$, then $fh \overset{r}{\sim} gh$.
                    \item If $X$ is cofibrant, $f \overset{r}{\sim} g$ and $h: Y \rightarrow Z$, then $hf \overset{r}{\sim} hg$.
                    \item If $Y$ is fibrant, then left homotopy is an equivalence relation on $\mathcal{C}(X, Y)$.
                    \item If $Y$ is fibrant and $f \overset{r}{\sim} g$, then $f \overset{l}{\sim} g$.
                \end{enumerate}
            \end{corollary}

            \begin{corollary}\label{cor: homotopy-is-eq-rel}
                Homotopy is a congruence relation on $\mathcal{C}_{cf}$. Thus the category $\mathcal{C}_{cf}/\sim$ is well-defined, exists, and inverts every homotopy equivalence.
            \end{corollary}

            \begin{lemma}[Weird Whitehead]\label{lem: Weird-Whitehead}
                Let $\mathcal{C}$ be a model category. Suppose that $C$ is cofibrant and $h: X \rightarrow Y$ is an acyclic fibration or a weak equivalence between fibrant objects, then $h$ induces an isomorphism:
                \begin{center}
                    \begin{tikzcd}
                        \sfrac{\mathcal{C}(C,X)}{\overset{l}{\sim}} \ar[]{r}[]{\overset{h_*}{\simeq}} & \sfrac{\mathcal{C}(C, Y)}{\overset{l}{\sim}}
                    \end{tikzcd}
                \end{center}

                Dually, if $X$ is fibrant and $h: C \rightarrow D$ is an acyclic cofibration or a weak equivalence between cofibrant objects, then $h$ induces an isomorphism:
                \begin{center}
                    \begin{tikzcd}
                        \sfrac{\mathcal{C}(D, X)}{\overset{r}{\sim}} \ar[]{r}[]{\overset{h^*}{\simeq}} & \sfrac{\mathcal{C}(C, X)}{\overset{r}{\sim}}
                    \end{tikzcd}
                \end{center}
            \end{lemma}

            \begin{proof}
                We assume $\mathcal{C}$ to be cofibrant and $h: X\rightarrow Y$ to be an acyclic fibration. We first prove that $h$ is surjective. Let $f:C\rightarrow Y$. By RLP of $h$, there is a morphism $f':C\rightarrow X$ such that $f = hf'$.
                \begin{center}
                    \begin{tikzcd}
                        \emptyset \ar[]{r}[]{} \ar[]{d}[]{} & X \ar[]{d}[]{h} \\
                        C \ar[]{r}[]{f} \ar[dashed]{ru}[]{f'} & Y
                    \end{tikzcd}
                \end{center}
                
                To show injectivity, we assume $f,g: C\rightarrow X$ such that $hf\overset{l}{\sim} hg$, in particular, there is a left homotopy $H: C\wedge I \rightarrow Y$. Remember that since $C$ is cofibrant, the map $p_0$ is a cofibration. We find a left homotopy $H: C\wedge I \rightarrow X$ witnessing $f\overset{l}{\sim} g$ by the following lift.
                \begin{center}
                    \begin{tikzcd}
                        C\coprod C \ar[]{r}[]{f+g} \ar[]{d}[]{p_0} & X \ar[]{d}[]{h} \\
                        C\wedge I \ar[dashed]{ru}[]{H'} \ar[]{r}[]{H} & Y
                    \end{tikzcd}
                \end{center}

                If we instead assume that both $X$ and $Y$ are fibrant, then the functor $\sfrac{\mathcal{C}(C,\argument)}{\overset{l}{\sim}}$ sends acyclic fibrations to isomorphisms by Corollary~\ref{cor: basic-homotopy-op}. Ken Brown's lemma, Lemma \ref{lem: Ken-Brown}, tells us then that $\sfrac{\mathcal{C}(C,\argument)}{\overset{l}{\sim}}$ sends weak equivalences between fibrant objects to isomorphisms.
            \end{proof}

            \begin{thm}[Generalized Whitehead's theorem]\label{thm: Whitehead}
                Let $\mathcal{C}$ be a model category. Suppose that $f: X \rightarrow Y$ is a morphism of bifibrant objects. Then $f$ is a weak equivalence if and only if $f$ is a homotopy equivalence.
            \end{thm}

            \begin{proof}
                Suppose first that $f$ is a weak equivalence. Pick a bifibrant object $A$, then by Lemma~\ref{lem: Weird-Whitehead} $f_*:\sfrac{\mathcal{C}(A,X)}{\sim}\rightarrow \sfrac{\mathcal{C}(A,Y)}{\sim}$ is an isomorphism. Letting $A = Y$, we know that there is a morphism $g: Y\rightarrow X$, such that $f_*g = fg \sim id_Y$. Furthermore, by Proposition~\ref{prop: basic-homotopy}, since $X$ is bifibrant, composing on the right preserves homotopy equivalence, e.g., $fgf \sim f$. By letting $A = X$, we get that $f_*gf = fgf \sim f = f_*id_X$, thus $gf \sim id_X$.

                For the opposite direction, assume that $f$ is a homotopy equivalence. We factor $f$ into an acyclic cofibration $f_\gamma$ and a fibration $f_\delta$, i.e. $X \overset{f_\gamma}{\rightarrow} Z \overset{f_\delta}{\rightarrow} Y$. Observe that $Z$ is bifibrant as $X$ and $Y$ is, in particular, $f_\gamma$ is a weak equivalence of bifibrant objects, so it is a homotopy equivalence. 

                It is enough to show that $f_\delta$ is a weak equivalence. Let $g$ be the homotopy inverse of $f$, and $H: Y\wedge I \rightarrow Y$ is a left homotopy witnessing $fg \sim id_Y$. Since $Y$ is bifibrant, the following square has a lift.
                \begin{center}
                    \begin{tikzcd}
                        Y \ar[]{r}[]{f_\gamma g} \ar[]{d}[]{i_0} & Z \ar[]{d}[]{f_\delta} \\
                        Y\wedge I \ar[]{r}[]{H} \ar[dashed]{ru}[]{H'} & Y
                    \end{tikzcd}
                \end{center}
                Let $h = H'i_1$, and then by definition, we know that $f_\delta H'i_1 = id_Y$. Moreover, $H$ is a left homotopy witnessing $f_\gamma g \sim h$. Let $g': Z\rightarrow X$ be the homotopy inverse of $f_\gamma$. We have the following relations $f_\delta \sim f_\delta f_\gamma g' \sim fg'$, and $hf_\delta \sim (f_\gamma g)(fg') \sim f_\gamma g' \sim id_Z$. Let $H'':Z\wedge I\rightarrow Z$ be a left homotopy witnessing this homotopy. Since $Z$ is bifibrant, $i_0$ and $i_1$ are weak equivalences. By the $2$-out-of-$3$ property, $H''$ and $hf_\delta$ are weak equivalences. Since $f_\delta h = id_Y$, it follows that $f_\delta$ is a retract of $h f_\delta$ and is thus a weak equivalence.
            \end{proof}

            \begin{corollary}
                The category $\sfrac{\mathcal{C}_{cf}}{\sim}$ satisfies the universal property of the localization of $\mathcal{C}_{cf}$ by the weak equivalences. I.e. there is a categorical equivalence $\tt{Ho}\mathcal{C}_{cf} \simeq \sfrac{\mathcal{C}_{cf}}{\sim}$.
            \end{corollary}

            \begin{proof}
                By generalized Whitehead's theorem, Theorem \ref{thm: Whitehead} weak equivalences and homotopy equivalences coincide. The corollary follows steadily from the universal property of the localization and quotient categories. 
            \end{proof}

            We collect the results from above in the following theorem.

            \begin{thm}[Fundamental theorem of model categories]\label{thm: Fundamental-thm-model}
                Let $\mathcal{C}$ be a model category and denote $L: \mathcal{C} \rightarrow \tt{Ho}\mathcal{C}$ the localization functor. Let $X$ and $Y$ be objects of $\mathcal{C}$.
                \begin{enumerate}
                    \item There is an equivalence of categories $\tt{Ho}\mathcal{C}\simeq \sfrac{\mathcal{C}_{cf}}{\sim}$.
                    \item There are natural isomorphisms $\sfrac{\mathcal{C}_{cf}}{\sim}(QRX,QRY)\simeq \tt{Ho}\mathcal{C}(X, Y) \simeq \sfrac{\mathcal{C}_{cf}}{\sim}(RQX, RQY)$. Additionally, $\tt{Ho}\mathcal{C}(X, Y)\simeq \sfrac{\mathcal{C}_{cf}}{\sim}(QX, RY)$.
                    \item The localization $L$ identifies left or right homotopic morphisms.
                    \item A morphism $f: X \rightarrow Y$ is a weak equivalence if and only if $qf$ is an isomorphism.
                \end{enumerate}
            \end{thm}

            \begin{proof}
                 theorem is clear by the results above.
            \end{proof}

        \subsection{Quillen adjoints}

            We now want to study morphisms, or certain functors, between model categories. Like in the case of homotopical functors, we want these morphisms to induce a functor between the homotopy categories. However, we also want them to respect the cofibration and fibration structure, not just weak equivalences. In this way, we will instead look toward derived functors to be able to define this extension to the homotopy category. We recall the definition of a total (left/right) derived functor. In the case of model categories, we get a simple description of some of these derived functors.

            \begin{definition}[Total derived functors]
                Let $\mathcal{C}$ and $\mathcal{D}$ be homotopical categories, and $F:\mathcal{C}\rightarrow\mathcal{D}$ a functor. Whenever it exists, a total left derived functor of $F$ is a functor $\mathbb{L}F:\tt{Ho}\mathcal{C}\rightarrow \tt{Ho}\mathcal{D}$ with a natural transformation $\varepsilon:\mathbb{L}F\circ L \Rightarrow L\circ F$ satisfying the universal property: If $G:\tt{Ho}\mathcal{C}\rightarrow \tt{Ho}\mathcal{D}$ is a functor. There is a natural transformation $\alpha: G\circ L \Rightarrow L\circ F$, then it factors uniquely up to unique isomorphism through $\varepsilon$.
                \begin{center}
                    \begin{tikzcd}
                        \mathcal{C} \ar[]{r}[]{F} \ar[]{d}[]{L} & \mathcal{D} \ar[]{d}[]{L} \\
                        \tt{Ho}\mathcal{C} \ar[dashed]{r}[]{\mathbb{L}F} \ar[Rightarrow]{ru}[]{\varepsilon} & \tt{Ho}\mathcal{D}
                    \end{tikzcd}\qquad
                    \begin{tikzcd}
                        \mathcal{C} \ar[]{r}[]{F} \ar[]{d}[]{L} & \mathcal{D} \ar[phantom, ""{name = Y}]{}[]{} \ar[]{d}[]{L} \\
                        \tt{Ho}\mathcal{C} \ar[bend left, ""{name = V, below}, ""{name=X, above}]{r}[description]{\mathbb{L}F} \ar[bend right, ""{name = U, above}]{r}[description]{G} \ar[Rightarrow, from = U, to = V]{}[]{\exists !} \ar[Rightarrow, from = X, to = Y, end anchor = {[yshift = -0.5ex, xshift = -1ex]south west}]{}[]{\varepsilon} & \tt{Ho}\mathcal{D}
                    \end{tikzcd}
                \end{center}

                Dually, whenever it exists, a total right derived functor of $F$ is a functor $\mathbb{R}F:\tt{Ho}\mathcal{C}\rightarrow \tt{Ho}\mathcal{D}$ with a natural transformation $\eta: L\circ F \Rightarrow \mathbb{R}F \circ L$ having the opposite universal property.
                \begin{center}
                    \begin{tikzcd}
                        \mathcal{C} \ar[]{r}[]{F} \ar[]{d}[]{L} & \mathcal{D} \ar[]{d}[]{L} \ar[Rightarrow]{ld}[]{\eta} \\
                        \tt{Ho}\mathcal{C} \ar[dashed]{r}[]{\mathbb{R}F} & \tt{Ho}\mathcal{D}
                    \end{tikzcd}\qquad
                    \begin{tikzcd}
                        \mathcal{C} \ar[]{r}[]{F} \ar[]{d}[]{L} & \mathcal{D} \ar[phantom, ""{name = Y}]{}[]{} \ar[]{d}[]{L} \\
                        \tt{Ho}\mathcal{C} \ar[bend left, ""{name = V, below}, ""{name=X, above}]{r}[description]{\mathbb{R}F} \ar[bend right, ""{name = U, above}]{r}[description]{G} \ar[Rightarrow, from = V, to = U]{}[left]{\exists !} \ar[Rightarrow, from = Y, to = X, start anchor = {[yshift = -0.5ex, xshift = -1ex]south west}]{}[left, yshift = 0.5ex]{\eta} & \tt{Ho}\mathcal{D}
                    \end{tikzcd}
                \end{center}
            \end{definition}

            \begin{definition}[Deformation]
                A left (right) deformation on a homotopical category $\mathcal{C}$ is an endofunctor $Q$ $(R)$ together with a natural weak equivalence $q: Q \Rightarrow Id_\mathcal{C}$ ($r: Id_\mathcal{C}\Rightarrow R$).

                A left (right) deformation on a functor $F:\mathcal{C}\rightarrow\mathcal{D}$ between homotopical categories is a left (right) deformation $Q$ on $\mathcal{C}$ such that $F$ preserves weak equivalences in the image of $Q$.
            \end{definition}

            \begin{remark}[Cofibrant and fibrant replacement]
                If $\mathcal{C}$ is a model category, then we have a left and a right deformation. The cofibrant replacement $Q$ defines a left deformation, and the fibrant replacement defines a right deformation. Notice that this is only because the factorization system is functorial.
            \end{remark}

            \begin{proposition}
                Let $F:\mathcal{C}\rightarrow\mathcal{D}$ be a functor between homotopical categories. If $F$ has a left deformation $Q$, then the total left derived functor $\mathbb{L}F$ exists. Moreover, the functor $FQ$ is homotopical, and $\mathbb{L}F$ is the unique extension of $FQ$.
            \end{proposition}

            \begin{proof}
                Since we already have a candidate for the derived functor, we must check that it has the universal property. This follows by \cite[Proposition 6.4.11][207]{Riehl16}.
            \end{proof}

            \begin{remark}
                There is a somewhat weaker statement by Dwyer and Spalinski \cite[Proposition 9.3][111]{Dwyer95}. If we instead ask for functors $F$, which have the cofibrant replacement $Q$ (fibrant replacement $R$) as a left (right) deformation, we may make this proof more explicit.
            \end{remark}

            With the above proposition and remark, it makes sense to define Quillen functors as left and right Quillen functors. A left Quillen functor should be left deformable by the cofibrant replacement. Moreover, for the composition of two left Quillen functors to make sense, we also need weak equivalences between cofibrant objects to be mapped to weak equivalences between cofibrant objects. We make the following definition.

            \begin{definition}[Quillen adjunction]
                Let $\mathcal{C}$ and $\mathcal{D}$ be model categories. \begin{enumerate}
                    \item A left Quillen functor is a functor $F:\mathcal{C}\rightarrow\mathcal{D}$ such that it preserves cofibrations and acyclic cofibrations.
                    \item A right Quillen functor is a functor $F:\mathcal{C}\rightarrow\mathcal{D}$ such that it preserves fibrations and acyclic fibrations.
                    \item Suppose that $(F,U)$ is an adjunction where $F:\mathcal{C}\rightarrow\mathcal{D}$ is left adjoint to $U$. $(F,U)$ is called a Quillen adjunction if $F$ is a left Quillen functor and $U$ is a right Quillen functor.
                \end{enumerate}
            \end{definition}

            \begin{remark}
                By Ken Brown's lemma, Lemma \ref{lem: Ken-Brown}, we see that a left Quillen functor $F$ is left deformable to the cofibrant replacement functor $Q$. Thus the total left derived functor is given by $\mathbb{L}F = \tt{Ho} FQ$.
            \end{remark}

            We will think of a morphism of model categories as a Quillen adjunction to eliminate the choice of left or right derivedness. We can choose the direction of the arrow to be along either the left or right adjoints, and we make the convention of following the left adjoint functors. We summarize the following properties.

            \begin{lemma}\label{lem: Quill-adj}
                Let $\mathcal{C}$ and $\mathcal{D}$ be model categories, and suppose there is an adjunction $F:\mathcal{C}\rightleftharpoons\mathcal{D}:U$. The following are equivalent:
                \begin{enumerate}
                    \item $(F,U)$ is a Quillen adjunction.
                    \item $F$ is a left Quillen functor.
                    \item $U$ is a right Quillen functor.
                \end{enumerate}
            \end{lemma}

            \begin{proof}
                This lemma follows from the naturality of the adjunction. I.e., any square in $\mathcal{C}$, with the right side from $\mathcal{D}$ is commutative if and only if any square in $\mathcal{D}$ with the left side from $\mathcal{C}$ is commutative. Now, $f$ has LLP with respect to $Ug$ if and only if $Ff$ has LLP with respect to $g$.
                \begin{center}
                    \begin{tikzcd}
                        A \ar[]{r}[]{k} \ar[]{d}[left]{f} & UX \ar[]{d}[]{Ug} \\
                        B \ar[]{r}[]{l} \ar[dotted]{ru}[]{h} & UY
                    \end{tikzcd} $\rightsquigarrow$
                    \begin{tikzcd}
                        FA \ar[]{r}[]{k^T} \ar[]{d}[left]{Ff} & X \ar[]{d}[]{g} \\
                        FB \ar[]{r}[]{l^T} \ar[dotted]{ru}[]{h^T} & Y
                    \end{tikzcd}
                \end{center} 
            \end{proof}

            \begin{remark}
                We say that $h^T$ is the transpose of $h$ along the unique natural isomorphism witnessing the adjunction between $F$ and $U$. With this notion, $(h^T)^T = h$.
            \end{remark}

            \begin{proposition}
                Suppose that $(F,U):\mathcal{C}\rightarrow\mathcal{D}$ is a Quillen adjunction. The functors $\mathbb{L}F:\tt{Ho}\mathcal{C}\rightarrow \tt{Ho}\mathcal{D}$ and $\mathbb{R}U: \tt{Ho}\mathcal{D}\rightarrow \tt{Ho}\mathcal{C}$ forms an adjoint pair.
            \end{proposition}

            \begin{proof}
                We must show that $\tt{Ho}\mathcal{D}(\mathbb{L}FX, Y) \simeq \tt{Ho}\mathcal{D}(X, \mathbb{R}UY)$. By using the fundamental theorem of model categories, Theorem \ref{thm: Fundamental-thm-model}, we have the following isomorphisms: $\tt{Ho}\mathcal{D}(\mathbb{L}FX,Y)\simeq \sfrac{\mathcal{C}(FQX,RY)}{\sim}$ and $\tt{Ho}\mathcal{D}(X,\mathbb{R}UY)\simeq\sfrac{\mathcal{D}(QX,URY)}{\sim}$. In other words, if we assume $X$ to be cofibrant and $Y$ to be fibrant, we must show that the adjunction preserves homotopy equivalences.

                We show it in one direction. Suppose that the morphisms $f,g: FA\rightarrow B$ are homotopic, witnessed by a right homotopy $H: FA\rightarrow B^I$. Since we assume $U$ to preserve products, fibrations, and weak equivalences between fibrant objects, $U(B^I)$ is a path object for $UB$. Thus the transpose $H^T:A\rightarrow U(B^I)$ is the desired homotopy witnessing $f^T \sim g^T$
            \end{proof}

            \begin{definition}[Quillen equivalence]
                Let $\mathcal{C}$ and $\mathcal{D}$ be model categories, and $(F,U):\mathcal{C}\rightarrow\mathcal{D}$ be a Quillen adjunction. $(F,U)$ is called a Quillen equivalence if for any cofibrant $X$ in $\mathcal{C}$, fibrant $Y$ in $\mathcal{D}$ such that any morphism $f:FX\rightarrow Y$ is a weak equivalence if and only if its transpose $f^T:X\rightarrow UY$ is a weak equivalence.
            \end{definition}

            \begin{proposition}\label{prop: Quill-Eq}
                Suppose that $(F,U):\mathcal{C}\rightarrow\mathcal{D}$ is a Quillen adjunction. The following are equivalent:
                \begin{enumerate}
                    \item $(F,U)$ is a Quillen equivalence.
                    \item Let $\eta :Id_\mathcal{C}\Rightarrow UF$ denote the unit, and $\varepsilon :FU\Rightarrow Id_\mathcal{D}$ denote the counit. The composite $Ur_{F} \circ \eta : Id_{\mathcal{C}_c} \Rightarrow URF|_{\mathcal{C}_c}$, and $\varepsilon \circ Fq_{U}:FQU|_{\mathcal{D}_f} \Rightarrow Id_{\mathcal{D}_f}$ are natural weak equivalences.
                    \item The derived adjunction $(\mathbb{L}F, \mathbb{R}U)$ is an equivalence of categories.
                \end{enumerate}
            \end{proposition}

            \begin{proof}
                Firstly observe that $2.\implies3.$ by definition. Secondly, observe that equivalences both preserves and reflect isomorphisms. From this, we get $3. \implies 1.$. We now show $1.\implies 2.$. Pick $X$ in $\mathcal{C}$ such that $X$ is cofibrant. Since $(F, U)$ is assumed to be a Quillen adjunction, $FX$ is still cofibrant. The fibrant replacement $r_{FX}: FX\rightarrow RFX$ gives us a weak equivalence. Furthermore, since $(F, U)$ is assumed to be a Quillen equivalence, its transpose $r_{FX}^T: X \rightarrow URFX$ is a weak equivalence. Unwinding the definition of the transpose, we get that $r_{FX}^T = Ur_{FX}\circ \eta_X$.

            \end{proof}

            We have the following refinement.

            \begin{corollary}\label{cor: Quill-Eq}
                Suppose that $(F,U):\mathcal{C}\rightarrow\mathcal{D}$ is a Quillen adjunction. The following are equivalent:
                \begin{enumerate}
                    \item $(F,U)$ is a Quillen equivalence.
                    \item $F$ reflects weak equivalences between cofibrant objects, and $\varepsilon \circ Fq_{U} : FQU|_{\mathcal{D}_f} \Rightarrow Id_{\mathcal{D}_f}$ is a natural weak equivalence.
                    \item $U$ reflects weak equivalences between fibrant objects, and $Ur_{F} \circ \eta : Id_{\mathcal{C}_c} \Rightarrow URF|{\mathcal{C}_c}$ is a natural weak equivalence.
                \end{enumerate}
            \end{corollary}

            \begin{proof}
                We start by showing $1. \implies 2.$ and $3.$. We already know that the derived unit and counit are isomorphisms in homotopy, so we only need to show that $F$ ($U$) reflects weak equivalences between cofibrant (fibrant) objects. Suppose that $Ff: FX\rightarrow FY$ is a weak equivalence between cofibrant objects. Since $F$ preserves weak equivalences between cofibrant objects, we get that $FQf$ is a weak equivalence; that $\mathbb{L}Ff$ is an isomorphism. By assumption, $\mathbb{L}F$ is an equivalence of categories, so $f$ is a weak equivalence as needed.

                We will show $2.\implies 1.$; the case $3.\implies 1.$ is dual. We assume that the counit map is an isomorphism in homotopy. By assumption, the derived unit $\mathbb{L}\eta$ is split-mono on the image of $\mathbb{L}F$. Moreover, the derived counit $\mathbb{R}\varepsilon$ is assumed to be an isomorphism. In particular, the derived unit $\mathbb{L}F\mathbb{L}\eta$ is an isomorphism. Unpacking this, we have a morphism, which we call $\eta_X': FQX \rightarrow FQURFQX$, which is a weak equivalence. Since $F$ and $Q$ reflect weak equivalences, we get that $\eta_X: X \rightarrow URFQX$ is a weak equivalence.
            \end{proof}

    \section{Model structures on Algebraic Categories}

            To understand $\infty$-quasi-isomorphism of strongly homotopy associative algebras, we will study different homotopy theories of various categories. Munkholm \cite{Munkholm78} successfully showed that the derived category of augmented algebras is equivalent to the derived category of augmented algebras equipped with $\infty$-morphisms. To be more precise, he showed that certain subcategories of augmented algebras had this property. Lefevre-Hasagawas Ph.D. thesis \cite{LefevreHasegawa03} builds upon this identification, but with the help of further development within the field. We will follow the approach of Lefevre-Hasegawa, by comparing the model structure for algebras and coalgebras,

        \subsection{DG-Algebras as a Model Category}

            Bousfield and Guggenheim \cite{Bousfield76} proved that the category of commutative dg-algebras had a model structure whenever the base field was a field of characteristic $0$. In a joint project, Jardine's paper from 1997 \cite{Jardine97} shows that this construction may be extended to dg-algebras over any commutative ring. On the other hand, Munkholm expanded on the ideas from Bousfield and Guggenheim to get an identification of derived categories. Also, Hinich's paper from 1997 \cite{Hinich97} details another method to obtain the model category we want. We will follow the approach of Hinich, as it will be helpful later on. Notice that where Hinich uses the theory of algebraic operads to show that the category of algebras is a model category, we will give a more explicit formulation.

            Let $\mathbb{K}$ be a field, and $\mathcal{C}$ be a category such that there is an adjunction $F : \tt{Ch}(\mathbb{K}) \rightleftharpoons \mathcal{C} : \#$, where $F$ is left adjoint to $\#$. Furthermore, suppose that $\mathcal{C}$ satisfies the $2$ conditions:
            \begin{itemize}
                \item[(H0)] $\mathcal{C}$ admits finite limits and every small colimit, and the functor $\#$ commutes with filtered colimits;
                \item[(H1)] For $M$ as the complex below, concentrated in $0$ and $1$,
                \begin{center}
                    \begin{tikzcd}
                        ... \ar[]{r}[]{} & 0 \ar[]{r}[]{} & \mathbb{K} \ar[]{r}[]{id} & \mathbb{K} \ar[]{r}[]{} & 0 \ar[]{r}[]{} & ...
                    \end{tikzcd}
                \end{center}
                we have that for any $d\in \mathbb{Z}$ and for any $A\in\mathcal{C}$, the injection $A \rightarrow A \coprod F(M[d])$ induces a quasi-isomorphism $A^\# \rightarrow (A\coprod F(M[d]))^\#$.
            \end{itemize}
                
            With this adjunction in mind, we define weak equivalences, fibrations, and cofibrations as follows:
            Let $f\in \mathcal{C}$ be a morphism
            \begin{itemize}
                \item $f\in \tt{Ac}$ if $f^\#$ is a quasi-isomorphism.
                \item $f\in \tt{Fib}$ if $f^\#$ is surjective on each component.
                \item $f\in \tt{Cof}$ if $f$ has LLP to acyclic fibrations.
            \end{itemize}

            \begin{thm}\label{thm: model-str-alg}
                The category $\mathcal{C}$ equipped with the weak equivalences, fibrations, and cofibrations as defined above is a model category.
            \end{thm}

            Before we show this theorem, we need to understand the cofibrations better. Let $A\in\mathcal{C}$, $M\in \tt{Ch}(\mathbb{K})$ and $\alpha : M \rightarrow A^\#$ a morphism in $\tt{Ch}(\mathbb{K})$. We define a functor 
            \begin{align*}
                h_{A,\alpha}(B) = \startset{(f,t)\mid f\in \mathcal{C}(A,B), t\in \tt{Hom}^{-1}_\mathbb{K}(M, B^\#) \tt{ s.t. } \partial t = f^\#\circ\alpha}.
            \end{align*}
            Note that $t$ is not a chain map. It is a homogenous morphism of degree $-1$. The differential then promotes this morphism to a chain map, and $t$ is thus a homotopy for the composite $f^\#\circ\alpha$.

            This functor is represented by an object of $\mathcal{C}$. We define this representing object $A\langle M, \alpha\rangle$ as the pushout:
            \begin{center}
                \begin{tikzcd}
                    F(A^\#) \ar[]{r}[]{\varepsilon_A} \ar[]{d}[]{} \ar[phantom]{rd}[near end]{\lrcorner} & A \ar[]{d}[]{a} \\
                    F(\tt{cone}(\alpha)) \ar[]{r}[]{e} & A\langle M,\alpha\rangle
                \end{tikzcd}
            \end{center}
            Let $i: M[1] \rightarrow \tt{cone}(\alpha)$ be a homogenous morphism which is the injection when considered as graded modules. Notice that we have a pair of morphisms $(a, e^Ti)\in h_{A,\alpha}(A\langle M,\alpha\rangle)$.
                
            \begin{proposition}\label{prop: universal-h}
                The functor $h_{A,\alpha}$ is represented by $A\langle M,\alpha\rangle$, i.e. $h_{A,\alpha}\simeq \mathcal{C}(A\langle M,\alpha\rangle,\argument)$ is a natural isomorphism. Moreover, the pair $(a, e^Ti)$ is the universal element of the functor $h_{A,\alpha}$, i.e., the natural isomorphism is induced by this element under Yoneda's lemma.
            \end{proposition}

            \begin{proof}
                Let $(f,t)\in h_{A,\alpha}(B)$ for some $B\in\mathcal{C}$. The condition that $\partial t = f^\#\alpha$ is equivalent to say that $f^\#$ extends to a morphism $f' : \tt{cone}(\alpha) \rightarrow B^\#$ along $t$, i.e. there is a vector of morphisms $f' = \begin{pmatrix}f^\# & t\end{pmatrix}$. This construction concludes the isomorphism part, as an element $(f,t)$ is equivalent to the diagram below, where $\widetilde{f}$ is uniquely determined.
                \begin{center}
                    \begin{tikzcd}
                        F(A^\#) \ar[]{r}[]{\varepsilon_A} \ar[]{d}[]{} & A \ar[]{d}[]{a} \ar[bend left]{ddr}[]{f} \\
                        F(\tt{cone}(\alpha)) \ar[bend right]{rrd}[]{f'^T} \ar[]{r}[]{e} & A\langle M,\alpha\rangle \ar[dashed]{rd}[]{\widetilde{f}} \\
                        & & B
                    \end{tikzcd}
                \end{center}
  
                We use the adjunction to observe that the element $(a, e^Ti)$ is universal to obtain naturality.
            \end{proof}

            We are now in a position to find some crucial cofibrations. We collect these morphisms into the "standard" cofibrations.

            \begin{definition}
                Let $f:A\rightarrow B$ be a morphism in $\mathcal{C}$. Suppose that $f$ factors as a transfinite composition of morphisms on the form $A_i \rightarrow A_i\langle M_i,\alpha_i\rangle$, i.e. $f$ factors into the diagram below, where $A_{i+1} = A_i\langle M_i,\alpha_i\rangle$.
                \begin{center}
                    \begin{tikzcd}
                        A \ar[]{r}[]{} & A_1 \ar[]{r}[]{} & A_2 \ar[]{r}[]{} & ... \ar[]{r}[]{} & B
                    \end{tikzcd}
                \end{center}
                \begin{itemize}
                    \item If every such $M_i$ is a complex consisting of free $\mathbb{K}$-modules and has a $0$-differential, we call $f$ a standard cofibration.
                    \item If every such $M_i$ is a contractible complex and $\alpha = 0$, we call $f$ a standard acyclic cofibration.
                \end{itemize}
            \end{definition}

            \begin{proposition}
                Every standard cofibration is a cofibration, and every standard acyclic cofibration is an acyclic cofibration.
            \end{proposition}

            \begin{remark}
                In some sense, we will see that these morphisms generate every (acyclic) cofibration.
            \end{remark}

            \begin{proof}
                Observe that every standard cofibration may be made iteratively from the chain complexes $\mathbb{K}[n]$, and likewise, every standard acyclic cofibration may be made iteratively from $M$ as in $(H1)$.

                We first prove that if $M \simeq \mathbb{K}[n]$, and $\alpha: M \rightarrow A^\#$ is any map, then the map $A \rightarrow A\langle M,\alpha\rangle$ is a cofibration; this amounts to show that it has LLP to every acyclic fibration. Suppose that $h: B \rightarrow C$ is an acyclic fibration and that there is a commutative square as below.
                \begin{center}
                    \begin{tikzcd}
                        A \ar[]{r}[]{f} \ar[]{d}[]{a} & B \ar[]{d}[]{h} \\
                        A\langle M,\alpha\rangle \ar[]{r}[]{g} & C
                    \end{tikzcd}
                \end{center}
                
                By the universal property of $h_{A,\alpha}$, Proposition \ref{prop: universal-h}, it suffices to find a pair $(f,t')\in h_{A,\alpha}$ which makes the lower triangle commute. That is, $t' : M \rightarrow B^\#$ is homogenous of degree $-1$, such that $\partial t' = f^\#\alpha$, and post composing $h$ with the morphism determined by $(f,t')$ is $g$. By the existence of $g$, there exists a $t : M \rightarrow C^\#$ such that $\partial t = g^\#a^\#\alpha = h^\#f^\#\alpha$. Since $h$ is an acyclic fibration, $h^\#$ is a surjective quasi-isomorphism. We assumed $M \simeq \mathbb{K}[n]$, so we can consider the morphism $t$ as an element of $(C^{\#})^{n-1}$. By surjectivity of $h^\#$ there is an element $u$ of $(B^{\#})^{n-1}$ such that $h^\#(u) = t$. Moreover, the difference $h^\#(\partial u - f^\#\alpha) = 0$, so $\partial u - f^\#\alpha$ factors through the kernel $\tt{Ker} h^\#$, which is assumed to be acyclic. This element is furthermore a cycle, so by acyclicity, there is another element $u'$ such that $\partial u' = \partial u - f^\#\alpha$. We may now see that $(f, u - u')$ is our desired factorization.
                
                Secondly, we see that it is enough to prove that if $M$ is as in (H1) and $\alpha = 0$, then the map $A \rightarrow A\langle M,\alpha\rangle$ is an acyclic cofibration. By (H1), we know that the map is already a weak equivalence, so we show that it has LLP to every acyclic fibration.
                
                Suppose that $h: B \rightarrow C$ is an acyclic fibration and that there is a commutative square as below.
                \begin{center}
                    \begin{tikzcd}
                        A \ar[]{r}[]{f} \ar[]{d}[]{a} & B \ar[]{d}[]{h} \\
                        A\langle M,\alpha\rangle \ar[]{r}[]{g} & C
                    \end{tikzcd}
                \end{center}

                We will again use \ref{prop: universal-h}, so it suffices to find a $t'$ such that $\partial t' = f^\#\alpha = 0$. By the existence of $g$, there is a $t : M \rightarrow C^\#$ such that $\partial t = g^\#a^\#\alpha = h^\# f^\# \alpha = 0$. Since $h^\#$ is surjective $t$ admits a linear homogenous lift $u : M \rightarrow B^\#$ such that $t = h^\#u$. We see that the map $\partial u$ factors through the kernel of $h^\#$ as $h^\# \partial u = \partial h^\# u = \partial t = 0$. As $\partial u = 0$ is a cycle of $\tt{Ker}h^\#$, there is a $u'$ such that $\partial u' = \partial u$. The result follows by picking $t' = u - u'$.

            \end{proof}

            Given the above proposition, we would like to make some more convenient notation. If $M\simeq \mathbb{K}[n]$ and $\alpha: M \rightarrow Z^n(A^\#)$, s.t. $\alpha(1) = a$, we write $A\langle M,\alpha\rangle$ as $A\langle T; dT = a\rangle$ instead. Hinich calls this "adding a variable to kill a cycle." If $M$ is the contractible complex as below and $\alpha = 0$, we write $A\langle T, S; dT = S\rangle$ for $A\langle M; dT = 0 \rangle$. This construction can be thought of as "adding a variable and a cycle to kill itself."

            \begin{center}
                \begin{tikzcd}
                    ... \ar[]{r}[]{} & 0 \ar[]{r}[]{} & \mathbb{K} \ar[]{r}[]{id} & \mathbb{K} \ar[]{r}[]{} & 0 \ar[]{r}[]{} & ...
                \end{tikzcd}
            \end{center}

            \begin{proof}[proof of Theorem \ref{thm: model-str-alg}]
                \textbf{MC1} and \textbf{MC2} are satisfied. By definition, we also have the first part of \textbf{MC3}. We start by checking \textbf{MC4}.

                Let $f:A\rightarrow B$ be a morphism in $\mathcal{C}$. Given any $b\in B^\#$, let $C_b = A\langle T_b,S_b; dT_b = S_b\rangle$. We define $g_b: C_b \rightarrow B$ by the conditions that it acts on $A$ as $f$, $g_b^\#(T_b) = b$ and $g_b^\#(S_b)=db$. By adding a "variable to kill a cycle" for every $b\in B$, we obtain an object $C$, such that the injection $A \rightarrow C$ is an acyclic standard cofibration, and the map $g: C \rightarrow B$ is a fibration. This is the desired factorization $f = f_\delta\circ f_\gamma$, where $f_\gamma$ is the injection and $f_\delta = g$.

                To obtain the other factorization, we want to make a standard cofibration. We already know that the map $A\rightarrow C$ is a standard cofibration, so let $C_0 = C$. From here on, we will make each $C_i$ inductively, such that $\varinjlim C_i$ has the factorization property we desire. Notice that from $C_0$, there is a morphism $g_0: C_0 \rightarrow B$, which is surjective and surjective on every kernel. This morphism may fail to be a quasi-isomorphism, so it is not an acyclic fibration.
                
                To construct $C_1$ we assign to every pair of elements $(c,b)$, such that $c\in ZC_0^\#$ and $g_0^\#(c) = db$, a variable to kill a cycle. If $(c,b)$ is such a pair, then we add a variable $T$ such that $dT = c$ and $g_1^\#(T)=b$. $C_1$ is then the complex where each cycle $c$ has been killed by adding a variable $T$. Now, if we suppose that we have constructed $C_i$, then $C_{i+1}$ is constructed similarly by adding a variable to kill each cycle which is a boundary in the image.

                When adding a variable, we have also updated the morphism $g_i$ by letting $g_{i+1}^\#(T) = b$. Thus in each step, we have also made a new morphism $g_{i+1}$. If $g$ denotes the morphism at the colimit, it is clear that it is still a fibration and has also become a quasi-isomorphism. We can see this as every cycle which have failed to be in the homology of $B$ has been killed. 
                
                It remains to check the last part of \textbf{MC3}. Suppose that $f: A\rightarrow B$ is an acyclic cofibration. By \textbf{MC4}, we know that it factors as $f = f_\delta \circ f_\gamma$, where $f_\delta$ is an acyclic fibration, and $f_\gamma$ is a standard acyclic fibration. We thus obtain that $f$ is a retract of $f_\gamma$ by the commutative diagram below.
                \begin{center}
                    \begin{tikzcd}
                        A \ar[]{r}[]{f_\gamma} \ar[]{d}[]{f} & C \ar[]{d}[]{f_\delta} \\
                        B \ar[equal]{r}[]{} \ar[dashed]{ru}[]{} & B
                    \end{tikzcd}
                \end{center}
            \end{proof}

            The following corollary will concretize what it means that the standard cofibrations generate every cofibration. This corollary is an emphasis on the last diagram in the previous proof.

            \begin{corollary}
                Any (acyclic) cofibration is a retract of a standard (acyclic) cofibration.
            \end{corollary}

            We may immediately apply this theorem to some familiar examples.

            \begin{corollary}\label{cor: Model-Mod}
                Let $A$ be a dg-algebra over the field $\mathbb{K}$. The category $\tt{Mod}_A$ of left modules is a model category.
            \end{corollary}

            \begin{proof}[sketch of proof]
                We establish the adjunction by letting $FM = A\otimes_\mathbb{K}\argument$. H0 is satisfied as this category is bicomplete, and we can think of filtered colimits as unions of sets. Moreover, since $\tt{Mod}_A$ is an Abelian category, the forgetful functor $\#$ commutes with coproducts, or direct sums, which makes H1 trivially satisfied.
            \end{proof}

            \begin{corollary}
                The categories $\tt{Alg}^\bullet_\mathbb{K}$ ($\tt{Alg}^\bullet_{\mathbb{K},+}$) are model categories.
            \end{corollary}

            \begin{proof}
                We establish the adjunction by letting $F = T(M)$, the tensor algebra of a cochain complex. For the same reasons as above, H0 is trivially satisfied. 

                Given a cochain complex $N^\bullet$, we may consider the free dg-algebra $T(N^\bullet)$. In this case, the coproduct $A\ast T(N^\bullet)$ has an easier description. We define a complex
                \begin{align*}
                    A[N^\bullet] = A \oplus (A \otimes N^\bullet \otimes A) \oplus (A \otimes N^\bullet \otimes A \otimes N^\bullet \otimes A) \oplus \cdots\tt{.}
                \end{align*}
                The differential on $A[N^\bullet]$ is the differential induced by the tensor product. We define a multiplication on $A[N^\bullet]$ by the following formula
                \begin{align*}
                    (a_1\otimes\cdots\otimes a_i)\cdot(a_1'\otimes\cdots\otimes a_j') = a_1\otimes\cdots\otimes a_ia_1'\otimes\cdots a_j'\tt{.}
                \end{align*}

                Let $i : A \rightarrow A[N^\bullet]$ denote the inclusion, and $\iota : T(N^\bullet) \rightarrow A[N^\bullet]$ is defined by interspersing the $N^\bullet$ tensors with $1$s. I.e. $\iota(n_1\otimes\cdots\otimes n_j) = 1 \otimes n_1 \otimes 1 \otimes \cdots \otimes 1 \otimes n_j \otimes 1$.

                To define a map $f: A[N^\bullet] \rightarrow T$ it is enough by the ring homomorphism property to define a map $g: A \rightarrow T$ and a map $h: T(N^\bullet) \rightarrow T$. This choice of $g$ and $h$ is unique for any $f$, establishing the universal property. I.e. $A[N^\bullet] \simeq A\ast T(N^\bullet)$.
                
                To see that the map $i^\#: A^\# \rightarrow A[M^\bullet]^\#$ is a quasi-isomorphism, it is enough to see that contractible complexes are stable under tensoring. Given any contractible complex $C^\bullet$, there is a homotopy $h: C^\bullet \rightarrow C^\bullet$ such that $\partial h = id_C$. Observe that $id_N\otimes h : N^\bullet\otimes C^\bullet \rightarrow N^\bullet\otimes C^\bullet$ is a homotopy witnessing $id_{N^\bullet\otimes C^\bullet} \sim 0$. Since $M$ is acyclic, we know that the homology of the inclusion is $H^*i = id_{H^*A}$, which shows H1.
            \end{proof}

            We summarize the last result:

            The category of augmented dg-algebras $\tt{Alg}^\bullet_{\mathbb{K},+}$ is a model category. Let $f: X\rightarrow Y$ be a homomorphism of augmented algebras. 
            \begin{itemize}
                \item $f\in \tt{Ac}$ if $f^\#$ is a quasi-isomorphism.
                \item $f\in \tt{Fib}$ if $f^\#$ is an epimorphism (surjective onto every component).
                \item $f\in \tt{Cof}$ if $f$ has LLP with respect to to every acyclic fibration.
            \end{itemize}
            
            The category of augmented dg-algebras has a zero object, and this is the stalk of $\mathbb{K}$. We see that every object is fibrant, as the forgetful functor preserves the augmentation map and, by definition, is a split-epimorphism.

            \begin{remark}
                In the process of showing that $\tt{Alg}_{\mathbb{K},+}$ is a model category, we have not cared about functorial factorization. One may see that we get this from the constructions used to prove \textbf{MC4}. This is a technical detail which we do not need to care too much about.
            \end{remark}
            
    \subsection{A Model Structure on DG-Coalgebras}

            We now want to equip the category of dg-coalgebras with a suitable model structure. This model structure should be suitable in the sense that conilpotent dg-coalgebras will have the same homotopy theory as dg-algebras. The bar-cobar construction will be crucial in this construction, as it is a Quillen adjunction. To this end, we will follow the setup as presented by Lefevre-Hasegawa \cite{LefevreHasegawa03}. His method modifies Hinich's paper \cite{Hinich01}.

            Let $f: C \rightarrow D$ be a morphism of coalgebras, the category of dg-coalgebras will be equipped with the three following classes of morphisms:
            \begin{itemize}
                \item $f\in \tt{Ac}$ if $\Omega f$ is a quasi-isomorphism.
                \item $f\in \tt{Fib}$ if $f$ has RLP with respect to every acyclic cofibration.
                \item $f\in \tt{Cof}$ if $f^\#$ is a monomorphism (injective in every component).
            \end{itemize}

            To see that these classes of morphisms do indeed define a model structure, we will get a better description of a subclass of weak equivalences. We can only check if a morphism is a weak equivalence by calculating homologies since $f$ is a weak equivalence if and only if $H^*\tt{cone}(\Omega f) \simeq 0$. Using spectral sequences to calculate these homologies is not crucial, but it gives us a method to handle the problems we will face.

            \begin{definition}
                A filtered chain map $f: M \rightarrow N$ of filtered complexes $M$ and $N$ is a graded quasi-isomorphism if $\tt{gr}f: \tt{gr}M \rightarrow \tt{gr}N$ is a quasi-isomorphism of the associated graded complexes.
            \end{definition}

            \begin{lemma}\label{lem: graded-qif-are-w}
                Let $f: C\rightarrow C'$ be a graded quasi-isomorphism between conilpotent dg-coalgebras, then $\Omega f: \Omega C \rightarrow \Omega C'$ is a quasi-isomorphism. 
            \end{lemma}

            \begin{proof}
                We do this by considering a spectral sequence. Endow $C$ with a grading (as a vector space) induced by the coradical filtration, i.e., $c\in C$ has degree $|c|=n$ if $n$ is the smallest number such that $\overline{\Delta}^nc = 0$. We define a filtration on $\Omega C$ by
                \begin{align*}
                    F_p\Omega C = \startset{\langle c_1 | \cdots | c_n \rangle \mid |c_1|+...+|c_n|\leq p}
                \end{align*}

                Since $C$ is a dg-coalgebra, the coradical filtration respects the differential. In other words, $F_p\Omega C$ is still a cochain complex, a subcomplex of $\Omega C$. This filtration is bounded below and exhaustive. Thus by the classical convergence theorem of spectral sequences, Theorem \ref{thm: class-conv}, the spectral sequence converges to the homology $E\Omega C \Rightarrow H^*\Omega C$.

                By definition, the $0$'th page is 
                \begin{align*}
                    E^0_{p,q}\Omega C = \sfrac{(F_p\Omega C)_{p+q}}{(F_{p-1}\Omega C)_{p+q}}\tt{.}
                \end{align*}
                Furthermore, notice that on this page we have the following isomorphism $E^0_{p,q}\Omega C \simeq (\Omega \tt{gr}C)^{(p)}_{p+q}$, where $(\Omega \tt{gr}C)^{(p)} = \startset{\langle c_1 | \cdots | c_n \rangle \mid |c_1|+...+|c_n| = p}$.

                Evaluating $f$ at the $0$'th page would look like $E^0\Omega f \simeq \Omega \tt{gr}f$. By the comparison theorem, Theorem \ref{thm: comp-thm}, it is enough to check that $\Omega \tt{gr}f$ is a quasi-isomorphism to see that $\Omega f$ is a quasi-isomorphism. We show that $\Omega \tt{gr}f$ is a quasi-isomorphism by inspecting every cochain complex $E^0_{p,\bullet}\Omega C$.

                Define a filtration $G_k$ on $E^0_{p,\bullet}\Omega C$ as
                \begin{align*}
                    G_k = \startset{\langle c_1 | \cdots | c_n \rangle \mid n \geq -k}.
                \end{align*}
                We see that $G_0 = E^0_{p, \bullet}\Omega C$ by definition and $G_{-p-1} \simeq 0$ on the coaugmentation quotient $\overline{C}$. The classical convergence theorem of spectral sequences defines a spectral sequence such that $EG \Rightarrow H^*E^0_{p, \bullet}\Omega C$.

                To see that $\Omega \tt{gr}f$ is a quasi-isomorphism, we will show that $E^0Gf$ is a quasi-isomorphism for any $p$. Notice that $E^0_{l,\bullet}G \subseteq (\tt{gr}C[-1])^{\otimes l}$ where the total grading is $p$. Since $f$ is a graded quasi-isomorphism, it follows by the K\"unneth-formula \cite[Theorem 3.6.3][88]{Weibel94} that $E^0Gf$ is a quasi-isomorphism.
            \end{proof}

            This proof will serve as a template for how we approach many of the proofs we encounter. With the lemma, to show that $f$ is a weak equivalence, it suffices to show that $f$ is a graded quasi-isomorphism. However, to show that $f$ is a graded quasi-isomorphism, we first need a good filtering, and once we have a filtering, we look at its spectral sequence. The mapping lemma says that it is enough to verify that a morphism becomes a quasi-isomorphism on any page to see that it is a quasi-isomorphism. We proceed then to calculate a page where we can assert that $f$ becomes a quasi-isomorphism. If there still are problems with calculations, we look at complexes within a page on a spectral sequence and define new filtrations on these complexes to calculate the next page. We will informally call this technique for an iterated spectral sequence argument.

            For completeness, we include the following statement.

            \begin{lemma}
                Let $f: A \rightarrow A'$ be a quasi-isomorphism between dg-algebras, then $Bf: BA \rightarrow BA'$ is a graded quasi-isomorphism.
            \end{lemma}

            \begin{proof}
                Notice that the homology of $BA$ may be calculated from the double complex used to define $BA$. In fact, at the $0$'th page of the canonical spectral sequence, we have $E^0_{p, \bullet}f \simeq f^{\otimes p}$. It follows that $f$ is a quasi-isomorphism on the $0$'th page from the K\"unneth formula, \cite[Theorem 3.6.3][88]{Weibel94}.            
            \end{proof}

            Let $A$ ($C$) be a filtered dg-algebra (coalgebra). Given an element $a\in A$ ($c\in C$) we say that its filtered degree $\tt{f-deg}(a)$ ($\tt{f-deg}(c)$) is the smallest number such that $a\in F_{\tt{f-deg}(a)}A$ ($c\in F_{\tt{f-deg}(c)}C$) but not $a\in F_{\tt{f-deg}(a)-1}A$ ($c\in F_{\tt{f-deg}(c)-1}C$). There is then an associated filtration on the bar (cobar) construction of this complex, defined as 
            \begin{align*}
                F_pBA & = \startset{[ a_1 \mid \cdots \mid a_n ] \mid \sum \tt{f-deg}(a_i)\leq p} \\
                (F_p\Omega C & = \startset{\langle c_1 \mid \cdots \mid c_n \rangle \mid \sum \tt{f-deg}(c_i) \leq p})\tt{.}
            \end{align*}
            We will call this the induced filtration on the bar or cobar construction.

            \begin{proposition}\label{prop: unit-counit-qif}
                Let $A$ be an augmented dg-algebra and $C$ a conilpotent dg-coalgebra. The counit $\varepsilon_A: \Omega BA \rightarrow A$ is a quasi-isomorphism. The unit $\eta_C: C \rightarrow B\Omega C$ is a graded quasi-isomorphism. Moreover, $\Omega \eta_C$ is a quasi-isomorphism.
            \end{proposition}

            The following proof is due to \cite{LefevreHasegawa03}, but with corrections given by \cite{Keller05}. Some minor modifications are given to the proof as it resembles a previous proof, using the method of iterated spectral sequences.

            \begin{proof}
               We start by showing that the counit is a quasi-isomorphism. Define the following filtration for $A$.
                \begin{align*}
                    F_0A & = \mathbb{K} \\
                    F_1A & = A \\
                    F_pA & = F_1A 
                \end{align*}
                We see that this filtration endows $A$ with the structure of a filtered dg-algebra. For $\Omega BA$, we will use the induced filtration from the coradical filtration of $BA$.

                The counit acts on $\Omega BA$ as tensor-wise projection, followed by multiplication in $A$. This morphism respects the filtration, so it is a filtered morphism. Notice that both filtrations are bounded below and exhaustive, so the classical convergence theorem of spectral sequences applies.

                Let $E_r\Omega BA$ and $E_rA$ be the spectral sequences given by these filtrations. We have that $E_1^{p}\Omega BA \simeq \tt{gr}_p \Omega BA$ and $E_1^{p}A \simeq \tt{gr}_pA$. For $p=1$, both complexes are isomorphic to the same complex, $\overline{A}$. Moreover, $E_1^{1}\varepsilon_A = id_{\overline{A}}$. Whenever $p\neq 1$, we get that $E_1^p A \simeq 0$, so it remains to show that $E_1^p \Omega BA \simeq \tt{gr}_p \Omega BA$ is acyclic for any $p \geq 2$.

                Three actions generate the differential of $\Omega BA$: the differential on $A$, the multiplication on $A$, and the comultiplication on $BA$. With the induced filtration on $\Omega BA$, we see that the multiplication on $A$ is the only action that maps $F_p\Omega BA \rightarrow F_{p-1}\Omega BA$. Thus this action is $0$ in the associated graded and the spectral sequence. 

                There is a homotopy of the identity given as $r: \tt{gr}_i \Omega BA \rightarrow \tt{gr}_i \Omega BA$, which is $0$ except if there is an element on the form $\langle [ a ] \mid [ \cdots ] \mid [ \cdots ] \rangle$. In this case, $r$ is 
                \begin{align*}                    
                    r\langle [a] \mid [\cdots] \mid \cdots \rangle = (-1)^{|a|+1} \langle [a \mid [\cdots] \mid \cdots \rangle
                \end{align*}
                We will show that this is a homotopy by induction on $i$.

                Let $i=2$. Then there are two cases we must handle, either an element is on the form $\langle [a_1] \mid [a_2] \rangle$ or $\langle [a_1 \mid a_2] \rangle$. We consider the latter case first. If we apply $r$ to this element, we are returned $0$. 
                \begin{align*}
                    (r\circ d_{\Omega BA}+d_{\Omega BA}\circ r)\langle [a_1 \mid a_2] \rangle = r (-1)^{|a_1|+1}\langle [a_1] \mid [a_2] \rangle = \langle [a_1 \mid a_2] \rangle
                \end{align*}
                Then we treat the former case
                \begin{multline*}
                    (r\circ d_{\Omega BA}+d_{\Omega BA}\circ r)\langle [a_1] \mid [a_2] \rangle \\ 
                    = r\langle [d_Aa_1] \mid [a_2] \rangle + (-1)^{|a_1|}r\langle [a_1] \mid [d_Aa_2] \rangle + d_{\Omega BA}(-1)^{|a_1|+1}\langle [a_1 \mid a_2]\rangle \\
                    = (-1)^{|a_1|}\langle [ d_Aa_1 \mid a_2] \rangle - \langle [a_1 \mid d_Aa_2] \rangle + \langle [a_1] \mid [a_2] \rangle \\
                    + (-1)^{|a|+1}\langle [d_Aa_1 \mid a_2] \rangle  + \langle [a_1 \mid d_Aa_2] \rangle = \langle [a_1] \mid [a_2] \rangle\tt{.}
                \end{multline*}
                This homotopy makes $id_{\tt{gr}_2 \Omega BA}$ null-homotopic.

                To extend this argument by induction, we will observe that the terms where the differential is applied will have opposite signs, such that they cancel. The result follows for any $i$ since the tensors far enough out to the right are not affected by $r$.

                If $C$ is a dg-coalgebra, we use the same technique as in Lemma~\ref{lem: graded-qif-are-w}. Consider the filtration on $B\Omega C$ given as
                \begin{align*}
                    F_pB\Omega C = \startset{[\langle sc_{1,1}\mid \cdots \mid sc_{1,n_1}\rangle \mid \cdots \mid \langle sc_{m, 1}\mid \cdots \mid sc_{m,n_m} \rangle] \mid |c_{1,1}| + \cdots + |c_{m,n_m}|\leq p}\text{.}
                \end{align*}
                
                This filtration is bounded below and exhaustive, so the classical convergence theorem says that the associated spectral sequence converges. We denote this sequence as $EF$, and then $EF \implies H^*B\Omega C$. Let $EC$ be the spectral sequence associated to $C$. Since $C$ is conilpotent, $EC \implies H^*C$. The unit $\eta_C: C \rightarrow B\Omega C$ is now a map acting on $EC^0$ as the identity, sending each element in $EC^0_{p,q}$ to itself in $EF^0_{p,q}$.
                
                On each row $EF^0_{p,\bullet}$, we make another filtration called $G$.
                \begin{align*}
                    G_kEF^0_{p,\bullet} = \startset{[\langle ... \rangle_1 \mid ... \mid \langle ... \rangle_n] \mid n \geq -k}
                \end{align*}

                Similarly, as in Lemma~\ref{lem: graded-qif-are-w}, this filtration is bounded below and exhaustive, so we may again apply the classical convergence theorem to obtain a spectral sequence $E_pG$ such that $E_pG \implies H^*EF^0_{p,\bullet} \simeq EF^1_{p,\bullet}$. Since the unit acts as the identity on $EC^0$, it descends to a morphism $\tt{gr}_pC \rightarrow E_pG^0_{k,\bullet}$ which is the identity when $k = -1$ and $0$ otherwise. Notice that this morphism does not hit every string of length $\geq 2$. However, by employing $r$ as above, we may show that these summands are acyclic. The unit is thus an isomorphism in homology.

                % To show that this map is a quasi-isomorphism, it is sufficient to show that $E_pG^0_{k, \bullet}$ is acyclic for $k \neq 1$; this looks like the same situation as for algebras. We use the same homotopy $r$ as in the first part to get a null homotopy of the identity.
            \end{proof}

            \begin{lemma}\label{lem: bar-cobar-Quill-adj}
                Let $f: C\rightarrow D$ be a morphism of dg-coalgebras, then:
                \begin{itemize}
                    \item if $f$ is a cofibration, then $\Omega f$ is a standard cofibration.
                    \item if $f$ is a weak equivalence, then $\Omega f$ is as well.
                \end{itemize}

                Almost dually, let $f: A\rightarrow B$ be a morphism of dg-algebras, then:
                \begin{itemize}
                    \item if $f$ is a fibration, then $B f$ is a fibration.
                    \item if $f$ is a weak equivalence, then $B f$ is as well.
                \end{itemize}
            \end{lemma}

            \begin{proof}
                First, suppose that $f: C\rightarrow D$ is a cofibration. We define a filtration on $D$ as the sum of the image of $f$ and the coradical filtration on $D$: $D_i = Imf + Fr_iD$. $f$ being a cofibration ensures us that $D_0 \simeq C$. Since $D$ is conilpotent, we know that $D \simeq \varinjlim D_i$, and since $\Omega$ commutes with colimits there is a sequence of algebras $\Omega C \rightarrow \Omega D_1 \rightarrow ... \rightarrow \Omega D$. It is enough to show that each morphism $\Omega D_i \rightarrow \Omega D_{i+1}$ is a standard cofibration. The quotient coalgebra $\sfrac{D_{i+1}}{D_i}$ only has a trivial comultiplication. Thus every element is primitive, and this means that as a cochain complex, $D_{i+1}$ is constructed from $D_i$ by attaching possibly very many copies of $\mathbb{K}$. We treat the case when there is only one such $\mathbb{K}$, here $D_{i+1} \simeq D_i \oplus \mathbb{K}\startset{x}$ where $dx = y$ for some $y\in D_i$, which is exactly the condition for the morphism $\Omega D_i \rightarrow \Omega D_{i+1}$ to be a standard cofibration.

                If $f$ is a weak equivalence, then $\Omega f$ is a quasi-isomorphism.

                By Lemma~\ref{lem: Quill-adj}, or adjointness, more specifically, the property that $B$ preserves fibrations is a consequence of $\Omega$ preserving cofibrations.

                It remains to show that if $f: A\rightarrow B$ is a quasi-isomorphism, then $Bf$ is a weak equivalence. Now, $Bf$ is a weak equivalence if and only if $\Omega Bf$ is a quasi-isomorphism. By Proposition~\ref{prop: unit-counit-qif}, the counit $A \rightarrow \Omega BA$ is a quasi-isomorphism, so $Bf$ is a weak equivalence by 2-out-of-3 property.

                \begin{center}
                    \begin{tikzcd}
                        A \ar[]{r}[]{f} & B \\
                        \Omega BA \ar[]{u}[]{\varepsilon_A} \ar[]{r}[]{\Omega Bf} & \Omega BB \ar[]{u}[]{\varepsilon_B}
                    \end{tikzcd}
                \end{center}
            \end{proof}

            We will need one more technical lemma.

            % \begin{center}
            %     \begin{tikzcd}[column sep = tiny]
            %         & BA \ast_{B\Omega D}D \ar[hook]{dr}[]{\pi} \ar[two heads]{dl}[]{} \\ 
            %         D \ar[hook]{dr}[]{\eta_D} & & BA \ar[two heads]{dl}[]{Bp} \ar[equal]{rr}[]{} & & BA \ar[equal]{rr}[]{} & & BA \ar[two heads]{d}[]{} \\
            %         & B\Omega D & & & BK \ar[hook]{u}[]{} & & B\Omega D
            %     \end{tikzcd}
            % \end{center}

            \begin{lemma}[Key lemma]\label{lem: tech-fac}
                Let $A$ be a dg-algebra, $D$ a dg-coalgebra, and $p: A \rightarrow \Omega D$ a fibration of algebras. The projection morphism $BA\ast_{B\Omega D}D \rightarrow BA$ is an acyclic cofibration.
                \begin{center}
                    \begin{tikzcd}
                        BA\ast_{B\Omega D}D \ar[]{d}[]{\pi} \ar[]{r}[]{} \ar[phantom]{rd}[near start]{\ulcorner} & D \ar[]{d}[]{\eta_D} \\
                        BA \ar[two heads]{r}[]{Bp} & B\Omega D
                    \end{tikzcd}
                \end{center}
            \end{lemma}

            This proof has a slightly troubled past. In \cite{LefevreHasegawa03}, Lefevre-Hasegawa made a proof which was a straightforward modification of Hinich's proof \cite[Key Lemma]{Hinich01}. However, this translation does not behave as well as one would like. Keller points out that this method may sometimes work but fails in its full generality \cite{Keller06}. The proof presented here is a modification of Vallette's proof of "A technical lemma" \cite[Appendix B]{Vallette20}.

            \begin{proof}
                $\pi$ being a cofibration is immediate by Corollary~\ref{cor: stable-cofib-base-change}. 
                
                To see that $\pi$ is a weak equivalence, We show that it is a graded quasi-isomorphism by Lemma~\ref{lem: graded-qif-are-w}. Since we assume $p$ to be a fibration onto a quasi-free algebra, we may realize the algebra $A$ as the following extension.
                \begin{center}
                    \begin{tikzcd}
                        \cdots \ar[hook]{r} & \tt{cone}(d') \ar[two heads]{r}[]{p} & \Omega D[1] \ar[dashed, out = -10, in = 170, looseness = 1.5]{dll}[]{d'[1]} \\
                        \tt{Ker}(p)[1] \ar[hook]{r} & \tt{cone}(d')[1] \ar[two heads]{r}[]{} & \cdots
                    \end{tikzcd}
                \end{center}
                Between each of the extensions, there is a connecting morphism $d'$, which comes from the differential of $\tt{cone}(d')$. As graded modules, $A \simeq \tt{cone}(d') \simeq \tt{Ker}(p) \oplus \Omega D$. We denote $K = \tt{Ker}(p)$, so that the differential of $A$ is then the differential coming from 
                \begin{align*}
                    d_K & : K \rightarrow K, \\
                    d_{\Omega D} & : \Omega D \rightarrow \Omega D \tt{ and} \\
                    d' & : \Omega D \rightarrow K\tt{.}
                \end{align*}

                In the category $\tt{Alg}_{\mathbb{K},+}^\bullet$, $\oplus$ is the product. Since $B : \tt{Alg}_{\mathbb{K},+}^\bullet \rightarrow \tt{coAlg}_{\mathbb{K}, conil}^\bullet$ is right adjoint, it necessarily preserves products. Thus
                \begin{align*}
                    & BA\simeq B(K\oplus \Omega D) \simeq BK \ast B\Omega D \tt{ and} \\
                    & BA \ast_{B\Omega D} D \simeq BK \ast D\tt{.}
                \end{align*}
                
                Using this identification of the underlying graded modules, we may identify the morphism $\pi$ with $id_{BK} \ast \eta_D$. If the differential of $BA$ was not perturbed by $d'$, then we could have appealed to the morphism $\pi$ being a graded quasi-isomorphism to conclude that it is a quasi-isomorphism. Instead, we will employ some smart filtrations onto $BA$ and $BA \ast_{B\Omega D}D$.

                Since $BK$ is quasi-free, by the comonadic presentation of $D$, we can obtain an identification of graded modules, $BK\ast D \subseteq T^c(\overline{K}[1]\oplus \overline{D})$. Likewise, since both $BK$ and $B\Omega D$ are quasi-free, we realize the product as $BK\ast B\Omega D \simeq T^c(\overline{K}[1]\oplus (\overline{\Omega}D)[1])$.

                With this description, we define filtrations as
                \begin{align*}
                    & F_n(BA\ast_{B\Omega D}D) \subseteq F_n(T^c(\overline{K}[1]\oplus \overline{D})) = \bigoplus_{k=0}^\infty\,\,\sum_{\substack{n_1+\cdots+n_k \\ \leq n}}\,\,\bigotimes_{i = 1}^k\,(\overline{K}[1]\oplus Fr_{n_i}\overline{D})\tt{ and} \\
                    & F_n(BA) = F_n(T^c(\overline{K}[1]\oplus(\overline{\Omega}D)[1])) = \bigoplus_{k=0}^\infty\,\,\sum_{\substack{n_1+\cdots+n_k \\ \leq n}}\,\,\bigotimes_{i=1}^k\,(\overline{K}[1]\oplus\widetilde{Fr}_{n_i}(\overline{\Omega}D)[1])\tt{.}
                \end{align*}
                Here $Fr$ and $\widetilde{Fr}$ refer to the coradical and induced coradical filtration. This filtration is made to be agnostic towards $K$. In other words, morphisms into $K$ are a priori filtered. Thus the part of the differential coming from $d_K$ and $d'$ are filtered. Likewise, the coradical filtration preserves the part of the differential coming from $d_{\Omega D}$. The differential coming from the multiplication of $K$ and $\Omega D$ is of $-1$ filtered degree. $\eta_{\overline{D}}$ preserves this filtration as it acts like the identity.

                The associated graded component reduces to the associated graded of $D$ and $B\Omega D$. If we lower the degree of a $n_i$ by $1$, this component lands in the lower degree of the filtration. By cocontinuity of the tensor, we may move the associated graded into each variable. The sum handles every other component.
                \begin{align*}
                    \tt{gr}_n(BA\ast_{B\Omega D}D) & \simeq BK \ast \tt{gr}_nD \\
                    \tt{gr}_n(BA) & \simeq BK\ast B\Omega \tt{gr}_nD
                \end{align*}
                In the same manner, the morphism $\pi$ then acts on each element as $id_{BK}\ast \tt{gr}(\eta_D)$.

                These filtrations are bounded below. Since $D$ and $B\Omega D$ are both conilpotent dg-coalgebras, the filtrations are also exhaustive. By the classical convergence theorem of filtered spectral sequences, we obtain spectral sequences $E(BA\ast_{B\Omega D}D) \implies \tt{H}^*(BA\ast_{B\Omega D}D)$ and \\ $E(BA) \implies \tt{H}^*(BA)$. We want to show that the morphism of spectral sequences $id_{BK}\ast_{B\Omega D}\tt{gr}\eta_D: E(BA\ast_{B\Omega D}D) \rightarrow E(BA)$ eventually becomes a quasi-isomorphism, and this will happen on the first page.

                To obtain this on the first page, we will define another spectral sequence $\widetilde{E}$ such that $\widetilde{E} \implies E_1$. We start by defining new filtrations,
                \begin{align*}
                    \widetilde{F}_n(BK\ast\tt{gr}D) & \subseteq \bigoplus_{k=0}^\infty\,\,\sum_{\substack{n_1+\cdots +n_k + k \\ \leq n}}\,\,\bigotimes_{i=1}^k(\overline{K}[1]\oplus \tt{gr}_{n_i}\overline{D})\tt{ and} \\
                    \widetilde{F}_n(BK\ast B\Omega\tt{gr}D) & = \bigoplus_{k=0}^\infty\,\,\sum_{\substack{n_1+\cdots +n_k \\ \leq n}}\,\,\bigotimes_{i=1}^k\,(\overline{K}[1]\oplus(\bigoplus_{t=1}^\infty\,\,\sum_{\substack{m_1+\cdots +m_t + t \\ \leq n_i}}\,\,\bigotimes_{j=1}^t\,\tt{gr}_{m_j}\overline{D}[-1])[1])\tt{.}
                \end{align*}

                Again, these filtrations are agnostic towards $K$, so both parts of the differential that comes from $d_K$ and $d'$ are filtered. The part of the differential which comes from $d_D$ naturally goes from $\tt{gr}_{n_i}\overline{D}$ to itself. The differential coming from the multiplication has already been dealt with, so these filtrations respect our differential. The morphism $id_{BK}\ast\tt{gr}(\eta_D)$ also preserves this filtration, as it acts like the identity on elements. In other words, the first filtered object is naturally a subobject of the second filtered object by identifying the elements $d$ with $[\langle d \rangle ]$.

                At the $0$'th page of $\widetilde{E}$, we want to show that the part of the differential coming from $d'$ acts like $0$. This is the same to say that $\tt{Im}d'\mid_{F_n} \subseteq F_{n-1}$. We calculate the $0$'th page of the double spectral sequence as below.
                \begin{align*}
                    \widetilde{E}_0^{-n}(BK\ast\tt{gr}D)[-n] \subseteq \tt{gr}_n(BK\ast\tt{gr}D) \simeq \bigoplus_{k=0}^\infty\,\,\sum_{\substack{n_1+\cdots + n_k + k \\ = n}}\,\,\bigotimes_{j=1}^k\,(\overline{K}[1]\oplus\tt{gr}_{n_i}\overline{D})
                \end{align*}
                \begin{multline*}
                    \widetilde{E}_0^{-n}(BK\ast B\Omega\tt{gr}D)[-n] = \tt{gr}_n(BK\ast B\Omega\tt{gr}D) \\ 
                    \simeq \bigoplus_{k=0}^\infty\,\,\sum_{\substack{n_1+\cdots + n_k \\ = n}}\,\,\bigotimes_{i=1}^k\,(\overline{K}[1]\oplus(\bigoplus_{t=1}^\infty\,\,\sum_{\substack{m_1+\cdots + m_t + t \\ = n_i}}\,\,\bigotimes_{j=1}^t\,\tt{gr}_{m_j}\overline{D}[-1])[1])
                \end{multline*}

                We now pick an element $([k_1] + d_1) \otimes \cdots \otimes ([k_k] + d_k) \in \tt{gr}_n(BK\ast \tt{gr}D)$. Then $|d_1| + \cdots + |d_k| + k = n$. The differential from $d'$ is the alternate sum of $d'$ at each tensor argument. We illustrate what happens at the $i$'th argument.
                \begin{align*}                    
                    & \widetilde{d'}(([k_1] + d_1) \otimes \cdots \otimes ([k_i] + d_i) \otimes \cdots \otimes ([k_k] + d_k)) \\ 
                    = & ([k_1] + d_1) \otimes \cdots \otimes ([k_i] + d'(d_i)) \otimes \cdots \otimes ([k_k] + d_k) 
                \end{align*}

                Since $|[k]+d'(d_i)| = 0$, the total degree of this element goes down at least $1$ if $d_i \neq 0$. If $d_i = 0$, then $d'(d_i) = 0$ anyway. In this manner, this morphism does not survive at the $\widetilde{E}_0$ page. Likewise, given an element on the form $[k_1 + \langle d_{1,1} \mid \cdots \mid d_{1,t_1} \rangle \mid \cdots \mid k_k + \langle d_{k,1} \mid \cdots \mid d_{k,t_k} \rangle ]$, then $|d'(\langle d_{i,1} \mid \cdots \mid d_{i,t_i} \rangle)| = 0$. So the phenomenon occurs at the other spectral sequence as well.
                
                In this way $\tt{gr}(id_{BK}\ast\tt{gr}\eta_D)$, is in fact a quasi-isomorphism between the sequences $\widetilde{E}(BK\ast\tt{gr}D) \rightarrow \widetilde{E}(BK\ast B\Omega\tt{gr}D)$ just as Lemma~\ref{prop: unit-counit-qif}. By the classical convergence theorem, this assembles into a quasi-isomorphism on the $E_1$ page of the previous spectral sequences, showing that $\pi$ is a graded quasi-isomorphism.
            \end{proof}

            \begin{thm}\label{thm: model-coalg}
                The category $\tt{coAlg}^\bullet_{\mathbb{K}, conil}$ is a model category with the classes $\tt{Ac}$, $\tt{Fib}$ and $\tt{Cof}$ as defined above.
            \end{thm}

            \begin{proof}
                The axioms \textbf{MC1} and \textbf{MC2} are immediate. Also, fibrations having RLP with respect to acyclic cofibrations is by definition.

                We show \textbf{MC4} first. Let $f: C\rightarrow D$ be a morphism of coalgebras. There is a factorization $\Omega f = pi$ of morphisms between algebras, where $i$ is a cofibration, $p$ is a fibration, and at least one of $i$ and $p$ are quasi-isomorphisms. Applying the bar construction, we get a factorization $B\Omega f = BiBp$, where $Bp$ is a fibration, and at least one of $Bi$ and $Bp$ are weak equivalences.
                \begin{center}
                    \begin{tikzcd}
                        \Omega C \ar[]{rr}[]{\Omega f} \ar[]{rd}[]{i} && \Omega D \\
                        & A \ar[]{ru}[]{p}
                    \end{tikzcd} $\rightsquigarrow$
                    \begin{tikzcd}
                        B\Omega C \ar[]{rr}[]{B\Omega F} \ar[]{rd}[]{Bi} && B\Omega D \\
                        & BA \ar[]{ru}[]{Bp}
                    \end{tikzcd}
                \end{center}

                We construct a pullback with $Bp$ and $\eta_D$. By Lemma~\ref{lem: tech-fac}, the morphism $\pi$ is an acyclic cofibration. We collect our morphisms in a big diagram. The dashed arrow exists since the rightmost square is a pullback.
                \begin{center}
                    \begin{tikzcd}
                        & BA\ast_{B\Omega D}D \ar[]{ddd}[]{} \ar[]{rd}[]{q} \ar[phantom]{ddddr}[very near start]{\ulcorner} \\
                        C \ar[dashed]{ru}[]{j} \ar[crossing over]{rr}[near start]{f} \ar[]{ddd}[]{\eta_C} && D \ar[]{ddd}[]{\eta_D} \\
                        \\
                        & BA \ar[]{rd}[]{Bp} \\
                        B\Omega C \ar[]{rr}[]{B\Omega f} \ar[]{ru}[]{Bi} && B\Omega D
                    \end{tikzcd}
                \end{center}
                
                First, notice that $q$ is a fibration since fibrations are stable under pullbacks. $j$ is a cofibration, or a monomorphism, as the composition $Bi\circ \eta_C$ is a monomorphism. Thus it remains to see that if $Bi$ ($Bp$) is a weak equivalence, then $j$ ($q$) is as well. We know this from the $2$-out-of-$3$ property, as $\eta$ is a natural weak equivalence, $\pi$ is a weak equivalence, and $Bi$ ($Bp$) is a weak equivalence.
                
                We now show \textbf{MC3}. Suppose there are morphisms as in the square below, where $i$ is a cofibration, and $t$ is an acyclic cofibration.
                \begin{center}
                    \begin{tikzcd}
                        E \ar[]{r}[]{} \ar[]{d}[]{i} & C \ar[]{d}[]{t} \\
                        F \ar[]{r}[]{} & D
                    \end{tikzcd}
                \end{center}
                
                We can factor $t$ as $t = qj$ by \textbf{MC4}. Notice that $t$ is a retract of $q$, i.e., there is a commutative diagram below.
                \begin{center}
                    \begin{tikzcd}
                        C \ar[equal]{r}[]{} \ar[]{d}[]{j} & C \ar[]{d}[]{t} \\
                        BA\ast_{B\Omega A}D \ar[]{r}[]{} \ar[]{r}[]{q} \ar[]{ru}[]{} & D
                    \end{tikzcd}
                \end{center}
                
                To find a lift to $C$, we may find a lift to $BA\ast_{B\Omega D}D$. Since $p$ is an acyclic fibration by construction and $\Omega i$ is a cofibration by Lemma~\ref{lem: bar-cobar-Quill-adj}, there is a lift $h: \Omega E \rightarrow A$ of algebras. We obtain our desired lift from the bar-cobar adjunction and the universal property of the pullback.
                \begin{center}
                    \begin{tikzcd}
                        E \ar[]{d}[]{i} \ar[]{r}[]{} & BA\ast_{B\Omega D}D \ar[crossing over]{d}[]{q} \ar[]{r}[]{\pi} \ar[phantom]{rd}[very near start]{\ulcorner} & BA \ar[]{d}[]{Bp} \\
                        F \ar[]{r}[]{} \ar[dotted]{rru}[near end, below]{h^T} \ar[dashed]{ru}[]{} & D \ar[]{r}[]{\eta_D} & B\Omega D
                    \end{tikzcd} $\leftrightsquigarrow$
                    \begin{tikzcd}
                        \Omega E \ar[]{d}[]{\Omega i} \ar[]{r}[]{} & A \ar[]{d}[]{p} \\
                        \Omega F \ar[]{r}[]{} \ar[dotted]{ru}[]{h} & \Omega D
                \end{tikzcd}
            \end{center}
        \end{proof}

        We restate the corollary of the adjunction.
        \begin{corollary}\label{cor: cobar-bar-quill-eq}
            The bar-cobar construction $\Omega : {coAlg}^\bullet_{\mathbb{K},conil} \rightleftharpoons \tt{Alg}^\bullet_{\mathbb{K},+} : B$ as a Quillen equivalence.
        \end{corollary}

        \begin{proof}
            We first observe that $(B, \Omega)$ is a Quillen adjunction by Lemma~\ref{lem: bar-cobar-Quill-adj}. Moreover, since the unit and counit are weak equivalences by Proposition~\ref{prop: unit-counit-qif}, it follows by either Proposition~\ref{prop: Quill-Eq} or its Corollary~\ref{cor: Quill-Eq} that $(B, \Omega)$ is a Quillen equivalence.
        \end{proof}

    \subsection{Homotopy theory of $A_{\infty}$-algebras}

        This section aims to finalize the discussion of the homotopy theory of $A_{\infty}$-algebras. We will look at the homotopy invertibility of every strongly homotopy associative quasi-isomorphism and its relation to ordinary associative algebras. This discussion will end with mentioning different results, which gives a more explicit description of fibrations, cofibrations, and homotopy equivalences. This section follows Lefevre-Hasegawa \cite{LefevreHasegawa03}. Before we get to the main theorem, we start by discussing a non-closed model structure on the category of $\tt{Alg}_\infty$.

        Let $f: A \rightsquigarrow B$ be a morphism between $A_\infty$-algebras, the category of $A_\infty$-algebras will be equipped with the three following classes of morphisms:
        \begin{itemize}
            \item $f\in \tt{Ac}$ if $f$ is an $\infty$-quasi-isomorphism, i.e. $f_1$ is a quasi-isomorphism.
            \item $f\in \tt{Fib}$ if $f_1$ is an epimorphism.
            \item $f\in \tt{Cof}$ if $f_1$ is a monomorphism.
        \end{itemize}

        This category does not make a model category in the sense of a closed model category, as we lack many finite limits. It does, however, come quite close to being such a category.

        \begin{thm}\label{thm: model-A-inf}
            The category $\tt{Alg}_\infty$ equipped with the three classes as defined above satisfies:
            \begin{itemize} 
                \item[a] The axioms \textbf{MC1} through \textbf{MC4}.
                \item[b] Given a diagram as below, where $p$ is a fibration, then its limit exists.
                \begin{center}
                    \begin{tikzcd}
                        & A \ar[]{d}[]{p} \\
                        B \ar[]{r}[]{} & C
                    \end{tikzcd}
                \end{center}
            \end{itemize}
        \end{thm}

        Before we are ready to prove this theorem, we will need some preliminary results. We will only prove the first lemma.

        \begin{lemma}\label{lem: inf-creator}
            let $A$ be an $A_\infty$-algebra, and $K$ an acyclic complex considered as an $A_\infty$-algebra. If $g: (A,m_1^A) \rightarrow (K, m_1^K)$ is a cochain map, then it extends to an $\infty$-morphism $f: A \rightsquigarrow K$.
        \end{lemma}

        \begin{proof}
            We construct each $f_i$ inductively. The case $i=1$ is degenerate as we have assumed $f_1 = g$.

            Assume that we have already constructed $f_1$ through $f_n$. We observe that the sum below is a cycle of $\tt{Hom}^*_\mathbb{K}(A, K)$.
            \begin{align*}
                \sum_{\substack{p + 1 + r = k \\ p + q + r = n}}(-1)^{pq+r}f_k\circ_{p+1}m^A_q - \sum_{\substack{k\geq 2 \\ i_1 + ... + i_k = n}}(-1)^{e}m^B_k \circ (f_{i_1}\otimes f_{i_2}\otimes ... \otimes f_{i_k})
            \end{align*}
            Thus since $K$ is acyclic, $\tt{Hom}^*_\mathbb{K}(A, K)$ is acyclic, and there exists some morphism $f_{n+1}$ such that $\partial (f{n+1})$ is the sum above, and this says that this extension does satisfy $(rel_{n+1})$.
        \end{proof}

        \begin{lemma}[{\cite[Lemma 1.3.3.3][44]{LefevreHasegawa03}}]\label{lem: strict-replacement}
            Let $j: A \rightsquigarrow D$ be a cofibration of $A_\infty$-algberas, and then there is an isomorphism $k: D\rightsquigarrow D'$ such that the composition $k\circ j: A \rightsquigarrow D'$ is a strict morphism of $A_\infty$-algebras.

            Dually, if $j: A \rightsquigarrow D$ is a fibration, then there is an isomorphism $l: A' \rightsquigarrow A$ such that the composition $j\circ l: A' \rightsquigarrow D$ is a strict morphism of $A_\infty$-algebras.
        \end{lemma}

        We will need the following lemma.

        \begin{proof}[Proof of Theorem~\ref{thm: model-A-inf}]
            We start by showing (b). Suppose we have a diagram of $A_\infty$-algebras, such that $g_1$ is an epimorphism.
            \begin{center}
                \begin{tikzcd}
                    & A \ar[two heads]{d}[]{g} \\
                    A' \ar[]{r}[]{f} & A''
                \end{tikzcd}
            \end{center}
            First, notice that as dg-coalgebras, this pullback exists and defines a new dg-coalgebra $BA \ast_{BA''}BA'$.

            Since $g_1$ is an epimorphism, $A[1]$ as a graded vector space splits into $A''[1] \oplus K$, where $K = \tt{Ker}g_1$. The pullback is then naturally identified with $BA \prod_{BA''}BA' \simeq \overline{T}^c(K)\prod \overline{T}^c(A'[1])$ as graded vector spaces. Since the cofree coalgebra is right adjoint to forget, it commutes with products, and we get $\overline{T}^c(A'[1])\prod \overline{T}^c(K) \simeq \overline{T}^c(A'[1]\oplus K)$. Thus the pullback is isomorphic to a cofree coalgebra as a graded coalgebra, i.e., an $A_\infty$-algebra.

            We now prove (a). \textbf{MC1} and \textbf{MC2} are immediate, so we will not prove them.

            We start by proving \textbf{MC3}. Suppose that there is a square of $A_\infty$-algebras as below, where $j$ is a cofibration, and $q$ is a fibration.
            \begin{center}
                \begin{tikzcd}
                    A \ar[]{r}[]{f} \ar[]{d}[]{j} & B \ar[]{d}[]{q} \\
                    C \ar[]{r}[]{g} & D
                \end{tikzcd}
            \end{center}

            By Lemma~\ref{lem: strict-replacement}, we may assume that both $j$ and $q$ are strict morphisms. We can assume that $q$ is an $\infty$-quasi-isomorphism since the proof will be analogous if $j$ is an $\infty$-quasi-isomorphism instead.

            Our goal is to construct a lifting in this diagram inductively. Having a lift means finding an $\infty$-morphism $a : C \rightsquigarrow B$, such that the following hold for any $n\geq 1$:
            \begin{itemize}
                \item $a$ satisfy $(rel_n)$.
                \item $a_n \circ j_1 = f_n$.
                \item $q_1\circ a_n = g_n$.
            \end{itemize}

            We start by showing there is such an $a_1$. Consider the diagram below of chain complexes over $\mathbb{K}$.
            \begin{center}
                \begin{tikzcd}
                    A \ar[tail]{d}[]{j_1} \ar[]{r}[]{f_1} & B \ar[two heads]{d}[]{q_1} \\
                    C \ar[]{r}[]{g_1} \ar[dashed]{ru}[]{a_1} & D
                \end{tikzcd}
            \end{center}
            The lift exists since the category $\tt{Ch}(\mathbb{K})$ is a model category, Corollary \ref{cor: Model-Mod}. Here $j_1$ is a cofibration, while $q_1$ is an acyclic fibration, so the lift $a_1$ exists.

            We now wish to extend this. Suppose that we have been able to create morphisms $a_1$ up to $a_n$, all satisfying the above points. A naive solution to make $a_{n+1}$ is \\ $b = f_{n+1}r^{\otimes n+1} + sg_{n+1} - s q_1f_{n+1}r^{\otimes n+1}$, where $r : C \rightarrow A$ is a splitting of $j_1$ and $s : D \rightarrow B$ is a splitting of $q_1$. Notice that this morphism satisfies the two last points by definition. We will augment $b$ to get an $a_{n+1}$ which also satisfies $(rel_{n+1})$.

            For our own convenience, let $-c(f_1, ..., f_n)$ denote the right hand side of $(rel_{n+1})$ formula. Since both $j$ and $q$ are strict $\infty$-morphisms we get the following identites:
            \begin{align*}
                & (\partial b + c(a_1, ..., a_n)) \circ j_1 = \partial (b\circ j_1) + c(a_1 \circ j_1, ..., a_n \circ j_1) = \partial f_{n+1} + c(f_1, ..., f_n) = 0 \\
                & q_1 \circ (\partial b + c(a_1, ..., a_n)) = \partial (q_1\circ b) + c(q_1\circ a_1, ..., q_1\circ a_n) = \partial {g_{n+1}} + c(g_1, ..., g_n) = 0
            \end{align*}

            We thus obtain that the cycle $\partial b + c(a_1, ..., a_n)$ factors through the cokernel of $j$ and the kernel of $q$. Let us say that it factors like the diagram below:
            \begin{center}
                \begin{tikzcd}
                    C \ar[]{r}[]{p} & \tt{Cok} j_1 \ar[]{r}[]{c'} & \tt{Ker} q_1 \ar[]{r}[]{i} & D
                \end{tikzcd}
            \end{center}

            Now, $c'$ is a morphism between two $A_\infty$-algebras. Since $q$ is assumed to be an $\infty$-quasi-isomorphism, it follows that $\tt{Ker}q_1$ is acyclic. Since $c'$ is a cycle in $\tt{Hom}_\mathbb{K}^*(\tt{Cok}j_1,\tt{Ker}q_1)$, it necessarily has to be in the image of the differential. Let $h$ be a morphism such that $\partial h = c'$, and define $a_{n+1} = b - i\circ h\circ p$. One may check that this morphism satisfies all three properties.

            We will now show \textbf{MC4}. Since the two properties have similar proofs, we will only show one direction. Let $f: A \rightsquigarrow B$ be an $\infty$-morphism, an $C = \tt{cone}(id_{B[-1]})$, where the complex $C$ is considered as an $A_\infty$-algebra. Let $j: A \rightsquigarrow A\prod C$ be the morphism induced by $id_A$ and $0:A \rightarrow C$. The canonical projection $q_1: A\oplus C \rightarrow B$ gives a lift of the following diagram.

            \begin{center}
                \begin{tikzcd}
                    A \ar[]{r}[]{f_1} \ar[]{d}[]{j_1} & B \ar[]{d}[]{} \\
                    A\oplus C \ar[]{ru}[]{q_1} \ar[]{r}[]{} & 0
                \end{tikzcd}
            \end{center}

            Since we have a morphism of chain complexes lodged between an acyclic cofibration and a fibration, we use the same technique as above to construct an $\infty$-morphism $q: A\prod C \rightarrow B$. $q$ is a fibration by construction. The morphism $f$ may be factored as $f = q\circ j$, where $j$ is an acyclic cofibration, and $q$ is a fibration.
        \end{proof}

        This model structure can characterize the fibrant and cofibrant conilpotent dg-coalgebras.

        \begin{proposition}
            Let $C$ be a conilpotent dg-coalgebra. Then $C$ is cofibrant, and $C$ is fibrant if and only if there is a cochain complex $V$, such that $C \simeq T^c(V)$ as complexes.
        \end{proposition}

        \begin{proof}
            To see that $C$ is cofibrant is the same as to verify that the map $\mathbb{K}\rightarrow C$ is a monomorphism, but this is clear.

            We start by assuming that $C$ is fibrant. Then there is a lift in the square below, making the unit split-mono.
            \begin{center}
                \begin{tikzcd}
                    C \ar[equal]{r}[]{} \ar[]{d}[]{\eta_C} & C \ar[]{d}[]{\varepsilon_C} \\
                    B\Omega C \ar[]{r}[]{\varepsilon_{B\Omega C}} \ar[dashed]{ru}[]{r} & \mathbb{K}
                \end{tikzcd}
            \end{center}
            Define the morphism $p_1^C: C \rightarrow Fr_1C$ as $p_1^C = Fr_1r\circ p_1\circ \eta_C$, where $p_1: B\Omega C \rightarrow Fr_1 B\Omega C$ is the canonical projection on the filtration induced by the coradical filtration on $C$. The morphism $r$ makes $p_1$ into a universal arrow in the category of conilpotent coalgebras, so $C \simeq T^c(\overline{Fr_1C})$.

            Assuming that $C$ is isomorphic to $T^c(V)$ as coalgebras for some cochain complex $V$. Note that, by definition, $C$ is an $A_\infty$-algebra. We have a commutative square of $A_\infty$-algebras. Since every $A_\infty$-algebra is bifibrant, we know that this diagram has a lift, exhibiting $C$ as a retract of $B\Omega C$.
            \begin{center}
                \begin{tikzcd}
                    C \ar[equal]{r}[]{} \ar[]{d}[]{} & C \ar[]{d}[]{} \\
                    B\Omega C \ar[]{r}[]{} \ar[dashed]{ru}[]{} & \mathbb{K}
                \end{tikzcd}
            \end{center}

            We know that $\Omega C$ is fibrant since the map $\Omega C \rightarrow \mathbb{K}$ is epi. By Lemma~\ref{lem: bar-cobar-Quill-adj}, we know that the bar construction preserves fibrations, so $B\Omega C$ is fibrant. Thus $C$ is fibrant as well.
        \end{proof}

        The model structure of $A_\infty$-algebras is compatible with the model structure of conilpotent dg-coalgebras in the following sense. If $f: A \rightsquigarrow A'$ is an $\infty$-morphism, we denote its dg-coalgebra counterpart as $Bf: BA \rightarrow BA'$. Remember that the bar construction is extended as an equivalence of categories on its image. We use this to realize $\tt{Alg}_\infty$ as a subcategory of $\tt{coAlg}_\mathbb{K}$ to obtain two different model structures on this category. The following proposition tells us that these structures do not differ.

        \begin{lemma}\label{lem: qif-infty-qi}
            Let $A$ and $A'$ be $A_\infty$-algebras. Suppose that $f: A \rightsquigarrow A'$ is an $\infty$-morphism and $Bf: BA \rightarrow BA'$ is a graded quasi-isomorphism, then $f$ is an $\infty$-quasi-isomorphism. 
        \end{lemma}

        \begin{proof}
            Given $Bf : BA \rightarrow BA'$, we may reconstruct $f_i = s \circ \pi_{B[1]} Bf \circ (\omega \circ \iota_A)^{\otimes i}$.

            We know that the unit $\eta_{BA}$ is a graded quasi-isomorphism from Proposition~\ref{prop: unit-counit-qif}. The inverse of the bar construction restricts this morphism to the first filtered degree, together with some shift; $B^{-1}\eta_{BA}: A \rightarrow \Omega BA$, which is is again a quasi-isomorphism by assumption.
        \end{proof}

        \begin{proposition}\label{prop: coherent-model-structure}
            Let $f : A \rightsquigarrow A'$ be an $\infty$-morphism. Then we have the following:
            \begin{itemize}
                \item $f$ is an $\infty$-quasi-isomorphism if and only if $Bf$ is a weak equivalence.
                \item $f_1$ is a monomorphism if and only if $Bf$ is a cofibration.
                \item $f_1$ is an epimorphism if and only if $Bf$ is a fibration.
            \end{itemize}
        \end{proposition}

        \begin{proof}
            Suppose that $f: A \rightarrow A'$ is an $\infty$-quasi-isomorphism. The K\"unneth theorem shows that $Bf: BA \rightarrow BA'$ is a graded quasi-isomorphism.

            Suppose that $Bf: BA \rightarrow BA'$ is a weak equivalence. Then $B\Omega Bf: B\Omega BA \rightarrow B\Omega BA'$ is a graded quasi-isomorphism. By Proposition~\ref{prop: unit-counit-qif}, we know that $\eta_{BA}$ and $\eta_{BA'}$ are both graded quasi-isomorphism. By Lemma~\ref{lem: qif-infty-qi}, we get that the $\infty$-morphisms $\Omega Bf$, $B^{-1}\eta_{BA}$ and $B^{-1}\eta_{BA'}$ are $\infty$-quasi-isomorphisms. By the $2$-out-of-$3$ property, we get that $f$ has to be as well.

            The cofibrations of $\tt{coAlg}_{\mathbb{K},conil}^\bullet$ are monomorphisms. Since $B$ is an equivalence of categories, it must preserve and reflect monomorphisms.

            Suppose that $Bf$ is a fibration. Then it has RLP to acyclic cofibrations $Bg$. By the previous points, we know that $g_1$ is a quasi-isomorphism and a monomorphism; in particular, $f$ has RLP to $g$.

            Suppose that $f_1$ is an epimorphism and that there exists morphism fitting inside a commutative diagram as below.
            \begin{center}
                \begin{tikzcd}
                    BC \ar[]{r}[]{Bh} \ar[tail]{d}[]{Bg} & BA \ar[two heads]{d}[]{Bf} \\
                    BD \ar[]{r}[]{Bi} & BB
                \end{tikzcd}
            \end{center}
            Assume that $Bg$ is an acyclic cofibration. We want to show that $Bf$ has RLP to $Bg$, then $Bf$ has to be a fibration. Notice that $BA$ and $BA'$ are fibrant, so the terminal morphism is a fibration. We find the lifting by considering the following diagram.
            \begin{center}
                \begin{tikzcd}
                    BC \ar[]{rr}[near start]{Bh} \ar[]{ddd}[]{Bg} \ar[]{rd}[]{B\eta_{BA}} & & BA \ar[]{ddd}[]{Bf} \\
                    & B\Omega BC \ar[dashed]{ru}[]{} \ar[tail]{ddd}[near start]{B\Omega Bg} \\
                    \\
                    BD \ar[]{rr}[near start]{Bi} \ar[]{rd}[]{B\eta_{BD}} & & BB \\
                    & B\Omega BD \ar[dashed]{ru}[]{} \ar[dashed]{ruuuu}[]{}
                \end{tikzcd}
            \end{center}

        \end{proof}

    \section{The Homotopy Category of $\tt{Alg}_\infty$}

    We now have many different notions of homotopy, coming from either homological algebra or the model categorical structure. In the case for $A_\infty$-algebras, these notions will luckily coincide.

    \begin{proposition}[{\cite[Proposition 1.3.4.1][49]{LefevreHasegawa03}}]\label{prop: homotopy-is-homotopy}
        Let $C$ and $D$ be two conilpotent dg-coalgebras, where $f,g: C \rightarrow D$ are two morphisms. Then:
        \begin{itemize}
            \item If $f-g$ is null homotopic by an $(f,g)$-coderivation $h$, then they are left homotopic.
            \item If $D$ is fibrant, then $f-g$ is null homotopic by an $(f,g)$-coderivation if and only if $f$ and $g$ are left homotopic.
        \end{itemize}
    \end{proposition}

    \begin{proof}[Sketch of proof]
        We construct a cylinder object for $C$. Consider the cochain complex below, called $I$,
        \begin{center}
            \begin{tikzcd}
                \cdots \ar[]{r}[]{} & \mathbb{K}\{e\} \ar[]{r}[]{\begin{pmatrix} 1 \\ -1 \end{pmatrix}} & \mathbb{K}\{e_1,e_2\} \ar[]{r}[]{} & \cdots
            \end{tikzcd}
        \end{center}
        concentrated in degree $-1$ and $0$. Its comulitplication is given as
        \begin{align*}
            \Delta(e_0) = e_0 \otimes e_0, \quad \Delta(e_1) = e_1 \otimes e_1, \quad \Delta(e) = e \otimes e_1 + e_0 \otimes e
        \end{align*}

        The object $C \otimes I$ is now a cylinder object of $C$. To define a left homotopy from $f$ to $g$ is the same as finding a morphism $H$ making the diagram below commute.
        \begin{center}
            \begin{tikzcd}
                C \ar[]{d}[]{} \ar[dotted]{rd}[]{i_0} \ar[bend left]{rrd}[]{f} \\
                C\coprod C \ar[]{r}[]{p_0} & C\wedge I \ar[]{r}[]{H} & D \\
                C \ar[]{u}[]{} \ar[dotted]{ru}[]{i_1} \ar[bend right]{rru}[]{g}
            \end{tikzcd}
        \end{center}

        Since we assume that $f$ and $g$ are homotopic, there is then an $(f,g)$-coderivation $h : C \rightarrow D$. To define $H$, there are essentially three different components we have to consider. Let $H$ be defined as
        \begin{align*}
            H\mid_{C \otimes e_0} = f, \quad H\mid_{C \otimes e_1} = g,\tt{ and }H\mid_{C \otimes e} = h
        \end{align*}

        We see that this morphism respect the comulitplication, as $h$ is an $(f,g)$-coderivation. We see that it respects the differential since $\partial h = f - g$, and that $f$ and $g$ are morphism of cochain complexes. Moreover, any such morphism $H : C \otimes I \rightarrow D$ defines an $(f,g)$-coderivation. This concludes that null homotopic morphisms are left homotopic.

        To see it the other way around if $D$ is fibrant, and the morphisms $f$ and $g$ are left homotopic, we may promote this homotopy to a homotopy $H : C \otimes I \rightarrow D$. The result follows by extracting the homotopy as $h = H\mid_{C \otimes e}$.
    \end{proof}

    \begin{remark}
        In the category $\tt{Alg}_\infty$, we are now able to say that the homotopies as defined in Section 1.3 are exactly the model categorical homotopies. This follows from the fact that bifibrant objects may promote their left homotopies to right homotopies, and right homotopies to left homotopies. By the above proposition, we know as well that left homotopies, may be promoted to ordinary homotopies.
    \end{remark}

    Due to this result, we may know think of homotopies to actually belong to the model categorical structure. We will make little distinction between these notions going forward.

    \begin{thm}\label{thm: model-Alg-inf}
        In the category $\tt{Alg}_\infty$ we have the following:
        \begin{itemize}
            \item Homotopy equivalence is an equivalence relation.
            \item A morphism is an $\infty$-quasi-isomorphism if and only if it is a homotopy equivalence.
            \item By abuse of notation, let $\tt{Alg}_\mathbb{K} \subseteq \tt{Alg}_\infty$ be the full subcategory consisting of dg-algebras considered as $A_\infty$-algebras. $\tt{Alg}_\mathbb{K}$ has an induced homotopy equivalence from $\tt{Alg}_\infty$, and the inclusion $\tt{Alg}_\mathbb{K} \rightarrow \tt{Alg}_\mathbb{K} \subset \tt{Alg}_\infty$ induces an equivalence in homotopy $\tt{Alg}[Qis^{-1}] \simeq \sfrac{\tt{Alg}_\mathbb{K}}{\sim}$.
        \end{itemize}
    \end{thm}

    \begin{proof}
        We observe the first point from Corollary~\ref{cor: homotopy-is-eq-rel}, and the second point is Whitehead's theorem, Theorem~\ref{thm: Whitehead}.

        To see the final point, observe that the inclusion functor is given by the bar construction $B$. By Corollary~\ref{cor: cobar-bar-quill-eq}, we know that the bar construction induces an equivalence on the homotopy categories, i.e., $\tt{Ho}\tt{Alg} \simeq \tt{Ho}\tt{coAlg}$. Moreover, we know that by Theorem~\ref{thm: Fundamental-thm-model} that $\tt{Ho}\tt{coAlg} \simeq \sfrac{\tt{Alg}_\infty}{\sim}$. Notice that the image of $B$ is $\tt{Alg}_\mathbb{K}$, so in homotopy, we get that the image $\sfrac{\tt{Alg}_\mathbb{K}}{\sim}$ is equivalent to the essential image $\tt{Ho}\tt{Alg}_\infty$.
    \end{proof}

\end{document}