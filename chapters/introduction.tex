\documentclass[../thesis.tex]{subfiles}

\begin{document}

    \newpage
    
    \section{Introduction}

        A differential graded algebra, or simply dg-algebra, is an associative algebra where the underlying object is a cochain complex. Any dg-algebra $A$ naturally carries a homotopical structure, as the homology, $H^*A$, defines an algebra where some elements are unique up to homology. Since homology algebras are determined by their dg-counterparts, we are very interested to understand quasi-isomorphisms; that is, morphisms $f:A\rightarrow~B$ between dg-algebras such that $H^*f:H^*A\rightarrow~H^*B$ is an isomorphism.

        It is quite known that quasi-isomorphisms $f:A\rightarrow~B$ between associative dg-algebras admit a homotopy inverse whenever we consider them as $A_\infty$-algebras. This allows us to think of homology algebras as homotopy algebras of $A_\infty$-algebras,
        \begin{align*}
            \tt{HoAlg}_\mathbb{K} \simeq \sfrac{\tt{Alg}_\infty}{\sim}\tt{.}
        \end{align*}

        This result is still true if we consider quasi-isomorphisms $f:M\rightarrow~N$ between $A$-modules. The morphism $f$ admits a homotopy inverse whenever we consider the modules as corresponding $A_\infty$-modules. With this in mind, there are equivalences of categories,
        \begin{align*}
            D_\infty A \simeq K_\infty A \simeq DA\tt{.}
        \end{align*}
        Here, $D_\infty A$ and $K_\infty A$ denote the derived and homotopy category of the category of $A_\infty$-modules, respectively.

        In this thesis we investigate a proof provided by Lef\`evre-Hasegawa \cite{LefevreHasegawa03} while taking a lot of inspiration from Loday and Vallette \cite{Loday12}. The thesis is split into three different chapters.

        \subsubsection*{Chapter 1 - The Bar and Cobar Construction}
            In Chapter 1, we develop the theory of dg-algebras and dg-coalgebras. We try to make the theory of coalgebras more intuitive by comparing how they differ from algebras. The augmented algebras and conilpotent coalgebras are of utmost importance in this thesis.

            The essential tool developed in this chapter is the bar and cobar construction, denoted as $B$ and $\Omega$, respectively. Twisting morphisms play a unique role as they define a functor, represented by the bar and cobar construction. Thus, we have an adjoint pair of functors,
            \begin{center}
                \begin{tikzcd}
                    \tt{coAlg}_{\mathbb{K}, \tt{conil}}^\bullet \ar[yshift = 1ex]{r}[]{\Omega} \ar[phantom]{r}[]{\bot} & \tt{Alg}_{\mathbb{K},+}^\bullet \ar[yshift = -1ex]{l}[below]{B}
                \end{tikzcd}
            \end{center}

            Lastly, we define $A_\infty$-algebras in terms of the bar construction. We will think of these as the algebras which make the bar construction fully faithful on the image of quasi-free conilpotent dg-coalgebras. We can thus think of an $A_\infty$-algebra in two different ways, either as a dg-algebra with strong homotopy associativity or as a conilpotent dg-coalgebra. Both points of view will be fruitful.

        \subsubsection*{Chapter 2 - Homotopy Theory of Algebras}
            Chapter 2 aims to explain some of the homotopy theories of dg-algebras, conilpotent dg-coalgebras, and $A_\infty$-algebras. We start by giving an exposition on model categories, having a special interest in Whitehead's theorem, the fundamental theorem of model categories, and Quillen equivalences.

            We upgrade the cobar-bar adjunction into a Quillen equivalence, identifying the homotopy category of dg-algebras and conilpotent dg-coalgebras. The category of $A_\infty$-algebra will be equivalent to the bifibrant conilpotent dg-coalgebras. This will allow us to show the first claim,
            \begin{align*}
                \tt{HoAlg}_\mathbb{K} \simeq \sfrac{\tt{Alg}_\infty}{\sim}\tt{.}
            \end{align*}

        \subsubsection*{Chapter 3 - Derived Categories of Strongly Homotopy Associative Algebras}
            In the final chapter, we investigate the homotopy theory of modules over dg-algebras and comodules over dg-coalgebras. We will further develop the theory of twisting morphisms to obtain Quillen equivalences,

            \begin{center}
                \begin{tikzcd}
                    \tt{coMod}^C \ar[yshift = 1ex]{r}[]{L_\alpha} \ar[phantom]{r}[]{\bot} & \tt{Mod}^A \ar[yshift = -1ex]{l}[below]{R_\alpha}
                \end{tikzcd}
            \end{center}

            We prove the fundamental theorem of twisting morphisms, which allows us to characterize whenever a twisting morphism defines a Quillen equivalence.

            $A_\infty$-modules of $A$, called $A$-polydules are defined to be objects being the converse of $R_\alpha$ whenever $C = BA$. We may then see that $A$-polydules are the bifibrant $BA$-comodules. We will then define the derived category of polydules, $D_\infty A$. The thesis will conclude by showing that,
            \begin{align*}
                D_\infty A \simeq K_\infty A \simeq DA\tt{.}
            \end{align*}

\end{document}