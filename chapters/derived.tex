\documentclass[../thesis.tex]{subfiles}

\begin{document}

    In this chapter we wish to study the derived categories of $A_\infty$-algebras. At the heart of homological algebra is the derived category of algebras, so it is only natural to ask how this category looks like in the $A_\infty$ case. In the last chapter we studied the relationship between the category of algebras and coalgebras to understand how quasi-isomorphisms between $A_\infty$-algebras worked. In this chapter we will instead study the relationship between module and comodule categories in order to understand how quasi-isomorphisms between $A_\infty$-modules will work. At the heart of this discussion are twisting morphisms $\alpha : C \rightarrow A$, which allows us to study the relationship between $Mod^A$ and $CoMod^C$.

    From twisting morphisms we will obtain functors $L_\alpha : CoMod^C \rightarrow Mod^A$ and $R_\alpha : Mod^A \rightarrow CoMod^C$ which create an adjoint pair of functors. Whenever the twisting morphism $\alpha$ is acyclic, this will in fact become a Quillen Equivalence.

    We wish to reuse all of the methods we have gained and acquired thorughout this thesis. This chapter will mostly be reformulation and recontextualization of previous definitions, concepts and techniques. 

    \section{Twisting Morphisms}

        Twisting morphisms were already introduced in chapter 1. There, they were used mostly to be represented by the bar and cobar construction. Now we want twisting morphisms and twisting tensors to play a bigger role. In order to define the functors $L_\alpha$ and $R_\alpha$, these constructions will be crucial.  

        \subsection{Twisted Tensor Products}

            Let $A$ be an augmented dg-algebra, $C$ a conilpotent dg-coalgebra and $\alpha : C \rightarrow A$ a twisting morphism. The right (left) twisted tensor product is the complex $C \otimes_\alpha A$ ($A\otimes_\alpha C$) together with the differential $d_\alpha^\bullet = d_{C\otimes A}^\bullet + d_\alpha^r$. The perturbation is defined as
            \begin{align*}
                d_\alpha^r = (\nabla_A\otimes id_C) \circ (id_A \otimes \alpha \otimes id_C) \circ (id_A \otimes \Delta_C).
            \end{align*}

            If $M$ is a right $A$-module and $N$ is a left $C$-comodule then the tensor product $M\otimes_\mathbb{K} N$ exists and is a $\mathbb{K}$-module with differential $d_{M\otimes N}$. We may define a perturbation to this differential as 
            \begin{align*}
                d_\alpha^r = (\mu_M\otimes id_N) \circ (id_M \otimes \alpha \otimes id_N) \circ (id_M \otimes \nu_N).
            \end{align*}
            By using the same line of thought as proposition \ref{prop: twisted-differential}, there is a twisted tensor product $M\otimes_\alpha N$ with differential $d_\alpha^\bullet = d_{M\otimes N} + d_\alpha^r$.

            \begin{remark}
                Koszuls sign rule forces us to define the differential of the left twisted tensor product as $d_\alpha^\bullet = d_{N\otimes M} - d_\alpha^l$. 
            \end{remark}
            
            \begin{definition}
                Suppose that $M\in Mod^A$ ($M\in Mod_A$) and $N\in CoMod_C$ ($N\in CoMod^C$), then the left (right) twisted tensor product is the $\mathbb{K}$-module $M\otimes_\alpha N$ ($N\otimes_\alpha M$).
            \end{definition}

            In this setting right handedness and left handedness for the twisted tensor product is more clear in this setting, as we only have an action or coaction from one of the chosen sides. Trying to force the other handedness on the twisted tensors would just be ill-defined.

            \begin{definition}
                Let $A$ be an augmented dg-algebra and $C$ a conilpotent dg-coalgebgra, such that there is a twisting morphism $\alpha: C\rightarrow A$. Given a linear map $f: N \rightarrow M$ between a right $C$-comodule $N$ and a right $A$-module $M$ we say that it is an $\alpha$ right twisted linear homomorphism, or just an  $\alpha$ twisted morphism, if it satisfies the following equation:
                \begin{align*}
                    \partial f - f\star \alpha = 0
                \end{align*} 
            \end{definition}

            This definitions gives us a functor $Tw_\alpha : CoMod^C \times Mod^A \rightarrow Ab$ which is the collection of right twisting linear homomorphisms between a comodule and module.

            Suppose that $\alpha : C \rightarrow A$ is a twisting morphism. We define the functor $L_\alpha = \otimes_\alpha A : CoMod^C \rightarrow Mod^A$ as an arbitrary right twisted tensor product with $A$. This functor does indeed hit $Mod^A$ by using the free right $A$-module structure on $A$. Likewise, we define a functor $R_\alpha = \otimes_\alpha C : Mod^A \rightarrow CoMod^C$ as an arbitrary left twisted tensor product with $C$. This does also hit right $C$-comodules by using the free right $C$-comodule structure on $C$.

            \begin{proposition}
                Suppose that $\alpha : C \rightarrow A$ is a twisting morphism. The functor $L_\alpha$ and $R_\alpha$ form an adjoint pair of categories.
                \begin{center}
                    \begin{tikzcd}
                        CoMod^C \ar[bend left]{r}[]{L_\alpha} \ar[phantom]{r}[]{\bot} & Mod^A \ar[bend left]{l}[]{R_\alpha}
                    \end{tikzcd}
                \end{center}
            \end{proposition}

            \begin{proof}
                This proof breaks down to showing $CoMod^C(N, L_\alpha(M))\simeq Tw_\alpha(N,M) \simeq Mod^A(R_\alpha(N),M)$. This is a rutine calculation, much like the proof for \ref{thm: cobar-bar-adj}.
            \end{proof}

            Let $A$ be a dg-algebra, and $M$ a right $A$-module. Recall that by the Cobar-Bar adjunction \ref{thm: cobar-bar-adj} there exists a universal twisting morphism $\pi_A : BA \rightarrow A$. We define the bar-construction of $M$ as $B_AM = R_{\pi_A}M = M\otimes_{\pi_A}BA$. Likewise, given a conilpotent dg-coalgebra $C$ and $N$ a right $C$-comodule we define the cobar-construction as $\Omega_CN = L_{\iota_C}N = N\otimes_{\iota_C}\Omega C$. In these cases we obtain adjunctions $\Omega_{BA} \dashv B_A$ and $\Omega_C \dashv B_{\Omega C}$.

            Let $A$ and $B$ be two algebras and $f : A \rightarrow B$ is an algebra morphism. Then $f$ induces a functor between the module categories by restriction: $f^* : Mod^B \rightarrow Mod^A$. Since $A$ and $B$ considered as algebroids are small, and the category of abelian groups is cocomplete, so the left Kan extension (induction) along this functor exists.
            \begin{center}
                \begin{tikzcd}
                    Mod^B \ar[]{r}[]{f^*} & Mod^A \ar[bend right]{l}[above]{f_!}
                \end{tikzcd}
            \end{center}

            Dually, if $C$ and $D$ are two coalgebras and $g : C \rightarrow D$ is an coalgebra morphism. Then $g$ induces a functor between the module categories by composing: $g* : CoMod^C \rightarrow CoMod^D$. Since $C$ and $D$ considered as coalgebroids are small, and the category of abelian groups is complete, so the right Kan extension (co-induction) along this functor exists.
            \begin{center}
                \begin{tikzcd}
                    CoMod^C \ar[]{r}[]{g_*} & CoMod^D \ar[bend left]{l}[above]{g^!}
                \end{tikzcd}
            \end{center}

            \begin{lemma}
                Let $\tau : C \rightarrow A$ be a twisting morphism. The adjunction $(L_\tau, R_\tau)$ factors as $(f_{\tau !}, f_\tau^*)\circ (L_{\iota_C},R_{\iota_C})$ or $(L_{\pi_A},R_{\pi_A})\circ (g_{\tau *}, g_\tau^!)$.
            \end{lemma}

            \begin{proof}
                This follows from corollary \ref{cor: universal-twisting}, that is $\tau = f_\tau \circ \iota_C = \pi_A\circ g_\tau$.
            \end{proof}

            \begin{definition}
                A twisting morphism $f: C \rightarrow A$ is called acyclic if the counit of the adjunction $L_\alpha \dashv R_\alpha$ is a pointwise quasi-isomorphism.
            \end{definition}

            \begin{lemma}
                Let $A$ be an augmented dg-algebra and $C$ a conilpotent dg-coalgebra. The universal twisting morphisms $\pi_A$ and $\iota_C$ are acyclic.
            \end{lemma}

            \begin{proof}
                We start with $\pi_A$. Recall that $\pi_A$ is constructed as the twisting morphism corresponding to $id_{BA}$. This morphism is thus given as the projection onto the first dimension of $BA$, that is:
                \begin{align*}
                    & \pi_A{sa} = a \\
                    & \pi_A(sa\otimes ...) = 0
                \end{align*}

                We say that $\pi_A$ is acyclic if the counit $\varepsilon : L_{\pi_A}R_{\pi_A} \Rightarrow Id_{Mod^A}$ at each object $M$ is a quasi-isomorphism.

                For each $M$ in $Mod^A$, $L_{\pi_A}R_{\pi_A}M = M\otimes_{\pi_A}BA\otimes_{\pi_A}A$. We may split up the differential into two summands, $d_v$ and $d_h$. $d_v$ is the ordinary differential on the tensor product, while $d_h = (-d^l_{\pi_A}\otimes A) + M\otimes d_2 \otimes A + d^r_{\pi_A}$. Since $(d_v + d_h)^2 = 0$ and $d_v^2 = 0$ we can observe that $d_vd_h = -d_hd_v$ and $d_h^2 = 0$. This is evident as $d_v$ changes the homological degree while $d_h$ does not, so in order for both of the first equations to hold, the last two must hold as well. We almost obtain a double complex.
                \begin{center}
                    \begin{tikzcd}
                        ... \ar[]{r}[]{d_h} & M\otimes BA_{(i)}\otimes A \ar[]{r}[]{d_h} \ar[loop, distance = 2em]{}[above]{d_v} & ... \ar[]{r}[]{d_h} & M\otimes BA_{(1)}\otimes A \ar[]{r}[]{d_h} \ar[loop, distance=2em]{}[above]{d_v} & M \otimes A \ar[]{r}[]{0} \ar[loop, distance=2em]{}[above]{d_v} & 0
                    \end{tikzcd}
                \end{center}
                It is clear that the total complex of this "double complex" is in fact $L_{\pi_A}R_{\pi_A}M$. Moreover, the counit induces an augmentation to this complex resolution of $M$, denoted as $cone(\varepsilon_M)$. 
                \begin{center}
                    \begin{tikzcd}
                        ... \ar[]{r}[]{d_h} & M\otimes BA_{(i)}\otimes A \ar[]{r}[]{d_h} \ar[loop, distance = 2em]{}[above]{d_{M\otimes BA_{(i)}\otimes A}} & ... \ar[]{r}[]{d_h} & M\otimes BA_{(1)}\otimes A \ar[]{r}[]{d_h} \ar[loop, distance=2em]{}[above]{d_{M\otimes BA_{(1)}\otimes A}} & M \otimes A \ar[]{r}[]{\varepsilon_M} \ar[loop, distance=2em]{}[above]{d_{M\otimes A}} & M \ar[loop, distance = 2em]{}[above]{d_M} \ar[]{r}[]{0} & 0
                    \end{tikzcd}
                \end{center}
                
                To see that this is in fact a resolution we define a morphism $h : cone(\varepsilon_M) \rightarrow cone(\varepsilon_M)$ of degree $-1$. It works by the following formula:
                \begin{align*}
                    h(m \otimes (sa_1 \otimes ... \otimes sa_n) \otimes a) = m \otimes (sa_1 \otimes ... \otimes sa_n \otimes sa) \otimes 1
                \end{align*}
                It is clear that $id_{cone(\varepsilon_M)} = d_hh - hd_h$ and $d_vh = hd_v$. Thus to see that the cone is acyclic we let $c\in cone(\varepsilon_M)$ be a cycle, that is $(d_v + d_h)(c) = 0$. Our goal is to show that $h(c)$ is a preimage of $c$ along $d_v + d_h$.
                \begin{multline*}
                    (d_v + d_h)\circ h(c) = d_v\circ h(c) + d_h\circ h(c) = h\circ d_v(c) + c + h\circ d_h(c) = h\circ (d_v + d_h)(c) + c = c
                \end{multline*}

                We will now treat $\iota_C$. By construction $\iota_C (c) = s^{-1}c$ if $c\notin \mathbb{K}$, as it is the restriction to $\bar{C}$ of the identity on $\Omega C$. 

            \end{proof}

        \subsection{Model Structure on Module Categories}

            Let $A$ be an augmented dg-algebra, then we know that $Mod^A$ is a model category. By corollary \ref{cor: Model-Mod} we have a model structure where the fibrations, cofibrations and weak equivalences are given as follows:
            \begin{itemize}
                \item $f\in Ac$ is a weak equivalence if $f$ is a quasi-isomorphism.
                \item $f\in Fib$ is a fibration if $f^\#$ is an epimorphism.
                \item $f\in Cof$ is a cofibration if it has LLP to acyclic fibrations. 
            \end{itemize}

            Every object in this category is fibrant as the morphism $0 : M \rightarrow 0$ is always an epimorphism.

        \subsection{Model Structure on Comodule Categories}

            Unless stated otherwise, for this section we fix $A$ to be an augmented dg-algebra, $C$ as a conilpotent dg-coalgebra and $\tau : C \rightarrow A$ as an acyclic twisting morphism. We endow $CoMod^C_{conil}$ with three classes of morphisms:
            \begin{itemize}
                \item $f\in Ac$ is a weak equivalence if $L_\tau f$ is a quasi-isomorphism.
                \item $f\in Fib$ is a cofibration if $f^\#$ is a monomorphism.
                \item $f\in Cof$ is a fibration if it har RLP to acyclic cofibrations.
            \end{itemize}

            \begin{thm}\label{thm: model-comod}
              The category $CoMod^C_{conil}$ with the three classes as above forms a model category. Every object is cofibrant, and those objects which is a direct summand of $R_\tau M$ for some $M\in Mod^A$ are fibrant. The adjoint pair $(L_\tau, R_\tau)$ is a Quilllen equivalence.
            \end{thm}

            We will call this model structure for the canonical model structure on $CoMod^C_{conil}$.

            To be able to prove this we will need some lemmata. This proof is essentially the same as the case for dg-coalgebras. The main difference is to show independence of the choice of a twisting morphism $\tau$. To this end we must establish the relationship between graded quasi-isomorphisms and weak equivalences, as well as a technical lemma.

            Recall that given a coaugmented coalgebra $C$ we have a filtration called the coradical filtration, defined as $Fr_iC = Ker(\bar{\Delta_C})^i$. If $N$ is a right $C$-comodule we may define the coradical filtration of $N$ as $Fr_iN = Ker(\bar{\omega_N}^i)$. This filtration is admissable, meaning it is exhaustive and $Fr_0N=0$.

            \begin{lemma}
                Let $C$ be a conilpotent dg-coalgebra, $M$ and $N$ be right $C$-comodules. Then any graded quasi-isomorphism $f: M \rightarrow N$ is a weak equivalence.
            \end{lemma}

            \begin{proof}
                This proof is identical to \ref{lem: graded-qif-are-w}.   
            \end{proof}

            \begin{lemma}
                Let $M$ and $N$ be two objects of $Mod^A$. The functor $R_\tau$ sends a quasi-isomorphism $f: M \rightarrow N$ to a weak equivalence $R_\tau f: R_\tau M \rightarrow r_\tau N$.

                The unit of the adjunction $\eta : Id \rightarrow R_\tau L_\tau$ is a pointwise weak equivalence.
            \end{lemma}

            \begin{proof}
                $R_\tau f$ is a weak equivalence if $L_\tau R_\tau f$ is a quasi-isomorphism. By naturality of the counit we have the following commutative diagram.
                \begin{center}
                    \begin{tikzcd}
                        M \ar[]{d}[]{f} & L_\tau R_\tau M \ar[]{d}[]{L_\tau R_\tau f} \ar[]{l}[]{\varepsilon_M} \\
                        N & L_\tau R_\tau N \ar[]{l}[]{\varepsilon_N}
                    \end{tikzcd}
                \end{center}

                By assumption we know that all three of $f$, $\varepsilon_M$ and $\varepsilon_N$ are quasi-isomorphisms. It follows by the $2$-out-of-$3$ property that $L_\tau R_\tau f$ is a quasi-isomorphism as well.

                To show that $\eta : Id \rightarrow L_\tau R_\tau$ is a pointwise weak-equivalence, we must show that $L\eta$ is a pointwise quasi-isomorphism. Since $L_\tau$ is left adjoint to $R_\tau$ we know that $\eta$ is split on the image of $L_\tau$, i.e.
                \begin{align*}
                    \varepsilon_{L_\tau}\circ L_\tau\eta = id_{L_\tau}
                \end{align*}
                Since we know that the natural isomorphisms $\varepsilon$ and $id$ are pointwise quasi-isomorphisms, we get by the $2$-out-of-$3$ property that $L\eta$ is a pointwise quasi-isomorphism as well.
            \end{proof}

            \begin{lemma}
                The functor $L_\tau$ preserves cofibrations and sends weak-equivalences to quasi-isomorphisms.
            \end{lemma}

            \begin{proof}
                This proof is essentially the same as \ref{lem: bar-cobar-Quill-adj}.
            \end{proof}

            With the above lemmata we have now established that the adjunction $(L_\tau, R_\tau)$ forms a Quillen equivalence if $CoMod^C$ is a model category.

            The next lemma is a technical lemma which we need. There will not be given a proof for it, but this is lemma 2.2.2.9 in \cite{LefevreHasegawa03}.

            \begin{lemma}
                Let $M$ be a right $A$-module and $N$ a right $C$-comodule. Let $p : M \rightarrow L_\tau N$ be a fibration of modules. The projection $j : R_\tau M \prod_{R_\tau L_\tau N} N \rightarrow R_\tau M$ is an acyclic cofibration of comodules.
            \end{lemma}

            \begin{proof}
                This proof is omitted.
            \end{proof}

            \begin{proof}[Proof of \ref{thm: model-comod}]
                With the above lemmata established, this proof is identical to the proof of \ref{thm: model-coalg}.
            \end{proof}

        \subsection{Triangulation of Homotopy Categories}
            In this section we will show that the homotopy categories are triangulated. If we look at the category $Mod^A$ we will observe that the category $HoMod^A$ is our beloved derived category $\mathcal{D}(A)$. It is not quite the same for the cateogry $CoMod^C$. Here we want $HoCoMod^C$ to be equivalent to the derived category of a ring, so we will see that the derived category is a further localization of $HoCoMod^C$.

            We will use the theory of triangulated categories to show that the model structure on $CoMod^C$ is independent on the choice of acyclic twisting morphism. This breaks down to show that every acyclic twisting morphism induce an equivalence between derived categories.

            

        \subsection{The Fundamental Theorem of Twisting Morphisms}

    \section{Polydules}
\end{document}