\documentclass[../thesis.tex]{subfiles}

\begin{document}

    In this chapter we wish to study the derived categories of $A_\infty$-algebras. At the heart of homological algebra is the derived category of algebras, so it is only natural to ask how this category looks like in the $A_\infty$ case. In the last chapter we studied the relationship between the category of algebras and coalgebras to understand how quasi-isomorphisms between $A_\infty$-algebras worked. In this chapter we will instead study the relationship between module and comodule categories in order to understand how quasi-isomorphisms between $A_\infty$-modules will work. At the heart of this discussion are twisting morphisms $\alpha : C \rightarrow A$, which allows us to study the relationship between $Mod^A$ and $CoMod^C$.

    From twisting morphisms we will obtain functors $L_\alpha : CoMod^C \rightarrow Mod^A$ and $R_\alpha : Mod^A \rightarrow CoMod^C$ which create an adjoint pair of functors. Whenever the twisting morphism $\alpha$ is acyclic, this will in fact become a Quillen Equivalence.

    We wish to reuse all of the methods we have gained and acquired thorughout this thesis. This chapter will mostly be reformulation and recontextualization of previous definitions, concepts and techniques. 

    \section{Twisting Morphisms}

        Twisting morphisms were already introduced in chapter 1. There, they were used mostly to be represented by the bar and cobar construction. Now we want twisting morphisms and twisting tensors to play a bigger role. In order to define the functors $L_\alpha$ and $R_\alpha$, these constructions will be crucial.  

        \subsection{Twisted Tensor Products}

            Let $A$ be an augmented dg-algebra, $C$ a conilpotent dg-coalgebra and $\alpha : C \rightarrow A$ a twisting morphism. The right (left) twisted tensor product is the complex $C \otimes_\alpha A$ ($A\otimes_\alpha C$) together with the differential $d_\alpha^\bullet = d_{C\otimes A}^\bullet + d_\alpha^r$. The perturbation is defined as
            \begin{align*}
                d_\alpha^r = (\nabla_A\otimes id_C) \circ (id_A \otimes \alpha \otimes id_C) \circ (id_A \otimes \Delta_C).
            \end{align*}

            If $M$ is a right $A$-module and $N$ is a left $C$-comodule then the tensor product $M\otimes_\mathbb{K} N$ exists and is a $\mathbb{K}$-module with differential $d_{M\otimes N}$. We may define a perturbation to this differential as 
            \begin{align*}
                d_\alpha^r = (\mu_M\otimes id_N) \circ (id_M \otimes \alpha \otimes id_N) \circ (id_M \otimes \nu_N).
            \end{align*}
            By using the same line of thought as proposition \ref{prop: twisted-differential}, there is a twisted tensor product $M\otimes_\alpha N$ with differential $d_\alpha^\bullet = d_{M\otimes N} + d_\alpha^r$.

            \begin{remark}
                Koszuls sign rule forces us to define the differential of the left twisted tensor product as $d_\alpha^\bullet = d_{N\otimes M} - d_\alpha^l$. 
            \end{remark}
            
            \begin{definition}
                Suppose that $M\in Mod^A$ ($M\in Mod_A$) and $N\in CoMod_C$ ($N\in CoMod^C$), then the right (left) twisted tensor product is the $\mathbb{K}$-module $M\otimes_\alpha N$ ($N\otimes_\alpha M$).
            \end{definition}

            In this setting right handedness and left handedness for the twisted tensor product is more clear in this setting, as we only have an action or coaction from one of the chosen sides. Trying to force the other handedness on the twisted tensors would just be ill-defined.

            \begin{definition}
                Let $A$ be an augmented dg-algebra and $C$ a conilpotent dg-coalgebgra, such that there is a twisting morphism $\alpha: C\rightarrow A$. Given a linear map $f: N \rightarrow M$ between a right $C$-comodule $N$ and a right $A$-module $M$ we say that it is an $\alpha$ right twisted linear homomorphism, or just an  $\alpha$ twisted morphism, if it satisfies the following equation:
                \begin{align*}
                    \partial f - f\star \alpha = 0
                \end{align*} 
            \end{definition}

            This definitions gives us a functor $Tw_\alpha : CoMod^C \times Mod^A \rightarrow Ab$ which is the collection of right twisting linear homomorphisms between a comodule and module.

            Suppose that $\alpha : C \rightarrow A$ is a twisting morphism. We define the functor $L_\alpha = \otimes_\alpha A : CoMod^C \rightarrow Mod^A$ as an arbitrary right twisted tensor product with $A$. This functor does indeed hit $Mod^A$ by using the free right $A$-module structure on $A$. Likewise, we define a functor $R_\alpha = \otimes_\alpha C : Mod^A \rightarrow CoMod^C$ as an arbitrary left twisted tensor product with $C$. This does also hit right $C$-comodules by using the free right $C$-comodule structure on $C$.

            \begin{proposition}
                Suppose that $\alpha : C \rightarrow A$ is a twisting morphism. The functor $L_\alpha$ and $R_\alpha$ form an adjoint pair of categories.
                \begin{center}
                    \begin{tikzcd}
                        CoMod^C \ar[bend left]{r}[]{L_\alpha} \ar[phantom]{r}[]{\bot} & Mod^A \ar[bend left]{l}[]{R_\alpha}
                    \end{tikzcd}
                \end{center}
            \end{proposition}

            \begin{proof}
                This proof breaks down to showing $CoMod^C(N, L_\alpha(M))\simeq Tw_\alpha(N,M) \simeq Mod^A(R_\alpha(N),M)$. This is a rutine calculation, much like the proof for \ref{thm: cobar-bar-adj}.
            \end{proof}

            Let $A$ be a dg-algebra, and $M$ a right $A$-module. Recall that by the Cobar-Bar adjunction \ref{thm: cobar-bar-adj} there exists a universal twisting morphism $\pi_A : BA \rightarrow A$. We define the bar-construction of $M$ as $B_AM = R_{\pi_A}M = M\otimes_{\pi_A}BA$. Likewise, given a conilpotent dg-coalgebra $C$ and $N$ a right $C$-comodule we define the cobar-construction as $\Omega_CN = L_{\iota_C}N = N\otimes_{\iota_C}\Omega C$. In these cases we obtain adjunctions $\Omega_{BA} \dashv B_A$ and $\Omega_C \dashv B_{\Omega C}$.

            Let $A$ and $B$ be two algebras and $f : A \rightarrow B$ is an algebra morphism. Then $f$ induces a functor between the module categories by restriction: $f^* : Mod^B \rightarrow Mod^A$. Since $A$ and $B$ considered as algebroids are small, and the category of abelian groups is cocomplete, so the left Kan extension (induction) along this functor exists.
            \begin{center}
                \begin{tikzcd}
                    Mod^B \ar[]{r}[]{f^*} & Mod^A \ar[bend right]{l}[above]{f_!}
                \end{tikzcd}
            \end{center}

            Dually, if $C$ and $D$ are two coalgebras and $g : C \rightarrow D$ is an coalgebra morphism. Then $g$ induces a functor between the module categories by composing: $g* : CoMod^C \rightarrow CoMod^D$. Since $C$ and $D$ considered as coalgebroids are small, and the category of abelian groups is complete, so the right Kan extension (co-induction) along this functor exists.
            \begin{center}
                \begin{tikzcd}
                    CoMod^C \ar[]{r}[]{g_*} & CoMod^D \ar[bend left]{l}[above]{g^!}
                \end{tikzcd}
            \end{center}

            \begin{lemma}
                Let $\tau : C \rightarrow A$ be a twisting morphism. The adjunction $(L_\tau, R_\tau)$ factors as $(f_{\tau !}, f_\tau^*)\circ (L_{\iota_C},R_{\iota_C})$ or $(L_{\pi_A},R_{\pi_A})\circ (g_{\tau *}, g_\tau^!)$.
            \end{lemma}

            \begin{proof}
                This follows from corollary \ref{cor: universal-twisting}, that is $\tau = f_\tau \circ \iota_C = \pi_A\circ g_\tau$.
            \end{proof}

            \begin{definition}
                A twisting morphism $f: C \rightarrow A$ is called acyclic if the counit of the adjunction $L_\alpha \dashv R_\alpha$ is a pointwise quasi-isomorphism.
            \end{definition}

            \begin{lemma}
                Let $A$ be an augmented dg-algebra and $C$ a conilpotent dg-coalgebra. The universal twisting morphisms $\pi_A$ and $\iota_C$ are acyclic.
            \end{lemma}

            \begin{proof}
                
            \end{proof}

        \subsection{Model Structure on Module Categories}

            Let $A$ be an augmented dg-algebra, then we know that $Mod^A$ is a model category. By corollary \ref{cor: Model-Mod} we have a model structure where the fibrations, cofibrations and weak equivalences are given as follows:
            \begin{itemize}
                \item[Weak equivalences] $f: M \rightarrow N$ is a weak equivalence if $f$ is a quasi-isomorphism.
                \item[Fibration] $f : M\rightarrow N$ is a fibration if $f^\#$ is an epimorphism.
                \item[Cofibration] $f : M\rightarrow N$ is a cofibration if it has LLP to acyclic fibrations. 
            \end{itemize}

        \subsection{Model Structure on Comodule Categories}

            Let $A$ be an augmented dg-algebra, $C$ a conilpotent dg-coalgebra and $\tau : C \rightarrow A$ an acyclic twisting morphism. We endow $CoMod^C_{conil}$ with three classes of morphisms:
            \begin{itemize}
                \item[Weak equivalences] $f : M \rightarrow N$ is a weak equivalence if $L_\tau f$ is a quasi-isomorphism.
                \item[Cofibration] $f : M \rightarrow N$ is a cofibration if $f^\#$ is a monomorphism.
                \item[Fibration] $f : M \rightarrow N$ is a fibration if it har RLP to acyclic cofibrations.
            \end{itemize}

            \begin{thm}
              The category $CoMod^C_{conil}$ with the three classes as above forms a model category. Every object is cofibrant, and those objects which is a direct summand of $R_\tau M$ for some $M\in Mod^A$ are fibrant. The adjoint pair $(L_\tau, R_\tau)$ is a Quilllen equivalence. Finally, the model structure of $CoMod^C_{conil}$ is independent of the choice of $\tau$.
            \end{thm}

            We will call this model structure for the canonical model structure on $CoMod^C_{conil}$.

        \subsection{Triangulation of HoCoMod}

        \subsection{The Fundamental Theorem of Twisting Morphisms}

    \section{Polydules}
\end{document}