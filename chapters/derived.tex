\documentclass[../thesis.tex]{subfiles}

\begin{document}

    In this chapter we wish to study the derived categories of $A_\infty$-algebras. This category lies at the heart of homological algebra, so it is only natural to ask how this category looks like in the case of an $A_\infty$-algebra. In the last chapter we studied the relationship between the category of dg-algebras and dg-coalgebras to understand how quasi-isomorphisms between $A_\infty$-algebras worked. In this chapter we will instead study the relationship between module and comodule categories in order to understand how quasi-isomorphisms between $A_\infty$-modules will work. Twisting morphisms $\alpha : C \rightarrow A$ will reappear, and they allow us to study the relationship between $\tt{Mod}^A$ and $\tt{coMod}^C$.

    From twisting morphisms we obtain functors $L_\alpha : \tt{coMod}^C \rightarrow \tt{Mod}^A$ and $R_\alpha : \tt{Mod}^A \rightarrow \tt{coMod}^C$, which creates an adjoint pair of functors. Whenever the twisting morphism $\alpha$ is acyclic, this adjoint pair will in fact become a Quillen Equivalence.

    We wish to reuse all of the methods we have gained and acquired thorughout this thesis. The first part of this chapter will mostly be reformulations and recontextualizations of previous definitions, concepts and techniques. 

    \section{Twisting Morphisms}

        Twisting morphisms were already introduced in chapter 1. There, they were used mostly to be represented by the bar and cobar construction. Now we want twisting morphisms and twisting tensors to play a bigger role. In order to define the functors $L_\alpha$ and $R_\alpha$, the choice of a given morphism will be crucial.  

        \subsection{Twisted Tensor Products}

            Let $A$ be an augmented dg-algebra, $C$ a conilpotent dg-coalgebra and $\alpha : C \rightarrow A$ a twisting morphism. The right (left) twisted tensor product is the complex $C \otimes_\alpha A$ ($A\otimes_\alpha C$) together with the differential $d_\alpha^\bullet = d_{C\otimes A}^\bullet + d_\alpha^r$. The perturbation is defined as
            \begin{align*}
                d_\alpha^r = (\nabla_A\otimes id_C) \circ (id_A \otimes \alpha \otimes id_C) \circ (id_A \otimes \Delta_C).
            \end{align*}

            If $M$ is a right $A$-module and $N$ is a left $C$-comodule then the tensor product $M\otimes_\mathbb{K} N$ exists and is a $\mathbb{K}$-module with differential $d_{M\otimes N}$. We may define a perturbation to this differential as 
            \begin{align*}
                d_\alpha^r = (\mu_M\otimes id_N) \circ (id_M \otimes \alpha \otimes id_N) \circ (id_M \otimes \nu_N).
            \end{align*}
            By using the same line of thought as in Proposition~\ref{prop: twisted-differential}, there is a twisted tensor product $M\otimes_\alpha N$ with differential $d_\alpha^\bullet = d_{M\otimes N} + d_\alpha^r$.

            \begin{remark}
                Koszul's sign rule forces us to define the differential of the left twisted tensor product as $d_\alpha^\bullet = d_{N\otimes M} - d_\alpha^l$. This is due to the skewness of left twisted tensor products.
            \end{remark}
            
            \begin{definition}
                Suppose that $M\in \tt{Mod}^A$ ($M\in \tt{Mod}_A$) and $N\in \tt{coMod}_C$ ($N\in \tt{coMod}^C$), then the left (right) twisted tensor product is the $\mathbb{K}$-module $M\otimes_\alpha N$ ($N\otimes_\alpha M$).
            \end{definition}

            In this setting, right handedness and left handedness for the twisted tensor product is clearer, as we only have an action or coaction from one of the chosen sides. Trying to force the other handedness on the twisted tensors would just be ill-defined.

            \begin{definition}
                Let $A$ be an augmented dg-algebra and $C$ a conilpotent dg-coalgebgra, such that there is a twisting morphism $\alpha: C\rightarrow A$. Given a linear map $f: N \rightarrow M$ between a right $C$-comodule $N$ and a right $A$-module $M$ we say that it is an $\alpha$ right twisted linear homomorphism, or just an  $\alpha$ twisted morphism, if it satisfies
                \begin{align*}
                    \partial f - f\star \alpha = 0\tt{.}
                \end{align*} 
            \end{definition}

            This definition essentially describes a functor $\tt{Tw}_\alpha^r : \tt{coMod}^C \times \tt{Mod}^A \rightarrow \tt{Mod}_\mathbb{K}$, which is the collection of right twisted linear homomorphisms between a comodule and module.

            Suppose that $\alpha : C \rightarrow A$ is a twisting morphism. Define the functor $L_\alpha = \otimes_\alpha A : \tt{coMod}^C \rightarrow \tt{Mod}^A$ as an arbitrary right twisted tensor product with $A$. This functor does indeed hit $\tt{Mod}^A$ by using the free right $A$-module structure on $A$. Likewise, we define a functor $R_\alpha = \otimes_\alpha C : \tt{Mod}^A \rightarrow \tt{coMod}^C$ as an arbitrary left twisted tensor product with $C$. This does also hit right $C$-comodules by using the cofree right $C$-comodule structure on $C$.

            \begin{proposition}
                Suppose that $\alpha : C \rightarrow A$ is a twisting morphism. The functor $L_\alpha$ and $R_\alpha$ form an adjoint pair of categories.
                \begin{center}
                    \begin{tikzcd}
                        \tt{coMod}^C \ar[bend left]{r}[]{L_\alpha} \ar[phantom]{r}[]{\bot} & \tt{Mod}^A \ar[bend left]{l}[]{R_\alpha}
                    \end{tikzcd}
                \end{center}
            \end{proposition}

            \begin{proof}
                This proof breaks down to showing $\tt{coMod}^C(N, L_\alpha(M))\simeq \tt{Tw}_\alpha(N,M) \simeq \tt{Mod}^A(R_\alpha(N),M)$. This is a rutine calculation, much like the proof for \ref{thm: cobar-bar-adj}.
            \end{proof}

            Let $A$ be a dg-algebra, and $M$ a right $A$-module. Recall that by the Cobar-Bar adjunction, Theorem~\ref{thm: cobar-bar-adj}, there exists a universal twisting morphism $\pi_A : BA \rightarrow A$. We define the bar construction of $M$ as $B_AM = R_{\pi_A}M = M\otimes_{\pi_A}BA$. Likewise, given a conilpotent dg-coalgebra $C$ and $N$ a right $C$-comodule we define the cobar construction as $\Omega_CN = L_{\iota_C}N = N\otimes_{\iota_C}\Omega C$. In these cases we obtain adjunctions $\Omega_{BA} \dashv B_A$ and $\Omega_C \dashv B_{\Omega C}$.

            Let $A$ and $B$ be two algebras and $f : A \rightarrow B$ is an algebra morphism. Then $f$ induces a functor between the module categories by restriction: $f^* : \tt{Mod}^B \rightarrow \tt{Mod}^A$. Since $A$ and $B$ considered as algebroids are small, and the category of abelian groups is cocomplete, the left Kan extension (induction) along this functor exists.
            \begin{center}
                \begin{tikzcd}
                    \tt{Mod}^B \ar[]{r}[]{f^*} & \tt{Mod}^A \ar[bend right]{l}[above]{f_!}
                \end{tikzcd}
            \end{center}

            Dually, if $C$ and $D$ are two coalgebras and $g : C \rightarrow D$ is an coalgebra morphism. Then $g$ induces a functor between the module categories by composing: $g* : \tt{coMod}^C \rightarrow \tt{coMod}^D$. Since $C$ and $D$ considered as coalgebroids are small, and the category of abelian groups is complete, the right Kan extension (co-induction) along this functor exists.
            \begin{center}
                \begin{tikzcd}
                    \tt{coMod}^C \ar[]{r}[]{g_*} & \tt{coMod}^D \ar[bend left]{l}[above]{g^!}
                \end{tikzcd}
            \end{center}

            \begin{lemma}\label{lem: twist-fac}
                Let $\tau : C \rightarrow A$ be a twisting morphism. The adjunction $(L_\tau, R_\tau)$ factors as $(f_{\tau !}, f_\tau^*)\circ (L_{\iota_C},R_{\iota_C})$ or $(L_{\pi_A},R_{\pi_A})\circ (g_{\tau *}, g_\tau^!)$.
            \end{lemma}

            \begin{proof}
                This follows from Corollary~\ref{cor: universal-twisting}, that is $\tau = f_\tau \circ \iota_C = \pi_A\circ g_\tau$.
            \end{proof}

            \begin{definition}
                A twisting morphism $f: C \rightarrow A$ is called acyclic if the counit of the adjunction $L_\alpha \dashv R_\alpha$ is a pointwise quasi-isomorphism.
            \end{definition}

            \begin{lemma}\label{lem: uni-twist-ac}
                Let $A$ be an augmented dg-algebra and $C$ a conilpotent dg-coalgebra. The universal twisting morphisms $\pi_A$ and $\iota_C$ are acyclic.
            \end{lemma}

            \begin{proof}
                We start with $\pi_A$. Recall that $\pi_A$ is constructed as the twisting morphism corresponding to $id_{BA}$. This morphism is thus given as the projection onto the first dimension of $BA$, that is:
                \begin{align*}
                    & \pi_A{sa} = a \\
                    & \pi_A(sa\otimes ...) = 0
                \end{align*}

                We say that $\pi_A$ is acyclic if the counit $\varepsilon : L_{\pi_A}R_{\pi_A} \Rightarrow Id_{\tt{Mod}^A}$ at each object $M$ is a quasi-isomorphism.

                For each $M$ in $\tt{Mod}^A$, $L_{\pi_A}R_{\pi_A}M = M\otimes_{\pi_A}BA\otimes_{\pi_A}A$. We may split up the differential into two summands, $d_v$ and $d_h$. $d_v$ is the ordinary differential on the tensor product, while $d_h = (-d^l_{\pi_A}\otimes A) + M\otimes d_2 \otimes A + d^r_{\pi_A}$. Since $(d_v + d_h)^2 = 0$ and $d_v^2 = 0$ we can observe that $d_vd_h = -d_hd_v$ and $d_h^2 = 0$. This is evident as $d_v$ changes the homological degree while $d_h$ does not, so in order for both of the first equations to hold, the last two must hold as well. We almost obtain a double complex.
                \begin{center}
                    \begin{tikzcd}
                        ... \ar[]{r}[]{d_h} & M\otimes BA_{(i)}\otimes A \ar[]{r}[]{d_h} \ar[loop, distance = 2em]{}[above]{d_v} & ... \ar[]{r}[]{d_h} & M\otimes BA_{(1)}\otimes A \ar[]{r}[]{d_h} \ar[loop, distance=2em]{}[above]{d_v} & M \otimes A \ar[]{r}[]{0} \ar[loop, distance=2em]{}[above]{d_v} & 0
                    \end{tikzcd}
                \end{center}
                It is clear that the total complex of this "double complex" is in fact $L_{\pi_A}R_{\pi_A}M$. Moreover, the counit induces an augmentation to this complex resolution of $M$, denoted as $\tt{cone}(\varepsilon_M)$. 
                \begin{center}
                    \begin{tikzcd}
                        ... \ar[]{r}[]{d_h} & M\otimes BA_{(i)}\otimes A \ar[]{r}[]{d_h} \ar[loop, distance = 2em]{}[above]{d_{M\otimes BA_{(i)}\otimes A}} & ... \ar[]{r}[]{d_h} & M\otimes BA_{(1)}\otimes A \ar[]{r}[]{d_h} \ar[loop, distance=2em]{}[above]{d_{M\otimes BA_{(1)}\otimes A}} & M \otimes A \ar[]{r}[]{\varepsilon_M} \ar[loop, distance=2em]{}[above]{d_{M\otimes A}} & M \ar[loop, distance = 2em]{}[above]{d_M} \ar[]{r}[]{0} & 0
                    \end{tikzcd}
                \end{center}
                
                To see that this is in fact a resolution we define a morphism $h : \tt{cone}(\varepsilon_M) \rightarrow \tt{cone}(\varepsilon_M)$ of degree $-1$. It works by the following formula:
                \begin{align*}
                    h(m \otimes (sa_1 \otimes ... \otimes sa_n) \otimes a) = m \otimes (sa_1 \otimes ... \otimes sa_n \otimes sa) \otimes 1
                \end{align*}
                It is clear that $id_{\tt{cone}(\varepsilon_M)} = d_hh - hd_h$ and $d_vh = hd_v$. Thus to see that the cone is acyclic we let $c\in \tt{cone}(\varepsilon_M)$ be a cycle, that is $(d_v + d_h)(c) = 0$. Our goal is to show that $h(c)$ is a preimage of $c$ along $d_v + d_h$.
                \begin{multline*}
                    (d_v + d_h)\circ h(c) = d_v\circ h(c) + d_h\circ h(c) = h\circ d_v(c) + c + h\circ d_h(c) = h\circ (d_v + d_h)(c) + c = c
                \end{multline*}

                Next up we show that $\iota_C$ is acyclic. Equip $C$ with its coradical filtration, then this induces a filtration $F_p\Omega C$. We will freely use $|\argument |$ to denote the filtered degree of every element. This may lead to ambiguity, but we wish to use this notation for clarity.
                \begin{align*}
                    Fr_pC & = \startset{c \mid |c| \leq p} \\
                    f_p\Omega C & = \startset{\langle c_1 \mid \cdots \mid c_n \rangle \mid |c_1| + \cdots + |c_n| \leq p }
                \end{align*}
                Let $M \in \tt{Mod}^{\Omega C}$, we equip this module with a trivial filtration,
                \begin{align*}
                    F_pM = M\tt{.}
                \end{align*}

                $M$'s associated graded is then quite trivial, $\tt{gr}_0M \simeq M$ and every other $\simeq 0$.

                All of these three filtrations together induces a filtration on $L_{\iota_C}R_{\iota_C}M$,
                \begin{align*}
                    F_pL_{\iota_C}R_{\iota_C}M = \startset{m \otimes c \otimes \langle c_1 \mid \cdots \mid c_n \rangle \mid |m| + |c| + |c_1| + \cdots + |c_n| \leq p}\tt{.}
                \end{align*}

                We calculate the associated graded of this module.
                \begin{align*}
                    \tt{gr}_0L_{\iota_C}R_{\iota_C}M & \simeq M \\
                    \tt{gr}_pL_{\iota_C}R_{\iota_C}M & \simeq \bigoplus_{i_1 + i_2 = p} M \otimes \tt{gr}_{i_1}C \otimes \tt{gr}_{i_2}\Omega C
                \end{align*}

                The graded counit $\tt{gr}_p\varepsilon : \tt{gr}_pL_{\iota_C}R_{\iota_C}M \rightarrow \tt{gr}_pM$ becomes the identity on $M$ when $p = 0$. To see that $\tt{gr}\varepsilon$ is a quasi-isomorphism, it is enough to show that $\tt{gr}_pL_{\iota_C}R_{\iota_C}M$ is acyclic for every $p \geq 1$.

                Consider the graded differential component $\tt{gr}_pd^l_{\iota_C}$ when it acts as a morphism $\tt{gr}_{i_1}C \otimes \tt{gr}_{i_2}\Omega C \rightarrow \tt{gr}_{i_1+i_2}\Omega C$. This can be considered as a morphism 
                \begin{align*}
                    \rho : \bigoplus_{i_1 + i_2 = p} \tt{gr}_{i_1}C[-1] \otimes \tt{gr}_{i_2}\Omega C \rightarrow \tt{gr}_{p}\Omega C\tt{,}
                \end{align*}
                which is an isomorphism by reversing the operation.
                \begin{align*}
                    \rho (sc \otimes \langle \cdots \rangle) = \langle c \mid \cdots \rangle\tt{,} \\
                    \rho^{-1}(\langle c \mid \cdots \rangle) = sc \otimes \langle \cdots \rangle\tt{.}
                \end{align*}

                Since $\rho$ is an isomorphism, $\tt{cone}(\rho)$ is then acyclic. By construction, we have that \\ $\tt{cone}(\rho) \simeq \tt{gr}_pL_{\iota_C}R_{\iota_C}M$.

            \end{proof}

        \subsection{Model Structure on Module Categories}

            Let $A$ be an augmented dg-algebra, then we know that $\tt{Mod}^A$ is a model category. By Corollary~\ref{cor: Model-Mod} we have a model structure where the fibrations, cofibrations and weak equivalences are given as follows:
            \begin{itemize}
                \item $f\in \tt{Ac}$ is a weak equivalence if $f$ is a quasi-isomorphism,
                \item $f\in \tt{Fib}$ is a fibration if $f^\#$ is an epimorphism,
                \item $f\in \tt{Cof}$ is a cofibration if it has LLP to acyclic fibrations. 
            \end{itemize}

            Every object in this category is fibrant as the morphism $0 : M \rightarrow 0$ is always an epimorphism.

        \subsection{Model Structure on Comodule Categories}

            Unless stated otherwise, for this section we fix $A$ to be an augmented dg-algebra, $C$ as a conilpotent dg-coalgebra and $\tau : C \rightarrow A$ as an acyclic twisting morphism. We endow $\tt{coMod}^C_{\tt{conil}}$ with three classes of morphisms:
            \begin{itemize}
                \item $f\in \tt{Ac}$ is a weak equivalence if $L_\tau f$ is a quasi-isomorphism.
                \item $f\in \tt{Cof}$ is a cofibration if $f^\#$ is a monomorphism.
                \item $f\in \tt{Fib}$ is a fibration if it har RLP to acyclic cofibrations.
            \end{itemize}

            \begin{thm}\label{thm: model-comod}
              The category $\tt{coMod}^C_{\tt{conil}}$ with the three classes as above form a model category. Every object is cofibrant, and those objects which is a direct summand of $R_\tau M$ for some $M\in \tt{Mod}^A$ are fibrant. The adjoint pair $(L_\tau, R_\tau)$ is a Quilllen equivalence.
            \end{thm}

            We will call this model structure for the canonical model structure on $\tt{coMod}^C_{\tt{conil}}$. Under the hypothesis of this theorem, we may observe that every object of $\tt{coMod}^C_{\tt{conil}}$ is cofibrant. Since every $M\in \tt{Mod}^A$ is fibrant, and $R_\tau$ preserves fibrant objects we known that $R_\tau M$ is fibrant as well. Clearly every direct summand of $R_\tau M$ is fibrant. If $N\in \tt{coMod}^C_{\tt{conil}}$ is fibrant, then it is a direct summand of $R_\tau L_\tau N$. This shows that the bifibrant objects of $\tt{coMod}^C_{\tt{conil}}$ is exactly the thick image of $R_\tau$.

            To be able to prove this we will need some lemmata. This proof is essentially the same as the case for dg-coalgebras. The main difference is to show independence of the choice of twisting morphisms $\tau$. To this end we must establish the relationship between graded quasi-isomorphisms and weak equivalences, as well as a technical lemma.

            Recall that given a coaugmented coalgebra $C$ we have a filtration called the coradical filtration, defined as $Fr_iC = \tt{Ker}(\bar{\Delta}_C)^i$. If $N$ is a right $C$-comodule we may define the coradical filtration of $N$ as $Fr_iN = \tt{Ker}(\bar{\omega}_N^i)$. This filtration is admissable, meaning it is exhaustive and $Fr_0N=0$.

            \begin{lemma}
                Let $C$ be a conilpotent dg-coalgebra, $M$ and $N$ be right $C$-comodules. Then any graded quasi-isomorphism $f: M \rightarrow N$ is a weak equivalence.
            \end{lemma}

            \begin{proof}
                This proof is identical to Lemma~\ref{lem: graded-qif-are-w}.   
            \end{proof}

            \begin{lemma}
                Let $M$ and $N$ be two objects of $\tt{Mod}^A$. The functor $R_\tau$ sends a quasi-isomorphism $f: M \rightarrow N$ to a weak equivalence $R_\tau f: R_\tau M \rightarrow r_\tau N$.

                The unit of the adjunction $\eta : \tt{Id}_{\tt{coMod}^C} \rightarrow R_\tau L_\tau$ is a pointwise weak equivalence.
            \end{lemma}

            \begin{proof}
                $R_\tau f$ is a weak equivalence if $L_\tau R_\tau f$ is a quasi-isomorphism. By naturality of the counit we have the following commutative diagram.
                \begin{center}
                    \begin{tikzcd}
                        M \ar[]{d}[]{f} & L_\tau R_\tau M \ar[]{d}[]{L_\tau R_\tau f} \ar[]{l}[]{\varepsilon_M} \\
                        N & L_\tau R_\tau N \ar[]{l}[]{\varepsilon_N}
                    \end{tikzcd}
                \end{center}

                By assumption we know that all three of $f$, $\varepsilon_M$ and $\varepsilon_N$ are quasi-isomorphisms. It follows by the $2$-out-of-$3$ property that $L_\tau R_\tau f$ is a quasi-isomorphism as well.

                To show that $\eta : \tt{Id}_{\tt{coMod}} \rightarrow L_\tau R_\tau$ is a pointwise weak equivalence, we must show that $L\eta$ is a pointwise quasi-isomorphism. Since $L_\tau$ is left adjoint to $R_\tau$ we know that $\eta$ is split on the image of $L_\tau$, i.e.
                \begin{align*}
                    \varepsilon_{L_\tau}\circ L_\tau\eta = id_{L_\tau}
                \end{align*}
                Since we know that the natural isomorphisms $\varepsilon$ and $id$ are pointwise quasi-isomorphisms, we get by the $2$-out-of-$3$ property that $L\eta$ is a pointwise quasi-isomorphism as well.
            \end{proof}

            \begin{lemma}
                The functor $L_\tau$ preserves cofibrations and sends weak equivalences to quasi-isomorphisms.
            \end{lemma}

            \begin{proof}
                This proof is essentially the same as Lemma~\ref{lem: bar-cobar-Quill-adj}.
            \end{proof}

            With the above lemmata, we have now established that the adjunction $(L_\tau, R_\tau)$ form a Quillen equivalence if $\tt{coMod}^C$ is a model category.

            \begin{lemma}[{\cite[Lemma 2.2.2.9][74]{LefevreHasegawa03}}]
                Let $M$ be a right $A$-module and $N$ a right $C$-comodule. Let $p : M \rightarrow L_\tau N$ be a fibration of modules. The projection $j : R_\tau M \prod_{R_\tau L_\tau N} N \rightarrow R_\tau M$ is an acyclic cofibration of comodules.
            \end{lemma}

            \begin{proof}
                Let $K = \tt{Ker}p$. Then since $R_\tau$ is a right adjoint, it preserves kernels, so $R_\tau K \simeq \tt{Ker}R_\tau p$. Consider the pullback-square with the horizontal kernels
                \begin{center}
                    \begin{tikzcd}
                        R_\tau K \ar[tail]{r}[]{} \ar[]{d}[]{\simeq} & RM \prod_{R_\tau L_\tau N}N \ar[two heads]{r}[]{} \ar[]{d}[]{j} & N \ar[]{d}[]{\eta_N} \\
                        R_\tau K \ar[tail]{r}[]{} & R_\tau M \ar[]{r}[]{} & R_\tau L_\tau N 
                    \end{tikzcd}
                \end{center}

                Since $L_\tau N$ is a quasi-free module, we get that $M \simeq K \oplus L_\tau N$ as a graded module. In other words, the short exact sequences above are split when considered as graded sequences. If we apply $L_\tau$ this sequence, then $L_\tau$ has to preserve exactness at the graded level, since it is additive. Thus we obtain a morphism of exact sequences, and $L_\tau j$ is a quasi-isomorphism by $5$-Lemma.
                \begin{center}
                    \begin{tikzcd}
                        L_\tau R_\tau K \ar[tail]{r}[]{} \ar[]{d}[]{\simeq} & L_\tau (RM \prod_{R_\tau L_\tau N}N) \ar[two heads]{r}[]{} \ar[]{d}[]{j} & L_\tau N \ar[]{d}[]{\eta_N} \\
                        L_\tau R_\tau K \ar[tail]{r}[]{} & L_\tau R_\tau M \ar[]{r}[]{} & L_\tau R_\tau L_\tau N 
                    \end{tikzcd}
                \end{center}
            \end{proof}

            \begin{proof}[Proof of Theorem~\ref{thm: model-comod}]
                With the above lemmata established, this proof is identical to the proof of Theorem~\ref{thm: model-coalg}.
            \end{proof}

        \subsection{Triangulation of Homotopy Categories}
            In this section we will show that the homotopy categories are triangulated. If we look at the category $\tt{Mod}^A$ we will observe that the category $\tt{Ho} \tt{Mod}^A$ is our beloved derived category $\mathcal{D}(A)$. It is not quite the same for the cateogry $\tt{coMod}^C$. Here we want $\tt{Ho} \tt{coMod}^C$ to be equivalent to the derived category of a ring, so we will see that the derived category is a further localization of $\tt{Ho} \tt{coMod}^C$.

            Furthermore, by employing the theory of triangulated categories we will show that the model structure on $\tt{coMod}^C$ is independent on the choice of acyclic twisting morphism. This breaks down to show that every acyclic twisting morphism induce an equivalence between derived categories, as done by Bernhard Keller in \cite{Keller94}.

            $\tt{Mod}^A$ is a an abelian category, where we employ the maximal exact structure $\mathcal{E}'$ consisting of short exact sequences in $\tt{Mod}^A$. This translates to short exact sequences which is short exact in each degree. However, this category also has an exact structure $\mathcal{E}$ which makes $\tt{Mod}^A$ into a Frobenius category, which we will now describe.

            Let $f : M \rightarrow N$, be a chain map from $M$ to $N$. Then $\mathcal{E}$ contains a conflation on the form:
            \begin{center}
                \begin{tikzcd}
                    N \ar[tail]{r}[]{} & \tt{cone}(f) \ar[two heads]{r}[]{} & M[1]
                \end{tikzcd}
            \end{center}
            We define $\mathcal{E}$ to be the smallest exact structure on $\tt{Mod}^A$ which contains every conflation arising from a chain map $f$. Observe that these conflations are exactly the short exact sequences of $\tt{Mod}^A$ such that they are split when regarded as graded modules, i.e. forgetting the differential. Thus the smallest such $\mathcal{E}$ is exactly the collection of every conflation arising from a chain map $f$.

            Recall that an object $M$ is projective (injective) if the represented functor $\tt{Mod}^A(M,\argument)$ ($\tt{Mod}^A(\argument,M)$) is exact. For the category $(\tt{Mod}^A, \mathcal{E})$

            \begin{proposition}
                Let $M$ be an object of $\tt{Mod}^A$. The following are equivalent:
                \begin{itemize}
                    \item $M$ is $\mathcal{E}$-projective
                    \item $M$ is $\mathcal{E}$-injective
                    \item $M$ is contractible
                \end{itemize}
            \end{proposition}

            \begin{proof}
                This is a well known statement from literature. See Krause \cite{Krause21}, Happel \cite{Happel88} or Buehler \cite{Buhler10} for an account of this result.
            \end{proof}

            To see that $(\tt{Mod}^A, \mathcal{E})$ has both enough projectives and injectives we consider the following conflation:

            \begin{center}
                \begin{tikzcd}
                    M \ar[tail]{r}[]{} & \tt{cone}(id_M) \ar[two heads]{r}[]{} & M[1]
                \end{tikzcd}
            \end{center}

            The complex $\tt{cone}(id_M)$ is contractible for any complex $M$. By letting $M$ vary we can find an inflation or deflation from the identity cone to or from any complex. This concludes that $(\tt{Mod}^A, \mathcal{E})$ is a Frobenius category.
            
            Let \underline{$\tt{Mod}^A$} denote the injectively stable module category. Let $I(M,N)$ denote the set of chain maps from $M$ to $N$ which factors through an injective object. We define the injectively stable category as the quotient of abelian groups \underline{$\tt{Mod}^A$}$(M,N)=\sfrac{\tt{Mod}^A(M,N)}{I(M,N)}$.

            \begin{thm}
                Suppose that $(\mathcal{C},\mathcal{E})$ is a Frobenius category, then the injectively stable category \underline{$\mathcal{C}$} is triangulated. The additive auto-equivalence is given by cozyzygy, and the standard triangles is the image of the conflations into the quotient.
            \end{thm}

            \begin{proof}
                This is well known in literature. An account for it may also be found in Krause \cite{Krause21}, Happel \cite{Happel88} or Buehler \cite{Buhler10}.
            \end{proof}

            We thus obtain a triangulated category \underline{$\tt{Mod}^A$} associated to the Frobenius pair $(\tt{Mod}^A, \mathcal{E})$. This category is commonly denoted as $K(A)$, and we will do this as well. Notice that with the structure given by $\mathcal{E}$, the cozyzygy is defined by the shift functor $[1]$. Every standard triangle is also on the form:

            \begin{center}
                \begin{tikzcd}
                    M \ar[]{r}[]{f} & N \ar[]{r}[]{} & \tt{cone}(f) \ar[]{r}[]{} & M[1]
                \end{tikzcd}
            \end{center}

            To define the derived category $D(A)$ of $A$ we will consider the localization of $K(A)$ at the quasi-isomorphisms, $D(A) = K(A)[\tt{Qis}^{-1}]$. To see that the derived category is still triangulated we may realize it as a Verdier quotient of $K(A)$.

            \begin{proposition}
                The derived category of $A$ is equivalent to the Verdier quotient $\sfrac{K(A)}{\tt{Ac}}$, where $\tt{Ac}$ denotes the image of acyclic objects in $K(A)$.
            \end{proposition}

            \begin{proof}
                A proof may be found in Buehler \cite{Buhler10}.
            \end{proof}

            There is another way of telling the story of the derived category $D(A)$. That is to directly localize it at the quasi-isomorphisms. We may directly see that $D(A) \simeq \tt{Mod}^A[\tt{Qis}^{-1}]$ which we know is $\tt{Ho} \tt{Mod}^A$ by definition. This gives us our first important identification.

            \begin{thm}
                The homotopy category of $\tt{Mod}^A$ is triangulated, and moreover it is the derived category $D(A)$.
            \end{thm}

            \begin{proof}
                Follows from discussion above.
            \end{proof}

            The triangulated construction for the category $\tt{Ho} \tt{coMod}^C$ closely resembles that of $\tt{Ho} \tt{Mod}^A$. We start by studying the Frobenius pair $(\tt{coMod}^C, \mathcal{E})$, where $\mathcal{E}$ is the same exact structure. Notice that this exact structure only takes the underlying category of chain complexes into account, so this follows from the above description.

            We define the injectively stable category \underline{$\tt{coMod}^C$}$=K(C)$ in the same manner. The standard triangles and the additive auto-equivalence stays the same.

            At this point things start to differ. The definition for the homotopy category $\tt{Ho} \tt{coMod}^C$ is $\tt{coMod}^C[\tt{Ac}^{-1}]$, here $\tt{Ac}$ denotes the class of weak equivalences in $\tt{coMod}^C$. By abuse of notation we also let $\tt{Ac}\subset K(C)$ be the collection of objects which are cones of weak equivalences. This subcategory can be characterized by being the preimage of acyclic objects $\tt{Ac}\subset K(A)$ along $L_\tau : \tt{coMod}^C \rightarrow \tt{Mod}^A$. To see this look at the image of the triangle where the cone is in $\tt{Ac}$. This identification suffices to show that $\tt{Ac}\subset K(C)$ is a triangulated subcategory. In this manner $\tt{Ho} \tt{coMod}^C$ is the category $\sfrac{K(C)}{Ac}$, which is a triangulated category.

            \begin{remark}
                We may show that $\tt{Ac}\subset K(C)$ is a subcategory of acyclic objects. In this manner we get that $D(C) \simeq \tt{Ho} \tt{coMod}^C[\tt{Qis}^{-1}]$. This is done in Lefevre-Hasegawa as \cite[Proposition 1.3.5.1][51]{LefevreHasegawa03} \cite[Lemma 2.2.2.11][75]{LefevreHasegawa03}. This follows from the fact that we have an equivalence of categories $\tt{coMod}^C[\tt{fQis}^{-1}]\simeq \tt{Ho} \tt{coMod}^C$, here fQis means the collection of filtered quasi-isomorphisms. Since every filtered quasi-isomorphism is in fact a quasi-isomorphism by a spectral sequence argument we get the inclusion of triangulated subcategories $\langle \tt{cone}(\tt{fQis})\rangle \subseteq \langle \tt{cone}(\tt{Qis})\rangle \subseteq K(C)$.
            \end{remark}

            Let $\tau: C \rightarrow A$ and $\upsilon : C \rightarrow A'$ be two acyclic twisting morphisms. These independently defines two different model structures on $\tt{coMod}^C$ by the adjunctions $(L_\tau, R_\tau)$ and $(L_\upsilon, R_\upsilon)$. By Lemma~\ref{lem: twist-fac} we have the identification $(L_\tau, R_\tau) = (f_{\tau !},f_\tau^*)(L_{\iota_C},R_{\iota_C}) = (f_{\tau !}L_{\iota_C},R_{\iota_C}f_\tau^*)$, and likewise for $\upsilon$. In order to show that $\tau$ and $\upsilon$ defines equivalent model structures on $\tt{coMod}^C$, it is enough that both define the same structure as $\iota_C$. By symmetry, we may assume that $\upsilon = \iota_C$. From Lemma~\ref{lem: uni-twist-ac}, we know that $\iota_C$ is acyclic, so this assumption is well-founded.

            Since we already know that $(L_\tau, R_\tau)$ and $(L_{\iota_C}, R_{\iota_C})$ are Quillen equivalences, it remains to show that $(f_{\tau !},f_\tau^*)$ is a Quillen equivalence. This is shown if $f_\tau^*$ is a right Quillen functor, and that it induces a triangle equivalence between $D(A)$ and $D(\Omega C)$.

            We know that $f_\tau^*$ preserves fibrations (epimorphisms). This is because on morphisms, this functor acts as the identity. It only changes the ring action, so epimorphisms stay as epimorphisms. It remains to see that quasi-isomorphisms are preserved. We will show this by identifying the derived categories. This follows the methods given by Keller in \cite{Keller94}. We will however simplify this discussion by restricting our attention solely to dg-algebras.

            Let $A$ be a dg-algebra. $A$ is then free in the enriched sense; i.e. for any right $A$-module $M$, ${\tt{Hom}}^\bullet_A(A,M) \simeq M$. Recall that $P$ is projective if it is a direct summand of $A^n$ for some $n\in \mathbb{N}$. 
            
            Given a right bounded complex $M$, we know how to construct a projective resolution $p: pM \rightarrow M$. Associated to this resoultion there is a triangle in $K(\mathbb{K})$ consisting of the complexes $M$, $pM$ and $aM$, where $aM$ is an acyclic complex.

            \begin{center}
                \begin{tikzcd}
                    M \ar[]{r}[]{p} & pM \ar[]{r}[]{} & aM \ar[]{r}[]{} & M[1]
                \end{tikzcd}
            \end{center}

            In this sense we obtain an identification $M \simeq pM$ in $D(\mathbb{K})^-$. By following Kellers construction we are able to weaken this identification to all of $D(\mathbb{K})$, by weakening the structure of the projective resolution. In Kellers paper he calls these, complexes of property (P). We will however refer to them as homotopy projective complexes, since these complexes are built up from projective complexes in a manner respecting homotopy colimits.

            \begin{definition}
                Let $P$ be a complex of $\tt{Mod}^A$. We say that $P$ is homotopy projective if there exists a complex $P'$, a homotopy equivalence $P \simeq P'$ and a filtration of $P'$.
                \begin{align*}
                    0 = F_0 \subseteq F_1 \subseteq ... \subseteq F_n \subseteq ... \subseteq P'
                \end{align*}
                The filtration should satisfy these properties:
                \begin{itemize}
                    \item[F1] $P'$ is the colimit of the filtration.
                    \item[F2] Each inclusion $i_n : F_n \subseteq F_{n+1}$ is split as graded modules.
                    \item[F3] The quotient $\sfrac{F_{n+1}}{F_n}$ is projective.
                \end{itemize}
            \end{definition}

            \begin{remark}
                The properties $F1$ and $F2$ may be reformulated to require that $P'$ should be the homotopy colimit of the filtration. Thus there is a canonical triangle in $K(A)$:
                \begin{center}
                    \begin{tikzcd}
                        \bigoplus F_n \ar[]{r}[]{\Phi} & \bigoplus F_n \ar[]{r}[]{} & P' \ar[]{r}[]{} & \bigoplus F_p[1] 
                    \end{tikzcd}
                \end{center}
                $\Phi$ is given as the unique morphism which acts as the identity and the inclusion on each summand of $\bigoplus F_p$:
                \begin{align*}
                    \Phi_n = \begin{pmatrix}
                        id_{F_n} \\ -i_n
                    \end{pmatrix}
                \end{align*}
            \end{remark}

            In the definition of a homotopy projective complex we have required that each quotient is strictly projective. If only this was true, then these objects would be ill-behaved in the homotopy category. We can weaken this assumption to (F3'), the quotient $\sfrac{F_{n+1}}{F_n}$ is homotopy equivalent to a projective complex.

            \begin{lemma}\label{lem: homo-proj-homo-well-def}
                If $P$ is the colimit of a filtration admitting (F2) and (F3'), then $P$ is homotopy projective.
            \end{lemma}

            \begin{proof}
                Let $\startset{F_n}$ denote the filtration on $P$. Showing that $P$ is homotopy projective is the same as finding a homotopy equivalence to a complex $P'$, such that $P'$ is the homotopy colimit of a filtration admitting (F3).

                Suppose that $\sfrac{F_{n+1}}{F_n}\simeq Q_{n+1}$, where each $Q_{n+1}$ is projective. We wish to inductively define a filtration $\startset{F_n'}$ which has (F2) and (F3) and a pointwise homotopy equivalence of filtrations $f : \startset{F_n} \rightarrow \startset{F_n'}$. The object $P'$ is then defined to be the (homotopy)\todo{Merkelig formulert} colimit of this new filtration.
                
                Define $F_0' = Q_0$, and let $f_0 : F_0 \rightarrow F_0'$ be the projection onto $Q_0$. By assumption $f_0$ is a homotopy equivalence and we have a commutative square where the vertical arrows are homotopy equivalences. Moreover, each horizontal arrow splits as a graded arrow.

                \begin{center}
                    \begin{tikzcd}
                        0 \ar[]{r}[]{0} \ar[]{d}[]{0} & F_0 \ar[]{d}[]{f_0} \\
                        0 \ar[]{r}[]{0} & Q_0
                    \end{tikzcd}
                \end{center}

                Suppose that we have been able to constructed this filtration up to $F_p'$. By using our known homotopy equivalences there is an isomorphism of Ext groups:
                \begin{align*}
                    \tt{Ext}_A(\sfrac{F_{p}}{F_{p-1}}, F_{p-1})\simeq \tt{Ext}_A(Q_p, F_{p-1}')
                \end{align*}

                Given the triangle consisting of $F_{p-1}$, $F_p$ and $\sfrac{F_p}{F_{p-1}}$ there is an assosiated triangle with the morphisms as follows:
                \begin{center}
                    \begin{tikzcd}
                        F_{p-1} \ar[]{d}[]{f_{p-1}} \ar[]{r}[]{} & F_p \ar[]{r}[]{} \ar[dashed]{d}[]{}& \sfrac{F_p}{F_{p-1}} \ar[]{d}[]{\sim} 1
                        1
                        1
                        1\ar[]{r}[]{} & F_{p-1}[1] \ar[]{d}[]{f_{p-1}[1]} \\
                        F_{p-1}' \ar[]{r}[]{} & F_p' \ar[]{r}[]{} & Q_p \ar[]{r}[]{} & F_{p-1}' 
                    \end{tikzcd}
                \end{center}

                By the morphism axiom there is a morphism $f_p : F_p \rightarrow F_{p}'$ which is also a homotopy equivalence by the 2-out-of-3 property.

                This defines a filtration $\startset{F_p'}$, with (F3) and $P'$ as its homotopy colimit. To see that $P$ is homotopy equivalent to $P'$ we use the maps $f_p$ constructed to obtain a homotopy equivalence by the morphism axiom and the 2-out-of-3 property.

                \begin{center}
                    \begin{tikzcd}
                        \bigoplus F_p \ar[]{d}[]{\oplus f_p} \ar[]{r}[]{\Phi} & \bigoplus F_p \ar[]{d}[]{\oplus f_p} \ar[]{r}[]{} & P \ar[dashed]{d}[]{\sim} \ar[]{r}[]{} & \bigoplus F_p[1] \ar[]{d}[]{\oplus f_p[1]} \\
                        \bigoplus F_p' \ar[]{r}[]{\Phi '} & \bigoplus F_p' \ar[]{r}[]{} & P' \ar[]{r}[]{} & \bigoplus F_p'[1]
                    \end{tikzcd}
                \end{center}
            \end{proof}

            The projective complexes are the complexes which are generated by the free module $A$ in the sense that they are all in the smallest thick triangulated subcategory of $K(A)$ containing $A$. By definition, we may see that the homotopy projective complexes are the complexes in the smallest thick triangulated subcategory of $K(A)$ which is closed under well-ordered homotopy colimits and contains $K(A)$. By devissage we may extend fully-fatihfullness of functors on the set $\startset{A}$ to the class of homotopy projective objects.

            \begin{lemma}[Devissage]
                Let $F: \mathcal{T} \rightarrow \mathcal{U}$ be a triangulated functor between triangulated categories. Suppose $S\subseteq \mathcal{T}$ is a class of objects closed under shift, and denote $\langle S \rangle$ for the smallest thick triangulated subcategory (closed under well-ordered homotopy colimits). If $F|_S$ is fully faithful, then $F|_{\langle S \rangle}$ is fully faithful as well.
            \end{lemma}

            \begin{proof}
                This is straightforward by using the Yoneda embeddings and 5-lemma. More details may be found in \cite{Krause21}. To get closed under homotopy colimits we also need that $F$ commutes with infinite direct sums, and that the set $\startset{S}$ only contains small objects.
            \end{proof}

            \begin{lemma}
                Supppose we have $F$ and $S$ as above. If $F|_S = 0$, then it is $0$ on all of $\langle S \rangle$.
            \end{lemma}

            \begin{proof}
                The same argument as above, except we have to squeeze out zeros from exact sequences.
            \end{proof}

            The final ingredient to construct a homotopy projective resolution for our complexes is the acyclic assembly lemma \cite{Weibel94}.

            \begin{lemma}[acyclic assembly]
                Suppose that $C$ is a double complex of $R$-modules. Then $Tot^\oplus C$ is acyclic if either:
                \begin{itemize}
                    \item $C$ is a lower half-plane complex with exact rows.
                    \item $C$ is a left half-plane complex with exact columns.
                \end{itemize}
            \end{lemma}

            \begin{proof}
                This is proposition 2.7.3 in \cite{Weibel94}. We omit the proof as the next proof is in some sense very similar.
            \end{proof}

            \begin{corollary}\label{cor: ac-as-2}
                Suppose that $C$ is a double complex of $R$-modules such that every column is exact and that the kernels along the rows give rise to exact columns, then $Tot^\oplus C$ is acyclic.
            \end{corollary}

            \begin{proof}
                We want to realize the images along the rows as the coimage along the horizontal differential. Write $Z^n(C)$ for the n-th horizontal kernel and $B^n(C)$ for the n-th horizontal image. We have a short exact sequence of complexes:

                \begin{center}
                    \begin{tikzcd}
                        Z^n(C)^* \ar{r} & C^{n,*} \ar{r} & B^n(C)^*
                    \end{tikzcd}
                \end{center}

                Given that $C^{n,*}$ is acyclic we get that $Z^n(C)^*$ is acyclic if and only if $B^n(C)^*$ is acyclic.

                Assuming that all of these three constructions are acyclic we make a filtration on $C$. Let $F_nC^{p,*} = C$ if $p \in [-n, n-1]$, $F_nC^{n,*} = Z^nC$ and $F_nC^{p,*} = 0$ otherwise.

                This filtration is bounded below and exhaustive as colimits commute with colimits.
                \begin{align*}
                    Tot^\oplus C = Tot^\oplus \varinjlim F_nC \simeq \varinjlim Tot^\oplus F_nC
                \end{align*}
                We should be a bit careful here as the total complex is not really a coproduct, but since coproducts and cokernels are calculated pointwise we obtain the commutativity.

                Now we apply the classical convergence theorem to the filtration to obtain a converging spectral sequence $EF_2C \implies H^*(Tot^\oplus C)$. But since we assume each column to be exact in the filtration, the second page is $0$, so $H^*(Tot^\oplus C) \simeq 0$ as desired. 
            \end{proof}

            \begin{thm}
                Suppose that $P$ is homotopy projective, $N$ is acyclic. Then $K(A)(P, N)\simeq 0$.

                Given any module $M$, there is a homotopy projective object $pM$ and an acyclic object $aM$ giving rise to a triangle in $K(A)$.
                \begin{center}
                    \begin{tikzcd}
                        pM \ar{r} & M \ar{r} & aM \ar{r} & pM[1]
                    \end{tikzcd}
                \end{center}
            \end{thm}

            \begin{proof}
                We assume that $P \simeq A$. By a devissage argument we may extend the isomorphism to all homotopy projective $P$. 

                \begin{align*}
                    K(A)(A, N) \simeq H^0 {\tt{Hom}}^\bullet_A (A, N) \simeq H^0 N \simeq 0
                \end{align*}

                We want to construct two complexes $pM$ and $aM$ by taking total complexes. We show that $aM$ is acyclic by using Corollary~\ref{cor: ac-as-2}. To use it we will construct an exact sequence of complexes satisfying the assumptions. As described by MacLane \cite{MacLane94}, there is an exact structure $\mathcal{E}$ on $\tt{Mod}^R$ such that the collections on conflations are the short exact sequences such that the kernel functor is exact.
                \begin{center}
                    \begin{tikzcd}
                        L \ar[tail]{r}[]{f} & M \ar[two heads]{r}[]{g} & N \\
                        Z^*L \ar[tail]{r}[]{Z^*f} & Z^*M \ar[two heads]{r}[]{Z^*g} & Z^*N
                    \end{tikzcd}
                \end{center}

                Since limits commute with limits, the kernel functor preserves any limit. Thus the kernel is left exact, and its only obstruction for exactness is to preserve cokernels. We may thus characterize the conflations by inflations and deflations, which are monomorphisms and epimorphisms which are preserved by the kernel functor. Mac Lane calls these deflations for proper epimorphisms instead.

                We want to construct $\mathcal{E}$-projectives be on the form of homotopy projective complexes. $A[-n]$ is $\mathcal{E}$-projective by the following isomorphism.
                \begin{align*}
                    Z^0\tt{Hom}_A^\bullet (A[-n], M) \simeq M^n
                \end{align*}

                Define the trivialization $\tt{triv}M$ of $M$ be the underlying graded module $M$ endowed with a trivial differential. This trivial differential is in some sense the inclusion of graded modules into chain complexes. Thus we have the following isomorphism on hom-sets:
                \begin{align*}
                    Z^i\tt{Hom}_A^\bullet(\tt{triv}M, \tt{triv}N) \simeq \tt{Hom}_A^i(M, N)
                \end{align*}
                $\tt{triv}$ is then well-defined as a functor as every morphism between chain complexes uniquely defines a morphism between their trivializations. By using the isomorphisms from Keller \cite{Keller94} 2.2. we get that:

                \begin{multline*}
                    Z^0\tt{Hom}_A^\bullet (\tt{cone}(id_{\tt{triv}A}), M) \simeq Z^0\tt{Hom}_A^\bullet (\tt{cone}(id_{\tt{triv}A[-1]})[1], M) \\ \simeq \tt{Hom}_A^*(\tt{triv}A, \tt{triv}M[-1])^0 \simeq \tt{Hom}_A^*(A, M)^{-1} \simeq M^{-1}\tt{.}
                \end{multline*}

                This shows that if $P$ is homotopy projective, then $P$ and $\tt{cone}(id_{\tt{triv}P})$ are $\mathcal{E}$-projective. To see that there are enough $\mathcal{E}$-projectives pick an arbitrary module $M$. Since we know that there are enough projectives, let $P$ be a projective such that there is an epimorphism $p : P \rightarrow M$. We don't know if this morphism is a deflation, so pick another projective $Q$ such that there is an epimorphism $q : Q \rightarrow Z^*M$. Since $Z^*M$ has a trivial differential we know that $d_Qq = 0$. Thus this morphism extends to $q' = \begin{bmatrix}
                    q & 0
                \end{bmatrix} : \tt{cone}(id_{\tt{triv}Q}) \rightarrow M$ such that $Z^*q'$ is an epimorphism. The morphism $\begin{bmatrix}
                    p & q'
                \end{bmatrix} : P \oplus \tt{cone}(id_{\tt{triv}Q}) \rightarrow M$ is thus a deflation. $P' = P \oplus \tt{cone}(id_{\tt{triv}Q})$ shows that we have enough projectives. Moreover every $\tt{cone}(id_{\tt{triv}Q})$ is contractible, so $P' \simeq P$ in $K(A)$.

                Since we have enough $\mathcal{E}$-projective, we may construct a $\mathcal{E}$-projective resolution $P'^{*,*}$ of $M$ in the standard way. See Keller \cite{Keller90} for details. Such resolutions are then double complexes, and the augmented resolution below is $\mathcal{E}$-acyclic.

                \begin{center}
                    \begin{tikzcd}
                        ... \ar{r} & P'_1 \ar[]{r}[]{} & P'_0 \ar[two heads]{r}[]{} & M \ar[]{r}[]{0} & 0
                    \end{tikzcd}
                \end{center} 
                
                Having an $\mathcal{E}$-acyclic resolution means that each row is exact, and taking kernels along the columns preserve exactness of the rows.
                
                Denote the augmentation of $P'^{*,*}$ by $m : P'^{',*} \rightarrow M$. We define the complexes $pM = Tot^\oplus(P'^{*,*})$ and $aM = Tot^\oplus(\tt{cone}(m))$.
                
                $pM$ carries a natural filtration $F_npM$ from the double complex structure. Let $F_npM$ be the truncated complex:
                \begin{center}
                    \begin{tikzcd}
                        ... \ar[]{r}[]{} & 0 \ar[]{r}[]{} & P'^{n, *} \ar[]{r}[]{} & ... \ar[]{r}[]{} & P'^{1,*} \ar[]{r}[]{} & P'^{0,*} \ar[]{r}[]{} & 0 \ar[]{r}[]{} &  ...
                    \end{tikzcd}
                \end{center}
                
                The filtration $F_npM$ satisfies F1 and F2 by construction. The quotients $\sfrac{F_{n+1}pM}{f_npM} \simeq P'_n$ which is homotopy equivalent to a projective. By Lemma~\ref{lem: homo-proj-homo-well-def} $pM$ is homotopy projective.
                
                The complex $\tt{cone}(m)$ satisfies the conditions for Corollary~\ref{cor: ac-as-2}, thus $aM$ is acyclic and we have a triangle in $K(A)$ as desired.
            \end{proof}

            \begin{corollary}
                Let $M$ be an erbitrary module. If $P$ is homotopy projective, then $K(A)(P,M) \simeq K(A)(P, pM)$. If $N$ is acyclic, then $K(A)(M, N) \simeq (aM, N)$.

                $a$ and $p$ are well-defined functors which commutes with infinite direct sums. 
            \end{corollary}

            \begin{corollary}
                Let $\langle A \rangle$ denote the smallest thick triangulated subcategory of $D(A)$ which is closed under homotopy colimits and contains $\startset{A}$. Then $D(A) \simeq \langle A \rangle$.
            \end{corollary}

            \begin{corollary}\label{cor: ring-qiso-is-eq}
                Suppose that $f : A \rightarrow B$ is a dg-algebra homomorphism and a quasi-isomorphism between the dg-algebras, then $D(A) \simeq D(B)$.
            \end{corollary}

            \begin{proof}
                $f$ endows $B$ with both a left and right $A$-module structure. We will think of $B$ as a left $A$-module and right $B$ module. There is then a natural hom-tensor adjunction between the differential graded enriched categories.

                \begin{center}
                    \begin{tikzcd}
                        \tt{Mod}^A \ar[bend left]{r}[]{f_!} \ar[phantom]{r}[]{\bot} & \tt{Mod}^B \ar[bend left]{l}[]{f^*}
                    \end{tikzcd}
                \end{center}

                The restriction functor $f^*$ can naturally be identified with the hom functor $\tt{Hom}_A^\bullet(B,\argument)$, and then it is evident to realize $f_!$ as $\argument \otimes_A B$. In this way, $f_!(A) \simeq B$, so $f_! : \tt{Hom}_A^\bullet (A,A) \rightarrow \tt{Hom}_B^\bullet (B,B)$ is given by $f$. Since we assume $f$ to be a quasi-isomorphism, it follows that $\mathbb{L}f_! : D(A) \rightarrow D(B)$ is fully faithful on $\startset{A}$.

                By devissage the functor $\mathbb{L}f_!$ is fully-faithful on all of $D(A)$, since $D(A) \simeq \langle A \rangle$. As $f_!$ hits all of $D(B)$'s generators, $\mathbb{L}f_!$ is essentially surjective as well.
            \end{proof}

            \begin{remark}
                We have ignored smallness conditions for objects. This technique does not always work, since it depends on some unstated isomorphisms which we have implicitly used. We have these since the objects $A$ and $B$ are small. This detail is given more care in Keller \cite{Keller94}.
            \end{remark}
                
            With this result we are able to show that $\tt{Ho} \tt{Mod}^A$ and $\tt{Ho} \tt{Mod}^{\Omega C}$ are equivalent. Since we assumed the morphism $\tau: C \rightarrow A$ to be acyclic, we would expect the morphism $f_\tau : \Omega C \rightarrow A$ to be a quasi-isomorphism. If this is the case, we know that $D(\Omega C)\simeq D(A)$. 

        \subsection{The Fundamental Theorem of Twisting Morphisms}

            In this section we aim to finish what we started the previous section. We will prove a characterization for the acyclic twisting morphisms.

            \begin{thm}[Fundamental Theorem of Twisting Morphisms]\label{thm: thm-twist}
                Let $\tau : C \rightarrow A$ be a twisting morphism between augmented objects. The following are equivalent:
                \begin{enumerate}
                    \item $\tau$ is acyclic, i.e. the natural transformation $\varepsilon : L_\tau R_\tau \implies Id_{\tt{Mod}^A}$ is a pointwise quasi-isomorphism.
                    \item The unit transformation $\eta : Id_{\tt{coMod}^C} \implies R_\tau L_\tau$ is a pointwise weak equivalence.
                    \item The counit at $A$ is a quasi-isomorphism, i.e. $\varepsilon_A : L_\tau R_\tau A \rightarrow A$ is a quasi-isomorphism.
                    \item The unit at $\mathbb{K}$ is a weak equivalence, i.e. the algebra unit $\upsilon_A$ and coaugmentation $\upsilon_C$ assembles into a weak equivalence: $\upsilon_A \otimes \upsilon_C : \mathbb{K} \rightarrow A \otimes_\tau C$.
                    \item The morphism of algebras $f_\tau : \Omega C \rightarrow A$ is a quasi-isomorphism.
                    \item The morphism of coalgebras $g_\tau : C \rightarrow BA$ is a weak equivalence.
                \end{enumerate}
            \end{thm}

            \begin{proof}
                Notice that 1. is equivalent to 2. since $\mathbb{L}L$ and $\mathbb{R}R$ are quasi-inverse. 3. is a special case of 1. and 4. is a special case of 2. Observe that 5. and 6. are equivalent since the cobar-bar-adjunction is a Quillen equivalence, which is Corollary~\ref{cor: cobar-bar-quill-eq}.

                We show 3. $\implies$ 1. Let $\mathcal{T}\subseteq D(A)$ be the full subcategory consisting of objects M where $\varepsilon_M$ is a quasi-isomorphism. This subcategory is by assumption non-empty and contains $A$. By the $5$-lemma, making triangles (and smallness of $A$), this subcategory contains the smallest thick triangulated subcategory closed under homotopy colimits which contains $A$. We know this to be all of $D(A)$.

                To show 4. implies 5. we consider the twisting morphism $\iota_C$. Since $\iota_C$ is acyclic we know that the counit at $A$ is a quasi-isomorphism. 
                \begin{align*}
                    L_{\iota_C}R_{\iota_C}f_{\tau}^*A \rightarrow f_\tau^*A
                \end{align*}
                By assumption the unit morphism $\eta_\mathbb{K} : \mathbb{K} \rightarrow A \otimes_\tau C$ is a weak equivalence, so the morphism $L_{\iota_C}\eta_\mathbb{K} : \Omega C \rightarrow L_{\iota_C}R_\tau A = L_{\iota_C}R_{\iota_C}f_\tau^* A$ is a quasi-isomorphism. Let $\varepsilon'$ denote the counit of $L_{\iota_C} \dashv R_{\iota_C}$, then we see that $f_\tau = \varepsilon'_A \circ L_{\iota_C}\eta_\mathbb{K}$, so $f_\tau$ is a quasi-isomorphism by the 2-out-of-3 property.

                It remains to show that 5. implies 1. Let the counit of $f_{\tau *} \dashv f_\tau^*$ be denoted as $\tilde{\varepsilon}$. Since $f_\tau$ is a quasi-isomorphism, $f_\tau^*$ descends to an equivalence between the derived categories, which is Corollary~\ref{cor: ring-qiso-is-eq}. Thus $\tilde{\varepsilon} : f_{\tau !}f_\tau^* \implies \tt{Id}$ is a pointwise quasi-isomorphism. Observe that the counit factors as
                \begin{align*}
                    \varepsilon = \tilde{\varepsilon} \circ f_{\tau !}\varepsilon'_{f_\tau^*}
                \end{align*}
                By the 2-out-of-3 property it follows that $\varepsilon$ is a quasi-isomorphism.
            \end{proof}

            \begin{corollary}
                There is one canonical model structure on $\tt{coMod}^C$ defined by the acyclic twisting morphisms $\tau : C \rightarrow A$, for any algebra $A$. I.e. each acyclic twisting morphism defines the same model structure for $\tt{coMod}^C$. 
            \end{corollary}

            \begin{proof}
                Apply the fundamental theorem of twisting morphisms to the discussion of the last section.
            \end{proof}

    \section{Polydules}
        \subsection{The Bar Construction}
            In Section \ref{sec: 1.3} we saw that we could extend the domain of the bar construction to obtain an equivalence of categories. This converse led us to the definition of an $A_\infty$-algebra, as well as recognizing them as quasi-free dg-coalgebras. By employing the adjunction $(L_\tau, R_\tau) : \tt{coMod}^C \rightleftharpoons \tt{Mod}^A$ we can do something similar for modules.

            Let $A$ be an augmented dg-algebra. The bar construction of $A$ gives us a universal adjunction $(L_{\pi_A}, R_{\pi_A}) : \tt{coMod}^{BA} \rightleftharpoons \tt{Mod}^A$. We will call $R_{\pi_A}(\argument [1]) = \argument [1]\otimes_{\pi_A}BA$ for $B_A$, the bar construction on $\tt{Mod}^A$. In this manner every $A$-module $M$ give rise to a quasi-free $BA$-comodule $B_AM$, but does the converse of this construction work?

            Let us first look at what $B_A$ does to an $A$-module $M$. $B_AM$ is the dg-comodule which as a graded comodule is the free comodule $M[1] \otimes BA$. The differential of $B_AM$ is given by the $A$-module structure of $M$. That is, every elementary element $m'$ of $B_AM$ is an element of $M$ together with a finite string of elements of $A$.
            \begin{align*}
                m' = [ m \mid\mid a_1 \mid ... \mid a_n]
            \end{align*}
            The differential acts on $m'$ by using the differential of $d_{M[1]\otimes BA}$ and multiplication from the right.
            \begin{align*}
                d_{B_AM}(m') = d_{M[1]\otimes BA}(m') + (-1)^{|m|+|a|} [m\cdot a_1 \mid\mid a_2 \mid ... \mid a_n]
            \end{align*}
            By using delooping, we see that in turn that $d_{B_AM}$ defines an $A$-module structure for $M$. We may decompose $B_AM$ as:
            \begin{align*}
                B_AM = M[1] \oplus M[1] \otimes \bar{A} \oplus M[1] \otimes \bar{A}^{\otimes 2} \oplus ...
            \end{align*}
            Let $\pi_M : R_{\pi_A}M \rightarrow M$ be the linear map which kills anything not on the form $[m]$. We denote $(d_{B_AM})_i$ by $d_{B_AM} \circ \iota_i$, where $\iota_i : M[-1] \otimes \bar{A}^{\otimes i-1} \hookrightarrow B_AM$. Proposition \ref{prop: free-derivation} tells us that we may recover the structure of $M$ from the differential $d_{B_AM}$. This is done by conjugating the components of $d_{B_AM}$ with desuspension and applying projections appropriately. We recover the maps as:
            \begin{enumerate}
                \item The differential of $M$ is $d_M = s \circ \pi_{M[1]} \circ (d_{B_AM})_1 \omega$
                \item The right multiplication from $A$ is $\mu_M = s\circ \pi_{M[-1]} \circ (d_{B_AM})_2 \circ \omega^{\otimes 2}$
                \item For $i \geq 3$ we have $0 = s \circ \pi_{M[1]} \circ (d_{B_AM})_i \circ \omega^{\otimes i}$
            \end{enumerate}

            Now, let $\widetilde{N}$ be a quasi-free $BA$-comodule. That is, $\widetilde{N} = N[1] \otimes BA$ as a graded comodule. We would now like that $N$ carries an $A$-module structure. Unfortunately, this does not happen in general. However, like in the case of algebras, this defines a notion of $A_\infty$-modules to the algebra $A$. If we try to recover the same structure we obtain the following structure morphisms for $N$:
            \begin{align*}
                \text{A differential of degree }1\text{: }& m_1 = d_{N} = s\circ \pi_N (d_{\widetilde{N}})_1 \circ \omega\\
                \text{A 2-ary operation of degree }0\text{: }& m_2 = s\circ \pi_N (d_{\widetilde{N}})_2 \circ \omega^{\otimes 2}\\
                \text{A 3-ary operation of degree }-1\text{: }& m_3 = s\circ \pi_N (d_{\widetilde{N}})_3 \circ \omega^{\otimes 3}\\
                \text{A 4-ary operation of degree }-2\text{: }& \text{...}
            \end{align*}
            Let $\widetilde{m}_i$ be the looped versions of the $m_i$. Then the sum $\sum \widetilde{m}_i : \widetilde{N} \rightarrow N[1]$ extends to $d_{B_AN}$ by Proposition \ref{prop: free-derivation}, i.e. 
            \begin{align*}
                d_{B_AN} = (\sum \widetilde{m}_i \otimes id_{BA})(id_N \otimes \Delta_{BA}) + N[1]\otimes d_{BA}\tt{.}
            \end{align*}
            Since $d_{B_AN}^2 = 0$ we get the relations $(rel_n)$ as defined in Section \ref{sec: 1.3} imposed on the morphisms $m_i$. We summarize this in the next definition.

            \begin{definition}[$A$-polydule]
                Let $A$ be a dg-algebra and $M$ be a graded $\mathbb{K}$-module. We say that $M$ is a right $A$-polydule if there are morphisms
                \begin{align}
                    m_i : M \otimes A^{\otimes i - 1} \rightarrow M
                \end{align}
                of degree $|m_i| = 2 - i$ for any $i \geq 1$. Furthermore, the morphisms should satisfy the relations
                \begin{align*}
                    (rel_n) \qquad \partial(m_n) = - \sum_{\substack{n = p + q + r \\ k = p + 1 + r \\ k > 1, q > 1}}(-1)^{pq+r}m_k\circ_{p+1}m_q^?\tt{,}
                \end{align*}
                where $m_q^?$ is meant as either $m_q$ or $m_q^A$, that which is appropriate.
            \end{definition}
            
            A left $A$-polydule is defined analogously. If $M$ is an $A$-polydule, then it has the structure of an $A$-module where associativity is only well-defined up to strong homotopy. In other words, $m_3$ is a homotopy for the associator for $m_2$, $m_4$ is a homotopy for the associator of $m_3$ and so on. \todo{Er dette sant?}
            % Following Lefevre-Hasegawa \cite{LefevreHasegawa03}, we call the chain-complex $M$ an $A$-polydule, given it has maps $m_i$ as above.

            The category of $A$-polydules is denoted as $\tt{Mod}_\infty^A$. We have defined its objects in correspondance to the bar construction, thus every object has been uniquely defined from a quasi-free $B(A^+)$-comodule. Likewise, we will uniquely define every morphism to come from $B(A^+)$-comodule morphisms. In this manner $B_{A^+}$ defines a fully faithful functor $B_{A^+} : \tt{Mod}_\infty^A \rightarrow \tt{coMod}^{B(A^+)}$ which is an isomorphism on the full subcategory of quasi-free $B(A^+)$-comodules.
            
            \begin{definition}[$\infty$-morphisms]
                Let $A$ be a dg-algebra, and let $M$ and $N$ be two right $A$-polydules. We say that $f : M \rightsquigarrow N$ is an $\infty$-morphism if there are morphisms
                \begin{align*}
                    f_i : M \otimes A^{\otimes i - 1} \rightarrow N
                \end{align*}
                of degree $|f_i| = 1 - i$ for any $i \geq 1$. Furthermore, the morphism should satisfy the relations
                \begin{align*}
                    (rel_n)\qquad \sum_{p+q+r = n} (-1)^{pq+r}f_{p+1+r} \circ_{p+1} m^M_{q} = \sum_{p+q = n} m^N_{p+1} \circ_1 f_q
                \end{align*}
            \end{definition}
            
            %Our goal is to have that the converse bar construction defines an equivalence of categories, i.e. $B_A$ extends to a functor $B_A : Mod_\infty^A \rightarrow \tt{coMod}^{BA}$ is fully-faithful. This makes sense as every $A$-module $M$ is a non-full $A$-polydule by letting $m_1 = d_M$, $m_2 = \mu_M$ and $m_i = 0$ for any $i\geq 3$.

            Suppose that we have the $A$-polydules $M$, $N$ and $P$. If $f : M \rightsquigarrow N$ and $g : N \rightsquigarrow P$ are $\infty$-morphisms, then their composition is defined as
            \begin{align*}
                (gf)_n = \sum_{p+q=n}g_{p+1}\circ_1 f_q\tt{.}
            \end{align*}

            To illustrate what the bar construction does, suppose that $f : M \rightsquigarrow N$ is an $\infty$-morphism. The bar construction on $f$ is then defined as
            \begin{center}
                \begin{tikzpicture}[line cap=round,line join=round,>=triangle 45,x=1cm,y=1cm, thick, op/.style={circle, draw, scale=0.75}, scale=0.7]
                    \node [op, scale = 0.75] (f1) at (0.5, -0.5) {$b_{A^+}f$};

                    \draw [line width = 1pt] (0,1) -- (0,0) -- (f1) -- (0.5 , -1.5);
                    \draw [line width = 1pt] (1.5, 1) -- (1.5, 0.75) -- (1, 0.25) -- (1, 0) -- (f1);
                    \draw [line width = 1pt] (1.5, 0.75) -- (2, 0.25) -- (2, -1.5);
                \end{tikzpicture}
            \end{center}
            where $b_{A^+}f = \sum s \circ f_i \circ \omega^{\otimes i}$.

            There is a natural inclusion on objects $i : \tt{Mod}^A \rightarrow \tt{Mod}_\infty^A$. This functor acts as the identity on each object, letting every higher $m_i = 0$:
            \begin{align*}
                i : \tt{Mod}^A & \rightarrow \tt{Mod}_\infty^A\tt{,}\\
                (M, d_M, \mu_M) & \mapsto (M, d_M, \mu_M, 0, 0, \cdots)\tt{.}
            \end{align*}
            Suppose that $f : M \rightarrow N$ is a morphism between the $A$-modules $M$ and $N$. Then this defines an $\infty$-morphism $i\circ f : M \rightsquigarrow N$, such that $if_1 = f$ and $if_n = 0$ for every $n \geq 2$. Thus $i : \tt{Mod}^A \rightarrow \tt{Mod}_\infty^A$ is a functor.
            
            \begin{definition}[strict $\infty$-morphisms]
                Let $f : M \rightsquigarrow N$ be an $\infty$-morphism. We say that it is strict if $f_i = 0$ for every $i \geq 2$.
            \end{definition}

            The category $\tt{Mod}_{\infty, strict}^A$ is the non-full subcategory of $\tt{Mod}_{\infty}^A$ such that every $\infty$-morphism are strict. 
            % We see that the functor restricts $i : \tt{Mod}^A \rightarrow \tt{Mod}_{\infty, strict}^A$ and becomes an equivalence of categories as well.
            
            We will give some examples of $A$-polydules given an augmented algebra $A$.\todo{!!!}

        \subsection{Polydules of SHA-algebras}
            In the last section we developed the notion of a polydule to a augmented and ordinary algebras. We extend this notion to any $A_\infty$-algebra.

            Suppose that $A$ is an $A_\infty$-algebra. Recall the bar construction $BA$, this is a quasi-free coalgebra on the form
            \begin{align*}
                BA = \bigoplus_{i=1}^\infty A[1]^{\otimes i}\tt{,}
            \end{align*}
            where the differential comes from the $m_i : A^{\otimes i} \rightarrow A$. In order to define the $A$-polydules, we will consider the quasi-free comodules in $\tt{coMod}^{BA}$. This will be completely analogous to how it worked for ordinary dg-algbras.
            
            \begin{definition}[$A$-polydule]
                Let $A$ be an $A_\infty$-algebra, and $M$ a graded $\mathbb{K}$-module. We say that $M$ is a right $A$-polydule if there exists morphisms
                \begin{align*}
                    m_i : M \otimes A^{\otimes i - 1} \rightarrow M\tt{,}
                \end{align*}
                where the degree $|m_i| = 2 - i$ for any $i \geq 1$. Furthermore, the morphisms should satisfy the relations
                \begin{align*}
                    (rel_n) \qquad \partial(m_n) = - \sum_{\substack{n = p + q + r \\ k = p + 1 + r \\ k > 1, q > 1}}(-1)^{pq+r}m_k\circ_{p+1}m_q\tt{.}
                \end{align*}
            \end{definition}

            \begin{definition}[$\infty$-morphisms]
                Let $A$ be an $A_\infty$-algebra, and let $M$ and $N$ be two right $A$-polydules. We say that $f : M \rightsquigarrow N$ is an $\infty$-morphism if there are morphisms
                \begin{align*}
                    f_i : M \otimes A^{\otimes i - 1} \rightarrow N
                \end{align*}
                of degree $|f_i| = 1 - i$ for any $i \geq 1$. Furthermore, the morphism should satisfy the relations
                \begin{align*}
                    (rel_n)\qquad \sum_{p+q+r = n} (-1)^{pq+r}f_{p+1+r} \circ_{p+1} m^M_{q} = \sum_{p+q = n} m^N_{p+1} \circ_1 f_q
                \end{align*}
            \end{definition}

            %This is the collections of comodules of the form $M' = M[1] \otimes BA$. This will be an $A$-polydule. Since there is no obstruction to the above arguments, the differential $d_{M'}$ is determined by a collection of morphisms $m^M_n : M \otimes A^{\otimes n-1} \rightarrow M$ satisfying $(rel_n)$. Moreover, an $\infty$-morphism $f: M \rightsquigarrow N$ is a collection of morphisms $f_n : M \otimes A^{\otimes n-1} \rightarrow N$ satisfying $(rel_n)$.

            \begin{definition}
                Let $A$ be an $A_\infty$-algebra. The category $\tt{Mod}_\infty^A$ has $A$-polydules as objects and $\infty$-morphisms as morphisms.

            \end{definition}

            The quasi-isomorphisms in $\tt{Mod}_\infty^A$ are the $\infty$-morphisms $f$ such that $f_1$ is a quasi-isomorphism.

            \begin{remark}
                The isomorphisms of $\tt{Mod}_\infty^A$ are the $\infty$-morphisms $f$ where $f_1$ is an isomorphism.
            \end{remark}

            We say that an $\infty$-morphism is strict if $f_i = 0$ for any $i\geq 2$. The category $\tt{Mod}_{\infty, strict}^A$ is the non-full subcategory of $\tt{Mod}_\infty^A$ restricted to strict $\infty$-morphisms.

            Suppose now that $A$ is instead a strictly unital $A_\infty$-algebra, see Definition~(\ref{def: strict-unit}). We may define strictly unital $A$-polydules as an $A$-polydule $M$ such that
            \begin{align*}
                & m^M_2\circ (id_M \otimes \upsilon_A) = id_M \\
                \forall i\geq 3\quad & m^M_i\circ (id_M \otimes ... \otimes \upsilon_A \otimes ... \otimes id_A) = 0
            \end{align*}
            An $\infty$-morphism $f : M \rightsquigarrow N$ is strictly unital if
            \begin{align*}
                \forall i> 2 \quad & f_i(id_M \otimes ... \otimes \upsilon_A \otimes ... \otimes id_A) = 0 
            \end{align*}
            We define the categories of strictly unital polydules with stricly unital morphisms $\tt{suMod}_\infty^A$ and $\tt{suMod}_{\infty, strict}^A$. These categories are non-full subcategories of $\tt{Mod}_\infty^A$.

            Given an augmented $A_\infty$-algebra $A$, see Definition~(\ref{def: augmented-sha}), we obtain an equivalence of categories. Recall that the categories $\tt{Alg}_\infty$ and $\tt{Alg}_{\infty,+}$ were equivalent by taking the kernel of augmentation and applying the free augmentation as its quasi-inverse. In the same manner, given a strictly unital $A$-polydule $M$, then it defines a strictly unital $\bar{A}$-polydule $\bar{M}$ by restricting the structure maps to $\bar{A}^{\otimes n}$. This defines an equivalence of categories.
            \begin{center}
                \begin{tikzcd}
                    \tt{suMod}_{\infty}^A \ar[bend left]{r}[]{\overline{\underline{\phantom{A}}}}& \tt{Mod}_{\infty}^{\overline{A}} \ar[bend left]{l}[]{\argument^+}
                \end{tikzcd}
            \end{center}
            We may call its quasi-inverse for the free strict unitization. This takes an $\overline{A}$-polydule $M$ and turns it into a strictly unital $A$-polydule by defining the structure morphism as $0$ on the unit.

            The reduced bar construction allows us to translate an $A$-polydule $M$ to a quasi-free $BA$-comodule. We let $\overline{B}_AM = M[1] \otimes BA$, together with the differential coming from each $m_n : M \otimes A^{\otimes n - 1} \rightarrow M$
            \begin{align*}
                d_{\overline{B}_AM} = (\sum \widetilde{m}_i \otimes id_{BA})(id_{M[1]}\otimes \Delta_{BA}) + id_{M[1]}\otimes d_{BA} = d_m + id_{M[1]}\otimes d_{BA}\tt{.}
            \end{align*}
            Likewise, we may take a quasi-free $BA$-comodule to obtain an $A$-polydule by doing the reverse bar construction, as it is done in Proposition \ref{prop: free-derivation}.

            We will mostly restrict our attention to augmented $A_\infty$-algebras. The reason for this is that if $A$ is an arbitrary $A_\infty$-algebra, then studying $\tt{Mod}_\infty^A$ would be the same as studying $\tt{suMod}_\infty^{A^+}$. We extend the bar construction along this equivalence to a fully faithful functor $B_A : \tt{suMod}_\infty^A \rightarrow \tt{coMod}^{B\overline{A}}$. By abuse of equivalence we may write $B_{A^+} : \tt{Mod}_\infty^A \rightarrow \tt{coMod}^{BA}$.

            We may also lift homotopies between quasi-free $BA$-comodules and $A$-polydules. A homotopy $B_{A^+}h : B_{A^+}M \rightarrow B_{A^+}M$ is a morphism of degree $-1$. Thus the collection $h_n : M \otimes A^{\otimes n-1} \rightarrow N$ has morphisms of degree $-i$. Moreover, $h : M \rightsquigarrow N$ defines a homotopy of $f, g : M \rightsquigarrow N$ if we have
            \begin{align*}
                f_n - g_n = \sum_{p+q}(-1)^{p}m^N_{p+1}\circ_1 h_q - \sum_{p+q+r = n}(-1)^{pq+r}h_{p+1+r}\circ_{p+1} m^M_q
            \end{align*}

            We say that a homotopy is strictly unital if it is a strictly unital $\infty$-morphism.

        \subsection{Universal Enveloping Algebra}

            Given any augmented $A_\infty$-algebra $A$, there is a universal enveloping algebra $UA$. This algebra is universal in the sense that given any augmented algebra $A'$ and an $\infty$-morphism  $A' \rightarrow A$, then this factors through $UA$ by an algebra map $A' \rightarrow UA$. By the cobar-bar adjunction there is essentially only one way to define this algebra.

            \begin{definition}
                Let $A$ be an $A_\infty$-algebra. The universal enveloping algebra is the algebra defined as $\Omega BA$.
            \end{definition}
            \begin{remark}
                In this definition we have used the extended bar construction to $A_\infty$-algebras and the cobar construction on dg-coalgebras.
            \end{remark}

            \begin{lemma}\label{lem: Polydules-are-modules}
                There is an \textit{isomorphism} of categories $i : \tt{Mod}^{UA} \rightarrow \tt{suMod}_{\infty, strict}^{A}$ given by delooping.
            \end{lemma}
            \begin{proof}
                This is immediate by the definition of a $UA$-module. To have a $UA$-module $M[1]$, we must have structure maps $m_i^M : M \otimes A^{\otimes i-1} \rightarrow M$ of degree $2-i$ for any $i \geq 2$. Unwinding this definition and using the adjunction data establishes this isomorphism.
            \end{proof}

            We can generalize the universal enveloping algebra to the case of $A_\infty$-algebras. This construction is very non-trivial and requires to use the notion of the universal enveloping algebra relative to an operad. The necessary definitions may be found in Kriz and May's paper "Operads, algebras, modules and motives" \cite{Kriz95}.
            
            Given an $A_\infty$-algebra, we will denote its universal enveloping algebra $UA$. We have the following proposition due to Kriz and May.

            \begin{proposition}[{\cite[Proposition 4.10][19]{Kriz95}}]
                Let $A$ be an $A_\infty$-algebra. There is an equivalence of categories
                \begin{align*}
                    i : \tt{Mod}^{UA} \rightarrow \tt{suMod}_{\infty,\tt{strict}}^A\tt{.}
                \end{align*}
            \end{proposition}

            With the established equivalences we can now pull the model structure on $\tt{Mod}^{UA}$ onto $\tt{suMod}_{\infty,\tt{strict}}^A$. Recall that this is the model structure as defined in Theorem~\ref{thm: model-str-alg}.

        \subsection{Bipolydules}

            For ordinary algebras $A$ and $A'$, an $A$-$A'$-bimodule $M$ may serve as a kind of morphism from $\tt{Mod}^A$ to $\tt{Mod}^{A'}$. This is used in conjunction with the tensor product to form the correct functors. We will now look at this idea for $A_\infty$-algebras.

            \begin{definition}[$A$-$A'$-Bipolydule]
                Suppose that $A$ and $A'$ are $A_\infty$-algebras, and that $M$ is a graded $\mathbb{K}$-module. $M$ is an $A$-$A'$-bipolydule if there are morphisms
                \begin{align*}
                    m_{i,j} : A^{\otimes i} \otimes M \otimes {A'}^{\otimes j} \rightarrow M\tt{,}
                \end{align*}
                such that the degree $|m_{i,j}| = 1 - i - j$ for any $i,j \geq 0$. Furthermore, the morphisms should satisfy the relations
                \begin{align*}
                    (rel_n)\qquad \sum_{\substack{n = p + q + r \\ p + 1 + r = s + t \\ q = u + v \\ s,t,u,v \geq 0}}(-1)^{pq + r}m_{s,t}\circ_{p+1}m_{u,v} = 0
                \end{align*}
            \end{definition}

            \begin{definition}[Strictly Unital $A$-$A'$-Bipolydule]
                Suppose that $A$ and $A'$ are stricly unital $A_\infty$-algebras, and that $M$ is an $A$-$A'$-bipolydule. We say that $M$ is strictly unital if
                \begin{align*}
                    m_{i,j}(id^{\otimes p} \otimes \upsilon_? \otimes id^{\otimes q}) = 0\tt{;}
                \end{align*}
                where $?$ is either $A$ or $A'$, $p \neq i$ and $(i,j) \neq (0,1)$ nor $(i,j) \neq (1,0)$. Lastly,
                \begin{align*}
                    m_{1,0}(\upsilon_A \otimes id_M) = m_{0,1}(id_M \otimes \upsilon_{A'}) = id_M\tt{.}
                \end{align*}
            \end{definition}

            A morphism of bipolydules is a bit more complicated than for normal right polydules. This is because the left module structure induces some more signs.
            
            \begin{definition}[$\infty$-morphisms]
                Let $A$ and $A'$ be two $A_\infty$-algebras and let $M$ and $N$ be two $A$-$A'$-bipolydules. An $\infty$-morphism $f : M \rightsquigarrow N$ is a collection of morphisms
                \begin{align*}
                    f_{i,j} : A^{\otimes i} \otimes M \otimes {A'}^{\otimes j} \rightarrow N\tt{,}
                \end{align*}
                where the degree $|f_{i,j}| = -i-j$ for any $i,, \geq 0$. Furthermore, the morphisms should satisfy the following relations
                \begin{align*}
                    (rel_n)\qquad \sum_{\substack{n = p + q + r \\ q = s + t}}(-1)^{p(-s-t)}m_{p,q}\circ_{p+1}f_{s,t} = \sum_{n = p + q + r}(-1)^{pq + r}f_{p,r} \circ_{p+1} m^?_q\tt{,}
                \end{align*}
                where $m^?_q$ is meant to mean the appropriate structure morphism.
            \end{definition}

            This definition is well-defined. If $m^?_q$ is supposed to mean $m_{q_1,q_2} : A^{\otimes q_1} \otimes M \otimes B^{\otimes q_2} \rightarrow M$, then $q_1$ and $q_2$ are not uniquely determined. However, the sum will span every possibility of $q_1$ and $q_2$.

            We say that an $\infty$-morphism is strict if $f_{0,0}$ is the only non-zero component.

            The polydules assemble into categories $\tt{Mod}_{A,\infty}^{A'}$, $\tt{Mod}_{A,\infty,strict}^{A'}$, $\tt{suMod}_{A,\infty}^{A'}$ and $\tt{suMod}_{A,\infty,strict}^{A'}$ like in the usual sense. These definitions may seem somewhat more complicated. However, they almost reduce to the ordinary case by considering the category $\tt{coMod}^{BA^{op}\otimes BA'}$. We may derive a $2$-sided bar-construction $B_{A^+-{A'}^+} : \tt{Mod}_{A,\infty}^{A'} \rightarrow \tt{coMod}^{BA'}_{BA}$. However, we know that $\tt{coMod}_{BA}^{BA'} \simeq \tt{coMod}^{BA^{op}\otimes BA'}$. In this manner we may argue about bipolydules with the techniques we have developed for comodules.

        \subsection{A Tensor and Hom on $\tt{Mod}_\infty^A$}

            In order to understand the category $\tt{Mod}_\infty^A$ better we would like to construct a tensor product and a hom-functor on it. In its most generality, the tensor will be a bifunctor:
            \begin{align*}
                \argument \otimes^\infty_{A'} \argument : \tt{Mod}_{A,\infty}^{A'} \otimes \tt{Mod}_{A',\infty}^{A''} \rightarrow \tt{Mod}_{A,\infty}^{A''}\tt{.}
            \end{align*}
            In the usual sense, given a bipolydule $M \in \tt{Mod}_{A,\infty}^{A'}$, then it will act as a morphism
            \begin{align*}
                \argument \otimes_A^\infty M : \tt{Mod}_\infty^A \rightarrow \tt{Mod}_\infty^{A'}\tt{.}
            \end{align*}
            In particular, this functor will be a left adjoint to its correpsonding hom-functor. In its most general form, the hom functor will be a bifunctor:
            \begin{align*}
                \tt{Hom}^\infty_{A'} : \tt{Mod}_{A,\infty}^{A'}\otimes\tt{Mod}_{A',\infty}^{A''} \rightarrow \tt{Mod}_{A,\infty}^{A''}\tt{.}
            \end{align*}

            We start by describing the tensor product in the simplest case. Let $A$ be an $A_\infty$-algebra, and let $M$ and $N$ be a right and left $A$-polydule respectively. We define $M \otimes_A^\infty N$ as a cochain complex
            \begin{align*}
                M \otimes_A^\infty N = M \otimes T^c(A[1]) \otimes N\tt{.}
            \end{align*}
            Its structure comes from the cotensor product of quasi-free coalgebras. Consider instead the right and left $BA$ dg-comodules $B_{A^+}M = M[1] \otimes BA$ and $B_{A^+}N = BA \otimes N[1]$.
            \begin{align*}
                B_{A^+}M \square_{BA} B_{A^+}N = \tt{Ker}(\omega^r_{B_{A^+}M}\otimes B_{A^+}N - B_{A^+}M \otimes \omega^l_{B_{A^+}N})
            \end{align*}
            Then $B_{A^+}M \square_{BA} B_{A^+}N$ is a $\mathbb{K}$ dg-module. By taking the cotensor, we restrict our attention to solely those parts of this tensor in which comulitplication from the left is the same as comultiplication from the right. An element may then be seen to be of the form
            \begin{align*}
                & [m \mid\mid a_1 \mid \cdots \mid a_n] \otimes [n] \\ 
                + & [m \mid\mid a_1 \mid \cdots \mid a_{n-1}]\otimes [a_n \mid\mid n] \\
                + & \cdots \\
                + & [m \mid\mid a_1] \otimes [a_2 \mid \cdots \mid a_n \mid\mid n] \\ 
                + &
                 [m]\otimes [a_1\mid\cdots \mid a_n \mid\mid n]\tt{.}
            \end{align*}
            There is an evident isomorphism to $M[1]\otimes BA \otimes N[1]$ by sending each of the elements above to the elements
            \begin{align*}
                [m \mid\mid a_1 \mid \cdots \mid a_n \mid\mid n]\tt{.}
            \end{align*}
            Its differential is induced by the restriction of the differential on the cochain-complex $B_{A^+}M \otimes B_{A^+}N$. Since $d_{B_{A^+}M\otimes B_{A^+}N}$ is well-defined on each element in $B_{A^+}M \otimes B_{A^+}N$, the restricted differential $d_{m^M} \otimes id_{N[1]} + id_{M[1]}\otimes d_{BA} \otimes id_{N[1]} + id_{M[1]}\otimes d_{m^N}$ on $M[1] \otimes BA \otimes N[1]$ is well defined as well.

            \begin{definition}[The tensor product]
                Let $A$ be an $A_\infty$-algebra, and let $M$ and $N$ be respectively a right and a left $A$-polydule. The tensor $M \otimes^\infty_A N = M \otimes BA \otimes N$ is a cochain complex with differential
                \todo{Fortegn???}\begin{align*}
                    -(s\otimes id_{BA} \otimes s)(d_{m^M} \otimes id_{N[1]} + id_{M[1]}\otimes d_{BA} \otimes id_{N[1]} + id_{M[1]}\otimes d_{m^N})(\omega \otimes id_{BA} \otimes \omega)\tt{.}
                \end{align*}
                An element of $M \otimes^\infty_A N$ may be written on the form
                \begin{align*}
                    m[a_1 \mid \cdots \mid a_n]n\tt{.}
                \end{align*}
            \end{definition}

            Given $A$-polydules $M$, $M'$, $N$ and $N'$ and $\infty$-morphisms $f : M \rightsquigarrow M'$ and $g : N \rightsquigarrow N'$, we define $f \otimes^\infty_A g$ as
            \begin{align*}
                f\otimes^\infty_Ag(m[a_1 \mid \cdots \mid a_n]n) = \sum_{p+q+r = n+2} (-1)^s f_p(m, a_1, \cdots)[\cdots]g_r(\cdots, a_n, n)\tt{,}
            \end{align*}
            where $s$ is the appropriate sign derived from Koszul's sign rule. Note that as a $\mathbb{K}$-polydule, this morphism is a strict $\infty$-morphism. This fact will not change, even in the more general cases.

            We will extend this tensor to bipolydules. Suppose that $N$ now has the structure of an $A$-$A'$-bipolydule. The cotensor $B_{A^+}M \square_{BA} B_{A^+-A'^+}N \simeq (B_{A^+}M \square_{BA} B_{A^+}N) \otimes T^c(A'[1])$ as graded comodules. When we thus recover the structure morphisms, we may recover them at $T^c(A'[1])$. In other words, $m_{0,n} : N \otimes A'^{\otimes n-1} \rightarrow N$ induces morphisms $m_n : M \otimes^\infty_A N \otimes A'^{\otimes n-1} \rightarrow M \otimes^\infty_A N$. Thus, given a bipolydule such as $N$ we obtain a functor
            \begin{align*}
                \argument\otimes^\infty_AN : \tt{Mod}_\infty^A \rightarrow \tt{Mod}_\infty^{A'}\tt{.}
            \end{align*}

            We will now describe the hom functor in the simplest case. Let $A$ be an $A_\infty$-algebra, and let $M$ and $N$ be right $A$-polydules. We define $\tt{Hom}^\infty_A(M, N)$ as a cochain complex
            \begin{align*}
                \tt{Hom}^\infty_A(M,N) = \tt{Hom}_{BA}^\bullet(B_{A^+}M, B_{A^+}N)\tt{.}
            \end{align*}
            Its differential is the usual hom differential, i.e. given $f \in \tt{Hom}_{BA}^\bullet(B_{A^+}M,B_{A^+}N)$ then
            \begin{align*}
                \partial f = d_{B_{A^+}N}\circ f - (-1)^{|f|}f\circ d_{B_{A^+}M}\tt{.}
            \end{align*}

            Functoriality is given by post- and pre-composition in the usual sense for dg-comodules. If we are given $\infty$-morphisms, we will rather consider the dg-comodule counterpart and define functoriality purely through that. Because of this, when we regard this as $\mathbb{K}$-polydule, post- and pre-composition is a strict $\infty$-morphism. This will also hold in the more general case which we will soon get to. 

            In order to be able to get to a more complicated case, we first need a new way to encode the data of an $A$-polydule. The $\mathbb{K}$-module $\tt{Hom}_{BA}(B_{A^+}M,B_{A^+}N)$ carries a natural bimodule structure. There are actions on $\tt{Hom}_{BA}(B_{A^+}M,B_{A^+}N)$ on the right from the dg-endomorphism algebra $\tt{End}(B_{A^+}M)$, and on the left from $\tt{End}(B_{A^+}N)$ by composition. If we consider these dg-algebras as $A_\infty$-algebras, then we may give $\tt{Hom}_{BA}(B_{A^+}M,B_{A^+}N)$ the structure of a bipolydule.  The following lemma connects representations of $A_\infty$-algebras to $A$-polydules.

            \begin{lemma}[Representation lemma, {\cite[Lemme 5.3.0.1][140]{LefevreHasegawa03}}]\label{lem: rep}
                Let $A$ be an $A_\infty$-algebra, and let $M$ be a graded $\mathbb{K}$-module. The following are equivalent:
                \begin{itemize}
                    \item There is an $\infty$-morphism of $A_\infty$-algebras $\phi: A \rightsquigarrow \tt{End}(M)$,
                    \item $M$ is a left $A$-polydule.
                \end{itemize}
            \end{lemma}

            \begin{proof}
                We will only establish the bijection map. A proof of well-definedness may be found in the original source.
                
                The bijection is given the transpose of tensor. Notice that as $\mathbb{K}$-linear morphisms we have the following bijections
                \begin{align*}
                    \tt{Hom}_\mathbb{K}(A^{\otimes n-1}, \tt{End}(M)) \simeq \tt{Hom}_{\mathbb{K}}(A^{\otimes n-1}\otimes M, M)\tt{.}
                \end{align*}

                Thus if $\phi : A \rightarrow \tt{End}(M)$ is an $\infty$-morphism, then we may define
                \begin{align*}
                    m_n : A^{\otimes n-1} \otimes M & \rightarrow M \\
                    (a_1 \otimes \cdots \otimes a_{n-1})\otimes m & \mapsto \phi(a_1\otimes\cdots\otimes a_{n-1})(m)\tt{.}
                \end{align*}

                On the other hand, if we have structure morphisms $m_n : A^{\otimes n-1}\otimes M \rightarrow M$, then we may define $\phi$ by uncurrying:
                \begin{align*}
                    \phi_n : A^{\otimes n} & \rightarrow \tt{End}(M)\tt{,} \\
                    a_1 \otimes \cdots \otimes a_n & \mapsto (m \mapsto m_{n+1}(a_1 \otimes \cdots \otimes a_n \otimes m))\tt{.}
                \end{align*}
            \end{proof}

            \begin{remark}
                This lemma is well-known and even holds in many other aspects. One may for example recognize this as in representation theory of finite groups. A more general account on this lemma may be found as \cite[Proposition 5.2.2.][139]{Loday12}.
            \end{remark}

            \begin{corollary}
                Let $A$ and $A'$ be two $A_\infty$-algebras, and let $M$ be an $A$-$A'$-bipolydule. Then there is an $A_\infty$-morphism $\phi : A \rightsquigarrow \tt{End}(B_{A'^+}M)$. In particular, any $\tt{End}(B_{A'^+}M)$-modules is an $A$-polydule.
            \end{corollary}

            \begin{proof}
                By Lemma~\ref{lem: rep} we obtain the $\infty$-morphism $\phi : A \rightsquigarrow \tt{End}(B_{A'^+}M)$ by transposing the structure morphisms
                \begin{align*}
                    m_{i,j} : A^{\otimes i} \otimes M \otimes A'^{\otimes j} \rightarrow M\tt{.}
                \end{align*}
                In other words,
                \begin{align*}
                    \phi_n : A^{\otimes n} & \rightarrow \tt{End}(B_{A'^+}M)\tt{,} \\
                    a_1 \otimes \cdots \otimes a_n & \mapsto (\\
                    [m \mid\mid a'_1 \mid \cdots \mid a'_l] & \mapsto d_{B_{A^+-A'^+}M}\circ (\omega^{\otimes n}\otimes id_{M[1]} \otimes id_{A'[1]}^{\otimes l}) (a_1 \otimes \cdots \otimes a_n \otimes [m \mid\mid a'_1 \mid \cdots \mid a'_l]))\tt{.}
                \end{align*}
            \end{proof}

            We are now ready to describe the hom-functor. Suppose that $A$ and $A'$ are $A_\infty$-algebras, and that $M$ is an $A$-$A'$-polydule and $N$ a right $A'$-polydule. We define the $A$-polydule
            \begin{align*}
                \tt{Hom}^\infty_{A'}(M,N) = \tt{Hom}_{BA'}^\bullet(B_{A'^+}M,B_{A'^+}N)\tt{,}
            \end{align*}
            with structure map $\phi : A \rightsquigarrow \tt{End}(B_{A'^+}M)$ defined by the above corollary. In this way we obtain a functor
            \begin{align*}
                \tt{Hom}^\infty_{A'}(M,\argument) : \tt{Mod}_\infty^{A'} \rightarrow \tt{Mod}_\infty^A\tt{.}
            \end{align*}

            \begin{lemma}[Hom-Tensor adjunction, {\cite[Lemme 4.1.1.4][115]{LefevreHasegawa03}}]
                Let $A$ and $A'$ be two $A_\infty$-algebras and $M$ an $A$-$A'$-bipolydule. There is an adjoint pair of functors
                \begin{center}
                    \begin{tikzcd}
                        \tt{Mod}_\infty^A \ar[phantom]{r}[]{\bot} \ar[bend left]{r}[]{\argument\otimes^\infty_A M} & \tt{Mod}_\infty^{A'} \ar[bend left]{l}[]{\tt{Hom}_{A'}^\infty(M,\argument)}
                    \end{tikzcd}
                \end{center}
            \end{lemma}

            \begin{proof}
                We establish the natural bijection. We refer to \cite[Lemme 4.1.1.4]{LefevreHasegawa03} to see that it is well-defined.

                Consider an $\infty$-morphism $f : L \otimes^\infty_A M \rightsquigarrow R$ of right $A'$-polydules. By consider the bar construction of $A'$, this morphism is in correspondance with $B_{A'^+}f : L \otimes^\infty_A B_{A'^+}M \rightarrow B_{A'^+}R$. Through the ordinary tensor-hom adjunction we get a correspondance $f^T_i : L \otimes A^{\otimes i} \rightarrow \tt{Hom}_{BA'}(B_{A'^+}M,B_{A'^+}R)$.
            \end{proof}

        \subsection{Homologically Unital SHA-Algebras and Polydules}

            In this section we will define the notion of homologically unital $A_\infty$-algebras and polydules. These notions will be weaker than strictly unitary objects, but their definition may be easier to work with. As we will see, these notions almost coincide up to homotopy. This section will be given without proofs.

            If $A$ is an $A_\infty$-algebra, or $M$ is an $A$-polydule, we will use $\tt{H}^*A$ and $\tt{H}^*M$ to denote their homology. Note that $\tt{H}^*A$ is an associative algebra, as $m_i$ for $i \geq 3$ are homotopies, witnessing associativity of $\tt{H}^*m_2$. In the same manner, $\tt{H}^*M$, becomes a $\tt{H}^*A$-module, by considering $\tt{H}^*m_2^M$.

            \begin{definition}[Homologically unital $A_\infty$-algebra]
                Let $A$ be an $A_\infty$-algebra. A morphism $\upsilon_A : \mathbb{K} \rightarrow A$ is called a homological unit, if $\tt{H}^*\upsilon_A : \mathbb{K} \rightarrow \tt{H}^*A$ is a unit in homology. We say that $A$ equipped with a homological unit $\upsilon_A$ is a homologically unital $A_\infty$-algebra.

                An $\infty$-morphism $f : A \rightsquigarrow A'$ is homologically unital if it preserves the unit in homology, i.e. $\tt{H}^*f : \tt{H}^*A \rightarrow \tt{H}^*A'$ is also a morphism of graded algebras.

                Given two $\infty$-morphisms $f,f' : A \rightsquigarrow A'$, then they are homotopically unital if there is a homotopy $h : A \rightsquigarrow A'$ between $f$ and $f'$ which is strictly unital with respect to the homological unit $\upsilon_A$. 
            \end{definition}

            We let $\tt{suAlg}_\infty$ denote the non-full subcategory of strictly unital $A_\infty$-algebras with strictly unital $\infty$-morphisms, $\tt{huAlg}_\infty$ denote the non-full subcategory of homologically unital $A_\infty$-algebras with homologically unital $\infty$-morphism, and $\tt{uAlg}_\infty$ denote the full subcategory of strictly unital $A_\infty$-algebras with $\infty$-morphisms. Note that if $A$ is a strictly unital $A_\infty$-algebra, then it is also homologically unital. Thus we see that $\tt{suAlg}_\infty \subseteq \tt{huAlg}_\infty$.

            In order to obtain a stronger relationship between homologically unital $A_\infty$-algebras and strictly unital $A_\infty$-algebras we need the notion of a minimal model. 

            \begin{definition}[Minimal SHA-algebra/polydule]
                Let $A$ be an $A_\infty$-algebra, and $M$ an $A$-polydule. We say that $A$ is minimal if $m_1^A = 0$, and likewise $M$ is minimal if $m_1^M = 0$
            \end{definition}

            \begin{definition}[Minimal model]
                Let $A$ and $A'$ be $A_\infty$-algebras. We say that an $\infty$-quasi-isomorphism $f : A' \rightsquigarrow A$, is a minimal model of $A$.
            \end{definition}
            
            \begin{thm}[{\cite[Corollaire 1.4.1.4][54]{LefevreHasegawa03}}]\label{thm: minimal-models}
                Let $A$ be an $A_\infty$-algebra. The injection from the homology $\tt{H}^*A$ into $A$ is a minimal model of $A$.
            \end{thm}

            \begin{proof}
                We will only construct the first component of this injection.

                Since $\tt{Mod}^\mathbb{K}$ is semi-simple, $A$ splits naturally as $A \simeq \tt{H}^*A \oplus K$. By definition $K$ is acyclic, and the inclusion $\tt{H}^*A \rightarrow A$ is a quasi-isomorphism. 
            \end{proof}

            We now state the following relationship between homologically unital and strictly unital $A_\infty$-algebras.

            \begin{thm}[{\cite[Theoreme 3.2.1.1][99]{LefevreHasegawa03}}]\label{thm: hu-to-su}
                Any minimal homologically unital $A_\infty$-algebra is isomorphic to a minimal strictly unital $A_\infty$-algebra.
            \end{thm}

            \begin{corollary}[Unital strictification of $A_\infty$-algebras, {\cite[Corollaire 3.2.1.2][99]{LefevreHasegawa03}}]\label{cor: unit-strict-alg}
                Any homologically unital $A_\infty$-algebra is homotopy equivalent to a stricly unital $A_\infty$-algebra.
            \end{corollary}

            \begin{proof}
                This is obtained by combining the two above theorems, i.e. Theorem~\ref{thm: minimal-models} and Theorem~\ref{thm: hu-to-su}.
            \end{proof}

            \begin{thm}[Unital strictification of $\infty$-morphisms, {\cite[Theoreme 3.2.2.1][103]{LefevreHasegawa03}}]
                An homologically unital $\infty$-morphism of strictly unital minimal $A_\infty$-algebras is homotopic to an strictly unital $\infty$-morphism.
            \end{thm}

            \begin{thm}[Unital strictification of homotopies, {\cite[Theoreme 3.2.3.1][104]{LefevreHasegawa03}}]
                Let $A$ and $A'$ be two minimal strictly unital $A_\infty$-algebras. Let $f,g : A \rightsquigarrow A'$ be strictly unital $\infty$-morphisms which are homotopic, then there is a stricly unital homotopy witnessing them being homotopic.
            \end{thm}

            \begin{corollary}
                Let $A$ and $A'$ be two $A_\infty$-algebra, and let $f : A \rightsquigarrow A'$ be a strictly unital homotopy equivalence. Then there is a strictly unital homotopy equivalence $g : A' \rightsquigarrow A'$, with strictly unital homotopies witnessing that $g$ is the homotopy inverse of $f$.
            \end{corollary}

            With the above results we learn that the homotopic information of strictly unital $A_\infty$-algebras is essentially controlled by strictly unital $\infty$-morphism. In other words the non-full inclusion $\tt{suAlg}_\infty \rightarrow \tt{uAlg}_\infty$ induces an equivalence of categories
            \begin{align*}
                \sfrac{\tt{suAlg}_\infty}{\sim} \simeq \sfrac{\tt{uAlg}_\infty}{\sim}\tt{.}
            \end{align*}
            We also get that the unital strictification of homologically unital $A_\infty$-algebras induces an equivalence
            \begin{align*}
                \sfrac{\tt{huAlg}_\infty}{\sim} \simeq \sfrac{\tt{suAlg}_\infty}{\sim}\tt{.}
            \end{align*}

            We also have similar results for polydules.

            \begin{definition}
                Let $A$ be a homologically unital $A_\infty$-algebra, and let $M$ be an $A$-polydule. We say that $M$ is homologically unital if $\tt{H}^*M$ is a unital $\tt{H}^*A$-module.

                Let $M$ and $N$ be two homologically unital $A$-polydules, and $f : M \rightsquigarrow N$ be an $\infty$-morphism. We say that $f : M \rightsquigarrow N$ is homologically unital if $\tt{H}^*f_1 : \tt{H}^*M \rightarrow \tt{H}^*N$ is a $\tt{H}^*A$-linear morphism.
            \end{definition}

            We denote the category of homologically unital $A$-polydules with homologically unital $\infty$-morphisms by $\tt{huMod}_\infty^A$. This is a non-full subcategory of $\tt{Mod}_\infty^A$. Recall that we also have $\tt{suMod}_\infty^A$, the category of strictly unital $A$-polydules with strictly unital $\infty$-morphism. Let $\tt{uMod}_\infty^A$ denote the full subcategory of $\tt{Mod}_\infty^A$ consisting of strictly unital $A$-polydules. We have the same kind of results as for $A_\infty$-algebras.

            \begin{thm}[Unital strictification of $A$-polydules, {\cite[Theoreme 3.3.1.2][109]{LefevreHasegawa03}}]\label{thm: unit-strict-poly}
                Let $A$ be a strictly unital $A_\infty$-algebra. Any minimal homologically unital $A$-polydule is isomorphic to a strictly unital $A$-polydule.
            \end{thm}

            \begin{corollary}[{\cite[Corollaire 3.3.1.3][109]{LefevreHasegawa03}}]\label{cor: unit-strict-poly}
                Let $A$ be a minimal strictly unital $A_\infty$-algebra. Any homologically unital $A$-polydule is homotopy equivalent to a strictly unital $A$-polydule.
            \end{corollary}

            \begin{thm}[Unital strictification of $\infty$-morphisms, {\cite[Theoreme 3.3.1.4][109]{LefevreHasegawa03}}]
                Let $A$ be a strictly unital $A_\infty$-algebra, and let $M$ and $N$ be minimal strictly unital $A$-polydules. Any $\infty$-morphism $f : M \rightsquigarrow N$ is homotopic to a strictly unital $\infty$-morphism.
            \end{thm}

            \begin{thm}[Unital strictification of homotopies, {\cite[Theoreme 3.3.1.5][109]{LefevreHasegawa03}}]
                Let $A$ be a strictly unital $A_\infty$-algebra, and let $M$ and $N$ be minimal strictly unital $A$-polydules. Let $f, g : M \rightsquigarrow N$ be homotopic $\infty$-morhpisms, then there is a strictly unital homotopy between $f$ and $g$.
            \end{thm}

            \begin{proposition}[Minimal models, {\cite[Proposition 3.3.1.7][109]{LefevreHasegawa03}}]
                Let $A$ be a strictly unital $A_\infty$-algebra, and let $M$ be a strictly unital $A$-polydule. Then there is a minimal strictly unital $A$-polydule $N$ together with a strictly unital minimal model $f : N \rightsquigarrow M$. In particular $f_1$ is a quasi-isomorphism.
            \end{proposition}

            Suppose that $A$ is a minimal strictly unital $A_\infty$-akgebra. With the above results, we are now able to deduce that the non-full inclusion $\tt{suMod}_\infty^A \rightarrow \tt{uMod}_\infty^A$ induces an equivalence
            \begin{align*}
                \sfrac{\tt{suMod}_\infty^A}{\sim} \simeq \sfrac{\tt{uMod}_\infty^A}{\sim}\tt{,}
            \end{align*}
            and the non-full inclusion $\tt{huMod}_\infty^A \rightarrow \tt{suMod}_\infty^A$ induces an equivalence
            \begin{align*}
                \sfrac{\tt{huMod}_\infty^A}{\sim} \simeq \sfrac{\tt{suMod}_\infty^A}{\sim}\tt{.}
            \end{align*}

        \subsection{H-Unitary SHA-Algebras and Polydules}

            In this section we will define notions which will help us to calculate homologies. We will start with defining a twisting morphism between an augmented $A_\infty$-algebra and a conilpotent dg-coalgebra. For the second part we will define H-unitary $A_\infty$-algebras and polydules.

            \begin{definition}
                Let $A$ be an augmented $A_\infty$-algebra, and let $C$ be a conilpotent dg-coalgebra. $\tau : C \rightarrow A$ is a twisting morphism if it is of degree $1$, it is $0$ on the augmentation and the coaugmentation and
                \begin{align*}
                    \sum_{i \geq 1} m_i \otimes (\tau^{\otimes i}) \otimes \Delta_C^i = 0\tt{.}
                \end{align*}
            \end{definition}

            Let $M$ be an $A$-polydule, and $N$ a $C$-comodule. Given a twisting morphism $\tau : C \rightarrow A$, we define the twisted tensor products
            \begin{align*}
                \argument \otimes_\tau C & : \tt{Mod}_\infty^A \rightarrow \tt{coMod}^C\tt{,} \\
                \argument \otimes_\tau A & : \tt{coMod}^C \rightarrow \tt{Mod}_\infty^A\tt{.}
            \end{align*}
            The perturbations are given as
            \begin{align*}
                d_\tau^r = \sum_{i=1}^\infty (m_i \otimes C)(M \otimes \tau^{\otimes i-1} \otimes C)(M \otimes \Delta_C^i)\tt{,} \\
                d_\tau^l = \sum_{i=1}^\infty (N \otimes m_i) (N \otimes \tau^{\otimes i-1} \otimes A)(\nu_N^i \otimes A)\tt{.}
            \end{align*}

            We define the perturbated differential of the cochain complexes $M \otimes C$ and $N \otimes A$ as
            \begin{align*}
                d_\tau^\bullet = d_{M \otimes C} + d_\tau^r\tt{, and} \\
                d_\tau^\bullet = d_{N \otimes A} - d_\tau^l\tt{.}
            \end{align*}

            \begin{definition}[Twisted tensor products]
                Let $A$ be an augmented $A_\infty$-algebra, let $C$ be a conilpotent dg-coalgebra and let $\tau : C \rightarrow A$ be a twisting morphism. Given an $A$-polydule $M$ (a $C$-comodule $N$) we define the right (left) twisted tensor product as $M \otimes_{\tau} C$ ($N \otimes_\tau A$) together with the perturbated differential $d_\tau^\bullet$.
            \end{definition}

            Pick an augmented $A_\infty$-algebra $A$. The morphism
            \begin{align*}
                \tau = i \circ s\circ \pi_1 : B\overline{A} \rightarrow A
            \end{align*}
            is a twisting morphism. Here $\pi_1 : B\overline{A} \rightarrow \overline{A}[1]$ is the projection onto first component, and $i : \overline{A} \rightarrow A$ is the inclusion.

            \begin{lemma}\label{lem: twisting-acyclic-v.2}
                The morphism $\varepsilon_{B\overline{A}}\otimes_\tau \varepsilon_A : B\overline{A} \otimes_\tau A \rightarrow \mathbb{K}$ is a quasi-isomorphism.
            \end{lemma}

            \begin{proof}
                We have already seen this in Lemma~\ref{lem: uni-twist-ac}.
            \end{proof}

            Twisting morphisms will be important in understanding H-unitary $A_\infty$-algebras and polydules.

            \begin{definition}
                Let $A$ be an $A_\infty$-algebra. We say that $A$ is H-unitary if the bar construction $BA$ is acyclic.
            \end{definition}

            \begin{lemma}\label{lem: min-unit-strict-H-unit}
                Let $A$ be a minimal strictly unital $A_\infty$-algebra, then it is H-unitary.
            \end{lemma}

            \begin{proof}
                The unit map $id_BA \otimes \upsilon_A[1] : BA \rightarrow BA$ is a morphism of degree $-1$, and is a homotopy of the identity.
            \end{proof}

            \begin{corollary}\label{cor: hu-to-H-u}
                Any homologically unital $A_\infty$-algebra is H-unitary. 
            \end{corollary}

            \begin{proof}
                Pick any homologically unital $A_\infty$-algebra $A$. By Corollary~\ref{cor: unit-strict-alg}, there exists a strictly unital $A_\infty$-algebra $A'$ and an $\infty$-quasi-isomorphism $f: A' \rightsquigarrow A$. Applying the bar construction yields a quasi-isomorphism $Bf : BA' \rightarrow BA$. By Lemma~\ref{lem: min-unit-strict-H-unit}, $BA'$ is acyclic, so $BA$ has to be acyclic as well.
            \end{proof}

            We have the same kind of relationships between polydules.

            \begin{definition}
                Let $A$ be an augmented srictly unital $A_\infty$-algebra. Any $A$-polydule $M$ is H-unitary if $B_AM$ is acyclic.
            \end{definition}

            \begin{lemma}\label{lem: H-u-is-hu}
                Let $A$ be a strictly unital $A_\infty$-algebra. An $A^+$-polydule $M$ is H-unitary if and only if it is homologically unital as an $A$-polydule.
            \end{lemma}

            \begin{proof}
                Suppose first that $M$ is a homologically unital $A$-polydule. Then by Corollary~\ref{cor: unit-strict-poly}, there is a strictly unital $A$-polydule $M'$ together with an $\infty$-quasi-isomorphism $M' \rightsquigarrow M$. It is enough to show that $B_{A^+}M'$ is acyclic. The unit $\upsilon_{A}$ defines a homotopy of the identity
                \begin{align*}
                    id_{B_{A^+}M'}\otimes \upsilon_{A}[1] : B_{A^+}M' \rightarrow B_{A^+}M'\tt{.}
                \end{align*}

                For the other direction, suppose instead that $M$ is a H-unitary $A^+$-polydule. First note that we have an exact sequence
                \begin{center}
                    \begin{tikzcd}
                        0 \ar[]{r}[]{} & A \ar[]{r}[]{} & A^+ \ar[]{r}[]{} & \mathbb{K} \ar[]{r}[]{} & 0
                    \end{tikzcd}
                \end{center}
                Recall that $\tau = i\circ s \circ \pi_1 : BA \rightarrow A^+$. This sequence induces an exact sequence on the twisted tensors
                \begin{center}
                    \begin{tikzcd}
                        0 \ar[]{r}[]{} & M \otimes_\tau BA \otimes_\tau A \ar[]{r}[]{} & M \otimes_\tau BA \otimes_\tau A^+ \ar[]{r}[]{} & M \otimes_\tau BA \otimes_\tau \mathbb{K} \ar[]{r}[]{} & 0
                    \end{tikzcd}
                \end{center}
                By assumption $M \otimes_\tau BA \otimes_\tau \mathbb{K} \simeq (M[1] \otimes_\tau BA)[-1] \simeq (B_{A^+}M)[-1]$ which is acyclic by assumption. Thus $M \otimes_\tau BA \otimes_\tau A$ is quasi-isomorphic to $M \otimes_\tau BA \otimes A^+$. By Lemma~\ref{lem: twisting-acyclic-v.2}, $M \otimes_\tau BA \otimes A^+ \simeq M \otimes _\tau \mathbb{K} \simeq M$. Thus $M \simeq M\otimes_\tau BA \otimes_\tau A$, which is a strictly unital right $A$-polydule by freeness. 
            \end{proof}

    \section{The Derived Category $D_\infty A$}

        \subsection{The Derived Category of Augmented SHA-Algebras} 

            In this section we wish to define the derived category of strictly unital polydules of an augmented $A_\infty$-algebra. If $[\tt{Qis}]$ denote the class of $\infty$-quasi-isomorphisms, we want the derived category to be the localization at $\infty$-quasi-isomorphisms, e.g.
            \begin{align*}
                \mathcal{D}_\infty A = \tt{suMod}_\infty^A[\tt{Qis}^{-1}].
            \end{align*}

            Like in the case of algebras, we may understand the quasi-isomorphisms better. The category $\tt{suMod}_\infty^A$ is not complete, but we may in the same sense as before give it a model structure without limits. Within this structure we already know that every object is cofibrant, and the goal is to show that every object is fibrant as well. This will allow us to lift every $\infty$-quasi-isomorphism to a homotopy equivalence. With this we may see that the localization from $K_\infty A \to D_\infty A$ is given by the identity.

            Within the category $\tt{suMod}_\infty^A$ we define three classes of morphisms:
            \begin{itemize}
                \item $f\in Ac$ is a weak equivalence if $f_1$ is a quasi-isomorphism,
                \item $f\in Cof$ is a cofibration if $f_1$ is a monomorphism,
                \item $f\in Fib$ is a fibration if $f_1$ is an epimorphism,
            \end{itemize}

            \begin{thm}
                The category $\tt{suMod}_\infty^A$ is a model category without enough limits. Moreover, every object are bifibrant.
            \end{thm}

            \begin{proof}
                This is more or less identical to the proof of Theorem~\ref{thm: model-Alg-inf}.
            \end{proof}

            \begin{corollary}
                Homotopy equivalence defined in $\tt{suMod}_\infty^A$ is an equivalence relation, and every $\infty$-quasi-isomorphism is a homotopy equivalence.

                If $A$ is an ordinary associative augmented algebra, then $\tt{Mod}^A$ fully-faithfully embeds into $\tt{suMod}_{\infty,strict}^A$. This inclusion defines an equivalence
                \begin{align*}
                    DA \simeq \sfrac{\tt{suMod}^A_{\infty, strict}}{\sim_\infty}\tt{.}
                \end{align*}
            \end{corollary}

            We now want this model structure on $\tt{suMod}^A_\infty$ to respect the model structure on the category $\tt{coMod}_{conil}^{B\overline{A}}$. In other words, we want the functor $B_A : \tt{suMod}_\infty^A \to \tt{coMod}_{conil}^{B\overline{A}}$ to preserve and reflect the model structure of both categories.

            \begin{lemma}
                Let $M$ be an object of $\tt{suMod}_\infty^A$. The unit $B_AM \to R_{\iota_{B\overline{A}}}L_{\iota_{B\overline{A}}}B_AM$ is a quasi-isomorphism on the primitive elements.
            \end{lemma}

            \begin{proof}
                This proof uses the same trick as lemma \ref{lem: uni-twist-ac}. Equip $M$, the trivial filtraton, $B\overline{A}$ the coradical filtration and $\Omega B\overline{A} = UA$ the induced filtration.
                \begin{align*}
                    F_pM & = M\tt{,} \\
                    Fr_pB\overline{A} & = \startset{[a_1 \mid \cdots \mid a_n] \mid n \leq p}\tt{,} \\
                    F_p UA & = \startset{\langle [a_{1_1} \mid \cdots \mid a_{n_1}] \mid \cdots \mid [a_{1_k} \mid \cdots \mid a_{n_k}] \rangle \mid n_1 + \cdots + n_k \leq p}\tt{.}
                \end{align*} 

                We see that $\tt{gr}_0M[1] \simeq M[1]$ and otherwise $\simeq 0$. In the same way, $\tt{gr}_0\eta$ acts as the identity on $M[1]$. By the similar lemma, we know that each $\tt{gr}_p M[1] \otimes B\overline{A} \otimes UA$ is acyclic for $p \geq 1$. Thus $\tt{gr}\eta$, is a filtered quasi-isomorphism on the primitives.

            \end{proof}

            \begin{proposition}
                Let $M$ and $M'$ be objects of $\tt{suMod}_\infty^A$, together with an $\infty$-morphism $f : M \to M'$.
                \begin{itemize}
                    \item $f$ is an $\infty$-quasi-isomorphism if and only if $B_Af$ is a weak equivalence.
                    \item $f$ is a fibration if and only if $B_Af$ is a fibration.
                    \item $f$ is a cofibration if and only if $B_Af$ is a cofibration.
                \end{itemize}
            \end{proposition}

            \begin{proof}
                Recall from theorem \ref{thm: model-comod} that the morphism $\iota_{B\overline{A}} : B\overline{A} \to UA$ is an acyclic twisting morphism. Thus the adjoint pair $(L_{\iota_{BA}}, R_{\iota_{BA}})$ defines a Quillen equivalence.

                We show only the first bullet point. The last two are identical to the proof of proposition \ref{prop: coherent-model-structure}.

                If $f_1$ is a quasi-isomorphism, then $B_Af$ is a filtered quasi-isomorphism by definition. So suppose that $B_Af$ is a weak equivalence instead. The unit transformation gives us a natural square.

                \begin{center}
                    \begin{tikzcd}
                        B_AM \ar[]{r}[]{} \ar[]{d}[]{B_Af} & R_{\iota_{B\overline{A}}}L_{\iota_{B\overline{A}}}B_AM \ar[]{d}[]{R_{\iota_{B\overline{A}}}L_{\iota_{B\overline{A}}}B_Af} \\
                        B_AM' \ar[]{r}[]{} & R_{\iota_{B\overline{A}}}L_{\iota_{B\overline{A}}}B_AM'    
                    \end{tikzcd}
                \end{center}

                In this case $R_{\iota_{B\overline{A}}} = B_Ai$, so this diagram is in the image of $B_A$. Since $B_A$ is fully-faithful, we consider this diagram in $\tt{suMod}_\infty^A$ instead.

                \begin{center}
                    \begin{tikzcd}
                        M \ar[]{r}[]{} \ar[]{d}[]{f} & iL_{\iota_{B\overline{A}}}BM \ar[]{d}[]{iL_{\iota_{B\overline{A}}}Bf} \\
                        M' \ar[]{r}[]{} & iL_{\iota_{B\overline{A}}}BM'
                    \end{tikzcd}
                \end{center}

                Since $B_Af$ is a weak equivalence, $iL_{\iota_{B\overline{A}}}B_Af$ is an $\infty$-quasi-isomorphism by definition. By the above lemma, the horizontal maps are $\infty$-quasi-isomorphisms. Thus by the $2$-out-of-$3$ property, $f$ is an $\infty$-quasi-isomorphism as well.
            \end{proof}

            Associated to each augmented $A_\infty$-algebra there is also a homotopy category. Since homotopy equivalence $\sim_\infty$ in $\tt{suMod}_\infty^A$ defines a congruence relation we may construct the homotopy category $K_\infty A$.

            \begin{corollary}
                The localization $K_\infty A \to D_\infty A$ is given by the identity. Moreover, $K_\infty A = D_\infty A$.
            \end{corollary}

            \begin{remark}
                The name homotopy category comes from homological algebra and has a priori nothing to do with the homotopy category $\tt{Ho}(\tt{suMod}_\infty^A)$. However, in this particular case these naming conventions coincide.
            \end{remark}

            \begin{lemma}\label{lem: universal-enveloping-is-derived}
                The composition $J : \tt{Mod}^{UA} \to \tt{suMod}_{\infty, strict}^A \to \tt{suMod}_\infty^A$ given by $J = \iota \circ i$, induces an equivalence of categories:
                \begin{align*}
                    DUA \simeq D_\infty A.
                \end{align*}
            \end{lemma}

            \begin{proof}
                Consider the commutative square:
                \begin{center}
                    \begin{tikzcd}
                        \tt{Mod}^{UA} \ar[]{d}[]{R_{\iota_{B\overline{A}}}} \ar[]{r}[]{i} & \tt{suMod}_{\infty, strict}^A \ar[]{d}[]{\iota} \\
                        \tt{coMod}^{B\overline{A}} & \tt{suMod}_\infty^A \ar[]{l}[]{B_A}
                    \end{tikzcd}
                \end{center}

                Since the three functors $R_{\iota_{B\overline{A}}}$, $i$ and $B_A$ all induces equivalences on the derived categories, then $\iota$ has to as well.
            \end{proof}

            To summarize, we have established an equivalence between 4 different categories:
            \begin{itemize}
                \item $D_\infty A$, derived category of $A$,
                \item $\tt{suMod}_{\infty, strict}^A[Qis^{-1}]$ derived category of $A$ with only strict morphisms,
                \item $DB\overline{A}$, derived category of $B\overline{A}$ as a dg-coalgebra,
                \item $DUA$, derived category of the universal enveloping algebra of $A$.
            \end{itemize}

            In this sense, we may also see that in the derived category, all of the higher homotopic data of each morphism have been collapsed by the homotopy.

            The triangulated structure on $D_\infty A$ may be lifted along these equivalences making them triangulated as well. Note that $R_{\iota_{B\overline{A}}}$ is already triangulated, and there is only one way of forcing the triangulated structure on $\tt{suMod}_\infty^A$. Since $\tt{suMod}_{\infty}^A$ isn't complete it isn't easy to obtain a description of the triangles along any $\infty$-morphism $f$. However, this problem does not appear in $\tt{suMod}_{\infty, strict}^A$, so one should think of only strict morphisms instead, but in this case we are already working in the category $\tt{Mod}^{UA}$. 

        \subsection{The Derived Category of Strictly Unital SHA-Algebras}

            In this section we will generalize the construction of the derived category to any strictly unital $A_\infty$-algebra. Consider the strictly unital $A_\infty$-algebra $A$. If we look at the augmented algebra $A^+$, then the augmentation $\varepsilon_A : A^+ \rightarrow \mathbb{K}$ gives $\mathbb{K}$ the structure of an $A^+$-polydule. We construct the following functor
            \begin{align*}
                \argument\otimes_{A^+}^\infty\mathbb{K} : \tt{Mod}_\infty^{A^+} \rightarrow \tt{Mod}_\mathbb{K}^\infty\tt{.}
            \end{align*}
            We may observe that this functor maps strictly unital objects into strictly unital objects
            \begin{align*}
                \argument\otimes_{A^+}^\infty\mathbb{K} : \tt{uMod}_\infty^{A^+} \rightarrow \tt{uMod}_\mathbb{K}^\infty\tt{.}
            \end{align*}

            The derived category $D_\infty A^+$ is equivalent to $\sfrac{\tt{uMod}_\infty^{A^+}}{\sim}$. Since the functor above preserves $\infty$-quasi-isomorphisms, it induces a functor between the derived categories
            \begin{align*}
                \argument\otimes_{A^+}^\infty\mathbb{K} : D_\infty A^+ \rightarrow D_\infty \mathbb{K}\tt{.}
            \end{align*}

            \begin{definition}
                Let $A$ be an $A_\infty$-algebra. We define the derived category as the kernel
                \begin{align*}
                    D_\infty A = \tt{Ker}(\argument\otimes_{A^+}^\infty\mathbb{K} : D_\infty A^+ \rightarrow D_\infty\mathbb{K})\tt{.}
                \end{align*}
            \end{definition}

            \begin{thm}\label{thm: derived-is-well-defined}
                Let $A$ and $A'$ be two $A_\infty$-algebras, and let $f : A \rightarrow A'$ be an $\infty$-quasi-isomorphism. The restriction
                \begin{align*}
                    f^* : \tt{Mod}_\infty^{A'} \rightarrow \tt{Mod}_\infty^{A}
                \end{align*}
                induces an equivalence on the derived categories
                \begin{align*}
                    f^* : D_\infty A' \rightarrow D_\infty A\tt{.}
                \end{align*}
            \end{thm}

            \begin{proof}
                We have already seen a variant of this. Consider the diagram
                \begin{center}
                    \begin{tikzcd} 
                        D_\infty^{A'} \ar[tail]{r} \ar[]{d}[]{f^*} & D_\infty A'^+ \ar[]{r} \ar[]{d}[]{(f^+)^*} & D_\infty \mathbb{K} \ar[]{d}[]{\simeq} \\
                        D_\infty A \ar[tail]{r} & D_\infty A^+ \ar[]{r} & D_\infty \mathbb{K}
                    \end{tikzcd}
                \end{center}

                By Lemma~\ref{lem: universal-enveloping-is-derived}, we have a commutative square
                \begin{center}
                    \begin{tikzcd}
                        DU(A'^+) \ar[]{r}[]{\simeq} \ar[]{d}[]{U((f^+)^*)}[left]{\simeq} & D_\infty A'^+ \ar[]{d}[]{(f^+)^*} \\
                        DU(A^+) \ar[]{r}[]{\simeq} & D_\infty A^+
                    \end{tikzcd}
                \end{center}

                Since $U((f^+)^*)$ is an equivalence by Corollary~\ref{cor: ring-qiso-is-eq}, then $((f^+)^*)$ is an equivalence as well. By the first diagram, $f^*$ has to be an equivalence by the kernel property.
            \end{proof}

            A usefull property of the $\infty$-tensor is that it behaves like the ordinary tensor up to homotopy.

            \begin{lemma}\label{lem: tensor-hom-qiso}
                Let $A$ be an $A_\infty$-algebra. Let $M$ be a strictly unital $A$-polydule. On the category $\tt{uMod}_\infty^A$ we have the following:
                \begin{itemize}
                    \item There is an $\infty$-quasi-isomorphism $M \otimes_A^\infty A \rightsquigarrow M$,
                    \item and there is an $\infty$-quasi-isomorphism $M \rightsquigarrow \tt{Hom}_A^\infty(A,M)$.
                \end{itemize}
            \end{lemma}

            \begin{proof}
                Since the second point is the transpose of the first point, we will only prove that $M \otimes_A^\infty A \rightsquigarrow M$ is an $\infty$-quasi-isomorphism.

                We define the multiplication morphism compenentwise
                \begin{align*}
                    g_{i,j} : M \otimes_A^\infty A & \rightarrow M\tt{,} \\
                    m \otimes [a_1\mid \cdots \mid a_{j}] \otimes a \otimes a'_1 \otimes \cdots \otimes a'_{i-1} & \mapsto m_{1+j+1+i-1}(m,a_1,\cdots,a_j,a,a'_1,\cdots,a'_i)\tt{,}
                \end{align*}
                so that $g_i = \sum_{j=1}^\infty g_{i,j}$.

                To see that $g$ defines an $\infty$-quasi-isomorphism we calculate the homology of $\tt{cone}(g_1)$.

                One may observe that the morphism 
                \begin{align*}
                    id \otimes \upsilon_A[1] \otimes id : M \otimes (A[1])^{\otimes i} \otimes A \rightarrow M \otimes (A[1])^{\otimes i+1} \otimes A
                \end{align*}
                induces a homotopy between $id_{\tt{cone}(g_1)}$ and $0$, so $g_1$ is a quasi-isomorphism.
            \end{proof}

            We are now going to define other categories which will look very similar to the derived category in the augmented case. It is also true that these categories will be equivalent to the derived category in the strictly unital case.

            \begin{definition}[Compactly generated triangulated category]
                Let $A$ be a strictly unital $A_\infty$-algebra. We let $\langle A \rangle$ denote the smallest thick triangulated subcategory category of $D_\infty A^+$ containing $A$ and which is closed under infinite sums. 
            \end{definition}

            \begin{definition}[Homotopy category]
                Let $A$ be a strictly unital $A_\infty$-algebra. Let the homotopy category be
                \begin{align*}
                    K_\infty A = \sfrac{\tt{suMod}_\infty^A}{\sim}\tt{,}
                \end{align*}
                where $\sim$ is homotopy equivalence.
            \end{definition}

            A priori, we are not sure if the congruence relation generated by homotopy equivalence is strictly greater than just homotopy equivalence. However, by considering the restriction map
            \begin{align*}
                r = \begin{pmatrix}
                    id_A & \upsilon_A
                \end{pmatrix} : A^+ \rightarrow A\tt{,} 
            \end{align*}
            we obtain a faithfull functor
            \begin{align*}
                r^* : \tt{suMod}_\infty^A \rightarrow \tt{suMod}_\infty^{A^+}\tt{,}
            \end{align*}
            which respects homotopy equivalences. This functor also induces a fully-faithfull functor
            \begin{align}
                \sfrac{r^*}{\sim} : K_\infty A \rightarrow K_\infty A^+\tt{.}
            \end{align}
            Since homotopy equivalence is a congruence relation in the latter category, it necessarily have to be that in former category.

            \begin{thm}
                Let $A$ be a strictly unital $A_\infty$-algebra. The following categories are equivalent:
                \begin{itemize}
                    \item $D_\infty A$
                    \item $\langle A \rangle$
                    \item $K_\infty A$
                    \item $\tt{suMod}_\infty^A[Qis^{-1}]$
                    \item $\tt{Ho}(\tt{suMod}_{\infty,\tt{strict}}^A)$
                \end{itemize}
            \end{thm}

            \begin{proof}[Proof of $D_\infty A \simeq \langle A \rangle$]
                We start by showing that $D_\infty A \simeq \langle A \rangle$. In order to see this, we would like to have an exact sequence of triangulated categories
                \begin{center}
                    \begin{tikzcd}
                        \langle A \rangle \ar[tail]{r} & D_\infty A^+ \ar[two heads]{r} & D_\infty \mathbb{K}
                    \end{tikzcd}
                \end{center}
                By \cite[Proposition 3.2.8][81]{Krause21} it suffices to show that for any $A^+$-polydule $M$, in the triangle
                \begin{center}
                    \begin{tikzcd}
                        M \otimes_{A^+}^\infty A \ar[]{r} & M \ar[]{r} & M \otimes_{A^+}^\infty \mathbb{K} \ar[]{r} & (M \otimes_{A^+}^\infty A)[1]
                    \end{tikzcd}
                \end{center}
                the objects $M \otimes_{A^+}^\infty A \in \langle A \rangle$ and $M \otimes_{A^+}^\infty \mathbb{K}$ are $\langle A \rangle$-local. An object of $M \in D_\infty A^+$ is said to be $\langle A \rangle$-local if for any $L \in \langle A \rangle$
                \begin{align*}
                    D_\infty A^+(L, M) = 0 \tt{.}
                \end{align*}

                We start by observing that $M \otimes_{A^+} A = M \otimes BA^+ \otimes A$, so $M \otimes_{A^+}^\infty A$ is in fact contained in $\langle A \rangle$.

                To see that $M \otimes_{A^+} \mathbb{K}$ is $\langle A \rangle$-local, we start by considering the following triangle
                \begin{center}
                    \begin{tikzcd}
                        A \otimes_{A^+}^\infty \mathbb{K} \ar[]{r} & A^+ \otimes_{A^+}^\infty \mathbb{K} \ar[]{r} & \mathbb{K} \otimes_{A^+}^\infty \mathbb{K} \ar[]{r} & (A \otimes_{A^+}^\infty \mathbb{K})[1]
                    \end{tikzcd}
                \end{center}
                By assumption $A$ is strictly unital, so it is homologically unital as well, even considered as an $A$-polydule. By Lemma~\ref{lem: H-u-is-hu}, $A$ is $H$-unitary as an $A^+$-polydule. Notice that $A \otimes_{A^+} \mathbb{K} = A \otimes BA^+ \otimes \mathbb{K} \simeq B_{A^+}A$. Since $A$ is $H$-unitary, we get that $A \otimes_{A^+}^\infty \mathbb{K}$ is acyclic. Moreover, by thickness, any $L\in \langle A \rangle$ has the property that
                \begin{align*}
                    L \otimes_{A^+}^\infty \mathbb{K} \simeq 0\tt{.}
                \end{align*}
                By acyclicity of $A \otimes_{A^+}^\infty \mathbb{K}$, we obtain an $\infty$-quasi-isomorphism
                \begin{align}
                    A^+ \otimes_{A^+}^\infty \mathbb{K} \rightarrow \mathbb{K} \otimes_{A^+}^\infty \mathbb{K}\tt{.}
                \end{align}

                If we consider the projection
                \begin{align*}
                    A^+ \otimes_{A^+}\mathbb{K} \rightarrow \mathbb{K}\tt{,}
                \end{align*}
                we see that this is an $\infty$-quasi-isomorphism, since the cone is the bar construction of $A^+$. $BA^+$ is acyclic, as $A^+$ is strictly unital and thus H-unitary.

                By composing these morphisms in the derived category $D_\infty A^+$, we get an isomorphism
                \begin{align}
                    \mathbb{K} \rightarrow \mathbb{K} \otimes_{A^+}^\infty \mathbb{K}\tt{.}
                \end{align}

                Now, pick an arbitrary morphism $f : L \rightarrow M \otimes_{A^+}^\infty \mathbb{K}$. We have the following commutative diagram
                \begin{center}
                    \begin{tikzcd}
                        L \ar[]{r}[]{f} \ar[]{d} & M \otimes_{A^+}^\infty \mathbb{K} \ar[]{d}[]{\simeq} \\
                        L \otimes_{A+}^\infty \mathbb{K} \ar[]{r} & M \otimes_{A^+}^\infty \mathbb{K} \otimes_{A^+}^\infty \mathbb{K}
                    \end{tikzcd}
                \end{center}

                As $L \otimes_{A^+}^\infty \mathbb{K} \simeq 0$, the morphism $f$ factors $0$. Thus $f = 0$.
            \end{proof}

            \begin{proof}[Proof of $D_\infty A \simeq K_\infty A$]
                Let $M$ be an $A^+$-polydule. We evaluate $M \otimes_{A^+}^\infty \mathbb{K} = M \otimes BA^+ = B_{A^+}M$. In other words, $M$ is H-unitary if and only if $M \otimes_{A^+}^\infty \mathbb{K}$ is acyclic. By definition, $D_\infty A$ is thus made up of every H-unitary $A^+$-polydules. By Lemma~\ref{lem: H-u-is-hu}, we know that $D_\infty A$ is then formed by the homologically unital $A$-polydules. By Corollary~\ref{cor: unit-strict-poly}, every such $A$-polydule is $\infty$-quasi-isomorphic to a strictly unital $A$-polydule.

                For the augmented $A_\infty$-algebra $A^+$ we know already that $K_\infty A^+ \simeq D_\infty A^+$. Thus $K_\infty A$ is exactly the kernel in the following diagram
                \begin{center}
                    \begin{tikzcd}
                        K_\infty A \ar[tail]{r} & K_\infty A^+ \ar[two heads]{r} & D_\infty \mathbb{K}
                    \end{tikzcd}
                \end{center}
                as the inclusion sends strictly unital polydules to H-unitary polydules.
            \end{proof}

            \begin{proof}[Proof of $K_\infty A \simeq \tt{suMod}_\infty^A{[\tt{Qis}^{-1}]}$]
                Since there is a fully-faithfull functor
                \begin{center}
                    \begin{tikzcd}
                        K_\infty A \ar[tail]{r} & K_\infty A^+
                    \end{tikzcd}
                \end{center}
                it follows that every $\infty$-quasi-isomorphism in $\tt{suMod}_\infty^A$ is a homotopy equivalence. Thus $\tt{suMod}_\infty^A[\tt{Qis}^{-1}]\simeq K_\infty A$.
            \end{proof}

            We prove the final statement first in the case of ordinary associative algebras. 

            \begin{lemma}\label{lem: derived-category-coincides}
                Let $A$ be a differential graded algebra. The inclusion $i : \tt{Mod}^A \rightarrow \tt{suMod}_\infty^A$ induces an equivalence of categories
                \begin{align*}
                    DA \simeq \tt{suMod}_\infty^A[\tt{Qis}^{-1}]\tt{,}
                \end{align*}
                where the inverse is given by $\argument\otimes_A^\infty A$.
            \end{lemma}

            \begin{proof}
                Let $M$ be an $A$-polydule, then we already know that there is an $\infty$-quasi-isomorphism $M\otimes_A^\infty A \rightsquigarrow M$.

                Let instead $M$ be an $A$-module. Then we can consider it as an $A$-polydule by letting the higher multiplication $m_i = 0$ for any $i \geq 3$. Thus we see that the $\infty$-morphism $g$ defined as in Lemma~\ref{lem: tensor-hom-qiso} is a strict morphism. In other words $g = g_1$ defines a morphism of algebras. We have already seen that it is the component $g_1$ which is a quasi-isomorphism, so there is a quasi-isomorphism of modules $i(M) \otimes_A^\infty A \rightarrow M$.

                Thus we have proved that in the derived categories $D_\infty A$ and $DA$ composing the functors are isomorphic to applying the identity functors. Thus we get an equivalence
                \begin{align*}
                    DA \simeq D_\infty A\tt{.}
                \end{align*}
            \end{proof}

            Before the last proof we will need some technical lemmata.

            \begin{lemma}[{\cite[Proposition 7.5.0.2][171]{LefevreHasegawa03}}]\label{lem: reverse-minimal-model}
                Let $A$ be a strictly unital $A_\infty$-algebra, then there is a dg-algebra $A'$ and a strictly unital acyclic cofibration
                \begin{align*}
                    A \rightsquigarrow A'\tt{.}
                \end{align*}
            \end{lemma}

            \begin{lemma}{{\cite[Proposition 3.2.4.5][106]{LefevreHasegawa03}}}\label{lem: inf-cofib-is-split}
                Let $A$ and $A'$ be two strictly unital $A_\infty$-algebras. If $i : A \rightsquigarrow A'$ is a strictly unital acyclic cofibration, then there is a strictly unital acyclic fibration $p : A' \rightarrow A$, such that $p\circ i = id_A$ and $i \circ p \sim id_{A'}$.
            \end{lemma}

            \begin{lemma}{\cite[Lemme 4.1.3.15][128]{LefevreHasegawa03}}\label{lem: homotopic-mor-is-same}
                Let $A$ and $B$ be two unital differential graded algebras. Let $f, f' : A \rightarrow B$ be two morphisms of algebras, such that they are right homotopic $f \sim_r f'$. The restriction functors
                \begin{align}
                    f^*, f'^* : \tt{Mod}^B \rightarrow \tt{Mod}^A
                \end{align}
                induces equivalent functors on the derived category
                \begin{align*}
                    f^* \simeq f'^* : DB \rightarrow DA\tt{.}
                \end{align*}
            \end{lemma}

            \begin{proof}[Proof of $\tt{suMod}_\infty^A{[\tt{Qis}^{-1}]} \simeq \tt{Ho}(\tt{suMod}_{\infty, \tt{strict}}^A)$]
               Assume first that $A$ is a differential graded associative algebra. We have the following chain of faithfull inclusions
                \begin{center}
                    \begin{tikzcd}
                        \tt{Mod}^A \ar[tail]{r} & \tt{suMod}_{\infty, \tt{strict}}^A \ar[tail]{r} & \tt{suMod}_\infty^A\tt{.}
                    \end{tikzcd}
                \end{center}
                 By lemma \ref{lem: derived-category-coincides}, the composition is an equivalence on the derived categories, and then necessarily essentially surjective and fully-faithfull. The last inclusion is by definition essentially surjective, and also fully-faithfull on the derived categories. In this manner all three categories are equivalent.

                 We will now suppose that $A$ is an $A_\infty$-algebra. By Lemma~\ref{lem: reverse-minimal-model}, there exists a dg-algebra $A'$ and an acyclic cofibration
                 \begin{align*}
                    p: A \rightsquigarrow A'\tt{.}
                 \end{align*}
                 By Lemma~\ref{lem: inf-cofib-is-split}, there also exists an acyclic fibration $q : A' \rightsquigarrow A$, splitting $p$ as $q \circ p = id_A$ and $p \circ q \sim id_{A'}$. 

                 If we are using the model structures on $\tt{suMod}_{\infty,\tt{strict}}^A$ and $\tt{suMod}_{\infty,\tt{strict}}^{A'}$ induced by the universal enveloping algebras, the morphisms $p$ and $q$ induces functors
                 \begin{align*}
                    \tt{Ho}(p^*) : \tt{Ho}(\tt{suMod}_{\infty,\tt{strict}}^{A'}) \rightarrow \tt{Ho}(\tt{suMod}_{\infty,\tt{strict}}^A)\tt{ and,}\\
                    \tt{Ho}(q^*) : \tt{Ho}(\tt{suMod}_{\infty,\tt{strict}}^A) \rightarrow \tt{Ho}(\tt{suMod}_{\infty,\tt{strict}}^{A'})\tt{.}
                 \end{align*}

                 If we have that 
                 \begin{align*}
                    \tt{Ho}(p^*)\tt{Ho}(q^*) \simeq \tt{Id}_{\tt{Ho}(\tt{suMod}_{\infty,\tt{strict}}^A)}\tt{ and} \\
                    \tt{Ho}(q^*)\tt{Ho}(p^*) \simeq \tt{Id}_{\tt{Ho}(\tt{suMod}_{\infty,\tt{strict}}^{A'})}\tt{,}
                 \end{align*}
                 then we would be done. This is because $p^* : D_\infty A' \rightarrow D_\infty A$ induces an equivalence by Theorem~\ref{thm: derived-is-well-defined}. Thus we may consider the following commutative diagram
                 \begin{center}
                    \begin{tikzcd}
                        \tt{Ho}(\tt{suMod}_{\infty,\tt{strict}}^A) \ar[]{r}[]{\simeq} \ar[dashed]{d} & \tt{Ho}(\tt{suMod}_{\infty,\tt{strict}}^{A'}) \ar[]{d}[]{\simeq} \\
                        D_\infty A & D_\infty A' \ar[]{l}[]{\simeq} 
                    \end{tikzcd}
                 \end{center}
                 Here the equivalence on the right hand side is given by the case for ordinary algebras treated earlier. Finally, by previous results we know that $D_\infty A \simeq \tt{suMod}_{\infty}^A[\tt{Qis}^{-1}]$\tt{.}
                 
                 To see that we have the equivalences as claimed, we first note that the first one is automatic by the equation $q\circ p = id_A$. We must show that $p\circ q$ is isomorphic to the identity on $\tt{Ho}(\tt{suMod}_{\infty,\tt{strict}}^{A'})$. By the earlier argument, proving this will be the same as proving that $p\circ q$ induces an equivalence on $DA'$. Since $p\circ q$ is homotopic to $id_{A'}$ we get by Lemma~\ref{lem: homotopic-mor-is-same} that they induce the same functor
                 \begin{align*}
                    (p\circ q)^* \simeq \tt{Id}_{DA'} : DA' \rightarrow DA'\tt{.}
                \end{align*}
            \end{proof}
\end{document}