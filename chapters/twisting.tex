\documentclass[../thesis.tex]{subfiles}

\begin{document}
    \section{Algebras, Coalgebras and Twisting Morphisms}

        In this section we will look at a result of associative algebras over a field $\mathbb{K}$. Given a coassociative conilpotent differential graded coalgebra $C$ and a differential graded associative algebra $A$, we say that a homogenous linear transformation $\alpha: C\rightarrow A$ is twisting if it satisfies the Maurer-Cartan equation:
            \begin{equation*}
                \partial\alpha + \alpha\star\alpha = 0.
            \end{equation*}
        Let $Tw(C,A)$ be the set of twisting morphisms, then considering it as a functor $Tw : CoAlg_{\mathbb{K}}^{op}\times Alg_{\mathbb{K}} \rightarrow Ab$ we want to show that it is represented in both arguments. Moreover, this representation give rise to an adjoint pair of functors, called the Bar and Cobar construction.

            \begin{center}
                \begin{tikzcd}
                    Alg_{\mathbb{K}} \ar[bend left]{r}[pos=0.55]{B} \ar[phantom]{r}{\top} & \substack{Conil \\ CoAlg}_{\mathbb{K}} \ar[bend left]{l}[pos=0.45]{\Omega}
                \end{tikzcd}
            \end{center}

        To obtain this result we need to define a twisting morphism. Thus this section will define algebras, coalgebras and convolution algebras before we state the result of the Bar and Cobar construction.

        \subsection{Algebras}

            In this subsection we will look at associative algebras. We will define unital associative algebras and non-unital associative algebras, which we will call algebras and non-unital algebras respectively. The collection of algebras together with homomorphisms between them form the category $Alg_{\mathbb{K}}$ of algebras. Other types of algebras such as augmented and tensor algebras will be defined as well.

            \begin{definition}[Algebra]
                Let $\mathbb{K}$ be a field. An algebra $A$ over $\mathbb{K}$ is a $\mathbb{K}$-module with structure morphisms called multiplication and unit,
                \begin{align*}
                    (\nabla_A) & : A\otimes_{\mathbb{K}}A \rightarrow A \\
                    \upsilon_A & : \mathbb{K} \rightarrow A,
                \end{align*}
                satisfying the associativity and identity laws. 
                \begin{align*}
                    (associativity)\quad & (a \nabla_A b) \nabla_A c = a \nabla_A (b \nabla_A c) \\
                    (unitality)\quad & \upsilon_A(1) \nabla_A a = a = a \nabla_A \upsilon_A(1)
                \end{align*}
            \end{definition}
            \begin{remark}
                Whenever $A$ does not posess a unit morphism, we will call $A$ a non-unital algebra. Only the associativity law must hold.
            \end{remark}
            Alternatively, instead of using equations, we may represent the laws with commutative diagrams. 
            \begin{center}
                (associativity)\quad
                \begin{tikzcd}
                    A \otimes_{\mathbb{K}} A \otimes_{\mathbb{K}} A \ar[]{r}{(\nabla_A)\otimes id_{\mathbb{K}}} \ar[]{d}[]{id_{\mathbb{K}}\otimes (\nabla_A)} & A \otimes_{\mathbb{K}} A \ar[]{d}{(\nabla_A)} \\
                    A \otimes_{\mathbb{K}} A \ar[]{r}[]{(\nabla_A)} & A
                \end{tikzcd} \\
                (unitality) \quad
                \begin{tikzcd}
                    A \otimes_{\mathbb{K}} \mathbb{K} \ar[]{r}[]{id_A \otimes \upsilon_A} \ar[]{rd}[below]{\simeq} & A \otimes_{\mathbb{K}} A \ar[]{d}[]{(\nabla_A)} & \mathbb{K} \otimes_{\mathbb{K}} A \ar[]{l}[above]{\upsilon_A \otimes id_A} \ar[]{ld}[]{\simeq}\\
                    & A
                \end{tikzcd}
            \end{center}
            We may also present the structure of algebras by electric circuits. Such circuits are read from top to bottom, where morphisms are composed by lines. Morphisms in such diagrams may be highlighted with figures, conjunctions or twistings. E.g. The multiplication operator may be represented as a converging fork, and the unit as a source.
            \begin{center}
                \begin{tikzpicture}[line cap=round,line join=round,>=triangle 45,x=1cm,y=1cm, thick, op/.style={circle, draw, scale = 0.75}, scale = 0.7]
                    \node at (-2.25, 0) {(Multiplication)};
                    
                    \node (1) at (1,1) {};
                    \node (2) at (-1,1) {};
                    \node[op] (3) at (0,0) {$\nabla_A$};
                    \node (4) at (0,-1) {};

                    \graph {
                        (1) --[line width = 1pt] (3);
                        (2) --[line width = 1pt] (3);
                        (3) --[line width = 1pt] (4);
                    };

                    \node at (1.5,0) {$=$};

                    \draw [line width=1pt] (2,1)-- (3,0);
                    \draw [line width=1pt] (3,0)-- (3,-1);
                    \draw [line width=1pt] (4,1)-- (3,0);
                \end{tikzpicture} \quad
                \begin{tikzpicture}[line cap=round,line join=round,>=triangle 45,x=1cm,y=1cm, thick, op/.style={circle, draw, scale=0.75}, scale=0.7]
                    \node at (-1.5,0.5) {(Unit)};
                    
                    \node[op] (1) at (0,1) {$\upsilon_A$};
                    \draw [line width=1pt] (1) -- (0,0);

                    \node at (0.75, 0.5) {$=$};

                    \node[op, scale=1] (2) at (1.25,1) {};
                    \draw [line width=1pt] (2) -- (1.25,0);

                    \node at (0,-0.5) {};
                \end{tikzpicture}
            \end{center}
            With these operators we obtain the electric laws for an algebra.
            \begin{center}
                \begin{tikzpicture}[line cap=round,line join=round,>=triangle 45,x=1cm,y=1cm, thick, op/.style={circle, draw, scale=0.75}, scale=0.7]
                    \node at (-2.4,0) {(Associativity)};

                    \draw [line width=1pt] (0,1) -- (1,0);
                    \draw [line width=1pt] (1,0) -- (2,1);
                    \draw [line width=1pt] (0.5,0.5) -- (1,1);
                    \draw [line width=1pt] (1,0) -- (1,-1);

                    \node at (2.5,0) {$=$};

                    \draw [line width=1pt] (3,1) -- (4,0);
                    \draw [line width=1pt] (4,0) -- (5,1);
                    \draw [line width=1pt] (4,1) -- (4.5,0.5);
                    \draw [line width=1pt] (4,0) -- (4,-1);
                \end{tikzpicture} \\

                \begin{tikzpicture}[line cap=round,line join=round,>=triangle 45,x=1cm,y=1cm, thick, op/.style={circle, draw, scale=0.75}, scale=0.7]
                    \node at (-2,0) {(unitality)};

                    \node[op, scale=0.75] (1) at (0.25, 0.75) {};
                    \draw [line width=1pt] (1) -- (1,0);
                    \draw [line width=1pt] (1,0) -- (2,1);
                    \draw [line width=1pt] (1,0) -- (1,-1);

                    \node at (2.25,0) {$=$};

                    \draw [line width=1.5pt] (3,1) -- (3,-1);

                    \node at (3.75,0) {$=$};

                    \draw [line width=1pt] (4,1) -- (5,0);
                    \node[op, scale=0.75] (2) at (5.75,0.75) {};
                    \draw [line width=1pt] (5,0) -- (2);
                    \draw [line width=1pt] (5,0) -- (5,-1);
                \end{tikzpicture}
            \end{center}

            \begin{definition}[Algebra homomorphisms]
                Let $A$ and $B$ be algebras. Then $f: A\rightarrow B$ is an algebra homomorphism if
                \begin{enumerate}
                    \item $f$ is $\mathbb{K}$-linear
                    \item $f(ab)=f(a)f(b)$
                    \item $f(\upsilon_A) = \upsilon_B$
                \end{enumerate}
                Whenever $A$ and $B$ are non-unital, we only require 1 and 2 for a homomorphism of non-unital algebras.
            \end{definition}

            \begin{definition}[Category of algebras]
                \begin{itemize}
                    \item Let $Alg_{\mathbb{K}}$ denote the category of algebras. It's objects consists of every algebra $A$, and the morphisms are algebra homomorphisms. The sets of morphisms between $A$ and $B$ are denoted as $Alg_{\mathbb{K}}(A,B)$.
                    \item Let $nAlg_{\mathbb{K}}$ denote the category of non-unital algebras. It's objects consists of every non-unital algebra $A$, and the morphisms are non-unital algebra homomorphisms. The sets of morphisms between $A$ and $B$ are denoted as $nAlg_{\mathbb{K}}(A,B)$.
                \end{itemize}
            \end{definition}

            \begin{definition}[Augmented algebras]
                Let $A$ be an algebra. It is called augmented if there is an algebra homomorphism $\varepsilon : A \rightarrow \mathbb{K}$.
            \end{definition}

            If $A$ is an augmented algebra, then it decomposes into $\mathbb{K}\oplus Ker\varepsilon$ as a module. The splitting is given by unitality of the morphism $\varepsilon: A \rightarrow \mathbb{K}$, as we know that $\varepsilon(\upsilon_A) = id_{\mathbb{K}}$. The kernel of $\varepsilon$ is called the augmentation ideal or redecued algebra and we will denote it as $\bar{A}$. Taking kernels gives an equivalence of categories between augmented algebras and non-unital algebras, with unitization as the quasi-inverse.

            \begin{definition}[Tensor algebra]
                Let $V$ be a $\mathbb{K}$-module. We define the tensor algebra $T(V)$ of $V$ as the module
                \begin{align*}
                    T(V) = \mathbb{K}\oplus V\oplus V^{\otimes 2} \oplus V^{\otimes 3} \oplus ...
                \end{align*}
                Given two strings $v^1..v^i$ and $w^1...w^j$ in $T(V)$ we define the multiplication by the concatenation operation.
                \begin{align*}
                    \nabla_{T(V)} : T(V)\otimes_{\mathbb{K}} T(V) & \rightarrow T(V) \\
                    (v^1...v^i)\otimes(w^1...w^j) & \mapsto v^1...v^iw^1...w^j
                \end{align*}
                The unit is given by including $\mathbb{K}$ into $T(V)$.
                \begin{align*}
                    \upsilon_{T(V)} : \mathbb{K} & \rightarrow T(V) \\
                    1 & \mapsto 1
                \end{align*}
            \end{definition}

            Observe that the tensor algebra is augmented. The projection from $T(V)$ into $\mathbb{K}$ is an algebra homomorphism, so we may split the tensor algebra into its unit and its augmentation ideal $T(V) \simeq \mathbb{K}\oplus\bar{T(V)}$. We call $\bar{T(V)}$ the reduced tensor algebra.

            \begin{proposition}[Tensor algebra is free]
                The tensor algebra is the free algebra over the category of $\mathbb{K}$-modules, i.e. for any $\mathbb{K}$-module $V$ there is a natural isomorphism $Hom_{\mathbb{K}}(V,A)\simeq Alg_{\mathbb{K}}(T(V),A)$.

                The reduced tensor algebra is the fre non-unital algebra over the category of $\mathbb{K}$-modules, i.e. for any $\mathbb{K}$-module $V$ there is a natural isomorphism $Hom_{\mathbb{K}}(V,A)\simeq nAlg_{\mathbb{K}}(\bar{T(V)},A)$.
            \end{proposition}

            \begin{proof}
                This proposition should be evident from the description of an algebra homomorphism from a tensor algebra. If $f: T(V) \rightarrow A$ is an algebra homomorphism, then $f$ must satisfy the following conditions:
                \begin{itemize}
                    \item (Unitality)\quad $f(1) = 1$
                    \item (Homomorphism property)\quad Given $v,w\in V$, then $f(vw) = f(v)\nabla_Af(w)$
                \end{itemize}
                By induction, we see that $f$ is completely determined by where it sends the elements of $V$. Thus restriction by the inclusion of $V$ into $T(V)$ induces a bijection.
            \end{proof}

            \begin{definition}[Modules]
                Let $A$ be an algebra. A $\mathbb{K}$-module $M$ is said to be a left (right) $A$-module if there exists a structure morphism $\mu_M : A\otimes_{\mathbb{K}}M \rightarrow A$ ($\mu_M : M\otimes_{\mathbb{K}}A \rightarrow A$) called multiplication. We require that $\mu_M$ is associative with respect to the multiplication and preserves the unit of $A$, i.e. the electric laws are satisfied.
                \begin{center}
                    \begin{tikzpicture}[line cap=round,line join=round,>=triangle 45,x=1cm,y=1cm, thick, op/.style={circle, draw, scale=0.75}, scale=0.7]
                        \node at (-2.4,0) {(Associativity)};
                        
                        \node at (0,1.5) {A};
                        \node at (1,1.5) {A};
                        \node at (2,1.5) {M};

                        \node[op, scale = 0.5] (a) at (1,0) {$\mu_M$};

                        \draw [line width=1pt] (0,1) -- (a);
                        \draw [line width=1pt] (a) -- (2,1);
                        \draw [line width=1pt] (0.5,0.5) -- (1,1);
                        \draw [line width=1pt] (a) -- (1,-1);
                        
                        \node at (2.5,0) {$=$};
                        
                        \node at (3,1.5) {A};
                        \node at (4,1.5) {A};
                        \node at (5,1.5) {M};

                        \node[op, scale = 0.5] (b) at (4.5,0.5) {$\mu_M$};
                        \node[op, scale = 0.5] (c) at (4,0) {$\mu_M$};

                        \draw [line width=1pt] (3,1) -- (c);
                        \draw [line width=1pt] (c) -- (b) -- (5,1);
                        \draw [line width=1pt] (4,1) -- (b);
                        \draw [line width=1pt] (c) -- (4,-1);
                    \end{tikzpicture}\\

                    \begin{tikzpicture}[line cap=round,line join=round,>=triangle 45,x=1cm,y=1cm, thick, op/.style={circle, draw, scale=0.75}, scale=0.7]
                        \node at (-2,0) {(unitality)};
    
                        \node at (1.75,1) {M};

                        \node[op, scale=0.75] (1) at (0.25, 0.75) {};
                        \draw [line width=1pt] (1) -- (1,0);
                        \draw [line width=1pt] (1,0) -- (1.75,0.75);
                        \draw [line width=1pt] (1,0) -- (1,-1);
    
                        \node at (2.25,0) {$=$};
    
                        \node at (3,1) {M};

                        \draw [line width=1.5pt] (3,0.75) -- (3,-1);
                    \end{tikzpicture}
                \end{center}
            \end{definition}

            \begin{definition}[A-linear homomorphisms]
                Let $M,N$ be two left $A$-modules. A morphism $f:M\rightarrow N$ is called $A$-linear if it is $\mathbb{K}$-linear and for any $a$ in $A$, $f(am) = af(m)$.
            \end{definition}

            The category of left $A$-modules is denoted as $Mod_A$, where the morphisms $Hom_A(\_,\_)$ are $A$-linear. Likewise, the category of right $A$-modules is denoted as $Mod^A$. 

            \begin{proposition}
                Let $M$ be a $\mathbb{K}$-module. The module $A\otimes_{\mathbb{K}}M$ is a left $A$-module. Moreover, it is the free left module over $\mathbb{K}$-modules, i.e. there is an isomorphism $Hom_{\mathbb{K}}(M,N)\simeq Hom_{A}(A\otimes_{\mathbb{K}}M,N)$.
            \end{proposition}
            
        \subsection{Coalgebras}
            This subsection aims to dualize the definitions from last section. To this end we will define counital coassociative coalgebras and non-counital coassociative coalgebras, which will be called coalgebras and non-counital coalgebras respectively. The collection of coalgebras together with coalgebra homomorphisms is the category $CoAlg_{\mathbb{K}}$. Due to some ill-behavior, this dualization is only a true dualization under some finiteness conditions for the algebras. Thus we will see that the proper dual concept will be of conilpotent coalgebras. We will see that the cofree coalgebra is conilpotent.

            \begin{definition}[Coalgebra]
                Let $\mathbb{K}$ be a field. A coalgebra $C$ over $\mathbb{K}$ is a $\mathbb{K}$-module with structure morphisms called comultiplication and counit,
                \begin{align*}
                    (\Delta_C) & : C \rightarrow C\otimes_{\mathbb{K}}C \\
                    \varepsilon_C & : C \rightarrow \mathbb{K},
                \end{align*}
                satisfying the coassociativity and coidentity laws. 
                \begin{align*}
                    (coassociativity)\quad & (\Delta_C\otimes id_C)\circ\Delta_C(c) = (id_C\otimes\Delta_C)\circ\Delta_C(c) \\
                    (counitality) \quad & (id_C\otimes\varepsilon_C)\circ\Delta_C(c) = c = (\varepsilon_C\otimes id_C)\circ\Delta_C(c)
                \end{align*}
            \end{definition}

            We define repeated application of comultiplication as $\Delta_C^n = (\Delta_C\otimes id_C\otimes ...)\circ\Delta_C^{n-1}$. Notice that the choice of where we put comultiplication in the tensor does not matter, as coassociativity require all of the choices to be equal.
            
            We may dualize the electric circuits of an algebra to coalgebras. In this manner our structure morphisms would be upside down relative to the algebra morphisms. Thus comultiplication becomes a diverging fork and counit is a sink. 
            \begin{center}
                \begin{tikzpicture}[line cap=round,line join=round,>=triangle 45,x=1cm,y=1cm, thick, op/.style={circle, draw, scale = 0.75}, scale = 0.7]
                    \node at (-2.5, 0) {(Comultiplication)};
                    
                    \node (1) at (0,1) {};
                    \node (2) at (-1,-1) {};
                    \node[op] (3) at (0,0) {$\Delta_C$};
                    \node (4) at (1,-1) {};

                    \graph {
                        (1) --[line width = 1pt] (3);
                        (2) --[line width = 1pt] (3);
                        (3) --[line width = 1pt] (4);
                    };

                    \node at (1.5,0) {$=$};

                    \draw [line width=1pt] (2,-1)-- (3,0);
                    \draw [line width=1pt] (3,0)-- (3,1);
                    \draw [line width=1pt] (4,-1)-- (3,0);
                \end{tikzpicture} \quad
                \begin{tikzpicture}[line cap=round,line join=round,>=triangle 45,x=1cm,y=1cm, thick, op/.style={circle, draw, scale=0.75}, scale=0.7]
                    \node at (-1.5,-0.5) {(Counit)};
                    
                    \node[op] (1) at (0,-1) {$\varepsilon_C$};
                    \draw [line width=1pt] (1) -- (0,0);

                    \node at (0.75, -0.5) {$=$};

                    \node[op, scale=1] (2) at (1.25,-1) {};
                    \draw [line width=1pt] (2) -- (1.25,0);

                    \node at (0,0.5) {};
                \end{tikzpicture}
            \end{center}
            We then obtain the electric laws for a coalgebra by flipping the diagrams around.
            \begin{center}
                \begin{tikzpicture}[line cap=round,line join=round,>=triangle 45,x=1cm,y=1cm, thick, op/.style={circle, draw, scale=0.75}, scale=0.7]
                    \node at (-2.4,0) {(Coassociativity)};

                    \draw [line width=1pt] (0,-1) -- (1,0);
                    \draw [line width=1pt] (1,0) -- (2,-1);
                    \draw [line width=1pt] (0.5,-0.5) -- (1,-1);
                    \draw [line width=1pt] (1,0) -- (1,1);

                    \node at (2.5,0) {$=$};

                    \draw [line width=1pt] (3,-1) -- (4,0);
                    \draw [line width=1pt] (4,0) -- (5,-1);
                    \draw [line width=1pt] (4,-1) -- (4.5,-0.5);
                    \draw [line width=1pt] (4,0) -- (4,1);
                \end{tikzpicture} \\

                \begin{tikzpicture}[line cap=round,line join=round,>=triangle 45,x=1cm,y=1cm, thick, op/.style={circle, draw, scale=0.75}, scale=0.7]
                    \node at (-2,0) {(Counitality)};

                    \node[op, scale=0.75] (1) at (0.25, -0.75) {};
                    \draw [line width=1pt] (1) -- (1,0);
                    \draw [line width=1pt] (1,0) -- (2,-1);
                    \draw [line width=1pt] (1,0) -- (1,1);

                    \node at (2.25,0) {$=$};

                    \draw [line width=1.5pt] (3,1) -- (3,-1);

                    \node at (3.75,0) {$=$};

                    \draw [line width=1pt] (4,-1) -- (5,0);
                    \node[op, scale=0.75] (2) at (5.75,-0.75) {};
                    \draw [line width=1pt] (5,0) -- (2);
                    \draw [line width=1pt] (5,0) -- (5,1);
                \end{tikzpicture}
            \end{center}

            \begin{definition}[Coalgebra homomorphism]
                Let $C$ and $D$ be coalgebras. Then $f:C\rightarrow D$ is a coalgebra morphism if
                \begin{enumerate}
                    \item $f$ is $\mathbb{K}$-linear
                    \item $(f\otimes f)\circ\Delta_C(c) = \Delta_D(f(c))$
                    \item $\varepsilon_D(f) = \varepsilon_C$
                \end{enumerate}
                Whenever $C$ and $D$ are non-counital, we only require 1 and 2 for a homomorphism of non-counital coalgebras.
            \end{definition}

            \begin{definition}[Category of Coalgebras]
                \begin{itemize}
                    \item Let $CoAlg_{\mathbb{K}}$ denote the category of coalgebras. It's objects consists of every coalgebra $C$, and the morphisms are coalgebra homomorphisms. The sets of morphisms between $C$ and $D$ are denoted as $CoAlg_{\mathbb{K}}(C,D)$.
                    \item Let $nCoAlg_{\mathbb{K}}$ denote the category of non-unital algebras. It's objects consists of every non-unital algebra $C$, and the morphisms are non-unital algebra homomorphisms. The sets of morphisms between $C$ and $D$ are denoted as $nCoAlg_{\mathbb{K}}(C,D)$.
                \end{itemize}
            \end{definition}

            \begin{example}[The coalgebra $\mathbb{K}$]
                The field $\mathbb{K}$ can be given a coalgebra structure over itself. Since $\{1\}$ is a basis for $\mathbb{K}$ we define the structure morphisms as
                \begin{align*}
                    \Delta_{\mathbb{K}}(1) & = 1\otimes 1 \\
                    \varepsilon(1) & = 1.
                \end{align*}
                One may check that these morphisms are indeed coassociative and counital. Thus we may regard our field as either an algebra or coalgebra over itself.
            \end{example}

            \begin{definition}[Coaugmented coalgebras]
                Let $C$ be a coalgebra. $C$ is coagumented if there is a coalgebra homomorphism $\upsilon:\mathbb{K}\rightarrow C$.
            \end{definition}

            If $C$ is a coaugmented coalgebra, then it splits as $C\simeq \mathbb{K}\oplus Cok\upsilon$. The splitting is given by counitality of $\upsilon$, as $\varepsilon_C(\upsilon) = id_{\mathbb{K}}$. We call the cokernel $Cok\upsilon = \bar{C}$ for the coaugmentation quotient or reduced coalgebra, and its reduced coproduct may be explicitly given as
            \begin{align*}
                \bar{\Delta}_C(c) = \Delta_C(c) - 1\otimes c - c\otimes 1. 
            \end{align*}

            \begin{definition}[Tensor Coalgebras]
                Let $V$ be a $\mathbb{K}$-module. We define the tensor coalgebra $T^c(V)$ of $V$ as the module
                \begin{align*}
                    T^c(V) = \mathbb{K}\oplus V\oplus V^{\otimes 2}\oplus V^{\otimes 3}\oplus ...
                \end{align*}
                Given a string $v^1...v^i$ in $T(V)$ we define the comultiplication by the deconcatenation operation.
                \begin{align*}
                    \Delta_{T^c(V)}:T^c(V) & \rightarrow T^c(V)\otimes_{\mathbb{K}}T^c(V) \\
                    v^1...v^i & \mapsto 1\otimes(v^1...v^i) + (\sum_{j=1}^{n-1} (v^1...v^{j})\otimes(v^{j+1}...v^i)) + (v^1...v^i)\otimes 1
                \end{align*}
                The counit is given by projecting $T^c(V)$ onto $\mathbb{K}$.
                \begin{align*}
                    \varepsilon_{T^c(V)} : T^c(V) & \rightarrow \mathbb{K} \\
                    1 & \mapsto 1 \\
                    v^1...v^i & \mapsto 0
                \end{align*}
            \end{definition}

            Notice that the tensor coalgebra is coaugmented. Its coaugmentation is given by the inclusion of $\mathbb{K}$ into $T^c(V)$. We may split $T^c(V) \simeq \mathbb{K}\oplus \bar{T^c}(V)$, where $\bar{T^c}(V)$ is the reduced tensor coalgebra.

            In order to get cofreeness for the tensor coalgebra we need some finiteness conditions. This is one of the properties which is ill-behaved when we are dualizing the tensor algebra. The extra assumption which we will need is to assume that the coalgebras are conilpotent. Let $C \simeq \mathbb{K} \oplus \bar{C}$ be a coaugmented coalgebra, we define the coradical filtration of $C$ as a filtration $Fr_0C \subseteq Fr_1C \subseteq ... \subseteq Fr_rC \subseteq ...$ by the submodules:
            \begin{align*}
                Fr_0C & = \mathbb{K} \\
                Fr_rC & = \mathbb{K} \oplus \{c\in\bar{C}\mid \forall n\geq r \bar{\Delta}_C(c) = 0\}.
            \end{align*}

            \begin{definition}[Conilpotent coalgebras]
                Let $C$ be a coaugmented coalgebra. We say that $C$ is conilpotent if its coradical filtration is exhaustive, i.e. $\substack{\varinjlim \\ r}Fr_rC \simeq C$. The subcategory of conilpotent coalgebras will be denoted as $CoAlg^{Conil}_{\mathbb{K}}$.
            \end{definition}
            
            \begin{proposition}[Conilpotent tensor coalgebra]
                Let $V$ be a $\mathbb{K}$-module. The tensor coalgebra $T^c(V)$ is conilpotent.
            \end{proposition}

            \begin{proof}
                Let $v\in V$, then $\Delta_{T^c(V)}(v)=1\otimes v + v\otimes 1$ and $\bar{\Delta}_{T^c(V)}(v)=0$. We then observe the following:
                \begin{align*}
                    Fr_0T^c(V) & = \mathbb{K} \\
                    Fr_1T^c(V) & = \mathbb{K} \oplus V \\
                    Fr_rT^c(V) & = \bigoplus_{i\leq r} V^{\otimes i}
                \end{align*}
                This shows that the coradical filtration is exhaustive.
            \end{proof}

            \begin{proposition}[Cofree tensor coalgebra]
                The tensor coalgebra is the cofree conilpotent coalgebra over the category of $\mathbb{K}$-modules, i.e. for any $\mathbb{K}$-module $V$ and any conilpotent coalgebra $C$ there is a natural isomorphism $Hom_{\mathbb{K}}(\bar{C}, V)\simeq CoAlg^{Conil}_{\mathbb{K}}(C, T^c(V))$.
            \end{proposition}

            \begin{proof}
                This proposition should be evident from the description of a coalgebra homomorphism into the a tensor coalgebra. If $g:C\rightarrow T^c(V)$ is a coalgebra homomorphism, then $g$ must satisfy the following conditions:
                \begin{enumerate}
                    \item (Coaugmentation)\quad $g(1)=1$
                    \item (Counitality)\quad Given $c\in \bar{C}$ then $\varepsilon_{T^c(V)}\circ g(c)=0$
                    \item (Homomorphism property)\quad Given $c\in C$ then $\Delta_{T^c(V)}(g(c))=(g\otimes g)\circ\Delta_C(c)$
                \end{enumerate}

                We will construct the maps for the isomorphism explicitly. If $g:C\rightarrow T^c(V)$ is a coalgebra homomorphism, then composing with projection gives a map $\pi\circ g:C\rightarrow V$. Note that $\pi\circ g(1)=0$, so this is essentially a map $\pi\circ g:\bar{C}\rightarrow V$. For the other direction, let $\bar{g}:\bar{C}\rightarrow V$. Then we define $g$ as
                \begin{align*}
                    g = id_{\mathbb{K}} \oplus \sum_{i=1}^{\infty}(\otimes^{i}\bar{g})\bar{\Delta}_C^{i-1}.
                \end{align*}
                Observe that $g$ is well defined, since convergence of the sum follows from conilpotency of $C$. One may then check that $g$ is a coalgebra homomorphism, which yields the result.
            \end{proof}

            \begin{definition}[Comodules]
                Let $C$ be a coalgebra. A $\mathbb{K}$-module $M$ is said to ba left (right) $C$-comodule if there exist a structure morphism $\omega_M: M \rightarrow C\otimes_{\mathbb{K}}M$ ($\omega_M: M \rightarrow M\otimes_{\mathbb{K}}C$) called comultiplication. We require that $\omega_M$ is coassociative with respect to the comultiplication of $C$ and preserves the counit of $C$, i.e. the electric laws are satisfied.
                \begin{center}
                    \begin{tikzpicture}[line cap=round,line join=round,>=triangle 45,x=1cm,y=1cm, thick, op/.style={circle, draw, scale=0.75}, scale=0.7]
                        \node at (-2.4,0) {(Coassociativity)};
                        
                        \node at (1,1.5) {M};

                        \node[op, scale = 0.5] (a) at (1,0) {$\omega_M$};

                        \draw [line width=1pt] (0,-1) -- (a);
                        \draw [line width=1pt] (a) -- (2,-1);
                        \draw [line width=1pt] (0.5,-0.5) -- (1,-1);
                        \draw [line width=1pt] (a) -- (1,1);
                        
                        \node at (2.5,0) {$=$};
                        
                        \node at (4,1.5) {M};

                        \node[op, scale = 0.5] (b) at (4.5,-0.5) {$\omega_M$};
                        \node[op, scale = 0.5] (c) at (4,0) {$\omega_M$};

                        \draw [line width=1pt] (3,-1) -- (c);
                        \draw [line width=1pt] (c) -- (b) -- (5,-1);
                        \draw [line width=1pt] (4,-1) -- (b);
                        \draw [line width=1pt] (c) -- (4,1);
                    \end{tikzpicture}\\

                    \begin{tikzpicture}[line cap=round,line join=round,>=triangle 45,x=1cm,y=1cm, thick, op/.style={circle, draw, scale=0.75}, scale=0.7]
                        \node at (-2,0) {(Counitality)};
    
                        \node at (1,1.5) {M};

                        \node[op, scale=0.75] (1) at (0.25, -0.75) {};
                        \draw [line width=1pt] (1) -- (1,0);
                        \draw [line width=1pt] (1,0) -- (1.75,-0.75);
                        \draw [line width=1pt] (1,0) -- (1,1);
    
                        \node at (2.25,0) {$=$};
    
                        \node at (3,1.5) {M};

                        \draw [line width=1.5pt] (3,-0.75) -- (3,1);
                    \end{tikzpicture}
                \end{center}
            \end{definition}

            \begin{definition}[C-colinear homomorphism]
                Let $M,N$ be two left $C$-comodules. A morphism $g:M\rightarrow N$ is called $C$-colinear if it is $\mathbb{K}$-linear and for any $m$ in $M$, $\omega_N(g(m)) = (id_C\otimes g)\omega_M(m)$.
            \end{definition}

            The category of left $C$-comodules is denoted as $CoMod_C$, where the morphisms $CoHom_C(\_,\_)$ are $C$-colinear. Likewise, the category of right $C$-comodules is denoted as $CoMod^C$.

            \begin{proposition}
                Let $M$ be a $\mathbb{K}$-module. The module $C\otimes_{\mathbb{K}}M$ is a left $C$-comodule. Moreover, it is the cofree left comodule over $\mathbb{K}$-modules, i.e. there is an isomorphism $Hom_{\mathbb{K}}(N,M)\simeq CoHom_C(N,C\otimes_{\mathbb{K}}M)$. 
            \end{proposition}

        \subsection{Differentials and Convolution Algebras}
            In this subsection we will look at differential graded objects and convolution products. We will define derivations and coderivations to obtain differential graded algebras and coalgebras. Moreover we will see that the set of homogenous homomorphisms between differential graded objects is itself differential graded. Moreover, whenever we look at morphisms between dg coalgebras and dg algebras, we can give this object the convolution operator, making the set a dg algebra.

            \begin{definition}[Derivations and Coderivations]
                Let $M$ be an $A$-bimodule. A $\mathbb{K}$-linear morphism $d:A\rightarrow M$ is called a derivation if $d(ab)=d(a)b+ad(b)$, i.e. electrically:

                \begin{center}
                    \begin{tikzpicture}[line cap=round,line join=round,>=triangle 45,x=1cm,y=1cm, thick, op/.style={circle, draw, scale=0.75}, scale=0.7]
                        
                        \node at (-0.5, 1) {a};
                        \node at (0.5, 1) {b};
                        \node[op, scale = 0.75] (d) at (0,-0.5) {d};
                        
                        \draw [line width=1pt] (-0.5, 0.75) -- (-0.5, 0.5) -- (0,0) -- (0.5,0.5) -- (0.5, 0.75);
                        \draw [line width=1pt] (0,0) -- (d) -- (0,-1);
                        
                        \node at (1,0) {$=$};

                        \node at (1.5, 1) {a};
                        \node at (2.5, 1) {b};

                        \node[op, scale = 0.75] (e) at (1.5, 0.25) {d};
                        \node[op, scale = 0.5] (r) at (2,-0.5) {$\mu_M^r$};

                        
                        \draw [line width=1pt] (1.5, 0.75) -- (e) -- (1.5, 0) -- (r) -- (2.5,0) -- (2.5,0.75);
                        \draw [line width=1pt] (r) -- (2,-1);

                        \node at (3,0) {$+$};

                        \node at (3.5, 1) {a};
                        \node at (4.5, 1) {b};

                        \node[op, scale = 0.75] (f) at (4.5, 0.25) {d};
                        \node[op, scale = 0.5] (l) at (4,-0.5) {$\mu_M^l$};

                        
                        \draw [line width=1pt] (3.5, 0.75) --  (3.5, 0) -- (l) -- (4.5,0) -- (f) -- (4.5,0.75);
                        \draw [line width=1pt] (l) -- (4,-1);
                    \end{tikzpicture}
                \end{center}

                Let $N$ be a $C$-bicomodule. A $\mathbb{K}$-linear morphism $d:N\rightarrow C$ is called a coderivation if $\Delta_C\circ d = (d\otimes id_C)\circ\omega_N^r + (id_C\otimes d)\circ\omega_N^l$, i.e. electrically:
                \begin{center}
                    \begin{tikzpicture}[line cap=round,line join=round,>=triangle 45,x=1cm,y=1cm, thick, op/.style={circle, draw, scale=0.75}, scale=0.7]
                        
                        \node[op, scale = 0.75] (d) at (0,0.5) {d};
                        
                        \draw [line width=1pt] (-0.5, -0.75) -- (-0.5, -0.5) -- (0,0) -- (0.5,-0.5) -- (0.5, -0.75);
                        \draw [line width=1pt] (0,0) -- (d) -- (0,1);
                        
                        \node at (1,0) {$=$};

                        \node[op, scale = 0.75] (e) at (1.5, -0.25) {d};
                        \node[op, scale = 0.5] (r) at (2,0.5) {$\omega_N^r$};

                        
                        \draw [line width=1pt] (1.5, -0.75) -- (e) -- (1.5, 0) -- (r) -- (2.5,0) -- (2.5,-0.75);
                        \draw [line width=1pt] (r) -- (2,1);

                        \node at (3,0) {$+$};

                        \node[op, scale = 0.75] (f) at (4.5, -0.25) {d};
                        \node[op, scale = 0.5] (l) at (4,0.5) {$\omega_N^l$};

                        
                        \draw [line width=1pt] (3.5, -0.75) --  (3.5, 0) -- (l) -- (4.5,0) -- (f) -- (4.5,-0.75);
                        \draw [line width=1pt] (l) -- (4,1);
                    \end{tikzpicture}
                \end{center}
            \end{definition}

            \begin{proposition}
                Let $V$ be a $\mathbb{K}$-module and $M$ be a $T(V)$-bimodule. A $\mathbb{K}$-linear morphism $f:V\rightarrow M$ uniquely determines a derivation $d_f:T(V)\rightarrow M$, i.e. there is an isomorphism $Hom_{\mathbb{K}}(V,M)\simeq Der(T(V),M)$.


                Let $N$ be a $T^c(V)$-comodule. A $\mathbb{K}$-linear morphism $g:M\rightarrow V$ uniquely determines a coderivation $d_g^c:N\rightarrow T^c(V)$, i.e. there is an isomorphism $Hom_{\mathbb{K}}(N,V)\simeq Coder(N,T^c(V))$.
            \end{proposition}

            \begin{proof}
                ...
            \end{proof}

            \begin{definition}[Differential algebra]
                Let $A$ be an algebra. We say that $A$ is a differential algebra if it is equipped with at least one derivation $d:A\rightarrow A$. Dually, a coalgebra $C$ is called differential if it is equipped with at least one coderivation $d:C\rightarrow C$.
            \end{definition}

            \begin{definition}[A-derivation]
                Let $(A,d_A)$ be a differential algebra and $M$ a left $A$-module. A $\mathbb{K}$-linear morphism $d_M:M\rightarrow M$ is called an $A$-derivation if $d_M(am)=d_A(a)m + ad_M(m)$, or electrically:
                \begin{center}
                    \begin{tikzpicture}[line cap=round,line join=round,>=triangle 45,x=1cm,y=1cm, thick, op/.style={circle, draw, scale=0.75}, scale=0.7]
                        
                        \node at (-0.5, 1) {a};
                        \node at (0.5, 1) {m};
                        \node[op, scale = 0.5] (d) at (0,-0.5) {$d_M$};
                        
                        \draw [line width=1pt] (-0.5, 0.75) -- (-0.5, 0.5) -- (0,0) -- (0.5,0.5) -- (0.5, 0.75);
                        \draw [line width=1pt] (0,0) -- (d) -- (0,-1);
                        
                        \node at (1,0) {$=$};

                        \node at (1.5, 1) {a};
                        \node at (2.5, 1) {m};

                        \node[op, scale = 0.5] (e) at (1.5, 0.25) {$d_A$};

                        
                        \draw [line width=1pt] (1.5, 0.75) -- (e) -- (1.5, 0) -- (2,-0.5) -- (2.5,0) -- (2.5,0.75);
                        \draw [line width=1pt] (2,-0.5) -- (2,-1);

                        \node at (3,0) {$+$};

                        \node at (3.5, 1) {a};
                        \node at (4.5, 1) {m};

                        \node[op, scale = 0.5] (f) at (4.5, 0.25) {$d_M$};
                        
                        \draw [line width=1pt] (3.5, 0.75) --  (3.5, 0) -- (4,-0.5) -- (4.5,0) -- (f) -- (4.5,0.75);
                        \draw [line width=1pt] (4,-0.5) -- (4,-1);
                    \end{tikzpicture}
                \end{center}
                Dually, given a differential coalgebra $(C,d_C)$ and $N$ a left $C$-comodule, a $\mathbb{K}$-linear morphism $d_N:N\rightarrow N$ is a coderivation if $\omega_N\circ d_N = (d_C\otimes id_N + id_C\otimes d_N)\circ \omega_N$, or electrically:
                \begin{center}
                    \begin{tikzpicture}[line cap=round,line join=round,>=triangle 45,x=1cm,y=1cm, thick, op/.style={circle, draw, scale=0.75}, scale=0.7]
                        
                        \node[op, scale = 0.5] (d) at (0,0.5) {$d_N$};
                        
                        \draw [line width=1pt] (-0.5, -0.75) -- (-0.5, -0.5) -- (0,0) -- (0.5,-0.5) -- (0.5, -0.75);
                        \draw [line width=1pt] (0,0) -- (d) -- (0,1);
                        
                        \node at (1,0) {$=$};

                        \node[op, scale = 0.5] (e) at (1.5, -0.25) {$d_C$};
                        
                        \draw [line width=1pt] (1.5, -0.75) -- (e) -- (1.5, 0) -- (2,0.5) -- (2.5,0) -- (2.5,-0.75);
                        \draw [line width=1pt] (2,0.5) -- (2,1);

                        \node at (3,0) {$+$};

                        \node[op, scale = 0.5] (f) at (4.5, -0.25) {$d_N$};
                        
                        \draw [line width=1pt] (3.5, -0.75) --  (3.5, 0) -- (4,0.5) -- (4.5,0) -- (f) -- (4.5,-0.75);
                        \draw [line width=1pt] (4,0.5) -- (4,1);
                    \end{tikzpicture}
                \end{center}
            \end{definition}

            \begin{proposition}
                Let $A$ be a differential algebra and $M$ a $\mathbb{K}$-module. A $\mathbb{K}$-linear morphism $f:M\rightarrow A\otimes_{\mathbb{K}} M$ uniquely determines a derivation $d_f:A\otimes M\rightarrow A\otimes M$, i.e. there is an isomorphism $Hom_{\mathbb{K}}(M,A\otimes_{\mathbb{K}}M)\simeq Der(A\otimes_{\mathbb{K}}M)$. Moreover, $d_f$ is given as $(\nabla_A\otimes id_M)\circ (id_A\otimes f) + d_A\otimes id_M$.

                Dually, if $C$ is a differential coalgebra and $N$ is a $\mathbb{K}$-module, then a $\mathbb{K}$-linear morphism $g:C\otimes N\rightarrow N$ uniquely determines a coderivation $d_g:C\otimes_{\mathbb{K}}N\rightarrow C\otimes_{\mathbb{K}}N$. There is an isomorphism $Hom_{\mathbb{K}}(C\otimes_{\mathbb{K}}N,N)\simeq Coder(C\otimes_{\mathbb{K}}N)$, and $d_g$ is given as $(id_C\otimes g)\circ (\Delta_C\otimes id_N) + d_C\otimes id_N$.
            \end{proposition}

            \begin{proof}
                ...
            \end{proof}

            Recall that a module $M^*$ is $\mathbb{Z}$ graded if it decomposes as a sum $M^* = \substack{\bigoplus \\ z:\mathbb{Z}}M^z$. Let $M^*,N^*$ be graded modules and $f:M^*\rightarrow N^*$ is a homogenous $\mathbb{K}$-linear morphism of degree $n$ if it preserves the grading, that is $f(M^i) \subseteq N^{n+i}$. We denote the degree of $f$ as $|f|$. The category of graded modules will be denoted as $GrMod_{\mathbb{K}}$, and the graded module of morphisms between two graded objects is denoted as $Hom_{\mathbb{K}}^*(M^*,N^*)$. We will use the Koszul-sign convention for this category, so a graded derivation uses Koszul sign rule to determine the sign.
            

            $M^{\bullet}$ is called a chain complex if it comes equipped with a homogenous morphism of degree $1$, like $d_M^{\bullet}:M^{\bullet}\rightarrow M^{\bullet}$, such that ${d_M^{\bullet}}^2=0$. A chain morphism $f: M^{\bullet}\rightarrow N^{\bullet}$ is a homogenous $\mathbb{K}$-linear morphism of degree $0$, such that $f\circ d_M^{\bullet} = d_N^{\bullet}\circ f$. The category of chain complexes will be denoted as $ChMod_{\mathbb{K}}$, and the module of morphisms between two chain complexes is denoted as $Hom_{\mathbb{K}}^{\bullet}(M^{\bullet},N^{\bullet})$.

            \begin{proposition}
                Let $M^{\bullet}$ and $N^{\bullet}$ be two chain complexes. The graded module of morphisms $Hom_{\mathbb{K}}^*(M^{\bullet},N^{\bullet})$ is a chain complex, given by the differential $\partial(f) = d_N^{\bullet}\circ f - (-1)^{|f|}f\circ d_M^{\bullet}$.
            \end{proposition}

            \begin{proof}
                We observe that $\partial : Hom_{\mathbb{K}}^*(M^{\bullet},N^{\bullet}) \rightarrow Hom_{\mathbb{K}}^*(M^{\bullet},N^{\bullet})$ is a morphism of degree $1$. It remains to check that $\partial^2 = 0$. Pick any homogenous morphism $f : M^{\bullet}\rightarrow N^{\bullet}$.
                \begin{multline*}
                    \partial^2(f) = \partial(d_N^{\bullet}\circ f - (-1)^{|f|}f\circ d_M^{\bullet}) = \partial(d_N^{\bullet}\circ f) - (-1)^{|f|}\partial(f\circ d_M^{\bullet}) \\ = - (-1)^{|d_N^{\bullet}\circ f|}d_N^{\bullet}\circ f\circ d_M^{\bullet} + (-1)^{|f|}d_N^{\bullet}\circ f\circ d_M^{\bullet} = 0
                \end{multline*}
            \end{proof}

            Observe that $f:M^{\bullet}\rightarrow N^{\bullet}$ of degree $0$ is a chain morphism if and only if $\partial(f) = 0$. We then observe that $Hom_{\mathbb{K}}^{\bullet}(M^{\bullet},N^{\bullet})\simeq Z^0Hom_{\mathbb{K}}^*(M^{\bullet})$


            To complete the definitions of graded modules and chain complexes to algebras we would like the structure morphisms to respect the given structure. E.g. if $a$ and $b$ are homogenous elements, we would like that the degree of $ab$ is the sum of its parts, i.e. $|ab| = |a| + |b|$. Since multiplication by identity doesn't do anything, we want that the identity lives in the $0$'th degree, and so forth.

            \begin{definition}[Graded algebra]
                Let $A^*$ be a graded $\mathbb{K}$-module. We say that $A^*$ is a graded algebra if $A^*$ is an algebra such that $\nabla_A$ and $\upsilon_A$ are homogenous and of degree $0$.
                Dually, $C^*$ is a graded coalgebra if $\Delta_C$ and $\varepsilon_C$ are homogenous and of degree $0$.
            \end{definition}

            \begin{definition}[Differential graded algebra]
                Let $A^{\bullet}$ be a chain complex over $\mathbb{K}$. We say that $A^{\bullet}$ is a differential graded algebra, or dg algebra, if it is a graded algebra and the differential is a graded derivation, i.e. $d_A(ab) = d_A(a)b + (-1)^{|a|}ad_A(b)$.

                Dually, $C^{\bullet}$ is a differential graded coalgebra if $C^{\bullet}$ is a graded coalgebra and the differential is a graded coderivation.
            \end{definition}

            Let $C$ be a coalgebra and $A$ an algebra, then if $f,g:C\rightarrow A$ are $\mathbb{K}$-linear morphism we may define $f\star g = \nabla_A(f\otimes g)\Delta_C$. We call the operation $\star$ for convolution.

            \begin{proposition}[Convolution algebra]
                The $\mathbb{K}$-module $Hom_{\mathbb{K}}(C,A)$ is an associative algebra when equipped with convolution $\star:Hom_{\mathbb{K}}(C,A)\rightarrow Hom_{\mathbb{K}}(C,A)$. The unit is given by $1 \mapsto \upsilon_A\circ\varepsilon_C$.
            \end{proposition}

            \begin{proof}
                This proposition follows from (co)associativity and (co)unitality of (C) A.

                \begin{center}
                    \begin{tikzpicture}[line cap=round,line join=round,>=triangle 45,x=1cm,y=1cm, thick, op/.style={circle, draw, scale=0.75}, scale=0.7]
                        % \node at (-3, 0) {Associativity};
                        
                        \node at (-3,0) {$(f\star g) \star h$};

                        \node at (-1, 0) {$=$};

                        \node[op, scale = 0.75] (f1) at (0,0) {f};
                        \node[op, scale = 0.75] (g1) at (1,0) {g};
                        \node[op, scale = 0.75] (h1) at (2,0) {h};

                        \draw (1, 1.5) -- (1,1.25) -- (0.5,1) -- (0.5, 0.75) -- (0,0.5) -- (f1) -- (0,-0.5) -- (0.5, -0.75) -- (0.5,-1) -- (1,-1.25) -- (1, -1.5);
                        \draw (0.5, 0.75) -- (1,0.5) -- (g1) -- (1, -0.5) -- (0.5, -0.75);
                        \draw (1,1.25) -- (2,0.75) -- (h1) -- (2, -0.75) -- (1, -1.25);

                        \node at (3,0) {$=$};

                        \node[op, scale = 0.75] (f2) at (4,0) {f};
                        \node[op, scale = 0.75] (g2) at (5,0) {g};
                        \node[op, scale = 0.75] (h2) at (6,0) {h};

                        \draw (5, 1.5) -- (5,1.25) -- (4, 0.75) -- (f2) -- (4,-0.5) -- (4.5, -0.75) -- (4.5,-1) -- (5,-1.25) -- (5, -1.5);
                        \draw (5, 1.25) -- (5.5, 1) -- (5.5, 0.75) -- (5,0.5) -- (g2) -- (5, -0.5) -- (4.5, -0.75);
                        \draw (5.5, 0.75) -- (6, 0.5) -- (h2) -- (6, -0.75) -- (5, -1.25);
                        
                        \node at (7,0) {$=$};

                        \node[op, scale = 0.75] (f3) at (8,0) {f};
                        \node[op, scale = 0.75] (g3) at (9,0) {g};
                        \node[op, scale = 0.75] (h3) at (10,0) {h};

                        \draw (9, 1.5) -- (9,1.25) -- (8, 0.75) -- (f3) -- (8,-0.75) -- (9,-1.25) -- (9, -1.5);
                        \draw (9, 1.25) -- (9.5, 1) -- (9.5, 0.75) -- (9,0.5) -- (g3) -- (9, -0.5) -- (9.5, -0.75);
                        \draw (9.5, 0.75) -- (10, 0.5) -- (h3) -- (10, -0.5) -- (9.5, -0.75) -- (9.5, -1) -- (9,-1.25);

                        \node at (11, 0) {$=$};

                        \node at (13, 0) {$f\star (g\star h)$};
                    \end{tikzpicture}

                    \begin{tikzpicture}[line cap=round,line join=round,>=triangle 45,x=1cm,y=1cm, thick, op/.style={circle, draw, scale=0.75}, scale=0.7]
                        \node at (-6, 0) {$(\upsilon_A\circ\varepsilon_C)\star f$};

                        \node at (-4, 0) {$=$};

                        \node[op, scale = 0.75] (f3) at (-2,0) {f};
                        \node[op, scale = 0.5] (c') at (-3, 0.25) {};
                        \node[op, scale = 0.5] (u') at (-3, -0.25) {};

                        \draw (-2.5, 1.25) -- (-2.5, 1) -- (-2, 0.75) -- (f3) -- (-2,-0.75) -- (-2.5,-1) -- (-2.5,-1.25);
                        \draw (-2.5, 1) -- (-3, 0.75) -- (c');
                        \draw (u') -- (-3, -0.75) -- (-2.5, -1);

                        \node at (-1,0) {$=$};

                        \node[op, scale = 0.75] (f1) at (0,0) {f};

                        \draw (0,1) -- (f1) -- (0,-1);

                        \node at (1,0) {$=$};

                        \node[op, scale = 0.75] (f2) at (2,0) {f};
                        \node[op, scale = 0.5] (c) at (3, 0.25) {};
                        \node[op, scale = 0.5] (u) at (3, -0.25) {};

                        \draw (2.5, 1.25) -- (2.5, 1) -- (2, 0.75) -- (f2) -- (2,-0.75) -- (2.5,-1) -- (2.5,-1.25);
                        \draw (2.5, 1) -- (3, 0.75) -- (c);
                        \draw (u) -- (3, -0.75) -- (2.5, -1);

                        \node at (4,0) {$=$};

                        \node at (6, 0) {$f\star (\upsilon_A\circ\varepsilon_C)$};
                    \end{tikzpicture}
                \end{center}
            \end{proof}

        \subsection{Twisting Morphisms}


    \section{Strongly Homotopy Associative Algebras, Coalgebras and Twisting Morphisms}

        \subsection{Sha Algebras}

        \subsection{Sha Coalgebras}

        \subsection{Twisting Sha Morphisms}
\end{document}