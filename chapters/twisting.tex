\documentclass[../thesis.tex]{subfiles}

\begin{document}

        In Stasheff \cite{Stasheff63}, a strongly homotopy associative algebra, or $A_\infty$-algebra, over a field is a graded vector space together with homogenous linear maps $m_n: A^{\otimes n}\rightarrow A$ of degree $2-n$ satisfying some homotopical relations; this will be made precise later. We will regard $m_2$ as a multiplication of $A$, but it is not a priori associative. We choose $m_3$ to be a homotopy of $m_2$'s associator. In this manner, we know that the homotopy of $A$ is an associative algebra. The maps $m_n$ corresponds uniquely to a map $m^c:BA\rightarrow \overline{A}[1]$, which extends to a coderivation $m^c : BA\rightarrow BA$ of the bar construction of $A$. With this relation, we will define an $A_\infty$-algebra to be a coalgebra on the form $BA$, and we will prefer to do so in this thesis.


        To understand the bar construction, we will first study it on associative algebras. Given a differential graded coassociative coalgebra $C$ and a differential graded associative algebra $A$, we say that a homogenous linear transformation $\alpha: C\rightarrow A$ is twisting if it satisfies the Maurer-Cartan equation;
            \begin{equation*}
                \partial\alpha + \alpha\star\alpha = 0.
            \end{equation*}
        Let $\tt{Tw}(C, A)$ be the set of twisting morphisms from $C$ to $A$. It defines a functor $\tt{Tw}: \tt{coAlg}_{\mathbb{K}}^{op}\times \tt{Alg}_{\mathbb{K}} \rightarrow Ab$, which is represented in both arguments. Moreover, these representations give rise to an adjoint pair of functors called the bar and cobar construction.

        \begin{center}
            \begin{tikzcd}
                \tt{Alg}_{\mathbb{K}, +}^\bullet \ar[bend left]{r}[pos=0.55]{B} \ar[phantom]{r}{\top} & \tt{Coalg}_{\mathbb{K},\tt{conil}}^\bullet \ar[bend left]{l}[pos=0.45]{\Omega}
            \end{tikzcd}
        \end{center}

        This chapter will follow the notions and progression presented in Loday and Vallette \cite{Loday12} to develop the theory for the bar-cobar adjunction, which will be the basis for our discussion of $A_\infty$-algebras.

    \section{Algebras and Coalgebras}    
    \subsection{Algebras}

            This section reviews associative algebras over a field $\mathbb{K}$. We denote the category of such algebras $\tt{Alg}_\mathbb{K}$, and we will study some of its properties before dualizing these to the context of coalgebras.
            
            \begin{definition}[$\mathbb{K}$-Algebra]
                Let $\mathbb{K}$ be a field with unit $1$. A $\mathbb{K}$-algebra $A$, or an algebra $A$ over $\mathbb{K}$, is a vector space with structure morphisms called multiplication and unit,
                \begin{align*}
                    (\cdot_A) & : A\otimes_{\mathbb{K}}A \rightarrow A \\
                    1_A & : \mathbb{K} \rightarrow A,
                \end{align*}
                satisfying the associativity and identity laws. 
                \begin{align*}
                    \text{(associativity)}\quad & (a \cdot_A b) \cdot_A c = a \cdot_A (b \cdot_A c) \\
                    \text{(unitality)}\quad & 1_A(1) \cdot_A a = a = a \cdot_A 1_A(1)
                \end{align*}
                Whenever $A$ does not possess a unit morphism, we will call $A$ a non-unital algebra. In this case, only the associativity law must hold.
            \end{definition}

            By abuse of notation, we will confuse the unit of $\mathbb{K}$ with the unit of $A$. Since $1_A$ is a ring homomorphism, this is well-defined. However, when we use the unit as a morphism, we will stick to the $1_A$ notation. When there is no confusion, we will exchange the symbol $(\cdot_A)$ with words in $A$. In other words, variable concatenation replaces $(\cdot_A)$.

            \begin{definition}[Algebra homomorphisms]
                Let $A$ and $B$ be algebras. Then $f: A\rightarrow B$ is an algebra homomorphism if
                \begin{enumerate}
                    \item $f$ is $\mathbb{K}$-linear
                    \item $f(ab)=f(a)f(b)$
                    \item $f\circ 1_A = 1_B$
                \end{enumerate}
                Whenever $A$ and $B$ are non-unital, we must drop the condition that $f$ preserves units.
            \end{definition}

            \begin{definition}[Category of algebras]
                We let $\tt{Alg}_{\mathbb{K}}$ denote the category of $\mathbb{K}$-algebras. Its objects consist of every algebra $A$, and the morphisms are algebra homomorphisms. The sets of morphisms between $A$ and $B$ are denoted as $\tt{Alg}_{\mathbb{K}}(A,B)$.
                
                Let $\widehat{\tt{Alg}}_{\mathbb{K}}$ denote the category of non-unital algebras. Its objects consist of every non-unital algebra $A$, and the morphisms are non-unital algebra homomorphisms. The sets of morphisms between $A$ and $B$ are denoted as $\widehat{\tt{Alg}}_{\mathbb{K}}(A,B)$.
            \end{definition}

            There is an equivalent description of algebras by considering the symmetric monoidal category $(\tt{Mod}_\mathbb{K},\otimes_\mathbb{K},\mathbb{Z})$. Observe that given any algebra $A$ in $\tt{Mod}_\mathbb{K}$, the triple $(A,(\cdot_A),1_A)$ is a monoid. There is thus an isomorphism of categories, namely $\tt{Alg}_\mathbb{K}$ is the category of monoids in $\tt{Mod}_\mathbb{K}$. The algebra axioms are then equivalent to the commutative diagrams below.
            \begin{center}
                \begin{tikzcd}
                    A \otimes_{\mathbb{K}} A \otimes_{\mathbb{K}} A \ar[]{r}{(\cdot_A)\otimes id_{\mathbb{K}}} \ar[]{d}[]{id_{\mathbb{K}}\otimes (\cdot_A)} & A \otimes_{\mathbb{K}} A \ar[]{d}{(\cdot_A)} \\
                    A \otimes_{\mathbb{K}} A \ar[]{r}[]{(\cdot_A)} & A
                \end{tikzcd} \quad
                \begin{tikzcd}
                    A \otimes_{\mathbb{K}} \mathbb{K} \ar[]{r}[]{id_A \otimes 1_A} \ar[]{rd}[below]{\simeq} & A \otimes_{\mathbb{K}} A \ar[]{d}[]{(\cdot_A)} & \mathbb{K} \otimes_{\mathbb{K}} A \ar[]{l}[above]{1_A \otimes id_A} \ar[]{ld}[]{\simeq}\\
                    & A
                \end{tikzcd}
            \end{center}

            In any symmetric monoidal category $\mathcal{C}$, we may reformulate these definitions by using the monoidal structure. Section 3 will introduce electronic circuits inspired by some of the proofs found in \cite{Loday12}. These conventions will give us a graphical calculus of morphisms in $\mathcal{C}$.
            % The final method we will use to represent an algebra is electronic circuits. An electronic circuit is a diagram read from top to bottom, where each column represents a different vector space in a tensor. Morphisms in such diagrams are figures, conjunctions, twistings, etc. E.g., The multiplication operator may be represented as a converging fork and the unit as a source.
            % \begin{center}
            %     \begin{tikzpicture}[line cap=round,line join=round,>=triangle 45,x=1cm,y=1cm, thick, op/.style={circle, draw, scale=0.75}, scale=0.7]
            %         \node at (-3, 0) {(Multiplication)};
                    
            %         \node[op, scale = 0.75] (1) at (0,0) {$(\cdot_A)$};

            %         \draw [line width=1pt] (-0.5, 1) -- (-0.5, 0.5) -- (1) -- (0, -1);
            %         \draw [line width=1pt] (0.5, 1) -- (0.5, 0.5) -- (1);


            %         \node at (1.5,0) {$=$};

            %         \draw [line width=1pt] (2.5,1) -- (2.5, 0.5) -- (3,0) -- (3,-1);
            %         \draw [line width=1pt] (3.5,1) -- (3.5, 0.5) -- (3,0);
            %     \end{tikzpicture} \qquad
            %     \begin{tikzpicture}[line cap=round,line join=round,>=triangle 45,x=1cm,y=1cm, thick, op/.style={circle, draw, scale=0.75}, scale=0.7]
            %         \node at (-1.5,0.25) {(Unit)};
                    
            %         \node[op, scale = 0.75] (1) at (0,1) {$\upsilon_A$};
            %         \draw [line width=1pt] (1) -- (0,-0.75);

            %         \node at (1, 0.25) {$=$};

            %         \node[op, scale=1] (2) at (1.75,1) {};
            %         \draw [line width=1pt] (2) -- (1.75,-0.75);

            %         \node at (0,-0.5) {};
            %     \end{tikzpicture}
            % \end{center}
            % Using these operations, we can now reformulate the algebra laws. These are the electronic laws for an algebra:
            % \begin{center}
            %     \begin{tikzpicture}[line cap=round,line join=round,>=triangle 45,x=1cm,y=1cm, thick, op/.style={circle, draw, scale=0.75}, scale=0.7]
            %         \node at (-3,0.5) {(Associativity)};

            %         \draw [line width=1pt] (-0.75, 1.25) -- (-0.75, 1) -- (-0.5, 0.75) -- (-0.5, 0.5) -- (0,0) -- (0, -0.5);
            %         \draw [line width=1pt] (-0.25, 1.25) -- (-0.25, 1) -- (-0.5, 0.75);
            %         \draw (0.5, 1.25) -- (0.5, 0.5) -- (0,0);

            %         \node at (1.25,0.5) {$=$};

            %         \draw [line width=1pt] (2,1.25) -- (2, 0.5) -- (2.5, 0) -- (2.5, -0.5);
            %         \draw [line width=1pt] (2.75, 1.25) -- (2.75, 1) -- (3, 0.75) -- (3, 0.5) -- (2.5, 0);
            %         \draw [line width=1pt] (3.25, 1.25) -- (3.25, 1) -- (3, 0.75);
            %     \end{tikzpicture}
            % \end{center}
            % \begin{center}
            %     \begin{tikzpicture}[line cap=round,line join=round,>=triangle 45,x=1cm,y=1cm, thick, op/.style={circle, draw, scale=0.75}, scale=0.7]
            %         \node at (-2,0) {(unitality)};

            %         \node[op, scale=0.75] (1) at (-0.5, 1) {};

            %         \draw [line width=1pt] (1) -- (-0.5, 0.5) -- (0,0) -- (0, -0.5);
            %         \draw [line width=1pt] (0.5, 1) -- (0.5, 0.5) -- (0,0);

            %         \node at (1.25,0.) {$=$};

            %         \draw [line width=1pt] (2, 1) -- (2, -0.5);

            %         \node at (2.75,0) {$=$};

            %         \node[op, scale=0.75] (2) at (4.5, 1) {};
            %         \draw [line width=1pt] (3.5, 1) -- (3.5, 0.5) -- (4, 0) -- (4, -0.5);
            %         \draw [line width=1pt] (2) -- (4.5, 0.5) -- (4, 0);

            %     \end{tikzpicture}
            % \end{center}

            We supply some examples of algebras one may encounter in nature.

            \begin{example}
                Let $\mathbb{K}$ be any field. The field is trivially an algebra over itself.
            \end{example}

            \begin{example}
                The complex numbers $\mathbb{C}$ is an algebra over $\mathbb{R}$, as it is a vector space over $\mathbb{R}$, and complex multiplication respects scalar multiplication.
            \end{example}

            \begin{example}
                Let $\mathbb{K}$ be any field. The ring of n-dimensional matrices $M_n(\mathbb{K})$ is an algebra over $\mathbb{K}$. The multiplication is matrix multiplication, and the unit is the n-dimensional identity matrix.
            \end{example}

            Augmented algebras will be central to our discussion. An algebra $A$ is augmented if an algebra homomorphism splits the algebra into an augmentation ideal and a unit component. We make this precise with the following definition

            \begin{definition}[Augmented algebras]
                A $\mathbb{K}$-algebra $A$ is augmented if there is an algebra homomorphism $\varepsilon_A: A \rightarrow \mathbb{K}$. We refer to the pair $(A,\varepsilon_A)$ as the augmented algebra.
            \end{definition}

            Given this algebra homomorphism, we know it has to preserve the unit. Thus the kernel $\tt{Ker}\varepsilon_A \subseteq A$ is almost $A$, but without its unit. In the module category $Mod_\mathbb{K}$, the morphism $\varepsilon_A$ is automatically a split-epimorphism, where the splitting is the unit $1_A$. Thus as a module, we have $A \simeq \overline{A}\oplus\mathbb{K}$, where $\overline{A} = \tt{Ker}\varepsilon_A$. $\overline{A}$ is called the augmentation ideal or the reduced algebra of $A$.

            A morphism $f: A \rightarrow B$ of augmented algebras is an algebra homomorphism, but with the added condition that it must preserve the augmentation, i.e., $\varepsilon_B \circ f = \varepsilon_A$. The collection of all augmented algebras over $\mathbb{K}$ together with the morphisms defines the category of augmented algebras over $\mathbb{K}$, $\tt{Alg}_{\mathbb{K},+}$.

            Given an augmented algebra $A$, taking kernels of $\varepsilon_A$ gives a functor $\overline{\underline{\phantom{A}}} : \tt{Alg}_{\mathbb{K},+} \rightarrow \widehat{\tt{Alg}}_\mathbb{K}$. This functor is well-defined on morphisms of augmented algebras, as each morphism is required to preserve the splitting. This functor has a quasi-inverse, given by the free augmentation $\argument^+: \widehat{\tt{Alg}}_\mathbb{K} \rightarrow \tt{Alg}_{\mathbb{K},+}$. Given a non-unital algebra $A$, the free augmentation is defined as $A^+ = A\oplus\mathbb{K}$, where the multiplication is given by:
            \begin{align*}
                (a,k)(a',k') = (aa' + ak' + a'k, kk')\tt{.}
            \end{align*}
            The unit is given by the element $(0,1)$. We summarize this in the statement below.
            
            \begin{proposition}
                The functors $\overline{\underline{\phantom{A}}}$ and $\argument^+$ are quasi-inverse to each other.
            \end{proposition}

            \begin{proof}
                We show that the free augmentation functor is fully faithful and essentially surjective. 

                Let $A$ and $B$ be non-unital $\mathbb{K}$-algebras, and let $f,g : A \rightarrow B$ morphisms in $\widehat{\tt{Alg}}_\mathbb{K}$. Suppose that $f^+ = g^+$, then $f = \overline{f^+} = \overline{g^+} = g$. Now suppose that $h : A^+ \rightarrow B^+$, then $h = \overline{h}^+$.

                Suppose that $A\in \tt{Alg}_{\mathbb{K},+}$. We want to show that $A \simeq \overline{A}^+$. As $\mathbb{K}$-modules, $A = \overline{A}^+$, so we propose that $id_A : A \rightarrow \overline{A}^+$ induces an isomorphism. To see that $id_A$ is an algebra homomorphism is to see that the multiplication in $A$ decomposes as $(a_1 + k)(a_2 + l) = (a_1a_2 + a_1l + ka_2) + kl$, where $a_1,a_2\in \overline{A}$ and $k,l \in \mathbb{K}$. The second condition is equivalent to the existence of $\varepsilon_A$. $id_A$ also preserves the augmentation as $\overline{A} \simeq \overline{\overline{A}^+}$.
            \end{proof}

            There are many augmented algebras to encounter in nature. We will note some examples.

            \begin{example}[Group algebra]
                Pick any group $G$ and any field $\mathbb{K}$. The group ring $K[G]$ is an augmented algebra where the augmentation $\varepsilon_{\mathbb{K}[G]} : \mathbb{K}[G] \rightarrow \mathbb{K}$ is given as
                \begin{align*}
                    \varepsilon_{\mathbb{K}[G]}(\sum_{g\in G}k_gg) = \sum_{g\in G}k_g\tt{.}
                \end{align*}
            \end{example}

            Among our most important example of algebras is the tensor algebra, which is also the free algebra over $\mathbb{K}$.

            \begin{example}[Tensor algebra]
                Let $V$ be a $\mathbb{K}$-module. We define the tensor algebra $T(V)$ of $V$ as the module
                \begin{align*}
                    T(V) = \mathbb{K}\oplus V\oplus V^{\otimes 2} \oplus V^{\otimes 3} \oplus \cdots\tt{.}
                \end{align*}
                The tensor algebra is then the algebra consisting of words in $V$. Given two words $v^1..v^i$ and $w^1...w^j$ in $T(V)$ we define the multiplication by the concatenation operation,
                \begin{align*}
                    \nabla_{T(V)} : T(V)\otimes_{\mathbb{K}} T(V) & \rightarrow T(V)\tt{,} \\
                    (v^1...v^i)\otimes(w^1...w^j) & \mapsto v^1...v^iw^1...w^j\tt{.}
                \end{align*}
                The unit is given by including $\mathbb{K}$ into $T(V)$,
                \begin{align*}
                    \upsilon_{T(V)} : \mathbb{K} & \rightarrow T(V)\tt{,} \\
                    1 & \mapsto 1\tt{.}
                \end{align*}
            \end{example}

            Observe that the tensor algebra is augmented. The projection from $T(V)$ into $\mathbb{K}$ is an algebra homomorphism, and its splitting is the inclusion $\mathbb{K} \rightarrow T(V)$. We obtain a splitting of the tensor algebra into its unit component and its augmentation ideal $T(V) \simeq \mathbb{K}\oplus\overline{T}(V)$. $\overline{T}(V)$ is called the reduced tensor algebra.

            \begin{proposition}[Tensor algebras are free]\label{prop: free-tensor}
                The tensor algebras are the free algebras over the category of $\mathbb{K}$-modules, i.e., for any $\mathbb{K}$-module $V$, there is a natural isomorphism $\tt{Hom}_{\mathbb{K}}(V, A)\simeq \tt{Alg}_{\mathbb{K}}(T(V), A)$.

                The reduced tensor algebra is the free non-unital algebra over the category of $\mathbb{K}$-modules. That is, for any $\mathbb{K}$-module $V$ there is a natural isomorphism $\tt{Hom}_{\mathbb{K}}(V, A)\simeq \widehat{\tt{Alg}}_{\mathbb{K}}(\overline{T}(V), A)$.
            \end{proposition}

            \begin{proof}
                If $f: T(V) \rightarrow A$ is an algebra homomorphism, then $f$ must satisfy the following conditions:
                \begin{itemize}
                    \item Unitality:\quad $f(1) = 1$
                    \item Homomorphism property:\quad Given $v,w\in V$, then $f(v w) = f(v)\cdot_Af(w)$
                \end{itemize}
                By induction, we see that $f$ is determined by where it sends the elements of $V$. Thus, restriction along the inclusion of $V$ into $T(V)$ induces a bijection.
            \end{proof}

            \subsubsection*{Modules}

                As for rings, every algebra $A$ has a module category.

                \begin{definition}[Modules]
                    Let $A$ be an algebra over $\mathbb{K}$. A $\mathbb{K}$-module $M$ is said to be a left (right) $A$-module if there exists a structure morphism $\mu_M : A\otimes_{\mathbb{K}}M \rightarrow M$ ($\mu_M : M\otimes_{\mathbb{K}}A \rightarrow M$) called multiplication. We require that $\mu_M$ is associative and preserves the unit of $A$; i.e. we have the commutative diagrams in $\tt{Mod}_\mathbb{K}$,
                    \begin{center}
                        \begin{tikzcd}
                            A\otimes_\mathbb{K}A\otimes_\mathbb{K}M \ar[]{d}[]{(\cdot_A)\otimes M} \ar[]{r}[]{A\otimes\mu_M} & A\otimes_\mathbb{K}M \ar[]{d}[]{\mu_M}\\
                            A\otimes_\mathbb{K}M \ar[]{r}[]{\mu_M} & M
                        \end{tikzcd}
                        \begin{tikzcd}
                            \mathbb{K} \otimes_\mathbb{K} M \ar[]{r}[]{1_A\otimes M} \ar[]{rd}[]{\simeq} & A \otimes_\mathbb{K} M \ar[]{d}[]{\mu_M} \\
                            & M
                        \end{tikzcd}
                    \end{center}
                    % \begin{center}
                    %     \begin{tikzpicture}[line cap=round,line join=round,>=triangle 45,x=1cm,y=1cm, thick, op/.style={circle, draw, scale=0.75}, scale=0.7]
                    %         \node at (-2.4,0) {(Associativity)};
                            
                    %         \node at (0,1.75) {A};
                    %         \node at (1,1.75) {A};
                    %         \node at (2,1.75) {M};

                    %         \node[op, scale = 0.5] (a) at (1,-0.25) {$\mu_M$};

                    %         \draw [line width=1pt] (0,1.25) -- (0, 1) -- ((0.5, 0.5) -- (0.5, 0.25) -- (a);
                    %         \draw [line width=1pt] (a) -- (2,0.5) -- (2,1.25);
                    %         \draw [line width=1pt] (0.5,0.5) -- (1,1) -- (1, 1.25);
                    %         \draw [line width=1pt] (a) -- (1,-1);
                            
                    %         \node at (2.5,0) {$=$};
                            
                    %         \node at (3,1.75) {A};
                    %         \node at (4,1.75) {A};
                    %         \node at (5,1.75) {M};

                    %         \node[op, scale = 0.5] (b) at (4.5,0.5) {$\mu_M$};
                    %         \node[op, scale = 0.5] (c) at (4,-0.25) {$\mu_M$};

                    %         \draw [line width=1pt] (3,1.25) -- (3, 0.5) -- (c);
                    %         \draw [line width=1pt] (c) -- (b) -- (5,1) -- (5,1.25);
                    %         \draw [line width=1pt] (4, 1.25) -- (4,1) -- (b);
                    %         \draw [line width=1pt] (c) -- (4,-1);
                    %     \end{tikzpicture}\\

                    %     \begin{tikzpicture}[line cap=round,line join=round,>=triangle 45,x=1cm,y=1cm, thick, op/.style={circle, draw, scale=0.75}, scale=0.7]
                    %         \node at (-2,0) {(unitality)};
        
                    %         \node at (1.75,1.5) {M};

                    %         \node[op, scale=0.75] (1) at (0.25, 1) {};
                    %         \draw [line width=1pt] (1) -- (0.25, 0.75) -- (1,0);
                    %         \draw [line width=1pt] (1,0) -- (1.75,0.75) -- (1.75, 1);
                    %         \draw [line width=1pt] (1,0) -- (1,-0.75);
        
                    %         \node at (2.25,0) {$=$};
        
                    %         \node at (3,1.5) {M};

                    %         \draw [line width=1pt] (3,1) -- (3,-0.75);
                    %     \end{tikzpicture}
                    % \end{center}
                \end{definition}

                \begin{definition}[A-linear homomorphisms]
                    Let $M, N$ be two left $A$-modules. A morphism $f:M\rightarrow N$ is called $A$-linear if it is $\mathbb{K}$-linear and for any $a$ in $A$ $f(am) = af(m)$.
                \end{definition}

                The category of left $A$-modules is denoted as $\tt{Mod}_A$, where the morphisms $\tt{Hom}_A(\argument,\argument)$ are $A$-linear. Likewise, we denote the category of right $A$-modules as $\tt{Mod}^A$. There is a free functor from $\mathbb{K}$-modules to left $A$-modules.

                \begin{proposition}\label{prop: free-mod}
                    Let $M$ be a $\mathbb{K}$-module. The module $A\otimes_{\mathbb{K}}M$ is a left $A$-module. Moreover, it is the free left module over $\mathbb{K}$-modules, i.e. there is a natural isomorphism $\tt{Hom}_{\mathbb{K}}(M,N)\simeq \tt{Hom}_{A}(A\otimes_{\mathbb{K}}M,N)$.
                \end{proposition}

                \begin{proof}
                    We define natural transformations in each direction and then show that they are inverses.

                    We define morphisms $\phi$ and $\psi$ as
                    \begin{align*}
                        \phi : \tt{Hom}_A(A\otimes_\mathbb{K}M, N) & \rightarrow \tt{Hom}_\mathbb{K}(M,N) \\
                        f & \mapsto f \circ (1_A \otimes M)\tt{,} \\
                        \psi : \tt{Hom}_\mathbb{K}(M, N) & \rightarrow \tt{Hom}_A(A\otimes_\mathbb{K}M, N) \\
                        g & \mapsto \mu_N \circ (A \otimes g)\tt{.}
                    \end{align*}

                    Pick an $f \in \tt{Hom}_A(A\otimes_\mathbb{K}M, N)$, then
                    \begin{align*}
                            \psi\circ\phi (f) = \mu_N \circ (A \otimes \phi (f)) = \mu_N \circ (A \otimes f (1_A \otimes M)) = f (A \otimes M) = f\tt{.}
                    \end{align*}
                    Pick a $g \in \tt{Hom}_\mathbb{K}(M,N)$, then
                    \begin{align*}
                        \phi\circ\psi (g) = \phi (\mu_N \circ (A \otimes g)) = \mu_N \circ (1_A \otimes g) = g\tt{.}
                    \end{align*}
                \end{proof}

                \begin{corollary}
                    $A$ as a left $A$-module is the free left $A$-module over $\mathbb{K}$; i.e. for any left $A$-module $M$, $M \simeq \tt{Hom}_\mathbb{K}(\mathbb{K}, M) \simeq \tt{Hom}_A(A,M)$
                \end{corollary}
            
            \subsubsection*{Categorical structure}

                It is convenient to understand some of the most fundamental limits and colimits to understand the category of algebras. Unfortunately, the category of algebras does not have nice kernels and cokernels; therefore, we will restrict our attention to augmented algebras.

                The category of augmented algebras is pointed. Since every morphism of augmented algebras has to preserve both unit and counit, the algebra $\mathbb{K}$ is both initial and terminal.

                \begin{definition}
                    Let $A$ and $B$ be augmented algebras. We define their direct sum $A \oplus B$ as the following limit:
                    \begin{center}
                        \begin{tikzcd}
                            A \oplus B \ar[]{r}[]{} \ar[]{d}[]{} \ar[phantom]{rd}[near start]{\ulcorner} & B \ar[]{d}[]{\varepsilon_B} \\
                            A \ar[]{r}[]{\varepsilon_A} & \mathbb{K}
                        \end{tikzcd}
                    \end{center}
                \end{definition}

                Notably, $A \oplus B$ is the product in $\tt{Alg}_{\mathbb{K},+}$, since $\mathbb{K}$ is terminal. Calculating this limit as a kernel, it is a subobject of $A \oplus B$ in the sense of $\mathbb{K}$-modules. We have the following relation between the direct and the ordinary direct sum.

                \begin{lemma}
                    The direct sum of augmented algebras $A$ and $B$ is the free augmentation on the direct sum of the augmentation ideals, $A \oplus B \simeq (\overline{A} \oplus \overline{B})^+$.
                \end{lemma}

                \begin{proof}
                    This lemma is clear from the monadicity of the forgetful functor; see Theorem \ref{thm: limit-creation},
                    \begin{align*}
                        forget : \tt{Alg}_{\mathbb{K},+} & \rightarrow \tt{Mod}_{\mathbb{K}} \\
                        A & \mapsto \overline{A}\tt{.}
                    \end{align*}
                \end{proof}

                Observe that the injections $A \hookrightarrow A \oplus B$ and $B \hookrightarrow A \oplus B$ do not satisfy the universal property of the coproduct. Thus, the direct sum is no longer the coproduct in this category.

                \begin{definition}
                    Given two augmented algebras $A$ and $B$, the free product $A\ast B$ is defined as the following colimit:
                    \begin{center}
                        \begin{tikzcd}
                            \mathbb{K} \ar[]{r}[]{\upsilon_A} \ar[]{d}[]{\upsilon_B} \ar[phantom]{rd}[near end]{\lrcorner} & A \ar[]{d}[]{} \\
                            B \ar[]{r}[]{} & A \ast B
                        \end{tikzcd}
                    \end{center}
                \end{definition}
                
                Notice that the free product is definitionally the coproduct. In the case of groups, the free product consists of every formal word formed from letters from each group. We extend this construction to augmented algebras, following the main idea presented by Aambø \cite{Aambø21}.

                \begin{lemma}
                    Let $A$ and $B$ be augmented algebras. The free product is isomorphic to a quotient of the tensor algebra
                    \begin{align*}
                        A\ast B \simeq \sfrac{T(\overline{A}\oplus \overline{B})}{I}\tt{.}
                    \end{align*}
                    The right-hand side is the tensor algebra over the direct sum of the underlying non-unital algebras, and $I$ is an ideal generated by elements on the form $\langle a\otimes a' - a\cdot a', b\otimes b' - b\cdot b' \rangle$.
                \end{lemma}

                \begin{proof}
                    We have naturally injective linear morphisms
                    \begin{align*}
                        \iota_A : A & \hookrightarrow \sfrac{T(\overline{A}\oplus \overline{B})}{I}\tt{,} \\ 
                        a & \mapsto a\tt{,} \\
                        1 & \mapsto 1\tt{.}
                    \end{align*}
                    This is in fact a ring homomorphism since $\iota_A(aa') = aa' = a\otimes a' = \iota_A(a)\iota_A(a')$.

                    Suppose we have the following diagram.
                    \begin{center}
                        \begin{tikzcd}
                            A \ar[bend right]{rd}[below]{f} \ar[hook]{r}[below]{\iota_A} & \sfrac{T(\overline{A}\oplus \overline{B})}{I} \ar[dashed]{d}[]{h} \ar[hookleftarrow]{r}[below]{\iota_B} & B \ar[bend left]{ld}[]{g} \\
                            & T
                        \end{tikzcd}
                    \end{center}
                    By functoriality we obtain a morphism $h = T(\overline{f}\oplus \overline{g}) : T(\overline{A} \oplus \overline{B}) \rightarrow T$. Unitality and augmentation property force this to act as the identity on the respective identities. Clearly $f = h \iota_A$ and $g = h \iota_B$.

                    Assume there exists another $h' : \sfrac{T(\overline{A}\oplus \overline{B})}{I} \rightarrow T$ such that $f = h' \iota_A$ and $g = h' \iota_B$. Then $h = h'$ on $A \oplus B$ part of $\sfrac{T(\overline{A}\oplus \overline{B})}{I}$. Since $h'$ is a ring morphism, $h = h'$ on all of $\sfrac{T(\overline{A}\oplus \overline{B})}{I}$.
                \end{proof}

                The forgetful functor creates every small limit in $\tt{Alg}_{\mathbb{K},+}$, and the kernel is no exception to this.

                \begin{lemma}
                    Suppose that $f: A \rightarrow B$ is a morphism of augmented algebras. The kernel of $f$ is isomorphic to $\tt{Ker}f = (\overline{Ker}f)^+$.
                \end{lemma}

                \begin{proof}
                    This lemma is clear from the monadicity of the forgetful functor.
                \end{proof}

                On the other hand, $\tt{Alg}_{\mathbb{K},+}$ is cocomplete as well. However, the colimits are not as simple to describe. In some cases, we can give a simple description of it. E.g., we know that the cokernel of a morphism $f: A \rightarrow B$ exists and is $\sfrac{\overline{B}}{\overline{A}}^+$ if $A$ is an ideal of $A$. Thus $A$ is the kernel of the cokernel morphism $g : B \rightarrow \sfrac{\overline{B}}{\overline{A}}^+$. Conversely, if $f$ is the kernel morphism of $g$, then $A$ is an ideal of $B$. In other words, we may think of an ideal as a kernel.
                
                Given any morphism $f: A \rightarrow B$, we may consider its coimage-image factorization.
                \begin{center}
                    \begin{tikzcd}
                        & A \ar[two heads]{rd}[]{} \ar[]{rrr}[]{f} & & & B \ar[two heads]{rd}[]{} \\
                        \tt{Ker}f \ar[hook]{ru}[]{} \ar[]{rr}[]{0} & & \tt{coIm}f \ar[hook, two heads]{r}[]{\widetilde{f}} & \tt{Im}f \ar[hook]{ru}[]{} \ar[]{rr}[]{0} & & \tt{coKer}f
                    \end{tikzcd}
                \end{center}
                It is clear that $\tt{Im}f$ is an ideal of $B$, thus $\tt{coKer}f \simeq \sfrac{\overline{B}}{\overline{\tt{Im}f}}^+$. The problem is that in the category of algebras, we cannot be sure if $\widetilde{f}$ is an isomorphism, even if it is mono and epi. Thus the ordinary set-theoretic image, $\tt{coIm}f$, may not be the categorical image, $\tt{Im}f$. We define the image as the smallest ideal of $B$ such that $\tt{coIm}f \subseteq \tt{Im}f \subseteq B$, and $f$ is called regular whenever $\widetilde{f}$ is an isomorphism. In this case, the image is then the same as the set-theoretic image, and
                \begin{align*}
                    \tt{coKer}f \simeq \sfrac{\overline{B}}{\overline{\tt{Im}}f}^+\tt{.}
                \end{align*}

    \subsection{Coalgebras}
            A coalgebra is like an algebra, but we reverse every arrow. In this section, we dualize the definitions as given for algebras. For many purposes, this dualization is good, but as we will observe, some finiteness conditions are necessary. We will denote the category of coalgebras as $\tt{coAlg}_\mathbb{K}$.

            \begin{definition}[$\mathbb{K}$-Coalgebra]
                Let $\mathbb{K}$ be a field. A coalgebra $C$ over $\mathbb{K}$ is a $\mathbb{K}$-module with structure morphisms called comultiplication and counit,
                \begin{align*}
                    (\Delta_C) & : C \rightarrow C\otimes_{\mathbb{K}}C \\
                    \varepsilon_C & : C \rightarrow \mathbb{K},
                \end{align*}
                satisfying the coassociativity and coidentity laws. 
                \begin{align*}
                    \text{(coassociativity)} \quad & (\Delta_C\otimes id_C)\circ\Delta_C(c) = (id_C\otimes\Delta_C)\circ\Delta_C(c) \\
                    \text{(counitality)} \quad & (id_C\otimes\varepsilon_C)\circ\Delta_C(c) = c = (\varepsilon_C\otimes id_C)\circ\Delta_C(c)
                \end{align*}
            \end{definition}

            In the same way as for algebras, we say that a coalgebra is non-counital if it is without a counit.

            Like algebras, coalgebras admits a single intuitive method for writing repeated application of the comultiplication. To see this, pick an element $c\in C$, we may apply the comultiplication twice on $c$ in two different ways:
            \begin{align*}
                \Delta_{C,(1)}^2(c) & = (\Delta_C \otimes C)\Delta_C(c)\tt{,} \\
                \Delta_{C,(2)}^2(c) & = (C \otimes \Delta_C)\Delta_C(c)\tt{.}
            \end{align*}
            One should immediately note that $\Delta_{C,(1)}^2(c) = \Delta_{C,(2)}^2(c)$ is the coassociativity axiom. Hence there is a unique way to make repeated applications of $\Delta_C$ on $c$. We denote the $n$-fold repeated application of $\Delta_C$ by $\Delta_C^n$. Since the element $\Delta_C^n(c)$ represents a finite sum in $C^{\otimes n}$, we may use Sweedlers notation \cite{Loday12},
            \begin{align*}
                \Delta_C^n(c) = \sum c_{(1)}\otimes ... \otimes c_{(n)}\tt{.}
            \end{align*}
            
            % We may dualize the electronic circuits of an algebra to coalgebras. In this manner, our structure morphisms would be upside down relative to the algebra morphisms. Thus comultiplication becomes a diverging fork, and counit becomes a sink. 
            % \begin{center}
            %     \begin{tikzpicture}[line cap=round,line join=round,>=triangle 45,x=1cm,y=1cm, thick, op/.style={circle, draw, scale = 0.75}, scale = 0.7]
            %         \node at (-3.5, 0) {(Comultiplication)};
                    
            %         \node[op, scale=0.75] (1) at (0,0) {$\Delta_C$};

            %         \draw [line width=1pt] (0,1) -- (1) -- (-0.5, -0.5) -- (-0.5, -1);
            %         \draw [line width=1pt] (1) -- (0.5, -0.5) -- (0.5, -1);

            %         \node at (1.5,0) {$=$};

            %         \draw [line width=1pt] (3, 1) -- (3, 0) -- (2.5, -0.5) -- (2.5, -1);
            %         \draw [line width=1pt] (3, 0) -- (3.5, -0.5) -- (3.5, -1);

            %     \end{tikzpicture} \qquad
            %     \begin{tikzpicture}[line cap=round,line join=round,>=triangle 45,x=1cm,y=1cm, thick, op/.style={circle, draw, scale=0.75}, scale=0.7]
            %         \node at (-2.5, 0) {(Counit)};
                    
            %         \node[op, scale=0.75] (1) at (0,-1) {$\varepsilon_C$};
            %         \draw [line width=1pt] (0, 1) -- (1);

            %         \node at (1, 0) {$=$};

            %         \node[op, scale=1] (2) at (2, -1) {};
            %         \draw [line width=1pt] (2, 1) -- (2);

            %         \node at (0,0.5) {};
            %     \end{tikzpicture}
            % \end{center}
            % We then obtain the electronic laws for a coalgebra by flipping the circuits around.
            % \begin{center}
            %     \begin{tikzpicture}[line cap=round,line join=round,>=triangle 45,x=1cm,y=1cm, thick, op/.style={circle, draw, scale=0.75}, scale=0.7]
            %         \node at (-3.5,0) {(Coassociativity)};

            %         \draw [line width=1pt] (0, 1) -- (0, 0) -- (-0.5, -0.5) -- (-0.5, -0.75) -- (-0.75, -1) -- (-0.75, -1.25);
            %         \draw [line width=1pt] (-0.5, -0.75) -- (-0.25, -1) -- (-0.25, -1.25);
            %         \draw [line width=1pt] (0, 0) -- (0.5, -0.5) -- (0.5, -1.25);

            %         \node at (1.5,0) {$=$};

            %         \draw [line width=1pt] (3, 1) -- (3, 0) -- (3.5, -0.5) -- (3.5, -0.75) -- (3.25, -1) -- (3.25, -1.25);
            %         \draw [line width=1pt] (3.5, -0.75) -- (3.75, -1) -- (3.75, -1.25);
            %         \draw [line width=1pt] (3, 0) -- (2.5, -0.5) -- (2.5, -1.25);
            %     \end{tikzpicture}
            % \end{center}
            % \begin{center}
            %     \begin{tikzpicture}[line cap=round,line join=round,>=triangle 45,x=1cm,y=1cm, thick, op/.style={circle, draw, scale=0.75}, scale=0.7]
            %         \node at (-3,0) {(Counitality)};

            %         \node[op, scale=0.75] (1) at (-0.5, -1) {};

            %         \draw [line width=1pt] (0, 1) -- (0, 0) -- (-0.5, -0.5) -- (1);
            %         \draw [line width=1pt] (0, 0) -- (0.5, -0.5) -- (0.5, -1);

            %         \node at (1,0) {$=$};

            %         \draw [line width=1pt] (2,1) -- (2,-1);

            %         \node at (3,0) {$=$};
                    
            %         \node[op, scale=0.75] (2) at (4.5,-1) {};

            %         \draw [line width=1pt] (4, 1) -- (4, 0) -- (3.5, -0.5) -- (3.5, -1);
            %         \draw [line width=1pt] (4, 0) -- (4.5, -0.5) -- (2);
            %     \end{tikzpicture}
            % \end{center}

            \begin{definition}[Coalgebra homomorphism]
                Let $C$ and $D$ be coalgebras. Then $f:C\rightarrow D$ is a coalgebra morphism if
                \begin{enumerate}
                    \item $f$ is $\mathbb{K}$-linear
                    \item $(f\otimes f)\circ\Delta_C(c) = \Delta_D(f(c))$
                    \item $\varepsilon_D \circ f = \varepsilon_C$
                \end{enumerate}
                Whenever $C$ and $D$ are non-counital, we only require 1. and 2. for a homomorphism of non-counital coalgebras.
            \end{definition}

            \begin{definition}[Category of coalgebras]
                Let $\tt{coAlg}_{\mathbb{K}}$ denote the category of coalgebras. Its objects consist of coalgebras $C$, and the morphisms are coalgebra homomorphisms. The set of morphisms between $C$ and $D$ are denoted as $\tt{coAlg}_{\mathbb{K}}(C,D)$.
                
                Let $\widehat{\tt{coAlg}}_{\mathbb{K}}$ denote the category of non-counital algebras. Its objects consist of non-counital algebras $C$, and the morphisms are non-counital coalgebra homomorphisms. The set of morphisms between $C$ and $D$ are denoted as $\widehat{\tt{coAlg}}_{\mathbb{K}}(C,D)$.
            \end{definition}

            At first glance, coalgebras may seem weird and unnatural, but they appear in many places in nature.

            \begin{example}[$\mathbb{K}$ as a coalgebra]
                The field $\mathbb{K}$ can be given a coalgebra structure over itself. Since $\{1\}$ is a basis for $\mathbb{K}$ we define the structure morphisms as
                \begin{align*}
                    \Delta_{\mathbb{K}}(1) & = 1\otimes 1 \\
                    \varepsilon(1) & = 1.
                \end{align*}
                One may check that these morphisms are indeed coassociative and counital. Thus we may regard our field as an algebra or a coalgebra over itself.
            \end{example}

            \begin{example}[$\mathbb{K}{[G]}$ as a coalgebra]
                The group algebra has a natural coalgebra structure. We may take duplication of group elements as the comultiplication, i.e.
                \begin{align*}
                    \Delta_{\mathbb{K}[G]}(kg) = kg\otimes g\tt{.}
                \end{align*}
                Coincidentally we have already defined the counit, and this is the augmentation $\varepsilon_{\mathbb{K}[G]}$ for the group algebra $\mathbb{K}[G]$. Recall that this was
                \begin{align*}
                    \varepsilon_C(\sum k_gg) = \sum k_g\tt{.}
                \end{align*}
                One may see that these morphisms satisfy coassociativity and counitality.
            \end{example}
            \begin{example}[The linear dual coalgebra]
                Let $M$ be any finite-dimensional $\mathbb{K}$-module. There is a natural isomorphism $\xi : M^* \otimes_\mathbb{K} M^* \rightarrow (M \otimes_\mathbb{K} M)^*$, given on elementary tensors as
                \begin{align*}
                    \xi(f \otimes g)(m \otimes n) = f(m)g(n)\tt{.}
                \end{align*}

                Let $A$ be a finite-dimensional algebra, then its linear dual $A^*$ is a coalgebra. The linear dual of the multiplication $(\cdot_A)$ is defined as 
                \begin{align*}
                    (\cdot_A)^* : A^* \rightarrow (A\otimes_\mathbb{K}A)^*\tt{.}
                \end{align*}
                We define the comulitplication of $A^*$ as $\xi^{-1}(\cdot_A)^*$.

                The counit of $A^*$ is the morphism $1_A^*$.
            \end{example}

            Before we state our primary example, we will introduce its essential structure.

            \begin{definition}[Coaugmented coalgebras]
                Let $C$ be a coalgebra. $C$ is coaugmented if there is a coalgebra homomorphism $\eta_C:\mathbb{K}\rightarrow C$.
            \end{definition}

            Like augmented algebras, each coaugmented coalgebra splits in the category $\tt{Mod}_\mathbb{K}$. We first notice that given a coalgebra homomorphism $f$, the cokernel $\tt{Cok}f$ is also a coalgebra. Given a coaugmentation $\eta_C : \mathbb{K} \rightarrow C$, we call $\tt{Cok}\eta_C = \overline{C}$ for the coaugmentation quotient or reduced coalgebra of $C$. Thus, we obtain the splitting $C \simeq \overline{C}\oplus\mathbb{K}$. The reduced comultiplication, denoted $\overline{\Delta}_C$ may explicitly be given as
            \begin{align*}
                \overline{\Delta}_C(c) = \Delta_C(c) - 1\otimes c - c\otimes 1\tt{.}
            \end{align*}

            \begin{example}[Tensor Coalgebras]
                Let $V$ be a $\mathbb{K}$-module. We define the tensor coalgebra $T^c(V)$ of $V$ as the module
                \begin{align*}
                    T^c(V) = \mathbb{K}\oplus V\oplus V^{\otimes 2}\oplus V^{\otimes 3}\oplus \cdots\tt{.}
                \end{align*}
                Given a string $v^1...v^i$ in $T(V)$ we define the comultiplication by the deconcatenation operation,
                \begin{align*}
                    \Delta_{T^c(V)}:T^c(V) & \rightarrow T^c(V)\otimes_{\mathbb{K}}T^c(V) \\
                    v^1...v^i & \mapsto 1\otimes(v^1...v^i) + (\sum_{j=1}^{i-1} (v^1...v^{j})\otimes(v^{j+1}...v^i)) + (v^1...v^i)\otimes 1\tt{.}
                \end{align*}
                The counit is given by projecting $T^c(V)$ onto $\mathbb{K}$,
                \begin{align*}
                    \varepsilon_{T^c(V)} : T^c(V) & \rightarrow \mathbb{K} \\
                    1 & \mapsto 1 \\
                    v^1...v^i & \mapsto \tt{.}
                \end{align*}
            \end{example}

            We observe that the tensor coalgebra is coaugmented, and its coaugmentation is the inclusion of $\mathbb{K}$ into $T^c(V)$. We can split $T^c(V) \simeq \mathbb{K}\oplus \overline{T}^c(V)$, where $\overline{T}^c(V)$ denotes the reduced tensor coalgebra.

            Cofreeness does not come for free for the tensor coalgebra. Our problem is a mismatch in the behavior of algebras and coalgebras. The problem arises when we try to do an evaluation. Suppose that $A$ is an algebra and that we have $n$ elements of $A$, i.e., an element of $A^{\otimes n}$. On this element, we may apply the multiplication of $A$ a maximum of $n$-times; there is no non-trivial empty multiplication. However, given a single element in a coalgebra $C$, we may use the comultiplication on this element $n$ times, $n+1$ times, and so on ad infinitum. In the coalgebra, we may comultiply any element, possibly an infinite amount of times. This property is sometimes ill-behaved with our dualization of algebras to coalgebras. 
            
            However, the correct property was not lost when we dualized the tensor algebra to the tensor coalgebra. We did not lose the property that an element may only be comultiplied a finite number of times since $T^c(V)$ is a direct sum of $V^{\otimes n}$, i.e., any element is a finite sum of finite tensors.
            
            This extra assumption we need for coalgebras will be called conilpotent. Let $C \simeq \mathbb{K} \oplus \overline{C}$ be a coaugmented coalgebra. We define the coradical filtration of $C$ as a filtration $Fr_0C \subseteq Fr_1C \subseteq ... \subseteq Fr_rC \subseteq ...$ by the submodules:
            \begin{align*}
                Fr_0C & = \mathbb{K} \\
                Fr_rC & = \mathbb{K} \oplus \{c\in\overline{C}\mid \forall n\geq r, \overline{\Delta}_C(c) = 0\}.
            \end{align*}

            \begin{definition}[Conilpotent coalgebras]
                Let $C$ be a coaugmented coalgebra. We say that $C$ is conilpotent if its coradical filtration is exhaustive, i.e.
                \begin{align*}
                    \varinjlim_rFr_rC \simeq C\tt{.}
                \end{align*}
                The full subcategory of conilpotent coalgebras will be denoted as $\tt{coAlg}_{\mathbb{K},\tt{conil}}$.
            \end{definition}
            
            \begin{proposition}[Conilpotent tensor coalgebra]\label{prop: conilpotent-tensor}
                Let $V$ be a $\mathbb{K}$-module. The tensor coalgebra $T^c(V)$ is conilpotent.
            \end{proposition}

            \begin{proof}
                Let $v\in V$, then $\Delta_{T^c(V)}(v)=1\otimes v + v\otimes 1$ and $\overline{\Delta}_{T^c(V)}(v)=0$. We then observe the following:
                \begin{align*}
                    Fr_0T^c(V) & = \mathbb{K}\tt{,} \\
                    Fr_1T^c(V) & = \mathbb{K} \oplus V\tt{,} \\
                    Fr_rT^c(V) & = \bigoplus_{i\leq r} V^{\otimes i}\tt{.}
                \end{align*}
                Exhaustiveness is clear from the coradical filtration.
            \end{proof}

            \begin{proposition}[Cofree tensor coalgebra]\label{prop: cofree-tensor}
                The tensor coalgebra is the cofree conilpotent coalgebra over the category of $\mathbb{K}$-modules. That is, for any $\mathbb{K}$-module $V$ and any conilpotent coalgebra $C$, there is a natural isomorphism $\tt{Hom}_{\mathbb{K}}(\overline{C}, V)\simeq \tt{coAlg}_{\mathbb{K},\tt{conil}}(C, T^c(V))$.
            \end{proposition}

            \begin{proof}
                This proposition should be evident from the description of a coalgebra homomorphism into the tensor coalgebra. If $g:C\rightarrow T^c(V)$ is a coalgebra homomorphism, then $g$ must satisfy the following conditions:
                \begin{enumerate}
                    \item (Coaugmentation)\quad $g(1)=1$,
                    \item (Counitality)\quad Given $c\in \overline{C}$ then $\varepsilon_{T^c(V)}\circ g(c)=0$,
                    \item (Homomorphism property)\quad Given $c\in C$ then $\Delta_{T^c(V)}(g(c))=(g\otimes g)\circ\Delta_C(c)$.
                \end{enumerate}

                We will construct the maps for the isomorphism explicitly. If $g: C\rightarrow T^c(V)$ is a coalgebra homomorphism, then composing with projection gives a map $\pi\circ g: C\rightarrow V$. Note that $\pi\circ g(1)=0$, so this is essentially a map $\pi\circ g:\overline{C}\rightarrow V$. For the other direction, let $\overline{g}:\overline{C}\rightarrow V$. We will then define $g$ as
                \begin{align*}
                    g = id_{\mathbb{K}} \oplus \sum_{i=1}^{\infty}(\otimes^{i}\overline{g})\overline{\Delta}_C^{i-1}.
                \end{align*}
                Observe that $g$ is well-defined since the sum convergence follows from the conilpotency of $C$. One may check that $g$ is a coalgebra homomorphism, which yields the result.
            \end{proof}

            \subsubsection*{Comodules}

                Essential to our dualization is comodules. We provide a short definition.

                \begin{definition}[Comodules]
                    Let $C$ be a coalgebra. A $\mathbb{K}$-module $M$ is said to be left (right) $C$-comodule if there exist a structure morphism $\omega_M: M \rightarrow C\otimes_{\mathbb{K}}M$ ($\omega_M: M \rightarrow M\otimes_{\mathbb{K}}C$) called comultiplication. We require that $\omega_M$ is coassociative with respect to the comultiplication of $C$ and preserves the counit of $C$; i.e. we have the following commutative diagrams in $\tt{Mod}_\mathbb{K}$,
                    \begin{center}
                        \begin{tikzcd}
                            C \otimes_\mathbb{K}C\otimes_\mathbb{K}M & C\otimes_\mathbb{K}M \ar[]{l}[]{C\otimes\omega_M} \\
                            C\otimes_\mathbb{K}M \ar[]{u}[]{\Delta_C\otimes M} & M \ar[]{l}[]{\omega_M} \ar[]{u}[]{\omega_M}
                        \end{tikzcd}
                        \begin{tikzcd}
                            \mathbb{K}\otimes_\mathbb{K}M & C\otimes_\mathbb{K}M \ar[]{l}[]{\varepsilon_C\otimes M} \\
                            & M \ar[]{lu}[]{\simeq} \ar[]{u}[]{\omega_M}
                        \end{tikzcd}
                    \end{center}
                    
                    %, i.e., the electronic laws are satisfied.
                    % \begin{center}
                    %     \begin{tikzpicture}[line cap=round,line join=round,>=triangle 45,x=1cm,y=1cm, thick, op/.style={circle, draw, scale=0.75}, scale=0.7]
                    %         \node at (-3.5,0) {(Coassociativity)};

                    %         \node at (0, 1.5) {M};
                    %         \node[op, scale=0.5] (1) at (0, 0) {$\omega_M$};
        
                    %         \draw [line width=1pt] (0, 1) -- (1) -- (-0.5, -0.5) -- (-0.5, -0.75) -- (-0.75, -1) -- (-0.75, -1.25);
                    %         \draw [line width=1pt] (-0.5, -0.75) -- (-0.25, -1) -- (-0.25, -1.25);
                    %         \draw [line width=1pt] (1) -- (0.5, -0.5) -- (0.5, -1.25);
        
                    %         \node at (1.5,0) {$=$};

                    %         \node at (3, 1.5) {M};
                    %         \node[op, scale=0.5] (2) at (3, 0) {$\omega_M$};
                    %         \node[op, scale=0.5] (3) at (3.5, -0.75) {$\omega_M$};
        
                    %         \draw [line width=1pt] (3, 1) -- (2) -- (3) -- (3.25, -1) -- (3.25, -1.25);
                    %         \draw [line width=1pt] (3) -- (3.75, -1) -- (3.75, -1.25);
                    %         \draw [line width=1pt] (2) -- (2.5, -0.5) -- (2.5, -1.25);
                    %     \end{tikzpicture}
                    % \end{center}
                    % \begin{center}
                    %     \begin{tikzpicture}[line cap=round,line join=round,>=triangle 45,x=1cm,y=1cm, thick, op/.style={circle, draw, scale=0.75}, scale=0.7]
                    %         \node at (-3,0) {(Counitality)};
        
                    %         \node[op, scale=0.75] (1) at (-0.5, -1) {};
        
                    %         \draw [line width=1pt] (0, 1) -- (0, 0) -- (-0.5, -0.5) -- (1);
                    %         \draw [line width=1pt] (0, 0) -- (0.5, -0.5) -- (0.5, -1);
        
                    %         \node at (1,0) {$=$};
        
                    %         \draw [line width=1pt] (2,1) -- (2,-1);
        
                    %         \node at (3,0) {$=$};
                            
                    %         \node[op, scale=0.75] (2) at (4.5,-1) {};
        
                    %         \draw [line width=1pt] (4, 1) -- (4, 0) -- (3.5, -0.5) -- (3.5, -1);
                    %         \draw [line width=1pt] (4, 0) -- (4.5, -0.5) -- (2);
                    %     \end{tikzpicture}
                    % \end{center}
                \end{definition}

                \begin{definition}[C-colinear homomorphism]
                    Let $M, N$ be two left $C$-comodules. A morphism $g:M\rightarrow N$ is called $C$-colinear if it is $\mathbb{K}$-linear and for any $m$ in $M$, $\omega_N(g(m)) = (id_C\otimes g)\omega_M(m)$. In Sweedlers notation, this looks like
                    \begin{align*}
                        \sum g(m)_{(1)}\otimes g(m)_{(2)} = \sum c_{(1)}\otimes g(m_{(2)})\tt{.}
                    \end{align*}
                \end{definition}

                The category of left $C$-comodules is denoted as $\tt{CoMod}_C$, where the morphisms $\tt{Hom}_C(\argument,\argument)$ are $C$-colinear. We would also like to restrict our attention to those $C$-comodules that are conilpotent, i.e., the comodules with exhaustive coradical filtration. The coradical filtration is defined analogously, as we only care for the $\mathbb{K}$-module structure. Notice that for conilpotent coalgebras, this requirement is automatic. Likewise, we denote the category of right $C$-comodules as $\tt{CoMod}^C$.

                \begin{proposition}\label{prop: cofree-comod}
                    Let $M$ be a $\mathbb{K}$-module. The module $C\otimes_{\mathbb{K}}M$ is a left $C$-comodule. Moreover, it is the cofree left comodule over $\mathbb{K}$-modules, i.e. there is an isomorphism $\tt{Hom}_{\mathbb{K}}(N,M)\simeq \tt{Hom}_C(N,C\otimes_{\mathbb{K}}M)$. 
                \end{proposition}

                \begin{proof}
                    This proposition is dual to Proposition \ref{prop: free-mod}. We will only construct the isomorphism, as its validity is apparent.

                    \begin{align*}
                        \phi' : \tt{Hom}_C(N,C\otimes_\mathbb{K}M) & \rightarrow \tt{Hom}_\mathbb{K}(N,M) \\
                        f & \mapsto (\varepsilon_C \otimes M) \circ f\tt{,} \\
                        \psi' : \tt{Hom}_\mathbb{K}(N,M) & \rightarrow \tt{Hom}_C(N,C\otimes_\mathbb{K}M) \\
                        g & \mapsto (C \otimes g) \circ \omega_N\tt{.}
                    \end{align*}
                \end{proof}

                \begin{corollary}
                    $C$ as a left $C$-comodule is the cofree $C$-comodule over $\mathbb{K}$; i.e. for any left $C$-comodule $N$, $N^* \simeq \tt{Hom}_\mathbb{K}(N,\mathbb{K}) \simeq \tt{Hom}_C(N, C)$.
                \end{corollary}

            \subsubsection*{Categorical structure}

                Dual to augmented algebras, conilpotent coalgebras have colimits that are easy to calculate, while the limits are complicated. For this discussion, we will restrict our attention to $\tt{coAlg}_{\mathbb{K},\tt{conil}}$.

                Like for augmented algebras, $\tt{coAlg}_{\mathbb{K},\tt{conil}}$ is a pointed category. The initial and terminal object is $\mathbb{K}$.

                \begin{definition}
                    Let $C$ and $D$ be conilpotent coalgebras. Their direct sum $C \oplus D$ is defined as the following colimit:
                    \begin{center}
                        \begin{tikzcd}
                            \mathbb{K} \ar[]{r}[]{\eta_C} \ar[]{d}[]{\eta_D} & C \ar[]{d}[]{} \\
                            D \ar[]{r}[]{} & C \oplus D 
                        \end{tikzcd}
                    \end{center}
                \end{definition}

                As before, this is some abuse of notation. This direct sum will almost be the direct sum, except we have to fix the coaugmentation.

                \begin{lemma}
                    Given conilpotent coalgebras $C$ and $D$, their direct sum is the free coaugmentation on the direct sum of the coaugmentation quotients, $C \oplus D \simeq (\overline{C} \oplus \overline{D})^+$.
                \end{lemma}

                \begin{proof}
                    This lemma is clear from the comonadicity of the forgetful functor.
                \end{proof}

                Dually to before, the projection $C \oplus D \rightarrow C$ is not usually a coalgebra morphism.

                \begin{definition}
                    Let $C$ and $D$ be two augmented algebras, the free product $C \ast D$ is defined as the following limit:
                    \begin{center}
                        \begin{tikzcd}
                            C\ast D \ar[]{r}[]{} \ar[]{d}[]{} & C \ar[]{d}[]{\varepsilon_C} \\
                            D \ar[]{r}[]{\varepsilon_D} & \mathbb{K}
                        \end{tikzcd}
                    \end{center}
                \end{definition}

                We proceed to describe the free product of conilpotent coalgebras. Due to it being dual to the free product of augmented algebras, this will naturally be a subobject of the tensor coalgebra.

                \begin{lemma}
                    Given to conilpotent coalgebras $C$ and $D$, then $C\ast D \subseteq T^c(\overline{C}\oplus \overline{D})$ consists in words generated by letters in $\overline{C}$ or $\overline{D}$ on the form
                    \begin{align*}
                        \llbracket c \rrbracket & = \sum_{i=0}^\infty \Delta^i_C(c)\tt{, and}\\
                        \llbracket d \rrbracket & = \sum_{i=0}^\infty \Delta^i_D(d)\tt{.}
                    \end{align*} 
                \end{lemma}

                \begin{proof}
                    We define a projection $C\ast D \rightarrow C$ as the "identity" on the letters in $C$ and $0$ otherwise.
                    \begin{align*}
                        p_C : C\ast D & \rightarrow C \\
                        \llbracket c \rrbracket & \mapsto c \\
                        \underline{\phantom{A}} & \mapsto 0
                    \end{align*}
                    By definition, $p_C$ is a coalgebra morphism as 
                    \begin{align*}
                        p_C^{\otimes 2}(\Delta_{T^c(\overline{C}\oplus \overline{D})}\llbracket c \rrbracket) 
                        = p_C^{\otimes 2}(\sum \llbracket c_{(1)} \rrbracket \otimes \llbracket c_{(2)}\rrbracket)
                        = \sum c_{(1)} \otimes c_{(2)}\tt{.}
                    \end{align*}

                    The morphisms $p_C$ and $p_D$ define a cone over $C$ and $D$. It remains to check the universal property. Suppose there are morphisms $f: T \rightarrow C$ and $g: T \rightarrow D$.
                    \begin{center}
                        \begin{tikzcd}
                            T \ar[bend left]{rrd}[]{f} \ar[dashed]{rd}[]{h} \ar[bend right]{rdd}[]{g} \\
                            & C\ast D \ar[]{r}[]{p_C} \ar[]{d}[]{p_D} & C \ar[]{d}[]{\varepsilon_C} \\
                            & D \ar[]{r}[]{\varepsilon_D} & \mathbb{K}
                        \end{tikzcd}
                    \end{center}

                    We define the morphism $h$ as the following sum
                    \begin{align*}
                        h(t) = \sum_{i=1}^\infty \llbracket f(t_{(1)}) \rrbracket \otimes \llbracket g(t_{(2)}) \rrbracket \otimes \cdots \otimes \llbracket ?(t_{(i)}) \rrbracket + \llbracket g(t_{(1)}) \rrbracket \otimes \llbracket f(t_{(2)}) \rrbracket \otimes \cdots \otimes \llbracket ?(t_{(i)}) \rrbracket\tt{,}
                    \end{align*}
                    where $?$ means either $f$ or $g$, which is appropriate.

                    We have constructed this morphism to be a coalgebra morphism, and every other coalgebra morphism has to be on this form as well. Thus $h$ is unique.
                \end{proof}

                Opposite to augmented algebras, every small colimit of conilpotent coalgebras is created by the forgetful functor.

                \begin{lemma}
                    Suppose that $f: C \rightarrow D$ is a morphism of augmented algebras. The cokernel is isomorphic to $\tt{coKer}f \simeq (\overline{\tt{coKer}}f)^+ \simeq \sfrac{\overline{D}}{\overline{\tt{Im}}f}^+$.
                \end{lemma}

                \begin{proof}
                    This lemma is clear from the comonadicity of the forgetful functor.
                \end{proof}

                This time around, we will instead have a problem calculating kernels. Let $f: C \rightarrow D$ be a morphism of coalgebras. The set $\{c\in C \mid f(c) = 0\}$ is not necessarily closed under comultiplication. We require that $f^{\otimes 2}(\Delta_C(c)) = f(c_{(1)})\otimes f(c_{(2)}) = \Delta_D(f(c)) = \Delta_D(0) = 0$, but then only one of $f(c_{(1)})$ or $f(c_{(2)})$ has to be $0$.

                The abovementioned construction will sometimes work. If $f$ is a cokernel map, that is if $f : D \rightarrow \sfrac{\overline{D}}{\overline{C}}^+$, then $C = \{d\in D \mid f(d) = 0\}$. Whenever $f: C \rightarrow D$ is epi and regular, $f$ will then be a cokernel map. In particular, it is enough that $f: C \rightarrow D$ is regular, as we can consider the morphism $\pi: C \rightarrow \tt{coIm}f$ instead of $f$. Since $\widetilde{f}: \tt{coIm}f \rightarrow \tt{Im}f$ is an isomorphism, $\tt{Ker}f \simeq \tt{Ker}\pi$, so we can use the set-theoretic description instead,
                \begin{align*}
                    \tt{Ker}f = \{c\in C \mid f(c) = 0\}\tt{.}
                \end{align*}

    \subsection{Electronic Circuits}
            Calculations involving both algebras and coalgebras tend to become convoluted and unmanageable. Since we want to study the interplay between algebras and coalgebras, using other tools to write equations can be handy. We will develop a graphical calculus briefly mentioned in \cite{Loday12}. This graphical calculus will consist of string diagrams, referred to as electronic circuits, which describe the function composition on tensors. Since we only care about the interplay of tensors, we may develop this graphical calculus in any closed symmetric monoidal category. Why do we want to introduce this abstraction? A closed symmetric monoidal category is a good category to model functions, or morphisms, which may take several variables in its argument. Moreover, in the next section, we are going to switch categories. In this manner, we can reuse the same notions and proofs.
            
            This section will use closed symmetric monoidal categories to define electronic circuits. The definitions can be found in Appendix \ref{ch: sym-mon}. For our purposes, a closed symmetric monoidal category is a category $\mathcal{C}$ together with a bifunctor $\argument\otimes\argument: \mathcal{C}\times\mathcal{C}\rightarrow\mathcal{C}$ usually called tensor, and a unit object $Z\in \mathcal{C}$. Additionally, we have four natural isomorphisms relating the functors and the unit to what they are supposed to represent:
            \begin{align*}
                \tt{Associator} & \quad \alpha : (A \otimes B) \otimes C \rightarrow A \otimes (B \otimes C)\tt{.}\\
                \tt{Right unit} & \quad \rho : A \otimes Z \rightarrow A\tt{.}\\
                \tt{Left unit} & \quad \lambda : Z \otimes A \rightarrow A\tt{.}\\
                \tt{Braiding/Symmetry} & \quad \beta : A \otimes B \rightarrow B \otimes A\tt{.}
            \end{align*}
            These natural isomorphisms are supposed to satisfy some laws as well. See the appendix for the full definition.

            We want to rewrite equations into string diagrams with an electronic circuit, possibly involving tensors. To illustrate with some simple examples, let $f: A \rightarrow B$, $g: B \rightarrow C$ and $h : D \rightarrow E$. We may consider the composition
            \begin{align*}
                (g \otimes E) \circ (f \otimes h) : A \otimes D \rightarrow C \otimes E \tt{.}
            \end{align*}
            An electronic circuit is written from top to bottom and is composed of levels. The first morphisms we apply will be at the top, descending downwards with each function composition. We write each argument in the composition as a string. Thus this example above will look like the circuit below. Notice how $f$ and $h$ are at the same level, indicating that they are interpreted as $f \otimes h$. Thus an $\otimes$ indicates a change of string, while a $\circ$ indicates a change of level.
            
            \begin{center}
                \begin{tikzpicture}[line cap=round,line join=round,>=triangle 45,x=1cm,y=1cm, thick, op/.style={circle, draw, scale=0.75}, scale=0.7]
                    \node[op, scale = 0.75] (f1) at (0,0) {f};
                    \node[op, scale = 0.75] (g1) at (0,-1) {g};
                    \node[op, scale = 0.75] (h1) at (1,0) {h};
                    
                    \draw [line width = 1pt] (0,0.5) -- (f1) -- (g1) -- (0, -1.5);
                    \draw [line width = 1pt] (1,0.5) -- (h1) -- (1,-1.5);
                \end{tikzpicture}
            \end{center}

            Beware that when many tensors are in use, we should remember exactly how each string is tensored. We may call adding tensors for horizontal composition and composition of morphism for vertical composition. Both have a choice in how we associate them, but both have unique choices up to isomorphism given by the associator. 

            The true power of electronic circuit comes to light when we consider morphisms that, in some sense, "creates" or "destroys" strings. For example, a morphism of 2 variables "destroys" a string by applying them to each other. Consider now a morphism $f: A \otimes B \rightarrow C$; we represent this morphism in an electronic circuit using a converging fork. Likewise, "creation" of strings is seen as a diverging fork.
            \begin{center}
                \begin{tikzpicture}[line cap=round,line join=round,>=triangle 45,x=1cm,y=1cm, thick, op/.style={circle, draw, scale=0.75}, scale=0.7]
                    \node[op, scale = 0.75] (f1) at (0,0) {f};
                    
                    \draw [line width = 1pt] (-0.5,0.75) -- (-0.5, 0.5) -- (f1) -- (0, -0.5);
                    \draw [line width = 1pt] (0.5,0.75) -- (0.5, 0.5) -- (f1) -- (0,-0.5);
                \end{tikzpicture}
            \end{center}

            We may write the unit object $Z$ without any strings in a circuit. By right and left unitality, any object $A$ is isomorphic to $A \otimes Z \simeq A \simeq Z \otimes A$. In this manner, whenever a morphism enters or exits the unit $Z$, we start a new string using a source or a sink. For example, consider $f$ as before and a morphism $g: Z \rightarrow A$, then we may write $f \circ (g \otimes B)$ as the circuit below. Again, this is only well-defined up to isomorphism by right and left unitality.
            \begin{center}
                \begin{tikzpicture}[line cap=round,line join=round,>=triangle 45,x=1cm,y=1cm, thick, op/.style={circle, draw, scale=0.75}, scale=0.7]
                    \node[op, scale = 0.75] (f1) at (0,0) {f};
                    \node[op, scale = 0.75] (g1) at (-0.5, 1) {g};
                    \node[op, scale =  0.76] (s1) at (-0.5, 1.75) {};

                    \draw [line width = 1pt] (s1) -- (g1) -- (-0.5, 0.5) -- (f1) -- (0, -0.5);
                    \draw [line width = 1pt] (0.5,2) -- (0.5, 0.5) -- (f1) -- (0,-0.5);
                \end{tikzpicture}
            \end{center}

            The final operation we have is braiding. When we apply the braiding morphism on the tensors, we may denote this as interchanging the strings. For example, $\beta_{A, B}: A \otimes B \rightarrow B \otimes A$ is the circuit below. Notice that by the naturality of $\beta$, we may move a braiding along the circuit. In this manner, if we have two braids, they may sometimes undo each other. In either case, we can carry a braid to either end of the circuit to ignore them during calculations.
            \begin{center}
                \begin{tikzpicture}[line cap=round,line join=round,>=triangle 45,x=1cm,y=1cm, thick, op/.style={circle, draw, scale=0.75}, scale=0.7]
                    \draw [line width = 1pt] (0, 1.25) -- (0, 0.75) -- (1, 0.25) -- (1, -0.25);
                    \draw [line width = 1pt] (1,1.25) -- (1, 0.75) -- (0, 0.25) -- (0,-0.25);
                \end{tikzpicture}
            \end{center}

            With the language of electronic circuits, we may now write down the axioms of an algebra or coalgebra electronically. The axioms state the existence of morphisms. We give the structure maps of algebras and coalgebras special notation since we will use these often.

            For convenience we will let $\mathcal{C} = \tt{Mod}_\mathbb{K}$. This category is closed symmetric monoidal, with $\otimes_\mathbb{K}$ as the tensor. Recall that an algebra is a $\mathbb{K}$-module $A$ together with maps $(\cdot_A) : A \otimes A \rightarrow A$ and $1_A : \mathbb{K} \rightarrow A$. We denote these morphisms electronically, as shown in the diagrams below.
            \begin{center}
                \begin{tikzpicture}[line cap=round,line join=round,>=triangle 45,x=1cm,y=1cm, thick, op/.style={circle, draw, scale=0.75}, scale=0.7]

                    \node at (-2.25, 0) {$(\cdot_A)$};
                    \node at (-1.25, 0) {=};

                    \draw [line width = 1pt] (-0.5, 0.75) -- (-0.5, 0.25) -- (0,0) -- (0, -0.5);
                    \draw [line width  = 1pt] (0.5, 0.75) -- (0.5 , 0.25) -- (0,0);

                \end{tikzpicture}
                \qquad
                \begin{tikzpicture}[line cap=round,line join=round,>=triangle 45,x=1cm,y=1cm, thick, op/.style={circle, draw, scale=0.75}, scale=0.7]

                    \node at (-2, 0) {$1_A$};
                    \node at (-1, 0) {=};
                    \node[op, scale = 0.75] (s1) at (0,0.5) {};

                    \draw [line width = 1pt] (s1) -- (0, -0.25);
                \end{tikzpicture}
            \end{center}

            We write the electronic laws for an algebra as how one would write equations. Associativity and unitality then become as follows.
            \begin{center}
                \begin{tikzpicture}[line cap=round,line join=round,>=triangle 45,x=1cm,y=1cm, thick, op/.style={circle, draw, scale=0.75}, scale=0.7]

                    \node at (-4, 0) {Associativity};

                    \draw [line width = 1pt] (0, 1.25) -- (0, 1) -- (-0.5, 0.75) -- (-0.5, 0.25) -- (0,0) -- (0,-0.5);
                    \draw [line width = 1pt] (-1, 1.25) -- (-1, 1) -- (-0.5, 0.75);
                    \draw [line width = 1pt] (0.5, 1.25) -- (0.5, 0.25) -- (0,0);

                    \node at (1.5, 0) {$=$};

                    \draw [line width = 1pt] (2.5, 1.25) -- (2.5, 0.25) -- (3, 0) -- (3, -0.5);
                    \draw [line width = 1pt] (3, 1.25) -- (3, 1) -- (3.5, 0.75) -- (3.5, 0.25) -- (3, 0);
                    \draw [line width = 1pt] (4, 1.25) -- (4, 1) -- (3.5, 0.75);
                \end{tikzpicture}
            \end{center}
            \begin{center}
                \begin{tikzpicture}[line cap=round,line join=round,>=triangle 45,x=1cm,y=1cm, thick, op/.style={circle, draw, scale=0.75}, scale=0.7]

                    \node at (-6, 0) {Unitality};

                    \node[op, scale = 0.75] (r1) at (-2, 1) {};

                    \draw [line width = 1pt] (r1) -- (-2, 0.25) -- (-2.5, 0) -- (-2.5, -0.5);
                    \draw [line width = 1pt] (-3, 1) -- (-3, 0.25) -- (-2.5, 0);

                    \node at (-1, 0) {$=$};

                    \draw [line width = 1pt] (0, 1) -- (0, -0.5);

                    \node at (1, 0) {$=$};

                    \node[op, scale = 0.75] (l1) at (2, 1) {};

                    \draw [line width = 1pt] (l1) -- (2, 0.25) -- (2.5, 0) -- (2.5, -0.5);
                    \draw [line width = 1pt] (3, 1) -- (3, 0.25) -- (2.5, 0);
                \end{tikzpicture}
            \end{center}

            Dually, given a coalgebra $C$, we will make a similar notation. We denote the maps $\Delta_C: C \rightarrow C \otimes C$ and $\varepsilon_C: C \rightarrow \mathbb{K}$ as the following electronic circuits.
            \begin{center}
                \begin{tikzpicture}[line cap=round,line join=round,>=triangle 45,x=1cm,y=1cm, thick, op/.style={circle, draw, scale=0.75}, scale=0.7]

                    \node at (-2.25, 0) {$\Delta_C$};
                    \node at (-1.25, 0) {=};

                    \draw [line width = 1pt] (0, 0.75) -- (0, 0.25) -- (-0.5,0) -- (-0.5, -0.5);
                    \draw [line width  = 1pt] (0, 0.25) -- (0.5,0) -- (0.5, -0.5);

                \end{tikzpicture}
                \qquad
                \begin{tikzpicture}[line cap=round,line join=round,>=triangle 45,x=1cm,y=1cm, thick, op/.style={circle, draw, scale=0.75}, scale=0.7]

                    \node at (-2, 0) {$\varepsilon_C$};
                    \node at (-1, 0) {=};
                    \node[op, scale = 0.75] (s1) at (0,-0.25) {};

                    \draw [line width = 1pt] (s1) -- (0, 0.5);
                \end{tikzpicture}
            \end{center}
            The electronic laws for $C$ become the following diagrams.
            \begin{center}
                \begin{tikzpicture}[line cap=round,line join=round,>=triangle 45,x=1cm,y=1cm, thick, op/.style={circle, draw, scale=0.75}, scale=0.7]
                    \node at (-4.5, 0) {Coassociativity};

                    \draw [line width = 1pt] (-0.5, 0) -- (-1, -0.25) -- (-1, -0.5);
                    \draw [line width = 1pt] (0, 1.25) -- (0, 0.75) -- (-0.5, 0.5) -- (-0.5, 0) -- (0, -0.25) -- (0, -0.5);
                    \draw [line width = 1pt] (0, 0.75) -- (0.5, 0.5) -- (0.5, -0.5);

                    \node at (1.5, 0) {$=$};

                    \draw [line width = 1pt] (3, 1.25) -- (3, 0.75) -- (2.5, 0.5) -- (2.5, -0.5);
                    \draw [line width = 1pt] (3, 0.75) -- (3.5, 0.5) -- (3.5, 0) -- (3, -0.25) -- (3, -0.5);
                    \draw [line width = 1pt] (3.5, 0) -- (4, -0.25) -- (4, -0.5);

                \end{tikzpicture}
            \end{center}
            \begin{center}
                \begin{tikzpicture}[line cap=round,line join=round,>=triangle 45,x=1cm,y=1cm, thick, op/.style={circle, draw, scale=0.75}, scale=0.7]

                    \node at (-7, 0) {Counitality};

                    \node[op, scale = 0.75] (s1) at (-2, -0.5) {};

                    \draw [line width = 1pt] (-2.5, 1) -- (-2.5, 0.5) -- (-2, 0.25) -- (s1);
                    \draw [line width = 1pt] (-2.5, 0.5) -- (-3, 0.25) -- (-3, -0.5);

                    \node at (-1, 0) {$=$};

                    \draw [line width = 1pt] (0, 1) -- (0, -0.5);

                    \node at (1, 0) {$=$};

                    \node[op, scale = 0.75] (s2) at (2, -0.5) {};

                    \draw [line width = 1pt] (2.5, 1) -- (2.5, 0.5) -- (2, 0.25) -- (s2);
                    \draw [line width = 1pt] (2.5, 0.5) -- (3, 0.25) -- (3, -0.5);
                \end{tikzpicture}
            \end{center}

            This notation will be adopted for our algebras and coalgebras when convenient. The intuition for coalgebras is more accessible with electronic circuits, as we can work out a statement of algebras and then turn the diagram upside down to make it into a statement of coalgebras.

            Previously we talked about braiding and how that relates to interchanging strings. In the same manner that we have a horizontal and vertical associator, we also have vertical and horizontal braiding. Horizontal braiding is the usual notion of braiding strings. On the other hand, vertical braiding refers to the function composition of tensors, which manifests in electronic circuits as sliding a morphism along a string. Whenever the given braiding of $\mathcal{C}$ is nice enough, we can get away by ignoring it whenever we move a morphism along a string. For instance, look at the category of $\mathbb{K}$-modules where we may define the braiding on elementary tensors as $\beta (a\otimes b) = b\otimes a$. In this case, the braiding is agnostic to how we move our morphisms along a string, and this means that we have the following equality of circuits.
            \begin{center}
                \begin{tikzpicture}[line cap=round,line join=round,>=triangle 45,x=1cm,y=1cm, thick, op/.style={circle, draw, scale=0.75}, scale=0.7]
                    \node[op, scale = 0.75] (f1) at (0,0) {f};
                    \node[op, scale = 0.75] (g1) at (0,-1) {g};
                    \node[op, scale = 0.75] (h1) at (1,0) {h};
                    
                    \draw [line width = 1pt] (0,0.5) -- (f1) -- (g1) -- (0, -1.5);
                    \draw [line width = 1pt] (1,0.5) -- (h1) -- (1,-1.5);

                    \node at (2, -0.5) {$=$};

                    \node[op, scale = 0.75] (f2) at (3,0) {f};
                    \node[op, scale = 0.75] (g2) at (3,-1) {g};
                    \node[op, scale = 0.75] (h2) at (4,-1) {h};
                    
                    \draw [line width = 1pt] (3,0.5) -- (f2) -- (g2) -- (3, -1.5);
                    \draw [line width = 1pt] (4,0.5) -- (h2) -- (4,-1.5);
                \end{tikzpicture}
            \end{center}

            In nature, we may encounter braidings that are not as nice. In these cases, we should take a step back to figure out how to move morphisms along strings before we continue using this graphical calculation of function composition. We will meet such a braiding soon.
            

    \subsection{Derivations and DG-Algebras}
            This section aims to define differential graded algebras and their modules. Given an algebra $A$, we define a derivation as a map satisfying the Leibniz rule. In the dual case for a coalgebra, we may define a coderivation as a map satisfying the Zinbiel rule, but we will refer to these maps as derivations for brevity. Once we grasp how to make derivations, we introduce graded algebras and modules to equip these with derivations. Derivations will allow us to state the categories of differential graded algebras and cochain complexes. Throughout this section, we will also develop electronic circuits for these notions.

            \begin{definition}[Derivations and Coderivations]
                Let $M$ be an $A$-bimodule. A $\mathbb{K}$-linear morphism $d:A\rightarrow M$ is called a derivation if $d(ab)=d(a)b+ad(b)$, i.e. electronically,

                \begin{center}
                    \begin{tikzpicture}[line cap=round,line join=round,>=triangle 45,x=1cm,y=1cm, thick, op/.style={circle, draw, scale=0.75}, scale=0.7]
                        
                        % \node at (-0.5, 1) {a};
                        % \node at (0.5, 1) {b};
                        \node[op, scale = 0.75] (d) at (0,-0.5) {d};
                        
                        \draw [line width=1pt] (-0.5, 1) -- (-0.5, 0.5) -- (0,0.25) -- (0.5,0.5) -- (0.5, 1);
                        \draw [line width=1pt] (0,0.25) -- (d) -- (0,-1);
                        
                        \node at (1,0) {$=$};

                        % \node at (2, 1) {a};
                        % \node at (3, 1) {b};

                        \node[op, scale = 0.75] (e) at (2, 0.5) {d};
                        \node[op, scale = 0.5] (r) at (2.5,-0.25) {$\mu_M^r$};

                        
                        \draw [line width=1pt] (2, 1) -- (e) -- (2, 0) -- (r) -- (3,0) -- (3,1);
                        \draw [line width=1pt] (r) -- (2.5,-1);

                        \node at (3.5,0) {$+$};

                        % \node at (4, 1) {a};
                        % \node at (5, 1) {b};

                        \node[op, scale = 0.75] (f) at (5, 0.5) {d};
                        \node[op, scale = 0.5] (l) at (4.5,-0.25) {$\mu_M^l$};

                        
                        \draw [line width=1pt] (4, 1) --  (4, 0) -- (l) -- (5,0) -- (f) -- (5,1);
                        \draw [line width=1pt] (l) -- (4.5,-1);
                    \end{tikzpicture}
                \end{center}

                Let $N$ be a $C$-bicomodule. A $\mathbb{K}$-linear morphism $d:N\rightarrow C$ is called a coderivation if $\Delta_C\circ d = (d\otimes id_C)\circ\omega_N^r + (id_C\otimes d)\circ\omega_N^l$, i.e. electronically,
                \begin{center}
                    \begin{tikzpicture}[line cap=round,line join=round,>=triangle 45,x=1cm,y=1cm, thick, op/.style={circle, draw, scale=0.75}, scale=0.7]
                        
                        \node[op, scale = 0.75] (d) at (0,0.5) {d};
                        
                        \draw [line width=1pt] (-0.5, -1) -- (-0.5, -0.5) -- (0,-0.25) -- (0.5,-0.5) -- (0.5, -1);
                        \draw [line width=1pt] (0,-0.25) -- (d) -- (0,1);
                        
                        \node at (1,0) {$=$};

                        \node[op, scale = 0.75] (e) at (2, -0.5) {d};
                        \node[op, scale = 0.5] (r) at (2.5,0.25) {$\omega_N^r$};

                        
                        \draw [line width=1pt] (2, -1) -- (e) -- (2, 0) -- (r) -- (3,0) -- (3,-1);
                        \draw [line width=1pt] (r) -- (2.5,1);

                        \node at (3.5,0) {$+$};

                        \node[op, scale = 0.75] (f) at (5, -0.5) {d};
                        \node[op, scale = 0.5] (l) at (4.5,0.25) {$\omega_N^l$};

                        
                        \draw [line width=1pt] (4, -1) --  (4, 0) -- (l) -- (5,0) -- (f) -- (5,-1);
                        \draw [line width=1pt] (l) -- (4.5,1);
                    \end{tikzpicture}
                \end{center}
            \end{definition}

            We remark that this translation between equations and electronic circuits is not at the same level of generalization. Due to this, the electronic circuit description has more advantages as it allows us to think with elements when we are only dealing with morphisms. We will use these circuits to derive results independent of the given braiding on the category.

            A helpful fact about derivations is that they will always map the identity to $0$. We obtain this from the Leibniz rule as one would get $d(1) = 2d(1)$, and thus $d(1) = 0$.

            \begin{proposition}\label{prop: tensor-derivation}
                Let $V$ be a $\mathbb{K}$-module and $M$ be a $T(V)$-bimodule. A $\mathbb{K}$-linear morphism $f:V\rightarrow M$ uniquely determines a derivation $d_f:T(V)\rightarrow M$, i.e. there is an isomorphism $\tt{Hom}_{\mathbb{K}}(V,M)\simeq \tt{Der}(T(V),M)$.


                Let $N$ be a $T^c(V)$-bicomodule. A $\mathbb{K}$-linear morphism $g:M\rightarrow V$ uniquely determines a coderivation $d_g^c:N\rightarrow T^c(V)$, i.e. there is an isomorphism $\tt{Hom}_{\mathbb{K}}(N,V)\simeq \tt{Coder}(N,T^c(V))$.
            \end{proposition}

            \begin{proof}
                Let $a_1\otimes ... \otimes a_n$ be an elementary tensor of $T(V)$. We define a map $d_f : T(V) \rightarrow M$ as
                \begin{align*}
                    d_f(a_1\otimes ... \otimes a_n) & = \sum_{i=1}^n a_1...f(a_i)...a_n \\
                    d_f(1) & = 0\tt{.} 
                \end{align*}
                $d_f$ is a derivation by definition.
                
                Restriction to $V$ gives the natural isomorphism. Let $i : V\rightarrow T(V)$ be the inclusion, then $i^*d_f = f$. Let $d : T(V) \rightarrow M$ be a derivation, then $d_{i^*d}=d$. Suppose now that $g: M \rightarrow N$ is a morphism of $T(V)$-bimodules; then naturality follows from linearity.

                In the dual case, $d_g^c: N \rightarrow T^c(V)$ is a bit tricky to define. Let $\omega^l_N:N\rightarrow N\otimes T^c(V)$ and $\omega^r_N : N\rightarrow T^c(V) \otimes N$ denote the coactions on $N$. Since $T^c(V)$ is conilpotent, we get the same finiteness restrictions on $N$. Define the reduced coactions as $\overline{\omega}^l_N = \omega^l_N - \argument\otimes 1$ and $\overline{\omega}^r_N = \omega^r_N - 1\otimes\argument\ $, this is well-defined by coassociativity. Observe that for any $n\in N$ there are $k$ and $k'>0$ such that ${\overline{\omega}^{l^k}_N}(n) = 0$ and ${\overline{\omega}^{r^{k'}}_N}(n)=0$.

                Let $n_{(k)}^{(i)}$ denote the extension of $n$ by $k$ coactions at position $i$, i.e. 
                \begin{align*}
                    n_{(k)}^{(i)} = \overline{\omega}^{r^i}_N\overline{\omega}^{l^{k-i}}_N(n)\tt{.} 
                \end{align*}
                The extension of $n$ by $k$ coactions is then the sum over every position $i$,
                \begin{align*}
                    n_{(k)} = \sum_{i=0}^kn_{(k)}^{(i)}\tt{.}
                \end{align*} 
                Observe that $n_{(0)} = n$. The grade of $n$ is the smallest $k$ such that $n_{(k)}$ is zero. This grading gives us the coradical filtration of $N$, and it is exhaustive by the finiteness restrictions given above. With this notion, every element of $N$ has a finite grade.

                If $g: N \rightarrow V$ is a linear map, we may think of it as a map sending every element of $N$ to an element of $T^c(V)$ of grade $1$. We must extend the morphism to get a map that sends the element of grade $k$ to grade $k$. Let $\pi: T^c(V) \rightarrow V$ be the linear projection and define $g_{(k)}^{(i)} = \pi\otimes ... \otimes g \otimes \pi$ as a morphism which of $k$ tensors which is $g$ at the $i$-th argument, but the projection otherwise. We define $d_g^c$ as the sum over each coaction and coordinate,
                \begin{align*}
                    d_g^c(n) = \sum_{k=0}^\infty \sum_{i=0}^k g_{(k)}^{(i)}(n_{(k)}^{(i)})\tt{.}
                \end{align*}
                
                Upon closer inspection, we may observe this is the dual construction of the derivation morphism. It is well-defined as the sum is finite by the finiteness restrictions. The map is a coderivation by duality, and the natural isomorphism is post-composition with the projection map $\pi$.
            \end{proof}

            \begin{definition}[Differential algebra]
                Let $A$ be an algebra. We say that $A$ is a differential algebra if it is equipped with a derivation $d: A\rightarrow A$. Dually, a coalgebra $C$ is a differential coalgebra if it is equipped with a coderivation $d: C\rightarrow C$.
            \end{definition}

            \begin{definition}[A-derivation]
                Let $(A,d_A)$ be a differential algebra and $M$ a left $A$-module. A $\mathbb{K}$-linear morphism $d_M:M\rightarrow M$ is called an $A$-derivation if $d_M(am)=d_A(a)m + ad_M(m)$, or electronically,
                \begin{center}
                    \begin{tikzpicture}[line cap=round,line join=round,>=triangle 45,x=1cm,y=1cm, thick, op/.style={circle, draw, scale=0.75}, scale=0.7]
                        
                        % \node at (-0.5, 1) {a};
                        % \node at (0.5, 1) {m};
                        \node[op, scale = 0.5] (d) at (0,-0.5) {$d_M$};
                        
                        \draw [line width=1pt] (-0.5, 1) -- (-0.5, 0.5) -- (0,0.25) -- (0.5,0.5) -- (0.5, 1);
                        \draw [line width=1pt] (0,0.25) -- (d) -- (0,-1);
                        
                        \node at (1,0) {$=$};

                        % \node at (2, 1) {a};
                        % \node at (3, 1) {m};

                        \node[op, scale = 0.5] (e) at (2, 0.5) {$d_A$};

                        
                        \draw [line width=1pt] (2, 1) -- (e) -- (2, 0) -- (2.5,-0.25) -- (3,0) -- (3,1);
                        \draw [line width=1pt] (2.5,-0.25) -- (2.5,-1);

                        \node at (3.5,0) {$+$};

                        % \node at (4, 1) {a};
                        % \node at (5, 1) {m};

                        \node[op, scale = 0.5] (f) at (5, 0.5) {$d_M$};
                        
                        \draw [line width=1pt] (4, 1) -- (4, 0) -- (4.5,-0.25) -- (5,0) -- (f) -- (5,1);
                        \draw [line width=1pt] (4.5,-0.25) -- (4.5,-1);
                    \end{tikzpicture}
                \end{center}
                Dually, given a differential coalgebra $(C,d_C)$ and $N$ a left $C$-comodule, a $\mathbb{K}$-linear morphism $d_N:N\rightarrow N$ is a coderivation if $\omega_N\circ d_N = (d_C\otimes id_N + id_C\otimes d_N)\circ \omega_N$, or electronically,
                \begin{center}
                    \begin{tikzpicture}[line cap=round,line join=round,>=triangle 45,x=1cm,y=1cm, thick, op/.style={circle, draw, scale=0.75}, scale=0.7]
                        
                        \node[op, scale = 0.5] (d) at (0,0.5) {$d_N$};
                        
                        \draw [line width=1pt] (-0.5, -1) -- (-0.5, -0.5) -- (0,-0.25) -- (0.5,-0.5) -- (0.5, -1);
                        \draw [line width=1pt] (0,-0.25) -- (d) -- (0,1);
                        
                        \node at (1,0) {$=$};

                        \node[op, scale = 0.5] (e) at (2, -0.5) {$d_C$};
                        
                        \draw [line width=1pt] (2, -1) -- (e) -- (2, 0) -- (2.5,0.25) -- (3,0) -- (3,-1);
                        \draw [line width=1pt] (2.5,0.25) -- (2.5,1);

                        \node at (3.5,0) {$+$};

                        \node[op, scale = 0.5] (f) at (5, -0.5) {$d_N$};
                        
                        \draw [line width=1pt] (4, -1) -- (4, 0) -- (4.5,0.25) -- (5,0) -- (f) -- (5,-1);
                        \draw [line width=1pt] (4.5,0.25) -- (4.5,1);
                    \end{tikzpicture}
                \end{center}
            \end{definition}

            When there is no ambiguity, we will start to adopt writing the differential in electronic circuits as a triangle,

            \begin{center}
                \begin{tikzpicture}[line cap=round,line join=round,>=triangle 45,x=1cm,y=1cm, thick, op/.style={circle, draw, scale=0.75}, diff/.style={regular polygon, draw, regular polygon sides = 3, scale=0.75, rotate = 180}, scale=0.7]

                    \node[op, scale = 0.75] (1) at (0,0) {$d_M^\bullet$};

                    \draw [line width = 1pt] (0, 1) -- (1) -- (0, -1);

                    \node at (1, 0) {$=$};

                    \node[diff, scale=0.75] (2) at (2, 0) {};

                    \draw [line width = 1pt] (2, 1) -- (2) -- (2, -1);
                    
                \end{tikzpicture}
            \end{center}

            \begin{proposition}\label{prop: free-derivation}
                Let $A$ be a differential algebra and $M$ a $\mathbb{K}$-module. A $\mathbb{K}$-linear morphism $f:M\rightarrow A\otimes_{\mathbb{K}} M$ uniquely determines a derivation $d_f:A\otimes M\rightarrow A\otimes M$, i.e. there is an isomorphism $\tt{Hom}_{\mathbb{K}}(M,A\otimes_{\mathbb{K}}M)\simeq \tt{Der}(A\otimes_{\mathbb{K}}M)$. Moreover, $d_f$ is given as $((\cdot_A)\otimes id_M)\circ (id_A\otimes f) + d_A\otimes id_M$.

                Dually, if $C$ is a differential coalgebra and $N$ is a $\mathbb{K}$-module, then a $\mathbb{K}$-linear morphism $g:C\otimes N\rightarrow N$ uniquely determines a coderivation $d_g:C\otimes_{\mathbb{K}}N\rightarrow C\otimes_{\mathbb{K}}N$. There is an isomorphism $\tt{Hom}_{\mathbb{K}}(C\otimes_{\mathbb{K}}N,N)\simeq \tt{Coder}(C\otimes_{\mathbb{K}}N)$, and $d_g$ is given as $(id_C\otimes g)\circ (\Delta_C\otimes id_N) + d_C\otimes id_N$.
            \end{proposition}

            \begin{proof}
                We will only prove this proposition in the case of algebras. The case of coalgebras is dual.
                
                We have to prove that the morphism $d_{\argument} : \tt{Hom}_\mathbb{K}(M, A\otimes_\mathbb{K}M) \rightarrow \tt{Der}(A \otimes_\mathbb{K}M)$ is well-defined. To do this, we must check that for any morphism $f: M \rightarrow A \otimes_\mathbb{K} M$, the morphism $d_f$ satisfies the Leibniz rule.

                Assume that we have elements $a, b \in A$ and $m\in M$. Then $d_f(ab\otimes m) = d_f(a(b\otimes m))$. We abuse the notation to write equality between an element and a circuit. Recall that this means that we have to think of $a$, $b$, and $m$ as generalized elements,
                \begin{center}
                    \begin{tikzpicture}[line cap=round,line join=round,>=triangle 45,x=1cm,y=1cm, thick, op/.style={circle, draw, scale=0.75}, diff/.style={regular polygon, draw, regular polygon sides = 3, scale=0.75, rotate = 180}, scale=0.7]

                        \node at (-2, 0) {$d_f(ab\otimes m)$};

                        \node at (0,0) {$=$};

                        \node[op, scale = 0.75] (f1) at (3, 0.5) {$f$};

                        \draw [line width = 1pt] (1,1) -- (1, 0.5) -- (1.5, 0.25);
                        \draw [line width = 1pt] (2,1) -- (2, 0.5) -- (1.5, 0.25) -- (1.5, 0) -- (2, -0.25);
                        \draw [line width = 1pt] (3, 1) -- (f1) -- (2.5, 0.25) -- (2.5, 0) -- (2, -0.25) -- (2, -1);
                        \draw [line width = 1pt] (f1) -- (3.5, 0.25) -- (3.5, -1);

                        \node at (4, 0) {$+$};
                        \node[diff, scale = 0.75] (d1) at (5, 0) {};

                        \draw [line width = 1pt] (4.5, 1) -- (4.5, 0.75) -- (5, 0.5);
                        \draw [line width = 1pt] (5.5, 1) -- (5.5, 0.75) -- (5, 0.5) -- (d1) -- (5, -1);
                        \draw [line width = 1pt] (6, 1) -- (6, -1);

                        \node at (7, 0) {$=$};

                        \node[op, scale = 0.75] (f2) at (10, 0.5) {$f$};

                        \draw [line width = 1pt] (8, 1) -- (8, -0.5) -- (8.5, -0.75);
                        \draw [line width = 1pt] (8.5, 1) -- (8.5, 0) -- (9, -0.25);
                        \draw [line width = 1pt] (10, 1) -- (f2) -- (9.5, 0.25) -- (9.5, 0) -- (9, -0.25) -- (9, -0.5) -- (8.5, -0.75) -- (8.5, -1);
                        \draw [line width = 1pt] (f2) -- (10.5, 0.25) -- (10.5, -1);

                        \node at (11, 0) {$+$};

                        \node[diff, scale = 0.75] (d2) at (11.5, 0.5) {};

                        \draw [line width = 1pt] (11.5, 1) -- (d2) -- (11.5, 0) -- (12, -0.25) -- (12, -1);
                        \draw [line width = 1pt] (12.5, 1) -- (12.5, 0) -- (12, -0.25);
                        \draw [line width = 1pt] (13, 1) -- (13, -1);
                        
                        \node at (13.5, 0) {$+$};

                        \node[diff, scale = 0.75] (d3) at (15, 0.5) {};

                        \draw [line width = 1pt] (14, 1) -- (14, 0) -- (14.5, -0.25) -- (14.5, -1);
                        \draw [line width = 1pt] (15, 1) -- (d3) -- (15, 0) -- (14.5, -0.25);
                        \draw [line width = 1pt] (15.5, 1) -- (15.5, -1);

                        \node[anchor = west] at (-4, -2) {$= d_A(a)b\otimes m + ad_f(b\otimes m)$.};
                    \end{tikzpicture}
                \end{center}

                Next, we show that $d_{\argument}$ has an inverse, which is given by "restriction to $M$," also known as 
                \begin{align*}
                    (1_A\otimes M)^* : \tt{Hom}_\mathbb{K}(A\otimes_\mathbb{K}M, N) \rightarrow \tt{Hom}_\mathbb{K}(M, N)\tt{.}
                \end{align*}
                Let $f: M \rightarrow A\otimes_\mathbb{K} M$ be a linear map and $D: A\otimes_\mathbb{K} M \rightarrow A\otimes_\mathbb{K} M$ be a derivation, then a quick calculation verifies that $d_{\argument}$ is inverse to restriction.

                \begin{center}
                    \begin{tikzpicture}[line cap=round,line join=round,>=triangle 45,x=1cm,y=1cm, thick, op/.style={circle, draw, scale=0.75}, diff/.style={regular polygon, draw, regular polygon sides = 3, scale=0.75, rotate = 180}, scale=0.7]

                        \node at (-2,0) {$d_f\circ(1_A \otimes M)$};
                        \node at (0,0) {$=$};

                        \node[op, scale = 0.75] (s1) at (1, 0.75) {};
                        \node[op, scale = 0.75] (f1) at (2.5, 0.5) {$f$};

                        \draw [line width = 1pt] (s1) -- (1, 0) -- (1.5, -0.25) -- (1.5, -1);
                        \draw [line width = 1pt] (2.5, 1) -- (f1) -- (2, 0.25) -- (2, 0) -- (1.5, -0.25);
                        \draw [line width = 1pt] (f1) -- (3, 0.25) -- (3, -1);

                        \node at (3.5, 0) {$+$};

                        \node[op, scale = 0.75] (s2) at (4, 0.75) {};
                        \node[diff, scale = 0.75] (d1) at (4, 0.25) {};

                        \draw [line width = 1pt] (s2) -- (d1) -- (4, -1);
                        \draw [line width = 1pt] (4.5, 1) -- (4.5, -1);

                        \node at (5.5, 0) {$=$};
                        \node at (6, 0) {$f$};
                        
                    \end{tikzpicture}
                \end{center}
                \begin{center}
                    \begin{tikzpicture}[line cap=round,line join=round,>=triangle 45,x=1cm,y=1cm, thick, op/.style={circle, draw, scale=0.75}, diff/.style={regular polygon, draw, regular polygon sides = 3, scale=0.75, rotate = 180}, scale=0.7]
                        
                        \node at (-2, 0) {$d_{D \circ (1_A\otimes M)}$};
                        \node at (0,0) {$=$};

                        \node[op, scale = 0.75] (s1) at (2,0.75) {};
                        \node[op, scale=0.75] (D1) at (2.5, 0.25) {$D$};

                        \draw [line width = 1pt] (1, 1) -- (1, -0.25) -- (1.5, -0.5) -- (1.5, -1);
                        \draw [line width = 1pt] (s1) -- (2, 0.5) -- (D1) -- (2, 0) -- (2, -0.25) -- (1.5, -0.5);
                        \draw [line width = 1pt] (3, 1) -- (3, 0.5) -- (D1) -- (3, 0) -- (3, -1);

                        \node at (3.5, 0) {$+$};

                        \node[diff, scale=0.75] (d1) at (4, 0) {};

                        \draw [line width = 1pt] (4, 1) -- (d1) -- (4, -1);
                        \draw [line width = 1pt] (4.5, 1) -- (4.5, -1);

                        \node at (5.5, 0) {$=$};
                        \node at (6, 0) {$D$};
                    \end{tikzpicture}
                \end{center}

                Notice that we use the Leibniz rule in the last equation to get the equality to $D$. 

            \end{proof}

            We say that a $\mathbb{K}$-module $M^*$ admits a $\mathbb{Z}$-grading if it decomposes into either summands or factors
            \begin{align*}
                M^* = \bigoplus_{z:\mathbb{Z}}M^z\tt{ or } M^* = \prod_{z:\mathbb{Z}}M^z\tt{.}
            \end{align*}
            An element of $m\in M$ is said to be homogenous if it is properly contained in a single summand, i.e., $m\in M^n$. $m$ is then said to have degree $n$. We say that a morphism of graded modules $f: M^*\rightarrow N^*$ is homogenous of degree $n$ if it preserves the grading, that is $f(M^i) \subseteq N^{n+i}$. The degree of a homogenous element $m$ or morphism $f$ is denoted as $|m|$ or $|f|$.
            
            There is a distinction between the ordinary and self-enriched categories of graded modules. We are going to work with the self-enriched category, and its hom-objects are the graded module of homogenous morphisms. We denote a factor in the grading as $\tt{Hom}_\mathbb{K}^w(M^*,N^*)=\startset{f : M^* \rightarrow N^* \mid f\tt{ is homogenous and }|f|=w}$, so the graded hom is
            \begin{align*}
                \tt{Hom}_\mathbb{K}^* = \prod_{w\in\mathbb{Z}}\tt{Hom}_\mathbb{K}^w\tt{.}
            \end{align*}
            This category is denoted as $\tt{Mod}_\mathbb{K}^*$. In general, and whenever it makes sense, we write $\mathcal{C}^*$ as the category of $\mathbb{Z}$-graded objects from $\mathcal{C}$.

            The category $\tt{Mod}_\mathbb{K}^*$ is a closed symmetric monoidal category. The tensor is given by the following formula, using the ordinary tensor of $\tt{Mod}_\mathbb{K}$,
            \begin{align*}
                M^*\otimes N^* = \bigoplus_{n\in \mathbb{Z}}\bigoplus_{p\in\mathbb{Z}}M^p\otimes_\mathbb{K} N^q\tt{, where }q = n - p\tt{.}
            \end{align*}
            The associator of $\tt{Mod}_\mathbb{K}$ may be lifted to this tensor. The unit is the module $\mathbb{K}$ concentrated in degree $0$. Likewise, both the right and left unit transformation may be lifted from $\mathbb{K}$. 
            
            The category $\tt{Mod}_\mathbb{K}^*$ is closed, which means that the graded tensor fixed in one variable is left adjoint to the graded hom. We may obtain the graded hom as the right adjoint for the other variable by using the braiding, which we will define later. Showing closedness is done using the tensor-hom adjunction from $\tt{Mod}_\mathbb{K}$.
            \begin{multline*}
                \tt{Hom}_\mathbb{K}^*(A^*\otimes B^*, C^*) = \prod_{w\in\mathbb{Z}}\prod_{n\in\mathbb{Z}}\tt{Hom}_\mathbb{K}^w(\bigoplus_{p\in\mathbb{Z}}A^p\otimes_\mathbb{K}B^{n-p},C^n) \\
                = \prod_{w\in\mathbb{Z}}\prod_{n\in\mathbb{Z}}\prod_{p\in\mathbb{Z}}\tt{Hom}_\mathbb{K}(A^p\otimes_\mathbb{K} B^{n-(p+w)},C^n) \simeq \prod_{w\in\mathbb{Z}}\prod_{n\in\mathbb{Z}}\prod_{p\in\mathbb{Z}}\tt{Hom}_\mathbb{K}(A^p,\tt{Hom}_\mathbb{K}(B^{n-(p+w),C^n})) \\
                \simeq \prod_{w\in\mathbb{Z}}\prod_{p\in\mathbb{Z}}\tt{Hom}_\mathbb{K}(A^p, \prod_{n\in\mathbb{Z}}\tt{Hom}_\mathbb{K}(B^{n-(p+w),C^n})) = \prod_{w\in\mathbb{Z}}\prod_{p\in\mathbb{Z}}\tt{Hom}_\mathbb{K}(A^p,\tt{Hom}_\mathbb{K}^{p+w}(B^*,C^*)) \\
                \simeq \prod_{w\in\mathbb{Z}}\tt{Hom}_\mathbb{K}^w(A^*,\tt{Hom}_\mathbb{K}^*(B^*,C^*)) = \tt{Hom}_\mathbb{K}^*(A^*,\tt{Hom}_\mathbb{K}^*(B^*,C^*))\tt{.}
            \end{multline*}
            
            We give a braiding on homogenous elementary tensors as
            \begin{align*}
                \beta(a\otimes b) = (-1)^{|a||b|}b\otimes a\tt{.}
            \end{align*}
            It is immediate that $\beta_{A,B}$ is inverse to $\beta_{B,A}$. Observe that this category also admits a braiding where we don't introduce a sign. However, this does not work when we want to add differentials to our graded modules, so we stick with this sign. This braiding is also commonly known as the Koszul sign convention. 

            Since $\tt{Mod}_\mathbb{K}^*$ is a closed symmetric monoidal category, it admits electronic circuits. Thus the previous results we have proved by electronic circuits also apply to this category, as the proof is identical in this language. One should note that the specific implementation may differ as vertical braiding works differently. We will now study vertical braiding in more detail. Following Kelly \cite{Kelly05}, we must define the application of two homogenous morphisms $f : A \rightarrow A'$ and \\ $g : B \rightarrow B'$ on elements $a \in A$ and $b\in B$ on tensors as
            \begin{align*}
                (f\otimes g)(a\otimes b) = (-1)^{|g||a|}f(a)\otimes g(b)\tt{.}
            \end{align*}
            Viewing $a$ and $b$ as generalized elements again, we get Koszul's sign rule on morphisms. That is, given homogenous composable morphisms $f, f', g, g'$, we get that
            \begin{align*}
                (f' \otimes g')\circ(f\otimes g) = (-1)^{|g'||f|}(f'\circ f)\otimes (g'\circ g)\tt{.}
            \end{align*}
            Electronically we may represent this as a $2$-string circuit where a morphism on the left wants to downwards pass a morphism on its right,
            \begin{center}
                \begin{tikzpicture}[line cap=round,line join=round,>=triangle 45,x=1cm,y=1cm, thick, op/.style={circle, draw, scale=0.75}, diff/.style={regular polygon, draw, regular polygon sides = 3, scale=0.75, rotate = 180}, scale=0.7]

                    \node[op, scale = 0.75] (f1) at (0, 0.25) {$f$};
                    \node[op, scale = 0.75] (g1) at (0.5, -0.25) {$g$};

                    \draw[line width = 1pt] (0,1) -- (f1) -- (0,-1);
                    \draw[line width = 1pt] (0.5,1) -- (g1) -- (0.5,-1);

                    \node at (1.5, 0) {$=$};

                    \node at (3, 0) {$(-1)^{|g||f|}$};

                    \node[op, scale  = 0.75] (f2) at (4.5, -0.25) {$f$};
                    \node[op, scale = 0.75] (g2) at (5, 0.25) {$g$};

                    \draw[line width = 1pt] (4.5, 1) -- (f2) -- (4.5, -1);
                    \draw[line width = 1pt] (5, 1) -- (g2) -- (5, -1);
                    
                \end{tikzpicture}
            \end{center}

            A good way of thinking about moving components in a circuit is that whenever we move a component downwards, it has to pass over every component to the left on its current level and every component to the right of it on the level below. We introduce signs in a $2$-string circuit whenever a component is moved downwards to or completely past another component on its right. If we move a component upwards completely past another component to its left, we introduce a sign. In an $n$-string circuit, it gets more complicated as the component may have to move past several components on both the left and right.

            Unlike the other electronic equations in which we may substitute parts of an electronic circuit with other equal parts, this does not work a priori in this context because of how we defined levels. Within a $3$-string circuit, the formula changes, and this is because we want to manipulate every element on a level simultaneously. If we move a left-most component downwards past many components, we may regard them as a single component on a single string. We will use this interpretation to prove an interchange of components on an $n$-string circuit formula.
            
            \begin{proposition}\label{prop: multi-koszul-sign}
                Let $n \geq 1$ and suppose that we have $a_i\in A_i \rightarrow B_i$ and $b_i: B_i \rightarrow C_i$ for any $0 < i \leq n $. Then we get that
                \begin{align*}
                    (b_i\circ a_i) \otimes \cdots \otimes (b_n \circ a_n) & = (-1)^s(b_1 \otimes \cdots \otimes b_n)\circ (a_1 \otimes \cdots \otimes a_n)\tt{,} \\
                    \tt{where } s & = \sum_{i=1}^n|b_i|(\sum_{1\leq j < i}|a_j|)\tt{.}
                \end{align*}
            \end{proposition}

            \begin{proof}
                We prove this by induction. If $n=1$, this is true. $s = 0$ since the sum is empty, so $b_1 \circ a_1  =(-1)^s b_1 \circ a_1$.
                
                Assume that the conclusion holds for $n - 1$ and that we have $a_i$ and $b_i$ as in the hypothesis. Let $s' = \sum_{i=1}^{n-1}|b_i|(\sum_{1\leq j < i}|a_j|)$, then
                \begin{align*}
                    s = s' + |b_n|(\sum_{i = 1}^{n-1}|a_i|)\tt{.}
                \end{align*}
                The conclusion follows from this calculation.
                \begin{multline*}
                    (b_1 \circ a_1) \otimes \cdots \otimes (b_n \circ a_n) = (-1)^{s'}((b_1 \otimes \cdots \otimes b_{n-1}) \circ (a_1 \otimes \cdots \otimes a_{n-1})) \otimes (b_n \circ a_n) \\ = (-1)^{s' + |b_n|(\sum_{i = 1}^{n-1}|a_i|)}(b_1 \otimes \cdots \otimes b_n) \circ (a_1 \otimes \cdots \otimes a_n)\tt{.}
                \end{multline*}
            \end{proof}

            A final remark on this braiding is that it affects any scenario where we compose functions, and they move past each other. Since function composition factors through this tensor, moving functions around is a braiding. An important example of this is the pre-composition functor. If $f$ and $g$ are homogenous and composable, then
            \begin{align*}
                f^*(g) = (-1)^{|f||g|}g\circ f\tt{.}
            \end{align*}

            The graphical calculus we have developed will be the same for any symmetric monoidal category where the braiding is similar. What this means will soon be evident when we add extra structure to the objects of $\tt{Mod}_\mathbb{K}^*$.

            A graded $\mathbb{K}$-module $M^\bullet$ is called a cochain complex if it comes equipped with a differential $d_M: M^\bullet \rightarrow M^\bullet$. By a differential, we mean a homogenous morphism of degree $1$ such that $d_M^2 = 0$. Be cautious of bad notation, as $d^2_M$ might mean $d^2_M = d_M \circ d_M$ and $d^2_M : M^2 \rightarrow M^3$.

            Given a cochain complex $M^\bullet$, we know by definition that the image of the differential lies inside the kernel of the differential. We denote this at the $i$'th coordinate as $B^iM \subseteq Z^iM$. $B^*M$ is the graded submodule of images, also called boundaries. $Z^*M$ is the graded submodule of kernels, also called cycles. The graded cohomology module $H^*M$ is defined as the quotient $\sfrac{Z^*M}{B^*M}$. A cochain complex is said to be exact if $H^*M \simeq 0$.

            Cochain complexes are plentiful in nature.
            \begin{example}[$\mathbb{K}$ as a cochain complex]
                Let $\mathbb{K}^\bullet = (\mathbb{K}, 0)$ be the graded $\mathbb{K}$-module concentrated in degree $0$ together with a $0$ differential, and this is trivially a cochain complex.
            \end{example}
            \begin{example}[Trivial cochain complexes]
                Let $M^*$ be a graded $\mathbb{K}$-module. Let $M^\bullet = (M^*, 0)$ be the same graded module with the $0$ differential, and this is also a cochain complex.
            \end{example}
            \begin{example}
                We can create a cochain complex, as shown in the following diagram.
                \begin{center}
                    \begin{tikzcd}
                        \cdots \ar[]{r}[]{} & 0 \ar[]{r}[]{} & \mathbb{K} \ar[]{r}[]{id_\mathbb{K}} & \mathbb{K} \ar[]{r}[]{} & 0 \ar[]{r}[]{} & \cdots
                    \end{tikzcd}
                \end{center}
            \end{example}
            \begin{example}[Cone of a chain map]
                Suppose that $f : A^\bullet \rightarrow B^\bullet$ is a homogenous morphism of degree $0$ such that $f\circ d_A = d_B\circ f$. There is an associated cochain complex to $f$, which yields a short-exact sequence of cochain complexes. We define $\tt{cone}(f)$ at each degree by
                \begin{align*}
                    \tt{cone}(f)^n = A^{n+1} \oplus B^{n}\tt{,} \\
                    d_{\tt{cone}(f)}^n = \begin{pmatrix}
                        d_A^{n+1} & 0 \\ f^{n+1} & d_B^n
                    \end{pmatrix}\tt{.}
                \end{align*}
                This complex gives us a short exact sequence,
                \begin{center}
                    \begin{tikzcd}
                        B^\bullet \ar[tail]{r}[]{} & \tt{cone}(f) \ar[two heads]{r}[]{} & A^\bullet[1]\tt{.}
                    \end{tikzcd}
                \end{center}
            \end{example}
            \begin{example}[Normalized cochain complex]
                Let $A : \Delta^{op} \rightarrow \tt{Mod}_\mathbb{K}$ be a simplicial $\mathbb{K}$-module. We define a collection of diagrams $J^n$ as $J^0 = A_0$, and every other as
                \begin{center}
                    \begin{tikzcd}
                        J^n = A_n \ar[bend left, yshift = 2.5ex]{r}[]{0}\ar[yshift = 1.5ex]{r}[]{d_1} \ar[phantom]{r}[]{\vdots} \ar[yshift = -2ex]{r}[below]{d_{n}} & A_{n-1}
                    \end{tikzcd}
                \end{center}
                $A$'s normalized cochain complex is the complex given as
                \begin{align*}
                    NA^{-n} = \varprojlim J^n\tt{.}
                \end{align*}
                In a complete pointed category, such as $\tt{Mod}_\mathbb{K}$, the limit is the same as the intersection of every kernel:
                \begin{align*}
                    \varprojlim J^n = \bigcap_{i=1}^n\tt{Ker}d_i\tt{.}
                \end{align*}
                The differential of $NA$ is defined to be $d_0$. Since we have turned the complex around, this is a morphism of degree $1$. By taking the limit, we force $d_0^2 = 0$ as well.
            \end{example}
            \begin{example}[Associated cochain complex]\label{ex: ass-complex}
                Let $A : \Delta^{op} \rightarrow \tt{Mod}_\mathbb{K}$ be a simplicial $\mathbb{K}$-module. We define a differential as
                \begin{align*}
                    d = \sum_{i=0}^n(-1)^id_i\tt{.}
                \end{align*}
                Let $\tt{C}A$ be the complex given in each degree as
                \begin{align*}
                    \tt{C}A^{-n} = A_n\tt{.}
                \end{align*}
                $d$ defines a differential on $\tt{C}A$ of degree $1$.
            \end{example}
            \begin{example}[Singular chain complex with $\mathbb{K}$-coefficents]
                Let $M$ be a topological space. There is a simplicial set defined as $\tt{Sing}(M) = \tt{Top}(\Delta^{\argument},M) : \Delta^{op} \rightarrow \tt{Set}$. Here $\Delta^{[n]}$ in Top refers to the topological standard $n$-simplex. We get a simplicial $\mathbb{K}$-module by creating the free one, $\mathbb{K}\tt{Sing}(M)$. The above example defines a chain complex in $\tt{Mod}_\mathbb{K}$.
            \end{example}

            We make a distinction for some cochain complexes, which is of particular interest.
            
            \begin{definition}[Quasi-free cochain complexes]
                Suppose that $M^\bullet$ is a cochain complex. We say that $M^\bullet$ is quasi-free if the underlying graded module $M^*$ is free; in other words, $M^*$ is a tensor algebra.

                Likewise, we say that $M^\bullet$ is quasi-cofree if $M^*$ is cofree; in other words, $M^*$ is a tensor coalgebra.
            \end{definition}

            The category of cochain complexes will be denoted as $\tt{Mod}_\mathbb{K}^\bullet$. Note that this category is built upon $\tt{Mod}_\mathbb{K}^*$, and we inherit the braiding $\beta$. We want to entertain different collections of morphisms because the morphisms that respect the structure and the morphisms that make this category self-enriched are different. We will usually denote both of these categories as $\tt{Mod}_\mathbb{K}^\bullet$, but when we want to emphasize the structure-preserving maps, we will instead denote this as $\tt{Ch}(\mathbb{K})$.

            When $A^\bullet$ and $B^\bullet$ are cochain complexes the graded $\mathbb{K}$-module $\tt{Hom}_\mathbb{K}^*(A^\bullet, B^\bullet)$ admits a derivative. Let $f : A^\bullet \rightarrow B^\bullet$ be any homogenous morphism, then the derivative-, or boundary of $f$ is given by
            \begin{align*}
                \partial f = (d_{B*} + d_A^*)(f) = d_B \circ f - (-1)^{|f|}f\circ d_A\tt{.}
            \end{align*}
            We see that $|\partial| = |d_{B*} + d_A^*| = 1$, and
            \begin{align*}
                \partial^2f = (d_{B*} + d_A^*)(d_B \circ f - (-1)^{|f|}f\circ d_A) = d_B^2f + (-1)^{|f|}d_Bfd_A - (-1)^{|f|}d_Bfd_A - fd_A^2 = 0\tt{.}
            \end{align*}
            Thus, $\tt{Hom}_\mathbb{K}^\bullet(A^\bullet, B^\bullet) = (\tt{Hom}_\mathbb{K}^*(A^\bullet, B^\bullet), \partial)$ is a cochain complex. We endow $Mod_\mathbb{K}^\bullet$ with these hom-objects. In an electronic circuit, we write $\partial f$ as a sum of circuits,
            
            \begin{center}
                \begin{tikzpicture}[line cap=round,line join=round,>=triangle 45,x=1cm,y=1cm, thick, op/.style={circle, draw, scale=0.75}, diff/.style={regular polygon, draw, regular polygon sides = 3, scale=0.75, rotate = 180}, scale=0.7]

                    \node at (0,0) {$\partial f =$};

                    \node[op, scale=0.75] (f1) at (1, 0.5) {$f$};
                    \node[diff, scale=0.75] (d1) at (1, -0.5) {};

                    \draw [line width = 1pt] (1, 1) -- (f1) -- (d1) -- (1, -1);

                    \node at (2.5, 0) {$+(-1)^{|f|}$};

                    \node[diff, scale=0.75] (d2) at (4, 0.5) {};
                    \node[op, scale=0.75] (f2) at (4, -0.5) {$f$};

                    \draw [line width = 1pt] (4, 1) -- (d2) -- (f2) -- (4, -1);

                \end{tikzpicture}
            \end{center}

            Notice how this construction of $\tt{Hom}_\mathbb{K}^\bullet$ is the same as the (product) total complex of an anticommutative double complex. An anticommutative double complex is a graded module of cochain complexes, together with a differential between the cochain complexes. These different differentials are supposed to be anticommuting. We draw an anticommutative double complex, as shown below.
            \begin{center}
                \begin{tikzcd}
                    & \vdots & \vdots & \vdots \\
                    \cdots \ar[]{r}[]{d_C^h} & C^{-1, 1} \ar[]{u}[]{d_C^v} \ar[]{r}[]{d_C^h} & C^{0, 1} \ar[]{u}[]{d_C^v} \ar[]{r}[]{d_C^h} & C^{1, 1} \ar[]{u}[]{d_C^v} \ar[]{r}[]{d_C^h} & \cdots \\
                    \cdots \ar[]{r}[]{d_C^h} & C^{-1, 0} \ar[]{u}[]{d_C^v} \ar[]{r}[]{d_C^h} & C^{0,0} \ar[]{u}[]{d_C^v} \ar[]{r}[]{d_C^h} & C^{1, 0} \ar[]{u}[]{d_C^v} \ar[]{r}[]{d_C^h} & \cdots \\
                    \cdots \ar[]{r}[]{d_C^h} & C^{-1, -1} \ar[]{u}[]{d_C^v} \ar[]{r}[]{d_C^h} & C^{0, -1} \ar[]{u}[]{d_C^v} \ar[]{r}[]{d_C^h} & C^{1, -1} \ar[]{u}[]{d_C^v} \ar[]{r}[]{d_C^h} & \cdots \\
                    & \vdots \ar[]{u}[]{d_C^v} & \vdots \ar[]{u}[]{d_C^v} & \vdots \ar[]{u}[]{d_C^v}
                \end{tikzcd}
            \end{center}
            Another way of thinking of an anticommutative double complex $C^{\bullet, \bullet}$ is that it is a bigraded $\mathbb{K}$-module with a vertical and horizontal differential such that $d^v_C\circ d^h_C = -d^h_C \circ d^v_C$.

            \begin{definition}
                Let $C^{\bullet, \bullet}$ be an anticommutative double complex. We define the sum and product total complex. The differential at each $C^{p,q}$ is defined as $d_{\tt{Tot}C} = d^v_C + d^h_C$, and
                \begin{align*}
                    \tt{Tot}^{\oplus}(C^{\bullet, \bullet}) & = \bigoplus_{n\in\mathbb{Z}}\bigoplus_{p+q=n}C^{p,q}\tt{,} \\
                    \tt{Tot}^{\prod}(C^{\bullet, \bullet}) & = \prod_{n\in \mathbb{Z}}\prod_{p+q=n}C^{p,q}\tt{.}
                \end{align*}
            \end{definition}
            
            If $C^{\bullet, \bullet}$ is bounded, then $\tt{Tot}^\oplus(C^{\bullet, \bullet}) \simeq \tt{Tot}^{\prod}(C^{\bullet, \bullet})$.

            If we let $\tt{Hom}_\mathbb{K}(A^\bullet, B^\bullet)^{\bullet, \bullet} = (\prod_{p,q\in\mathbb{Z}}\tt{Hom}_\mathbb{K}(A^p,B^q), d_A^*, d_{B*})$, then it is clear that
            \begin{align*}
                \tt{Hom}_\mathbb{K}^\bullet(A^\bullet,B^\bullet) = \tt{Tot}^{\prod}(\tt{Hom}_\mathbb{K}(A^\bullet, B^\bullet)^{\bullet, \bullet})\tt{.}
            \end{align*}

            From this, we can deduce that $\tt{Mod}_\mathbb{K}^\bullet$ is a closed symmetric monoidal category. The tensor is collected from the data of $\tt{Hom}_\mathbb{K}^\bullet$. We do this by defining an anticommutative double complex $(A^\bullet \otimes_\mathbb{K} B^\bullet)^{\bullet, \bullet} = (\bigoplus_{n\in\mathbb{Z}}\bigoplus_{p+q=n}A^p\otimes B^q, d_A\otimes B,A\otimes d_B)$, then the tensor is defined as
            \begin{align*}
                A^\bullet \otimes B^\bullet = \tt{Tot}^\oplus((A^\bullet \otimes B^\bullet)^{\bullet, \bullet})\tt{.}
            \end{align*}
            This tensor is left adjoint to $\tt{Hom}_\mathbb{K}^\bullet$. All the structure morphisms for a closed symmetric monoidal category are inherited from
            s inherited from $\tt{Mod}_\mathbb{K}^*$, and this also means that $\tt{Mod}_\mathbb{K}^\bullet$ employs the same electronic circuits as $\tt{Mod}_\mathbb{K}^*$.

            The category of cochain complexes with chain maps $\tt{Ch}(\mathbb{K})$ is defined to have its hom-objects as $Z^0\tt{Hom}_\mathbb{K}^\bullet(A^\bullet, B^\bullet)$. By abuse of notation we may write $\tt{Ch}(\mathbb{K}) = Z^0\tt{Mod}_\mathbb{K}^\bullet$. Notice that this condition means that the derivative of any morphism $f: A^\bullet \rightarrow B^\bullet$ in $\tt{Ch}(\mathbb{K})$ is $0$; i.e., that $\partial f = 0$, or $f\circ d_A = d_B \circ f$. We will call these morphisms chain maps.

            The homotopy category $\tt{K}(\mathbb{K})$ is defined to be the quotient category of $\tt{Ch}(\mathbb{K})$ at null-homotopic chain maps. Observe that $\tt{K}(\mathbb{K}) = H^0\tt{Mod}_\mathbb{K}^\bullet$ because the chain maps $f, g: A^\bullet \rightarrow B^\bullet$ are homotopic if there is a homogenous morphism $h: A^\bullet \rightarrow B^\bullet$ of degree $-1$ such that $\partial h = f - g$.

            A chain map $f: A^\bullet \rightarrow B^\bullet$ induces homogenous morphisms of degree $0$.
            \begin{align*}
                B^*f & : B^*A \rightarrow B^*B \\
                Z^*f & : Z^*A \rightarrow Z^*B \\
                H^*f & : H^*A \rightarrow H^*B
            \end{align*}
            We say that $f$ is a quasi-isomorphism if $H^*f$ is an isomorphism, which is equivalent to saying that $\tt{cone}(f)$ is exact.

            A cochain complex $N^\bullet$ is said to be contractible if $id_N$ is null-homotopic. Then it follows for any other cochain complexes $M^\bullet$ that $H^0\tt{Hom}_\mathbb{K}^\bullet(M^\bullet, N^\bullet) \simeq 0$.

            We define the shift functor $\argument [n] : \tt{Mod}_\mathbb{K}^\bullet \rightarrow \tt{Mod}_\mathbb{K}^\bullet$ is defined on cochains $M^\bullet$ as
            \begin{align*}
                (M^\bullet, d_M)[n] = (M^\bullet[n], (-1)^nd_M)\tt{.}
            \end{align*}
            With this definition, shifting is naturally isomorphic to tensoring. That is if $\mathbb{K}[n]$ denotes the field concentrated in dimension $-n$, then
            \begin{align*}
                \mathbb{K}[n]\otimes_\mathbb{K} M^\bullet \simeq M^\bullet[n]\simeq M^\bullet\otimes_\mathbb{K}\mathbb{K}[n]\tt{.}
            \end{align*}
            One may see how the differential gets its sign by writing out the total tensor product. We usually call $\argument[1]$ shifting, desuspension or looping; and $\argument[-1]$ for inverse-shifting, suspension or delooping.

            We are now ready to talk about algebras in $\tt{Mod}_\mathbb{K}^\bullet$.

            \begin{definition}[Differential graded algebra]
                $(A^\bullet,d_A)$ is a differential graded algebra if:
                \begin{itemize}
                    \item $A^\bullet$ is a differential algebra in $\tt{Mod}_\mathbb{K}^\bullet$,
                    \item the structure morphisms $(\cdot_A)$ and $1_A$ are chain maps,
                    \item and the derivation and differential coincide.
                \end{itemize}
            \end{definition}

            \begin{example}[The unit]
                $\mathbb{K} = (\mathbb{K},0)$ is a differential graded algebra in the trivial way. It is concentrated in degree $0$, and the differential is the trivial derivation.
            \end{example}
            \begin{example}[De Rham complex]
                Given a manifold $M$, the exterior algebra $\Omega M$ is a differential graded algebra. See Tu \cite{Tu07} for a thorough explanation.
            \end{example}

            In the case of differential graded algebras, we can naively define homotopies like homotopies for cochain complexes. Given morphisms $f, g : A^\bullet \rightarrow B^\bullet$, a homotopy between $f$ and $g$ is a morphism $h : A^\bullet \rightarrow B^\bullet$ of degree $-1$ such that $\partial h = f - g$. We know that such morphisms allow us to say that these morphisms are isomorphic in homotopy on the underlying cochain complexes. However, the ring structure is no longer required to be preserved. We amend this problem by $(f,g)$-derivations.

            \begin{definition}
                Suppose there are morphisms $f,g: A^\bullet \rightarrow B^\bullet$. We say that $h : A \rightarrow B$ is an $(f,g)$-derivation if $|h| = -1$ and $h \circ (\cdot) = (\cdot)\circ (f \otimes h + g \otimes h)$.
            \end{definition}

            We will say that the morphisms $f$ and $g$ are homotopic whenever there is an $(f,g)$-derivation $h$ such that $\partial h = f - g$.

            Given a differential graded, or dg-algebra $A^\bullet$, we may form the category of left $A^\bullet$-modules, $\tt{Mod}_A$.
            \begin{definition}
                $M^\bullet$ is a left $A^\bullet$-module if
                \begin{itemize}
                    \item $M^\bullet$ is a cochain complex,
                    \item there is a chain map $\mu_M : A^\bullet\otimes_\mathbb{K} M^\bullet \rightarrow M^\bullet$ satisfying associativity and unitality,
                    \item $d_M$ is an $A^\bullet$-derivation.
                \end{itemize}
            \end{definition}
            The hom-objects are defined analogously. We use $\tt{Hom}_{A^\bullet}^\bullet$ to denote the $\mathbb{K}$-linear cochain complex.

            With this definition, the categories $\tt{Mod}_\mathbb{K}$ where $\mathbb{K}$ is considered as a cochain complex, and the category $\tt{Mod}_\mathbb{K}^\bullet$ is the same category because a chain complex already satisfies the first two bullet points by definition. Being a $\mathbb{K}^\bullet$-derivation is a trivial condition, so every map meets this.

            We also have the dual definition to obtain dg-coalgebras, $(f,g)$-coderivations and their comodules.
            
            \begin{definition}
                $C^{\bullet}$ is a differential graded coalgebra if
                \begin{itemize}
                    \item $C^\bullet$ is a differential coalgebra in $\tt{Mod}_\mathbb{K}^\bullet$,
                    \item the structure morphisms $\Delta_C$ and $\varepsilon_C$ are chain maps,
                    \item the coderivation and differential coincides
                \end{itemize}
            \end{definition}

            \begin{definition}
                Suppose that $f,g: C^\bullet \rightarrow D^\bullet$ are morphisms of dg-coalgebras. We say that $h$ is an $(f,g)$-coderivation if $\Delta h = (f \otimes h + g \otimes h) \Delta$.
            \end{definition}

            Two morphisms $f,g: C^\bullet \rightarrow D^\bullet$ are said to be homotopic if there is an $(f,g)$-coderivation such that $\partial h = f - g$.

            \begin{definition}
                $N^\bullet$ is a left $C^\bullet$-comodule if
                \begin{itemize}
                    \item $N^\bullet$ is a cochain complex,
                    \item there is a chain map $\omega_C : N^\bullet \rightarrow C^\bullet \otimes_\mathbb{K} N^\bullet$ satisfying coassociativity and counitality,
                    \item $d_N$ is a $C^\bullet$-coderivation.
                \end{itemize}
            \end{definition}

            By these definitions, we may extend proposition \ref{prop: free-derivation} to the category of cochain complexes.

            \begin{corollary}\label{cor: dg-free-derivation}
                Let $A^\bullet$ be a differential graded algebra and $M^\bullet$ a cochain complex. A homogenous $\mathbb{K}$-linear morphism $f:M\rightarrow A\otimes_{\mathbb{K}} M$ uniquely determines a derivation \\ $d_f:A\otimes M\rightarrow A\otimes M$ of same degree, i.e. there is an isomorphism \\ $\tt{Hom}_{\mathbb{K}}^*(M^\bullet,A^\bullet\otimes_{\mathbb{K}}M^\bullet)\simeq \tt{Der}^*(A^\bullet\otimes_{\mathbb{K}}M^\bullet)$. Moreover, $d_f$ is given as $(\nabla_{A^\bullet}\otimes id_M)\circ (id_A\otimes f) + d_{A\otimes M}$.

                Dually, if $C^\bullet$ is a differential graded coalgebra and $N^\bullet$ is a cochain complex, then a homogenous $\mathbb{K}$-linear morphism $g:C^\bullet\otimes N^\bullet\rightarrow N^\bullet$ uniquely determines a coderivation \\ $d_g:C^\bullet\otimes_{\mathbb{K}}N^\bullet\rightarrow C^\bullet\otimes_{\mathbb{K}}N^\bullet$. There is an isomorphism $\tt{Hom}_{\mathbb{K}}^*(C^\bullet\otimes_{\mathbb{K}}N^\bullet,N^\bullet)\simeq \tt{Coder}^*(C^\bullet\otimes_{\mathbb{K}}N^\bullet)$, and $d_g$ is given as $(id_C\otimes g)\circ (\Delta_{C^\bullet}\otimes id_N) + d_{C\otimes N}$.
            \end{corollary}

            \begin{proof}
                The same electronic circuits as in the proof of proposition \ref{prop: free-derivation} suffice to prove this statement.
            \end{proof}

            Notably, this statement carries an additional two duals. We have the same result when considering right modules, and the same proof applies in these cases.
    \section{Cobar-Bar Adjunction}
    \subsection{Convolution Algebras}

            Given a coalgebra $C$ and an algebra $A$, we obtain a particular product on the hom-object $\tt{Hom}_\mathbb{K}(C, A)$ by twisting the comultiplication and multiplication together. The convolution algebra forms the backbone of our proof of the cobar-bar adjunction.

            Let $C$ be a coalgebra and $A$ an algebra, then if $f,g:C\rightarrow A$ is a $\mathbb{K}$-linear morphism we may define $f\star g = (\cdot_A)(f\otimes g)\Delta_C$. This operation is called $\star$ convolution.

            \begin{center}
                \begin{tikzpicture}[line cap=round,line join=round,>=triangle 45,x=1cm,y=1cm, thick, op/.style={circle, draw, scale=0.75}, diff/.style={triangle, draw, scale=0.75}, scale=0.7]

                    \node at (-2.5, 0) {$f\star g$};

                    \node at (-1.5, 0) {$=$};

                    \node[op, scale=0.75] (a) at (-0.5, 0) {$f$};
                    \node[op, scale=0.75] (b) at (0.5, 0) {$g$};
                    
                    \draw [line width=1pt] (0, 1.25) -- (0, 1) -- (-0.5, 0.5) -- (a) -- (-0.5, -0.5) -- (0, -1) -- (0, -1.25);
                    \draw [line width=1pt] (0, 1) -- (0.5, 0.5) -- (b) -- (0.5, -0.5) -- (0, -1);

                \end{tikzpicture}
            \end{center}

            \begin{proposition}[Convolution algebra]
                The $\mathbb{K}$-module $\tt{Hom}_{\mathbb{K}}(C,A)$ is an associative algebra when equipped with convolution $\star:\tt{Hom}_{\mathbb{K}}(C,A)\rightarrow \tt{Hom}_{\mathbb{K}}(C,A)$. The unit is given by $1 \mapsto \upsilon_A\circ\varepsilon_C$.
            \end{proposition}

            \begin{proof}
                This proposition follows from (co)associativity and (co)unitality of (C) A.

                \begin{center}
                    \begin{tikzpicture}[line cap=round,line join=round,>=triangle 45,x=1cm,y=1cm, thick, op/.style={circle, draw, scale=0.75}, scale=0.7]
                        % \node at (-3, 0) {Associativity};
                        
                        \node at (-3,0) {$(f\star g) \star h$};

                        \node at (-1, 0) {$=$};

                        \node[op, scale = 0.75] (f1) at (0,0) {f};
                        \node[op, scale = 0.75] (g1) at (1,0) {g};
                        \node[op, scale = 0.75] (h1) at (2,0) {h};

                        \draw (1, 1.5) -- (1,1.25) -- (0.5,1) -- (0.5, 0.75) -- (0,0.5) -- (f1) -- (0,-0.5) -- (0.5, -0.75) -- (0.5,-1) -- (1,-1.25) -- (1, -1.5);
                        \draw (0.5, 0.75) -- (1,0.5) -- (g1) -- (1, -0.5) -- (0.5, -0.75);
                        \draw (1,1.25) -- (2,0.75) -- (h1) -- (2, -0.75) -- (1, -1.25);

                        \node at (3,0) {$=$};

                        \node[op, scale = 0.75] (f2) at (4,0) {f};
                        \node[op, scale = 0.75] (g2) at (5,0) {g};
                        \node[op, scale = 0.75] (h2) at (6,0) {h};

                        \draw (5, 1.5) -- (5,1.25) -- (4, 0.75) -- (f2) -- (4,-0.5) -- (4.5, -0.75) -- (4.5,-1) -- (5,-1.25) -- (5, -1.5);
                        \draw (5, 1.25) -- (5.5, 1) -- (5.5, 0.75) -- (5,0.5) -- (g2) -- (5, -0.5) -- (4.5, -0.75);
                        \draw (5.5, 0.75) -- (6, 0.5) -- (h2) -- (6, -0.75) -- (5, -1.25);
                        
                        \node at (7,0) {$=$};

                        \node[op, scale = 0.75] (f3) at (8,0) {f};
                        \node[op, scale = 0.75] (g3) at (9,0) {g};
                        \node[op, scale = 0.75] (h3) at (10,0) {h};

                        \draw (9, 1.5) -- (9,1.25) -- (8, 0.75) -- (f3) -- (8,-0.75) -- (9,-1.25) -- (9, -1.5);
                        \draw (9, 1.25) -- (9.5, 1) -- (9.5, 0.75) -- (9,0.5) -- (g3) -- (9, -0.5) -- (9.5, -0.75);
                        \draw (9.5, 0.75) -- (10, 0.5) -- (h3) -- (10, -0.5) -- (9.5, -0.75) -- (9.5, -1) -- (9,-1.25);

                        \node at (11, 0) {$=$};

                        \node at (13, 0) {$f\star (g\star h)$};
                    \end{tikzpicture}

                    \begin{tikzpicture}[line cap=round,line join=round,>=triangle 45,x=1cm,y=1cm, thick, op/.style={circle, draw, scale=0.75}, scale=0.7]
                        \node at (-6, 0) {$(\upsilon_A\circ\varepsilon_C)\star f$};

                        \node at (-4, 0) {$=$};

                        \node[op, scale = 0.75] (f3) at (-2,0) {f};
                        \node[op, scale = 0.5] (c') at (-3, 0.25) {};
                        \node[op, scale = 0.5] (u') at (-3, -0.25) {};

                        \draw (-2.5, 1.25) -- (-2.5, 1) -- (-2, 0.75) -- (f3) -- (-2,-0.75) -- (-2.5,-1) -- (-2.5,-1.25);
                        \draw (-2.5, 1) -- (-3, 0.75) -- (c');
                        \draw (u') -- (-3, -0.75) -- (-2.5, -1);

                        \node at (-1,0) {$=$};

                        \node[op, scale = 0.75] (f1) at (0,0) {f};

                        \draw (0,1) -- (f1) -- (0,-1);

                        \node at (1,0) {$=$};

                        \node[op, scale = 0.75] (f2) at (2,0) {f};
                        \node[op, scale = 0.5] (c) at (3, 0.25) {};
                        \node[op, scale = 0.5] (u) at (3, -0.25) {};

                        \draw (2.5, 1.25) -- (2.5, 1) -- (2, 0.75) -- (f2) -- (2,-0.75) -- (2.5,-1) -- (2.5,-1.25);
                        \draw (2.5, 1) -- (3, 0.75) -- (c);
                        \draw (u) -- (3, -0.75) -- (2.5, -1);

                        \node at (4,0) {$=$};

                        \node at (6, 0) {$f\star (\upsilon_A\circ\varepsilon_C)$};
                    \end{tikzpicture}
                \end{center}
            \end{proof}

            This proof does not rely on braiding and lifts to any closed symmetric monoidal category.

            Any algebra $A$ may be considered a differential algebra together with the trivial derivation. That is, $(A, 0)$ is a differential algebra. For such structures, the set of $A$-derivations is precisely the set of $A$-linear morphisms. Dually, we can consider every coalgebra $C$ as a differential coalgebra. 
            
            We may apply a trivialization of proposition \ref{prop: free-derivation} to $A$ and $C$ considered as differential (co)algebra. When we look at the module $C\otimes_\mathbb{K}A$, it is free over $A$ on the right and cofree over $C$ on the left. Consider a morphism $\alpha: C \rightarrow A$, and then there are two ways to extend $\alpha$ to obtain a (co)derivation. Precomposing with $C$'s comultiplication gives us a morphism from $C$ to the free $A$-module $C\otimes_\mathbb{K} A$,

            \begin{align*}
                (id_C\otimes \alpha) \circ \Delta_C : C \rightarrow C \otimes_\mathbb{K} A\tt{.}
            \end{align*}

            Postcomposing with the multiplication of $A$ gives us a morphism from to the cofree $C$-comodule $C\otimes_\mathbb{K}A$ to $A$,
            \begin{align*}
                (\cdot_A) \circ (\alpha\otimes id_A) : C \otimes_\mathbb{K} A \rightarrow A\tt{.}
            \end{align*}

            When we apply proposition \ref{prop: free-derivation} to both morphisms, it yields the same map. Therefore it is both a derivation and a coderivation, as
            \begin{align*}
                d_\alpha^r = (id_C\otimes (\cdot_A)) \circ (id_C \otimes \alpha \otimes id_A) \circ (\Delta_C\otimes id_A)
            \end{align*}

            \begin{center}
                \begin{tikzpicture}[line cap=round,line join=round,>=triangle 45,x=1cm,y=1cm, thick, op/.style={circle, draw, scale=0.75}, scale=0.7]

                    \node at (-3, 0) {$d_\alpha^r$};

                    \node at (-2, 0) {$=$};

                    \node[op, scale=0.75] (a) at (0,0) {$\alpha$};

                    \draw [line width=1pt] (-0.5, 1.25) -- (-0.5, 1) -- (0, 0.5) -- (a) -- (0, -0.5) -- (0.5, -1) -- (0.5, -1.25);
                    \draw [line width=1pt] (-0.5, 1) -- (-1, 0.5) -- (-1, -1.25);
                    \draw [line width=1pt] (1, 1.25) -- (1, -0.5) -- (0.5, -1);
                    
                \end{tikzpicture}
            \end{center}

            This coderivation will be very important for the rest of this thesis. In the ungraded case, we may transform it into a ring homomorphism.

            \begin{proposition}\label{prop: convolution to endomorphism}
                $d_{\argument}^r:\tt{Hom}_\mathbb{K}(C,A)\rightarrow \tt{End}(C\otimes_\mathbb{K}A)$ is a morphism of algebras. Moreover, if $\alpha\star \alpha = 0$, then $(d_\alpha^r)^2 = 0$.
            \end{proposition}

            \begin{proof}
                The proof follows from (co)associativity and (co)unitality.
                \begin{center}
                    \begin{tikzpicture}[line cap=round,line join=round,>=triangle 45,x=1cm,y=1cm, thick, op/.style={circle, draw, scale=0.75}, scale=0.7]
    
                        \node at (-3, 0) {$d_{\alpha\star \beta}^r$};
    
                        \node at (-2, 0) {$=$};
    
                        \node[op, scale=0.75] (a) at (-0.5, 0) {$\alpha$};
                        \node[op, scale=0.75] (b) at (0.5, 0) {$\beta$};
                        
                        \draw [line width=1pt] (-0.5, 2) -- (-0.5, 1.75) -- (0, 1.25) -- (0, 1) -- (-0.5, 0.5) -- (a) -- (-0.5, -0.5) -- (0, -1) -- (0, -1.25) -- (0.5, -1.75) -- (0.5, -2);
                        \draw [line width=1pt] (0, 1) -- (0.5, 0.5) -- (b) -- (0.5, -0.5) -- (0, -1);
                        \draw [line width=1pt] (-0.5, 1.75) -- (-1, 1.25) -- (-1, -2);
                        \draw (1, 2) -- (1, -1.25) -- (0.5, -1.75);
                        
                        \node at (2, 0) {$=$};

                        \node[op, scale=0.75] (a') at (4, 0) {$\alpha$};
                        \node[op, scale=0.75] (b') at (5, 0) {$\beta$};

                        \draw [line width=1pt] (4, 2) -- (4, 1.75) -- (3.5, 1.25) -- (3.5, 1) -- (3, 0.5) -- (3, -2);
                        \draw [line width=1pt] (3.5, 1) -- (4, 0.5) -- (a') -- (4, -0.75) -- (5, -1.75);
                        \draw [line width=1pt] (4, 1.75) -- (5, 0.75) -- (b') -- (5, -0.5) -- (5.5, -1);
                        \draw [line width=1pt] (6, 2) -- (6, -0.5) -- (5.5, -1) -- (5.5, -1.25) -- (5, -1.75) -- (5, -2);

                        \node at (7, 0) {$=$};

                        \node at (8.5, 0) {$d_\alpha^r \circ d_\beta^r$};

                    \end{tikzpicture}
                \end{center} 

                \begin{center}
                    \begin{tikzpicture}[line cap=round,line join=round,>=triangle 45,x=1cm,y=1cm, thick, op/.style={circle, draw, scale=0.75}, scale=0.7]

                        \node at (-3.5, 0) {$d_{\upsilon_A \circ\varepsilon_C}^r$};

                        \node at (-2, 0) {$=$};

                        \node[op, scale=0.5] (a) at (0, 0.25) {};
                        \node[op, scale=0.5] (b) at (0, -0.25) {};

                        \draw [line width=1pt] (-0.5, 1.25) -- (-0.5, 1) -- (0, 0.5) -- (a);
                        \draw [line width=1pt] (b) -- (0, -0.5) -- (0.5, -1) -- (0.5, -1.25);
                        \draw [line width=1pt] (-0.5, 1) -- (-1, 0.5) -- (-1, -1.25);
                        \draw [line width=1pt] (1, 1.25) -- (1, -0.5) -- (0.5, -1);
                        
                        \node at (2, 0) {$=$};

                        \draw [line width=1pt] (3, 1.25) -- (3, -1.25);
                        \draw [line width=1pt] (4, 1.25) -- (4, -1.25);

                        \node at (5, 0) {$=$};
                        \node at (6.5, 0) {$id_{C \otimes_\mathbb{K} A}$};

                    \end{tikzpicture}
                \end{center}
            \end{proof}

            This proof relies on braiding, so we will encounter problems when we try to lift this proposition to the graded case. We may observe that the above has no problem lifting, and this is because the $\beta$ has no morphisms of odd degrees to the right or over itself. However, the dual will introduce some signs when lifted.

            \begin{corollary}\label{cor: graded-conv-to-end}
                Suppose that $C$ and $A$ are differential graded (co)algebras. $d_{\argument}^r : \tt{Hom}_\mathbb{K}^*(C,A) \rightarrow \tt{End}^*(C\otimes_\mathbb{K} A)$ extends to a homogenous ring morphism of degree $0$.
            \end{corollary}

            Suppose that $C$ and $A$ are differential graded (co)algebras. We want to expect that the differential $\partial$ makes $(\tt{Hom}_\mathbb{K}^*(C,A), \star)$ into a dg-algebra.
            
            \begin{proposition}
                The convolution algebra $(\tt{Hom}_\mathbb{K}^*(C,A),\star)$ is a dg-algebra with differential $\partial$.
            \end{proposition}

            \begin{proof}
                We know that $(\tt{Hom}_\mathbb{K}^*(C,A),\star)$ is a convolution algebra and that $(\tt{Hom}_\mathbb{K}^*(C,A),\partial)$ is a chain complex. It remains to verify that the differential is compatible with the multiplication, i.e., $\partial(f\star g) = \partial{f}\star g + (-1)^{|f|}f\star\partial{g}$.

                Let $f,g\in \tt{Hom}_\mathbb{K}^*(C,A)$ be two homogenous morphisms. The key property to arrive at the result is that the differential in a dg-(co)algebra is a (co)derivation. We denote the degree of $f\star g$ as $|f\star g| = |f| + |g| = d$. Then

                \begin{center}
                    \begin{tikzpicture}[line cap=round,line join=round,>=triangle 45,x=1cm,y=1cm, thick, op/.style={circle, draw, scale=0.75}, diff/.style={regular polygon, draw, regular polygon sides = 3, scale=0.75, rotate = 180}, scale=0.7]

                        \node at (0,0) {$\partial (f\star g) =$};

                        \node at (1.35,0) {$\partial$};
                        \node[op, scale = 0.75] (f1) at (2, 0) {$f$};
                        \node[op, scale = 0.75] (g1) at (3, 0) {$g$};

                        \draw [line width = 1pt] (2.5, 1.25) -- (2.5, 1) -- (2, 0.5) -- (f1) -- (2, -0.5) -- (2.5, -1) -- (2.5, -1.25);
                        \draw [line width = 1pt] (2.5, 1) -- (3, 0.5) -- (g1) -- (3, -0.5) -- (2.5, -1);


                        \node at (4, 0) {$=$};


                        \node[diff, scale=0.75] (d1) at (5.5, -1) {};
                        \node[op, scale=0.75] (f2) at (5, 0.5) {$f$};
                        \node[op, scale=0.75] (g2) at (6, 0.5) {$g$};

                        \draw [line width = 1pt] (5.5, 1.75) -- (5.5, 1.5) -- (5, 1) -- (f2) -- (5, 0) -- (5.5, -0.5) -- (d1) -- (5.5, -1.5);
                        \draw [line width = 1pt] (5.5, 1.5) -- (6, 1) -- (g2) -- (6, 0) -- (5.5, -0.5);

                        \node at (7.5, 0) {$-(-1)^{d}$};

                        \node[op, scale = 0.75] (f3) at (9, -0.5) {$f$};
                        \node[op, scale = 0.75] (g3) at (10, -0.5) {$g$};
                        \node[diff, scale = 0.75] (d2) at (9.5, 1) {};

                        \draw [line width = 1pt] (9.5, 1.5) -- (d2) -- (9.5, 0.5) -- (9, 0) -- (f3) -- (9, -1) -- (9.5, -1.5) -- (9.5, -1.75);
                        \draw [line width = 1pt] (9.5, 0.5) -- (10, 0) -- (g3) -- (10, -1) -- (9.5, -1.5);

                    \end{tikzpicture}
                \end{center}

                \begin{center}
                    \begin{tikzpicture}[line cap=round,line join=round,>=triangle 45,x=1cm,y=1cm, thick, op/.style={circle, draw, scale=0.75}, diff/.style={regular polygon, draw, regular polygon sides = 3, scale=0.75, rotate = 180}, scale=0.7]

                        \node at (0,0) {$=$};

                        \node[op, scale = 0.75] (f1) at (1, 0.5) {$f$};
                        \node[op, scale = 0.75] (g1) at (2, 0) {$g$};
                        \node[diff, scale = 0.75] (d1) at (1, -0.25) {};

                        \draw [line width = 1pt] (1.5, 1.75) -- (1.5, 1.5) -- (1, 1) -- (f1) -- (d1) -- (1, -0.75) -- (1.5, -1.25) -- (1.5, -1.5); 
                        \draw [line width = 1pt] (1.5, 1.5) -- (2, 1) -- (g1) -- (2, -0.75) -- (1.5, -1.25);

                        \node at (3.5, 0) {$+(-1)^{|f|}$};

                        \node[op, scale = 0.75] (f2) at (5, 0) {$f$};
                        \node[op, scale = 0.75] (g2) at (6, 0.5) {$g$};
                        \node[diff, scale = 0.75] (d2) at (6, -0.25) {};

                        \draw [line width = 1pt] (5.5, 1.75) -- (5.5, 1.5) -- (5, 1) -- (f2) -- (5, -0.75) -- (5.5, -1.25) -- (5.5, -1.5);
                        \draw [line width = 1pt] (5.5, 1.5) -- (6, 1) -- (g2) -- (d2) -- (6, -0.75) -- (5.5, -1.25) -- (5.5, -1.5);

                        \node at (8.25, 0) {$-(-1)^{d}((-1)^{|g|}$};

                        \node[op, scale = 0.75] (f3) at (10.5, -0.25) {$f$};
                        \node[op, scale = 0.75] (g3) at (11.5, 0) {$g$};
                        \node[diff, scale = 0.75] (d3) at (10.5, 0.5) {};

                        \draw [line width = 1pt] (11, 1.75) -- (11, 1.5) -- (10.5, 1) -- (d3) -- (f3) -- (10.5, -0.75) -- (11, -1.25) -- (11, -1.5);
                        \draw [line width = 1pt] (11, 1.5) -- (11.5, 1) -- (g3) -- (11.5, -0.75) -- (11, -1.25);

                        \node at (12.25, 0) {$+$};

                        \node[op, scale = 0.75] (f4) at (13, 0) {$f$};
                        \node[op, scale = 0.75] (g4) at (14, -0.25) {$g$};
                        \node[diff, scale = 0.75] (d4) at (14, 0.5) {};

                        \draw [line width = 1pt] (13.5, 1.75) -- (13.5, 1.5) -- (13, 1) -- (f4) -- (13, -0.75) -- (13.5, -1.25) -- (13.5, -1.5);
                        \draw [line width = 1pt] (13.5, 1.5) -- (14, 1) -- (d4) -- (g4) -- (14, -0.75) -- (13.5, -1.25);
                        \node at (14.5, 0) {$)$};
                    \end{tikzpicture}
                \end{center}

                \begin{center}
                    \begin{tikzpicture}[line cap=round,line join=round,>=triangle 45,x=1cm,y=1cm, thick, op/.style={circle, draw, scale=0.75}, diff/.style={regular polygon, draw, regular polygon sides = 3, scale=0.75, rotate = 180}, scale=0.7]

                        \node at (0,0) {$=$};

                        \node[op, scale = 0.75] (f1) at (1, 0.5) {$f$};
                        \node[op, scale = 0.75] (g1) at (2, 0) {$g$};
                        \node[diff, scale = 0.75] (d1) at (1, -0.25) {};

                        \draw [line width = 1pt] (1.5, 1.75) -- (1.5, 1.5) -- (1, 1) -- (f1) -- (d1) -- (1, -0.75) -- (1.5, -1.25) -- (1.5, -1.5); 
                        \draw [line width = 1pt] (1.5, 1.5) -- (2, 1) -- (g1) -- (2, -0.75) -- (1.5, -1.25);
                        
                        \node at (3.5, 0) {$-(-1)^{|f|}$};

                        \node[op, scale = 0.75] (f3) at (5, -0.25) {$f$};
                        \node[op, scale = 0.75] (g3) at (6, 0) {$g$};
                        \node[diff, scale = 0.75] (d3) at (5, 0.5) {};

                        \draw [line width = 1pt] (5.5, 1.75) -- (5.5, 1.5) -- (5, 1) -- (d3) -- (f3) -- (5, -0.75) -- (5.5, -1.25) -- (5.5, -1.5);
                        \draw [line width = 1pt] (5.5, 1.5) -- (6, 1) -- (g3) -- (6, -0.75) -- (5.5, -1.25) -- (5.5, -1.5);

                        \node at (7.5, 0) {$+(-1)^{|f|}($};

                        \node[op, scale = 0.75] (f2) at (9, 0) {$f$};
                        \node[op, scale = 0.75] (g2) at (10, 0.5) {$g$};
                        \node[diff, scale = 0.75] (d2) at (10, -0.25) {};

                        \draw [line width = 1pt] (9.5, 1.75) -- (9.5, 1.5) -- (9, 1) -- (f2) -- (9, -0.75) -- (9.5, -1.25) -- (9.5, -1.5);
                        \draw [line width = 1pt] (9.5, 1.5) -- (10, 1) -- (g2) -- (d2) -- (10, -0.75) -- (9.5, -1.25) -- (9.5, -1.5);


                        \node at (11.5, 0) {$-(-1)^{|g|}$};

                        \node[op, scale = 0.75] (f4) at (13, 0) {$f$};
                        \node[op, scale = 0.75] (g4) at (14, -0.25) {$g$};
                        \node[diff, scale = 0.75] (d4) at (14, 0.5) {};

                        \draw [line width = 1pt] (13.5, 1.75) -- (13.5, 1.5) -- (13, 1) -- (f4) -- (13, -0.75) -- (13.5, -1.25) -- (13.5, -1.5);
                        \draw [line width = 1pt] (13.5, 1.5) -- (14, 1) -- (d4) -- (g4) -- (14, -0.75) -- (13.5, -1.25);
                        \node at (14.5, 0) {$)$};
                    \end{tikzpicture}
                \end{center}

                \begin{center}
                    \begin{tikzpicture}[line cap=round,line join=round,>=triangle 45,x=1cm,y=1cm, thick, op/.style={circle, draw, scale=0.75}, diff/.style={regular polygon, draw, regular polygon sides = 3, scale=0.75, rotate = 180}, scale=0.7]

                        \node at (0,0) {$=$};

                        \node[op, scale = 0.75] (df) at (1, 0) {$\partial f$};
                        \node[op, scale = 0.75] (g) at (2, 0) {$g$};

                        \draw [line width = 1pt] (1.5, 1.25) -- (1.5, 1) -- (1, 0.5) -- (df) -- (1, -0.5) -- (1.5, -1) -- (1.5, -1.25);
                        \draw [line width = 1pt] (1.5, 1) -- (2, 0.5) -- (g) -- (2, -0.5) -- (1.5, -1);

                        \node at (3.5, 0) {$+(-1)^{|f|}$};

                        \node[op, scale = 0.75] (f) at (5, 0) {$f$};
                        \node[op, scale = 0.75] (dg) at (6, 0) {$\partial g$};
                        
                        \draw [line width = 1pt] (5.5, 1.25) -- (5.5, 1) -- (5, 0.5) -- (f) -- (5, -0.5) -- (5.5, -1) -- (5.5, -1.25);
                        \draw [line width = 1pt] (5.5, 1) -- (6, 0.5) -- (dg) -- (6, -0.5) -- (5.5, -1); 

                        \node at (7, 0) {$=$};

                        \node at (10.5, 0) {$\partial (f) \star g + (-1) ^{|f|}f \star \partial (g)$};

                    \end{tikzpicture}
                \end{center}
            \end{proof}

            \begin{proposition}\label{prop: dr-dg-hom}
                The morphism $d_{\argument}^r : \tt{Hom}_\mathbb{K}^\bullet(C,A) \rightarrow \tt{End}^\bullet(C\otimes_\mathbb{K}A)$ is a chain map. 
            \end{proposition}

            \begin{proof}
                We already know from Corollary \ref{cor: graded-conv-to-end} that $d_{\argument}^r$ is a homogenous ring map. It remains to see that it commutes with the differentials. That is, $\partial d_\alpha^r = d_{\partial\alpha}^r$. We write out each summand in $\partial d_\alpha^r$,

                \begin{center}
                    \begin{tikzpicture}[line cap=round,line join=round,>=triangle 45,x=1cm,y=1cm, thick, op/.style={circle, draw, scale=0.75}, diff/.style={regular polygon, draw, regular polygon sides = 3, scale=0.75, rotate = 180}, scale=0.7]

                        \node at (0,0) {$d_{C\otimes_\mathbb{K} A}\circ d_\alpha^r=$};
                        
                        \node[diff, scale = 0.75] (dc1) at (2.25,0) {};
                        \node[op, scale = 0.75] (f1) at (3.25, 0) {$\alpha$};

                        \draw [line width = 1pt] (2.75, 1.75) -- (2.75, 1.5) -- (2.25, 1) -- (dc1) -- (2.25, -1.75);
                        \draw [line width = 1pt] (2.75, 1.5) -- (3.25, 1) -- (f1) -- (3.25, -1) -- (3.75, -1.5) -- (3.75, -1.75);
                        \draw [line width = 1pt] (4.25, 1.75) -- (4.25, -1) -- (3.75, -1.5);

                        \node at (4.75, 0) {$+$};

                        \node[op, scale = 0.75] (f2) at (6.25, 0.5) {$\alpha$};
                        \node[diff, scale = 0.75] (da1) at (6.25, -0.5) {};

                        \draw [line width = 1pt] (5.75, 1.75) -- (5.75, 1.5) -- (5.25, 1) -- (5.25, -1.75);
                        \draw [line width = 1pt] (5.75, 1.5) -- (6.25, 1) -- (f2) -- (da1) -- (6.25, -1) -- (6.75, -1.5) -- (6.75, -1.75);
                        \draw [line width = 1pt] (7.25, 1.75) -- (7.25, -1) -- (6.75, -1.5);

                        \node at (8.75, 0) {$+(-1)^{|\alpha|}$};

                        \node[op, scale = 0.75] (f3) at (11, 0) {$\alpha$};
                        \node[diff, scale = 0.75] (da2) at (12, 0) {};

                        \draw [line width = 1pt] (10.5, 1.75) -- (10.5, 1.5) -- (10, 1) -- (10, -1.75);
                        \draw [line width = 1pt] (10.5, 1.5) -- (11, 1) -- (f3) -- (11, -1) -- (11.5, -1.5) -- (11.5, -1.75);
                        \draw [line width = 1pt] (12, 1.75) -- (da2) -- (12, -1) -- (11.5, -1.5);

                    \end{tikzpicture}
                \end{center}

                \begin{center}
                    \begin{tikzpicture}[line cap=round,line join=round,>=triangle 45,x=1cm,y=1cm, thick, op/.style={circle, draw, scale=0.75}, diff/.style={regular polygon, draw, regular polygon sides = 3, scale=0.75, rotate = 180}, scale=0.7]

                        \node at (-0.75,0) {$d_\alpha^r\circ d_{C\otimes_\mathbb{K} A}=(-1)^{|\alpha|}$};
                        
                        \node[diff, scale = 0.75] (dc1) at (2.25,0) {};
                        \node[op, scale = 0.75] (f1) at (3.25, 0) {$\alpha$};

                        \draw [line width = 1pt] (2.75, 1.75) -- (2.75, 1.5) -- (2.25, 1) -- (dc1) -- (2.25, -1.75);
                        \draw [line width = 1pt] (2.75, 1.5) -- (3.25, 1) -- (f1) -- (3.25, -1) -- (3.75, -1.5) -- (3.75, -1.75);
                        \draw [line width = 1pt] (4.25, 1.75) -- (4.25, -1) -- (3.75, -1.5);

                        \node at (4.75, 0) {$+$};

                        \node[op, scale = 0.75] (f2) at (6.25, -0.5) {$\alpha$};
                        \node[diff, scale = 0.75] (da1) at (6.25, 0.5) {};

                        \draw [line width = 1pt] (5.75, 1.75) -- (5.75, 1.5) -- (5.25, 1) -- (5.25, -1.75);
                        \draw [line width = 1pt] (5.75, 1.5) -- (6.25, 1) -- (da1) -- (f2) -- (6.25, -1) -- (6.75, -1.5) -- (6.75, -1.75);
                        \draw [line width = 1pt] (7.25, 1.75) -- (7.25, -1) -- (6.75, -1.5);

                        \node at (7.75, 0) {$+$};

                        \node[op, scale = 0.75] (f3) at (9.25, 0) {$\alpha$};
                        \node[diff, scale = 0.75] (da2) at (10.25, 0) {};

                        \draw [line width = 1pt] (8.75, 1.75) -- (8.75, 1.5) -- (8.25, 1) -- (8.25, -1.75);
                        \draw [line width = 1pt] (8.75, 1.5) -- (9.25, 1) -- (f3) -- (9.25, -1) -- (9.75, -1.5) -- (9.75, -1.75);
                        \draw [line width = 1pt] (10.25, 1.75) -- (da2) -- (10.25, -1) -- (9.75, -1.5);

                    \end{tikzpicture}
                \end{center}

                When $\alpha$ is of even degree, $\partial d_\alpha^r = d_{C\otimes_\mathbb{K}A}\circ d^r_\alpha - d^r_\alpha\circ d_{C\otimes_\mathbb{K}A}$. The outer summands cancel, and we have
                \begin{align*}
                    \partial d_\alpha^r = d_{d_A \alpha - \alpha d_C} = d_{\partial \alpha}\tt{.}
                \end{align*}

                When $\alpha$ is of odd degree, $\partial d_\alpha^r = d_{C\otimes_\mathbb{K}A}\circ d^r_\alpha + d^r_\alpha\circ d_{C\otimes_\mathbb{K}A}$. The outer summands cancel, and we have
                \begin{align*}
                    \partial d_\alpha^r = d_{d_A \alpha + \alpha d_C} = d_{\partial \alpha}\tt{.}
                \end{align*}
            \end{proof}

    \subsection{Twisting Morphisms}

            In this section, we will define twisting morphisms from coalgebras to algebras. They are important as the bifunctor $\tt{Tw}(C, A)$ is represented in both arguments. To understand the elements of $\tt{Tw}$, we start this section by reviewing the Maurer-Cartan equation.

            Suppose that $C$ is a coaugmented dg-coalgebra and $A$ is an augmented dg-algebra. We say that a morphism $\alpha\in \tt{Hom}_\mathbb{K}^*(C,A)$ is twisting if it is of degree $1$, is $0$ on the coaugmentation of $C$, is $0$ on the augmentation of $A$ and satisfies the Maurer-Cartan equation:
            \begin{align*}
                \partial\alpha + \alpha\star\alpha = 0\text{.}
            \end{align*}
            We say that $\alpha$ is an element of $\tt{Tw}(C,A)\subset \tt{Hom}_\mathbb{K}^{1}(C,A)\subset \tt{Hom}_\mathbb{K}^*(C,A)$. Notice that these requirements means that $\tt{Im}\alpha |_{\overline{C}} \subseteq \overline{A}$. In light of proposition \ref{prop: convolution to endomorphism}, every morphism between (coalgebras) algebras extends to a unique (co)derivation on the tensor product $C\otimes_\mathbb{K}A$. Let $d_\alpha^r$ denote this unique morphism. In the case of dg-coalgebras and dg-algebras, we perturb the total differential on the tensor with $d_\alpha^r$, as in proposition \ref{prop: free-derivation}. We call this derivation for the perturbated derivative,
            \begin{align*}
                d_\alpha = d_{C\otimes_\mathbb{K}A} + d_\alpha^r = d_C\otimes id_A + id_C\otimes d_A + d_\alpha^r\tt{.}
            \end{align*}
            \begin{proposition}\label{prop: twisted-differential}
                Suppose that $C$ is a dg-coalgebra and $A$ is a dg-algebra, and $\alpha\in \tt{Hom}_\mathbb{K}^{1}(C,A)$. The perturbated derivation satisfies the following relation.
                \begin{align*}
                    {d_\alpha}^2 = d^r_{\partial \alpha + \alpha\star\alpha}
                \end{align*}
                Moreover, a morphism satisfies the Maurer-Cartan equation if and only if its associated perturbated derivative is a differential.
            \end{proposition}

            \begin{proof}
                ${d_\alpha}^2 = d_{C\otimes_\mathbb{K}A} \circ d_\alpha^r + d_\alpha^r \circ d_{C\otimes_\mathbb{K}A} + {d_\alpha^r}^2$. The result is immediate by proposition \ref{prop: dr-dg-hom}.
            \end{proof}

            \begin{corollary}
                If $\alpha: C\rightarrow A$ is a twisting morphism, then $(C\otimes_\mathbb{K}A, d_\alpha^\bullet)$ is a chain complex which is also a left $C$-comodule and a right $A$-module. We call this the right twisted tensor product, denoted as $C\otimes_\alpha A$.
            \end{corollary}

            Normally $A\otimes C$ and $C\otimes A$ are isomorphic as modules. In general, it is not true that $C\otimes_\alpha A$ and $A\otimes_\alpha C$ are isomorphic since we have to choose a particular side to perform the twisting. However, if $A$ is commutative and $C$ is cocommutative, they are isomorphic. To illustrate, we realize the unique derivation above as a right derivative. The left derivative $d_\alpha^l$ is then defined analogously,
            \begin{center}
                \begin{tikzpicture}[line cap=round,line join=round,>=triangle 45,x=1cm,y=1cm, thick, op/.style={circle, draw, scale=0.75}, diff/.style={regular polygon, draw, regular polygon sides = 3, scale=0.75, rotate = 180}, scale=0.7]

                    \node at (0, 0) {$d_\alpha^l =$};

                    \node[op, scale = 0.75] (a) at (2, 0) {$\alpha$};

                    \draw [line width = 1pt] (1, 1.25) -- (1, -0.5) -- (1.5, -1) -- (1.5, -1.25);
                    \draw [line width = 1pt] (2.5, 1.25) -- (2.5, 1) -- (2, 0.5) -- (a) -- (2, -0.5) -- (1.5, -1) -- (1.5, -1.25);
                    \draw [line width = 1pt] (2.5, 1) -- (3, 0.5) -- (3, -1.25);
                    
                \end{tikzpicture}
                \tt{.}
            \end{center}

            $d_{\argument}^l : \tt{Hom}_\mathbb{K}^\bullet(C,A) \rightarrow \tt{End}^\bullet(C,A)$ does no longer define a ring morphism. Note that this still commutes with the differential. The problem lies in the ring homomorphism property. Observe that we get
            \begin{align*}
                d_{\alpha\star\beta}^l = (-1)^{|\alpha||\beta|}d_\beta^l\circ d_\alpha^l\tt{.}
            \end{align*}
            We summarize this in the next proposition.

            \begin{proposition}
                The morphism $d_{\argument}^l : \tt{Hom}_\mathbb{K}^\bullet(C,A) \rightarrow \tt{End}^\bullet(C,A)$ is a skew chain map. 
            \end{proposition}

            \begin{proof}
                This proposition is clear from the previous discussion.
            \end{proof}
            
            \begin{remark}
                The functoriality of the right twisted tensor at the level of chain maps does not work. To show where it may go wrong, pick two twisting morphisms $\alpha: C \rightarrow A$ and $\beta: C' \rightarrow A'$. Given a pair of morphisms $f: C \rightarrow C'$ and $g: A \rightarrow A'$, it is unclear if $f\otimes g$ will preserve the perturbed differential, and it is not valid in general.

                However, it is the case that the right twisted tensor product defines a tri-functor from the category of elements to cochain complexes,
                \begin{align*}
                    \argument\otimes_{\argument}\argument : \sum_{\tt{Coalg}\otimes\tt{Alg}}\tt{Tw} \rightarrow \tt{Mod}_C^A\tt{.}
                \end{align*}
                Any commutative square as below gets mapped to a morphism of its right twisted tensors. Here $f$ is a morphism of coalgebras, and $g$ is a morphism of algebras,
                \begin{center}
                    \begin{tikzcd}
                        C \ar[]{r}[]{\alpha} \ar[]{d}[]{f} & A \ar[]{d}[]{g} \\
                        C' \ar[]{r}[]{\alpha '} & A'
                    \end{tikzcd}\quad\rightsquigarrow\quad
                    \begin{tikzcd}
                        C\otimes_\alpha A \ar[]{d}[]{f\otimes g} \\
                        C'\otimes_{\alpha '} A'
                    \end{tikzcd}
                \end{center}
                The important property to obtain this is that $f$ and $g$ are morphisms in their respective categories, allowing us to collapse the different compositions to the same map up to sign.
            \end{remark}

            % \begin{remark}
            %     Functoriality of $\otimes_\alpha$ is obtained from the category of elements. I propose that there is an equivalence of categories, that is:
            %     \begin{align*}
            %         \int_{(C,A)} \tt{Tw}(C,A) \simeq \text{right twisted tensors.}
            %     \end{align*}
            % \end{remark}

    \subsection{Bar and Cobar Construction}

            Eilenberg and Mac Lane first formalized the bar construction for augmented skew-commutative dg-rings \cite{Eilenberg53}. The bar construction then served as a method to calculate the homology of Eilenberg-Mac Lane spaces. This construction was later dualized by Adams \cite{Adams56} to obtain the cobar construction. Its first purpose was to serve as a method for constructing an injective resolution to calculate the cotor resolution \cite{Eilenberg65}. With time, the bar-cobar construction has been subjected to many generalizations, such as a fattened tensor product on simplicially enriched, tensored, and cotensored categories \cite{Riehl14}. We will mainly follow the work of \cite{Loday12} to obtain the one-sided algebraic bar and cobar construction. The approach we will take is also slightly inspired by MacLane's canonical resolutions of comonads \cite{MacLane71}.

            For our purposes, the bar construction of an augmented algebra is a simplicial resolution as a cofree coalgebra structure. Given a dg-algebra, we will realize this as the total complex of its resolution. Dually, the cobar construction of a conilpotent coalgebra is a cosimplicial resolution as a free algebra structure. We will see that these constructions define an adjoint pair of functors.

            % A monad is a monoid in the monoidal category of endofunctors. This is a functor $M:\mathcal{C}\rightarrow\mathcal{C}$, with natural transformations $\mu:M^2\implies M$ and $\eta:Id_\mathcal{C}\implies M$ called multiplication and unit. The triple $(M, \mu, \eta)$ is a monad whenever it is a monoid, i.e., multiplication satisfies associativity, and the unit is the unit of the multiplication. Dually a comonad $W:\mathcal{C}\rightarrow\mathcal{C}$ is a triple $(W, \nu, \varepsilon)$ such that it is a comonoid. 

            % \begin{proposition}
            %     Suppose that $W:\mathcal{C}\rightarrow\mathcal{C}$ is comonad. The sequence of functors $(W^{n})_\mathbb{N}$ is an augmented simplicial functor with face and degeneracy maps
            %     \begin{align*}
            %         d_n^i & = W^i\varepsilon_{W^{n-i}} \\
            %         s_n^i & = W^i\nu_{W^{n-i}}\text{.}
            %     \end{align*}

            %     Suppose that $M:\mathcal{C}\rightarrow\mathcal{C}$ is a monad. The sequence of functors $(M^{n})_\mathbb{N}$ is a coaugmented cosimplicial functor with coface and codegeneracy maps
            %     \begin{align*}
            %         d^n_i & = M^i\eta_{M^{n-i}} \\
            %         s^n_i & = M^i\mu_{M^{n-i}}\text{.}
            %     \end{align*}
            % \end{proposition}

            % \begin{proof}
            %     Follows from the universal property of the simplex category, see \cite{MacLane71}.
            % \end{proof}

            % Let $W^?:\Delta^{op}\rightarrow\mathcal{C}$ denote the simplicial functor. If $C\in\mathcal{C}$ is an object, then there is a simplicial object, denoted $W_C^?$. The face and degeneracy maps are obtained by applying $C$, i.e., $(d_n^i)_C$ and $(s_n^i)_C$. Dually, if $M$ is a monad, we obtain a cosimplicial object $M_C^?$.

            % \begin{definition}
            %     Suppose that $\mathcal{A}$ is an abelian category. Let $W:\mathcal{A}\rightarrow\mathcal{A}$ be a comonad and $A$ and object of $\mathcal{A}$. The canonical $W$-projective resolution of $A$ is the chain complex $W^\bullet_A$ together with an augmentation $\varepsilon_A: W^\bullet_A \rightarrow A$.

            %     Dually, suppose that $M:\mathcal{A}\rightarrow\mathcal{A}$ is a monad. A's canonical $M$-injective resolution is the chain complex $M_A^\bullet$ together with an augmentation $\eta_A: A\rightarrow MA$.
            % \end{definition}

            % \begin{remark}
            %     It makes sense to call these canonical resolutions projective or injective. Whenever the object $A$ is $W$-projective, we get that the augmentation is a quasi-isomorphism. See Weibel for more information.
            % \end{remark}

            % \begin{lemma}\label{lem: adjoints-make-monads}
            %     Suppose that there is an adjoint pair of functors $F: \mathcal{C}\rightleftharpoons\mathcal{D}:G$, where $F$ is left adjoint to $G$. The composite $GF$ is a monad, and $FG$ is a comonad.
            % \end{lemma}

            % We will use this lemma to construct the canonical resolution for algebras. One may observe that this is the same as the Hochschild complex.

            % \begin{example}
            %     Let $A$ be an algebra over the ring $\mathbb{K}$. The functor $A\otimes_\mathbb{K}\argument:Mod_\mathbb{K}\rightarrow Mod_A$ is the free $A$-module over a $\mathbb{K}$-module. Let $U: Mod_A\rightarrow Mod_\mathbb{K}$ denote the forgetful functor. 

            %     \begin{minipage}[c]{0.3\textwidth}
            %         \begin{center}
            %             \begin{tikzcd}
            %                 Mod_\mathbb{K} \ar[phantom]{r}{\top} \ar[bend right]{r}[below]{A\otimes_\mathbb{K}\argument} & Mod_A \ar[bend right]{l}[above]{U}
            %             \end{tikzcd}
            %         \end{center}
            %     \end{minipage}
            %     \begin{minipage}[c]{0.7\textwidth}
            %         The unit of the adjunction is given by adjoining the unit of $A$, $\eta_M(m) = 1_A \otimes m$. The counit is given by the structure morphism of each $A$-module, i.e., $\varepsilon_M: A\otimes_\mathbb{K}M\rightarrow M$ is the algebra action. These morphisms are, by definition, natural and satisfy the triangle identities.
            %     \end{minipage}

            %     By lemma \ref{lem: adjoints-make-monads} $A\otimes_\mathbb{K}U : Mod_A \rightarrow Mod_A$ is a comonad. $\varepsilon$ is the counit and the comultiplication is given by $A\otimes_\mathbb{K}\eta_U$. If $M$ is an $A$-module we get the canonical free-resolution of $M$ as $A\otimes_\mathbb{K}U_M^\bullet$:
            %     \begin{center}
            %         \begin{tikzcd}
            %             ... \ar[]{r}[]{} & A\otimes_\mathbb{K}A\otimes_\mathbb{K}M \ar[]{r}[]{\substack{\varepsilon_{A\otimes\mathbb{K}M} \\ -A\otimes_\mathbb{K}\varepsilon_M}} & A\otimes_\mathbb{K}M \ar[]{r}[]{0} & 0 \ar[]{r}[]{} & ...
            %         \end{tikzcd}
            %     \end{center}
            % \end{example}

            An algebra $A$ is a monoid in the monoidal category $(\tt{Mod}_\mathbb{K}, \otimes_\mathbb{K}, \mathbb{K})$. By proposition \ref{prop: universal-monoid}, we may think of $A$ as an augmented cosimplicial object $A:\Delta_+ \rightarrow \tt{Mod}_\mathbb{K}$. Notice that all of the cosimplicial identities follow from associativity and unitality. If $A$ is an augmented algebra, we may instead give it the structure of an augmented simplicial set. Let $d^0_0 = \varepsilon_A$ be the augmentation. We define $d^n_n = A^{\otimes n-1}\otimes\varepsilon_A$ and set $d^i_n = A^{i-1}\otimes (\cdot_A) \otimes A^{\otimes n-i-1}$. The degeneracies are chosen to be the units, that is, the morphisms $s^i_n = A^{\otimes i}\otimes \upsilon_A \otimes A^{\otimes n-i-1}$. One may check that this structure defines an augmented simplicial object $A:\Delta_+^{op}\rightarrow \tt{Mod}_\mathbb{K}$. Observe that the chain complex $\tt{C}A$ is exactly the Hochschild complex of $A$. We depict the simplicial object in the following diagram:
            \begin{center}
                \begin{tikzcd}
                    \mathbb{K} & A \ar[]{l}[above]{\varepsilon_A} & A^{\otimes 2} \ar[yshift = 0.5ex]{l}[above]{(\cdot_A)} \ar[yshift = -0.5ex]{l}[]{A\otimes \varepsilon_A} & A^{\otimes 3} \ar[yshift = 0.75ex]{l}[above]{(\cdot_A)} \ar[]{l}[]{} \ar[yshift = -0.75ex]{l}[]{A^{\otimes 2}\otimes \varepsilon_A} & ... \ar[yshift = 1ex]{l}[above]{(\cdot_A)} \ar[yshift = 0.33ex]{l}[]{} \ar[yshift = -0.33ex]{l}[]{} \ar[yshift = -1ex]{l}[]{A^{\otimes 4}\otimes\varepsilon_A} 
                \end{tikzcd}

                \begin{tikzcd}
                    \mathbb{K} & A \ar[]{r}[]{s^1} & A^{\otimes 2} \ar[yshift = 0.5ex]{r}[]{s^i} \ar[yshift = -0.5ex]{r}[]{} & A^{\otimes 3} \ar[yshift = 0.75ex]{r}[]{s^i} \ar[]{r}[]{} \ar[yshift = -0.75ex]{r}[]{} & ...
                \end{tikzcd}
            \end{center}

            The augmentation ideal $\overline{A}$ carries a natural semi-simplicial structure induced by $A$. As in Example \ref{ex: ass-complex}, there is an associated cochain complex to $\overline{A}$ by restricting each of the face maps, ${\overline{d}}^i = d^i {\mid}_{\overline{A}}:\overline{A}^{\otimes n} \rightarrow \overline{A}^{\otimes n-1}$. The associated cochain complex is the non-unital Hochschild complex of $A$. We depict the semi-simplicial object as shown in the following diagram:
            \begin{center}
                \begin{tikzcd}
                    \mathbb{K} & \overline{A} \ar[]{l}[above]{0} & \overline{A}^{\otimes 2} \ar[yshift = 0.5ex]{l}[above]{(\cdot_A)} \ar[yshift = -0.5ex]{l}[]{0} & \overline{A}^{\otimes 3} \ar[yshift = 0.75ex]{l}[above]{(\cdot_A)} \ar[]{l}[]{} \ar[yshift = -0.75ex]{l}[]{0} & ... \ar[yshift = 1ex]{l}[above]{(\cdot_A)} \ar[yshift = 0.33ex]{l}[]{} \ar[yshift = -0.33ex]{l}[]{} \ar[yshift = -1ex]{l}[]{0} 
                \end{tikzcd}
            \end{center}

            As graded modules, the chain complex $\tt{C}\overline{A}$ is isomorphic to $T^c(\overline{A})$. Here we think of the grading $T^c(\overline{A})$ as starting at $0$ and going down to negative degrees. Consider instead the looped non-unital algebra $\overline{A}[1]$. There is a natural grading on every algebra, concentrating it in degree $0$. The shift functor then changes the degree to which we concentrate the algebra. However, $\overline{A}[1]$ is no longer an associative algebra. To understand this looped multiplication, we will first consider $\mathbb{K}\startset{\omega}$, where $|\omega| = -1$. We define a looped multiplication $(\cdot):\mathbb{K}\startset{\omega}^{\otimes 2} \rightarrow \mathbb{K}\startset{\omega}$ as
            \begin{align*}
                \omega \cdot \omega = \omega\tt{.}
            \end{align*}
            Given an algebra $A$, the looped multiplication of $A[1]$ is defined as the composite 
            \begin{align*}
                (\cdot_{A[1]}) = ((\cdot) \otimes (\cdot_A)) \circ (\mathbb{K}\startset{\omega} \otimes \beta \otimes \overline{A})\tt{.} 
            \end{align*}
            As an example, suppose that $\omega a_1$ and $\omega a_2$ are elements of $A[1]$, then their multiplication would look like
            \begin{align*}
                (\cdot_{A[1]})(\omega a_1 \otimes \omega a_2) = (-1)^{|a_1||\omega|}((\cdot) \otimes \cdot_{A})(\omega^{\otimes 2} \otimes a_1 \otimes a_2) = (-1)^{|a_1|}\omega a_1 a_2\tt{.}
            \end{align*}
            Observe that the resulting morphism $(\cdot_{A[1]})$ is of degree $1$.

            \begin{proposition}\label{prop: a-to-dgc}
                Suppose that $A$ is an augmented algebra. The differential $d_{\overline{A}[1]}$ is a coderivation for the cofree coalgebra $T^c(\overline{A}[1])$. Thus $(\tt{C}\overline{A}[1], d_{\overline{A}[1]})$ is a dg-coalgebra.
            \end{proposition}

            \begin{proof}
                By injecting $\overline{A}[1]$ into $T^c(\overline{A}[1])$, we may think of $(\cdot_{\overline{A}[1]}) : \overline{A}[1]^{\otimes 2} \rightarrow T^c(\overline{A}[1])$ as a morphism into the tensor coalgebra. By using Proposition \ref{prop: tensor-derivation}, $(\cdot_{\overline{A}[1]})$ extends uniquely into a coderivation:
                \begin{align*}
                    d_{\overline{A}[1]}^c = \sum_{n=0}^{\infty}\sum_{i=0}^n(\cdot_{\overline{A}[1]})_{(i)}^{(n)} = d_{\overline{A}[1]}\text{.}
                \end{align*}
            \end{proof}

            If $(A, d_A)$ is an augmented dg-algebra, then $A$ is a simplicial object of $\tt{Mod}_\mathbb{K}^\bullet$. There is also an associated complex $\tt{C}A$ of $A$ by taking the alternate sum of face maps. The complex $\tt{C}A$ may be seen as the total complex of the double complex represented below. 
            \begin{center}
                \begin{tikzcd}
                    \cdots \ar[]{r}[below]{0} & 0 \ar[]{r}[below]{0} & \mathbb{K} \ar[]{r}[below]{0} & 0 \ar[]{r}[below]{0} & \cdots \\
                    \cdots \ar[]{r}[below]{-d_{A}} & A^{-1} \ar[]{r}[below]{-d_{A}} \ar[]{u}[]{\varepsilon_A} & A^0 \ar[]{r}[below]{-d_{A}} \ar[]{u}[]{\varepsilon_A} & A^{1} \ar[]{r}[below]{-d_{A}} \ar[]{u}[]{\varepsilon_A} & \cdots \\
                    \cdots \ar[]{r}[below]{d_{A^{\otimes 2}}} & (A^{\otimes 2})^{-1} \ar[]{r}[below]{d_{A^{\otimes 2}}} \ar[xshift = -0.5ex]{u}[left]{(\cdot_A)} \ar[xshift = 0.5ex]{u}[right]{A\otimes\varepsilon_A} & (A^{\otimes 2})^0 \ar[]{r}[below]{d_{A^{\otimes 2}}} \ar[xshift = -0.5ex]{u}[left]{(\cdot_A)} \ar[xshift = 0.5ex]{u}[right]{A\otimes\varepsilon_A} & (A^{\otimes 2})^{1} \ar[]{r}[below]{d_{A^{\otimes 2}}} \ar[xshift = -0.5ex]{u}[left]{(\cdot_A)} \ar[xshift = 0.5ex]{u}[right]{A\otimes\varepsilon_A} & \cdots \\
                    & \vdots \ar[xshift = -1ex]{u}[left]{} \ar[]{u}[]{} \ar[xshift = 1ex]{u}[right]{} & \vdots \ar[xshift = -1ex]{u}[left]{} \ar[]{u}[]{}x \ar[xshift = 1ex]{u}[right]{} & \vdots \ar[xshift = -1ex]{u}[left]{} \ar[]{u}[]{} \ar[xshift = 1ex]{u}[right]{} & 
                \end{tikzcd}
            \end{center}
        
            For simplicity, we will write $d_1$ for the horizontal differential and $d_2$ for the vertical differential. $\tt{C}A$ is thus the total complex of the double complex above. Instead of considering the abovementioned double complex, we will consider the double complex associated with the looped algebra $\overline{A}[1]$. The following lemma states that this double complex is well-defined.
            
            \begin{proposition}
                Let $A$ be an augmented dg-algebra. The bar complex $BA$ is the total associated chain complex of the augmentation ideal $\overline{A}$. $(BA, d_{BA}^\bullet)$ is the cofree conilpotent coalgebra equipped with $d_{BA}^\bullet = d_1 + d_2$ as coderivation.
            \end{proposition}

            \begin{proof}
                    $d_1$ and $d_2$ are coderivations with respect to deconcatenation as comultiplication. Since the multiplication $(\cdot_A)$ is a chain map, we should have ${d_{BA}^\bullet}^2 = d_1 \circ d_2 + d_2 \circ d_1= 0$. We will show this for each element in $A^{\otimes 2}$, and the result may be extended to all of $BA$. Instead of decorating each $a_i$ with an $\omega$, we will follow Eilenberg and MacLane's notation, using brackets and bars, $\omega a_1 \otimes \omega a_2 = [a_1 | a_2]$ \cite[73]{Eilenberg53}. The bars in this notation are what gave this coalgebra its name.

                \begin{multline*}
                    d_1 \circ d_2 [a_1 | a_2] = (-1)^{|a_1|}d_1 [a_1a_2] = (-1)^{|a_1|}d_{A[1]}[a_1a_2] \\ = (-1)^{|a_1|+1}[d_A(a_1a_2)] = (-1)^{|a_1|+1}([d_A(a_1)a_2] + (-1)^{|a_1|}[a_1d_A(a_2)]) \\ = (-1)^{|a_1|+1}[d_A(a_1)a_2] - [a_1d_A(a_2)]
                \end{multline*}
                \begin{multline*}
                    d_2\circ d_1 [a_1 | a_2] = d_2\circ (d_{A[1]}\otimes id_{A[1]} + id_{A[1]}\otimes d_{A[1]}) [a_1\otimes a_2] \\ = -d_2 \circ ([d_A(a_1) | a_2] + (-1)^{|a_1|+1}[a_1 | d_A(a_2)]) \\ = (-1)^{|d_A(a_1)|+1}[d_A(a_1)a_2] + (-1)^{2|a_1|+2}[a_1d_A(a_2)] \\ = (-1)^{|a_1|}[d_A(a_1)a_2] + [a_1d_A(a_2)] = -d_1\circ d_2 [a_1 | a_2]
                \end{multline*}
            \end{proof}

            \begin{remark}
                We don't need to show that $BA$ is a functor. This property follows from $BA$ representing the object of $\tt{Tw}(\argument, A)$.
            \end{remark}

            On the other hand, a coalgebra $C$ is a comonoid in $\tt{Mod}_\mathbb{K}$. By the dual of proposition \ref{prop: universal-monoid}, we may think of it as an augmented simplicial object $C:(\Delta_+)^{op} \rightarrow Mod_\mathbb{K}$. Dually, all of the simplicial identities follow from coassociativity and counitality. A coaugmented coalgebra $C$ may be given an augmented cosimplicial structure in the opposite way of algebras. We then get that the coaugmentation quotient $\overline{C}$ is a semi-cosimplicial object of $\tt{Mod}_\mathbb{K}$. Observe that $\overline{C}$ has an associated chain complex like $\overline{A}$, but every arrow goes in the opposite direction.

            \begin{center}
                \begin{tikzcd}
                    \mathbb{K} \ar[]{r}[]{\upsilon_C} & C \ar[yshift = 0.5ex]{r}[]{\Delta_C} \ar[yshift = -0.5ex]{r}[below]{A\otimes \upsilon_C} & C^{\otimes 2} \ar[yshift = 0.75ex]{r}[]{\Delta_C} \ar[]{r}[]{} \ar[yshift = -0.75ex]{r}[below]{C^{\otimes 2}\otimes \upsilon_C} & C^{\otimes 3} \ar[yshift = 1ex]{r}[]{\Delta_C} \ar[yshift = 0.33ex]{r}[]{} \ar[yshift = -0.33ex]{r}[]{} \ar[yshift = -1ex]{r}[below]{C^{\otimes 4}\otimes\upsilon_C} & ... 
                \end{tikzcd}

                \begin{tikzcd}
                    \mathbb{K} & C & C^{\otimes 2} \ar[]{l}[above]{s_1} & C^{\otimes 3} \ar[yshift = 0.5ex]{l}[above]{s_i} \ar[yshift = -0.5ex]{l}[]{} & ... \ar[yshift = 0.75ex]{l}[above]{s_i} \ar[]{l}[]{} \ar[yshift = -0.75ex]{l}[]{}
                \end{tikzcd}
            \end{center}
            
            The cobar construction is made from the suspended dg-coalgebra $C[-1]$. We may also denote suspension by tensoring with a formal generator $s$, such that $|s| = 1$. Then we have an isomorphism $C[-1] \simeq \mathbb{K}\startset{s}\otimes C$. The cobar construction is realized as the free tensor algebra $T(\overline{C}[-1])$, where the comultiplication $\Delta_{\overline{C}[-1]}$ induces a derivation $d_{\overline{C}[-1]}$ by Proposition \ref{prop: tensor-derivation}.

            \begin{remark}
                As we have chosen to define $(\cdot_{A[1]})(a_1\otimes a_2)=(-1)^{|a_1|}a_1a_2$, we are forced by the linear dual to define $\Delta_{C[-1]}(c)=-(-1)^{|c_{(1)}|}c_{(1)}\otimes c_{(2)}$. Here we use Sweedler's notation without sums to denote the comultiplication. Note that this really should be a sum of many different elementary tensors. Lastly, observe that this definition also agrees with Koszuls's sign rule.
            \end{remark}

            The associated cochain complex $\tt{C}C$ is the total complex of the double complex below. Similarly, we want to study $C[-1]$ to obtain a similar result to the bar construction.

            \begin{center}
                \begin{tikzcd}
                    & \vdots & \vdots & \vdots \\
                    ... \ar[]{r}[]{d_{\overline{C}^{\otimes 2}}} & (\overline{C}^{\otimes 2})^{-1} \ar[]{r}[]{d_{\overline{C}^{\otimes 2}}} \ar[xshift = -0.5ex]{u}[]{\Delta_C\otimes \overline{C}} \ar[xshift = 0.5ex]{u}[]{} & (\overline{C}^{\otimes 2})^{0} \ar[]{r}[]{d_{\overline{C}^{\otimes 2}}} \ar[xshift = -0.5ex]{u}[]{\Delta_C\otimes\overline{C}}  \ar[xshift = 0.5ex]{u}[]{} & (\overline{C}^{\otimes 2})^1 \ar[]{r}[]{d_{\overline{C}^{\otimes 2}}} \ar[xshift=-0.5ex]{u}[]{\Delta_C\otimes \overline{C}} \ar[xshift=0.5ex]{u}[]{} & ... \\
                    ... \ar[]{r}[]{d_{\overline{C}}} & \overline{C}^{-1} \ar[]{r}[]{d_{\overline{C}}} \ar[]{u}[]{\Delta_C} & \overline{C}^0 \ar[]{r}[]{d_{\overline{C}}} \ar[]{u}[]{\Delta_C} & \overline{C}^1 \ar[]{r}[]{d_{\overline{C}}} \ar[]{u}[]{\Delta_C} & ... \\
                    ... \ar[]{r}[]{} & 0 \ar[]{r}[]{} \ar[]{u}[]{} & \mathbb{K} \ar[]{u}[]{0} \ar[]{r}[]{} & 0 \ar[]{r}[]{} \ar[]{u}[]{} & ...
                \end{tikzcd}
            \end{center}

            \begin{proposition}
                Let $C$ be a coaugmented dg-coalgebra. The cobar complex $\Omega C$ is the total associated chain complex of the suspended coaugmentation quotient $\overline{C}[-1]$. $(\Omega C, d_{\Omega C})$ is the free algebra equipped with the differential $d_{\Omega C} = d_1 + d_2$ as derivation.
            \end{proposition}

            \begin{proof}
                This proof is similar to the one given for the bar construction.
            \end{proof}

            Given a string of elements in the cobar $sc_1 \otimes \cdots$, we write it by using pointed brackets and bars instead,
            \begin{align*}
                sc_1 \otimes sc_2 \otimes \cdots \otimes sc_n = \langle c_1 | c_2 | \cdots | c_n \rangle\tt{.}
            \end{align*}

            The bar and cobar construction defines an adjoint pair of functors. We want to show that for any conilpotent dg-coalgebra $C$, the object $\Omega C$ represents a functor in the category of augmented algebras. By Yoneda's lemma, $\Omega$ does truly define a functor.

            \begin{thm}\label{thm: cobar-bar-adj}
                Let $C$ be a conilpotent dg-coalgebra and $A$ an augmented dg-algebra. The functor $\tt{Tw}(C,A)$ is represented in both arguments, i.e.
                \begin{align*}
                    \tt{Alg}_{\mathbb{K},+}^\bullet(\Omega C, A)\simeq \tt{Tw}(C, A) \simeq \tt{coAlg}_{\mathbb{K},\tt{conil}}^\bullet(C, BA)\text{.}
                \end{align*}
            \end{thm}

            \begin{proof}
                We will show that $\Omega C$ represents the set of twisting morphisms in the first argument, and this shows that $BA$ represents the second argument by using every dual proposition. Thus, $C$ must be conilpotent to dualize the results.

                Suppose that $f:\Omega C \rightarrow A$ is an augmented dg-algebra homomorphism. $f$ is then a morphism of degree $0$. By freeness, $f$ is uniquely determined by a morphism $f\mid_{\overline{C}[-1]}:\overline{C}[-1]\rightarrow \overline{A}$ of degree $0$, which corresponds to a morphism $f':C\rightarrow A$ of degree $1$ which is $0$ on the augmentation and coaugmentation. 

                Since $f$ is a morphism of chain complexes, it commutes with the differential, i.e. 
                \begin{align*}
                    f\circ d_{\Omega C} & = d_A\circ f \\
                    \Leftrightarrow\quad f\circ (d_1 + d_2) & = d_A\circ f 
                \end{align*}
                By \ref{prop: free-tensor}, to establish these conditions, it is enough to consider the summand where $d_1 = -d_C$ and $d_2 = \overline{\Delta}_{C[-1]}$. Then the right hand side becomes $-f' \circ d_C - (-1)^{|f|}(\cdot_A)(f' \otimes f')\Delta_C$. This is equivalent to saying that $-f'\circ d_C - f'\star f' = d_A\circ f'$. Thus $f'$ is a twisting morphism as desired.

                Since every step to establish that $f'$ is a twisting morphism was a logical equivalence, we arrive at the desired conclusion.
            \end{proof}

            For our convenience, we will give these isomorphisms some names. Whenever $\tau: C \rightarrow A$ is a twisting morphism, we denote the induced morphism of algebras as $f_\tau: \Omega C \rightarrow A$, and the induced morphism of coalgebras as $g_\tau: C \rightarrow BA$.

            \begin{remark}
                We could have defined a twisting morphism from any coalgebra $C$ to algebra $A$. In this case, we could have defined a twisting morphism as a morphism of degree $1$, which satisfies the Cartan-Maurer equation. However, the cobar and bar construction on augmented algebras does not represent this definition of twisting morphisms. The subclass of twisting morphisms which also (co)restricts to twisting morphisms on its coaugmentation quotient and augmentation ideal, would be represented in this manner, which is what our definition requires. 

                The cobar-bar adjunction consists of a composition with the augmentation ideal (quotient) and then the (co)free tensor (co)algebra. By reversing these operations, we obtain another adjunction that is more or less the same. By abuse of language, we will call these functors for the bar and cobar construction as well, and they establish an adjoint pair between non-unital dg-algebras and reduced conilpotent dg-coalgebras. In other words, given a non-unital dg-algebra $A$ and a reduced conilpotent dg-coalgebra $C$, $BA = \overline{T}^c(A[1])$ and $\Omega C = \overline{T}(C[-1])$.
                \begin{center}
                    \begin{tikzcd}
                        \widehat{\tt{Alg}}_\mathbb{K}^\bullet \ar[bend right]{r}[below]{B} \ar[phantom]{r}[pos = 0.78]{\bot} & \tt{coAlg}_{\mathbb{K},\tt{conil},-}^\bullet \ar[bend right]{l}[above]{\Omega}
                    \end{tikzcd}
                \end{center}
            \end{remark}

            We obtain universal elements and universal properties associated with this adjunction. Let $A$ be an augmented dg-algebra, then the identity of the coalgebras $id_{BA}: BA \rightarrow BA$, the counit $\varepsilon_A: \Omega BA \rightarrow A$ and a twisting morphism $\pi_A: BA \rightarrow A$ are equivalent by the adjunction and representation. Dually, the identity of algebras $id_{\Omega C} : \Omega C\rightarrow\Omega C$, the unit $\eta_C : C \rightarrow B\Omega C$ and the twisting morphism $\iota_C : C\rightarrow \Omega C$ are equivalent. The morphisms $\pi_A$ and $\iota_C$ are called the universal elements. We summarize their universal property in the following corollary.

            \begin{corollary}\label{cor: universal-twisting}
                Let $A$ be an augmented dg-algebra and $C$ a conilpotent dg-coalgebra. Any twisting morphism $\alpha: C \rightarrow A$ factors uniquely through either $\pi_A$ or $\iota_C$.
                
                \begin{center}
                    \begin{tikzcd}
                        & \Omega C \ar[dashed]{rd}[]{g_\alpha}\\
                        C \ar[]{rr}[]{\alpha} \ar[]{ru}[]{\iota_C} \ar[dashed]{rd}[]{f_\alpha} & & A \\
                        & BA \ar[]{ru}[]{\pi_A}
                    \end{tikzcd}
                \end{center}
                Moreover, the morphism $f_\alpha$ is a morphism of dg-coalgebras, and $g_\alpha$ is a morphism of dg-algebras.
            \end{corollary}

            \begin{definition}[Augmented Bar-Cobar construction]
                Let $A$ be an augmented dg-algebra. The (right) augmented bar construction is the right twisted tensor product $BA \otimes_{\pi_A} A$, where $\pi_A$ is the universal twisting morphism.

                Let $C$ be a conilpotent dg-coalgebra. The (right) augmented cobar construction is the right twisted tensor product $C \otimes_{\iota_C} \Omega C$, where $\iota_C$ is the universal twisting morphism.
            \end{definition}

            \begin{remark}
                We could have defined the augmented bar-cobar construction as the left twisted tensor product. There is no preference for handedness. It will be specified whenever we wish to be precise about which handedness we will use. For instance, the left augmented bar construction of $A$.
            \end{remark}

            \begin{proposition}\label{prop: aug-bar-ac}
                The augmentation ideal and quotient of the augmented bar and cobar construction are acyclic, i.e., $BA \overline{\otimes}_{\pi_A} A$ ($A \overline{\otimes}_{\pi_A} BA$) and $C \overline{\otimes}_{\iota_C}\Omega C$ ($\Omega C \overline{\otimes}_{\iota_C} C$) are acyclic.
            \end{proposition}

            \begin{proof}
                We will postpone this proof until chapter 3; this is a part of the fundamental theorem of twisting morphisms and will not be relevant until then.
            \end{proof}

    % \subsection{Comparison Lemma}

        % In this section, we wish to deduce the fundamental theorem of twisting morphisms. In order to do this, we will need a result by Henri Cartan \cite{Cartan55}, called the comparison lemma. This section follows Loday \cite{Loday12} and takes inspiration from Lefevre-Hasagawa \cite{LefevreHasegawa03}.
        
        % In order to state the comparison lemma, we need to understand what it means for a dg-algebra to be connected. We define weight, which is a second grading for the objects.

        % \begin{definition}[Weight graded cochain complexes]
        %     Let $N = \startset{0, 1, ...}$ be an indexing set, which is possibly finite. A cochain complex $M$ is weight graded if it splits as a direct sum $M\simeq \bigoplus_{n\in N} M_{(n)}$ indexed over $N$. Let $m\in M$, then $m$ has weight $n$ if $m\in M_{(n)}$, has homological degree $n'$ if $m\in M^{n'}$ and (total) degree $|m| = n + n'$.

        %     A dg-(co)algebra is weight graded if the weight on the cochain complex respects the (co)multiplication. A weight-graded dg-(co)algebra will be called a wdg-(co)algebra.
        % \end{definition}

        % \begin{definition}[Connected cochain complexes]
        %     A weight graded cochain complex $M$ is called connected if $M_(0)\simeq \mathbb{K}$ and is concentrated in homological degree $0$. 
        % \end{definition}

        % \begin{lemma}[Comparison Lemma]\label{lem: comparison}
        %     Let $g:A\rightarrow A'$ be a morphism of connected wdg-algebras, and $f:C\rightarrow C'$ be a morphism of connected wdg-coalgebras. Suppose there are twisting morphisms $\alpha: C \rightarrow A$ and $\alpha ':C'\rightarrow A'$ such that $f$ and $g$ becomes a morphism of twisted tensors. If two out of $f$, $g$, and $f\otimes g$ are quasi-isomorphisms, then so is the third.
        % \end{lemma}

        % \begin{proof}
        %     A proof may be found in either Cartans paper \cite{Cartan55} or Loday and Valletes book \cite{Loday12}.
        % \end{proof}

        % \begin{thm}[Fundamental Theorem of Twisting Morphisms I]\label{thm: fundamental-thm-of-twisting}
        %     Let $A$ be a connected wdg-algebra, $C$ be a connected wdg-coalgebra and $\alpha : C\rightarrow A$ a twisting morphism. The following are equivalent:
        %     \begin{enumerate}
        %         \item The augmentation ideal of the right twisted tensor product $C \overline{\otimes}_\alpha A$ is acyclic
        %         \item The augmentation ideal of the left twisted tensor product $A \overline{\otimes}_\alpha C$ is acyclic
        %         \item The morphism $f_\alpha : C \rightarrow BA$ is a quasi-isomorphism
        %         \item The morphism $g_\alpha : \Omega C \rightarrow A$ is a quasi-isomorphism
        %     \end{enumerate}
        % \end{thm}

        % \begin{proof}
        %     In order to do this proof, we must first observe that the bar and cobar construction preserve the weighted grading and, therefore, the connectedness.
            
        %     We prove 1. $\iff$ 3., the other equivalences are analogous. By corollary \ref{cor: universal-twisting}, the morphism $id_C \otimes g_\alpha : C \otimes_{\iota_C} \Omega C \rightarrow C \otimes_{\alpha} A$ is a morphism of twisting tensor products. Since $id_C$ is a quasi-isomorphism, we get by \ref{lem: comparison} that $id_C \otimes g_\alpha$ is a quasi-isomorphism if and only if $g_\alpha$ is a quasi-isomorphism if and only if $\overline{g}_\alpha$ is a quasi-isomorphism. By \ref{prop: aug-bar-ac} $C \overline{\otimes}_{\iota_C} \Omega C$ is acyclic, so $\overline{g}_\alpha$ is a quasi-isomorphism if and only if $C\overline{\otimes}_\alpha A$ is acyclic.
        % \end{proof}

        % \begin{corollary}
        %     Let $A$ be a connected wdg-algebra, $C$ be a connected wdg-coalgebra. The counit $\varepsilon_A : \Omega BA \rightarrow A$, and the unit $\eta_C : C \rightarrow B\Omega C$ are quasi-isomorphisms. 
        % \end{corollary}

        % We now know some cases where the unit and the counit are quasi-isomorphisms. However, we would like to promote this result to every (conilpotent) augmented dg-(co)algebra. To this end, we will find suitable filtrations to realize the associated graded as a connected wdg-(co)algebra and then get isomorphisms on the associated graded. The problem would then be to lift such quasi-isomorphisms back to our original objects.

    \section{Strongly Homotopy Associative Algebras and Coalgebras}\label{sec: 1.3}
    \subsection{SHA-Algebras}
        We have seen from Corollary \ref{prop: a-to-dgc} that any dg-algebra $A$ defines a dg-coalgebra $T^c(A[1])$, the bar construction, with a coderivation $m^c$ of degree $1$. Does this work in reverse? I.e., if $A$ is a vector space such that the coalgebra $T^c(A[1])$ together with a coderivation $m^c$ is a dg-coalgebra, is then $A$ an algebra? The answer is no, but it leads to the definition of a strongly homotopy associative algebra.

        \begin{definition}
            An $A_\infty$-algebra is a graded vector space $A$ together with a differential $m:\overline{T}^c(A[1])\rightarrow\overline{T}^c(A[1])$ that is a coderivation of degree $1$.
        \end{definition}

        The differential $m$ induces structure morphisms on $A[1]$. By Proposition \ref{prop: tensor-derivation}, there is a natural bijection $\tt{Hom}_\mathbb{K}(\overline{T}^c(A[1]),A[1])\simeq \tt{Coder}(\overline{T}^c(A[1]),\overline{T}^c(A[1]))$ given by the projection onto $A[1]$. Thus $m:\overline{T}^c(A[1])\rightarrow\overline{T}^c(A[1])$ corresponds to maps $\widetilde{m}_n:A[1]^{\otimes n}\rightarrow A[1]$ of degree $1$ for any $n\geq 1$. We define maps $m_n : A^{\otimes n}\rightarrow A$ by the composite $s\widetilde{m}_n \omega^{\otimes n}$. Since $\omega^{\otimes n}$ is of degree $-n$, $\widetilde{m}_n$ and $s$ is of degree $1$, we get that $m_n$ is of degree $2-n$.
        \begin{center}
            \begin{tikzcd}
                A^{\otimes n} \ar[]{r}[]{m_n} \ar[]{d}[]{\simeq}[left]{\omega^{\otimes n}} & A \\
                A[1]^{\otimes n} \ar[]{r}[]{\widetilde{m}_n} & A[1] \ar[]{u}[right]{\simeq}[]{s}
            \end{tikzcd}
        \end{center}

        \begin{remark}
            The choice of isomorphisms here is not canonical. Different choices may lead to different signs in the following formulas. We will follow the sign convention of Loday and Vallette \cite{Loday12}. This will give us the same signs as in Lef\`evre-Hasegawa \cite{LefevreHasegawa03}, as his signs always come in a pair to cancel each other out.
        \end{remark}

        \begin{proposition}\label{prop: A-infinity def}
            An $A_\infty$-algebra is equivalent to a graded vector space $A$ together with homogenous morphisms $m_n: A^{\otimes n}\rightarrow A$ of degree $2-n$. Moreover, the morphism must satisfy the following relations for any $n\geq 1$:
            \begin{align*}
                (\text{rel}_n)\qquad \sum_{p+q+r = n}(-1)^{pq+r}m_{p+1+r}\circ (id^{\otimes p}\otimes m_q \otimes id^{\otimes r}) = 0
            \end{align*}
        \end{proposition}

        \begin{remark}
            We make a more convenient notation for $(\text{rel}_n)$, called partial composition $\circ_i$,
            \begin{align*}
                & m_{p+1+r} \circ_{p+1} m_q = m_k \circ (id^{\otimes p}\otimes m_q \otimes id^{\otimes r})\tt{.}
            \end{align*}
            With this noation we may rewrite each $(\tt{rel}_n)$ as
            \begin{align*}
                (\text{rel}_n)\qquad & \sum_{p+q+r = n} (-1)^{pq+r} m_{p+1+r} \circ_{p+1} m_q = 0\tt{.}
            \end{align*}
        \end{remark}

        Before starting with the proof, we will need a lemma for checking whether a coderivation \\ $m: T^c(A) \rightarrow T^c(A)$ is a differential.

        \begin{lemma}\label{lem: coderivation-is-diff?}
            Let $m: T^c(A) \rightarrow T^c(A)$ be a coderivation, and denote $m_n = m|_{A^{\otimes n}}$. $m$ is a differential if and only if the following relations are satisfied,
            \begin{align*}
                & \sum_{p+q+r = n}m_{p+1+r}\circ_{p+1}m_q = 0\tt{.}
            \end{align*}
        \end{lemma}

        \begin{proof}
            By Proposition \ref{prop: tensor-derivation} we may write $m = \sum_{n = 0}^\infty \sum_{i = 0}^n m_{(n)}^{(i)}$. By using partial composition, we rewrite its $n$'th component as,
            \begin{align*}
                m_n = \sum_{q=1}^n\sum_{p = 1}^n id^{\otimes (n-q)}\circ_{p} m_q = \sum_{p + q + r = n}id^{\otimes (p+1+r)}\circ_{p+1}m_q\tt{.}
            \end{align*}

            For $m^2$, we denote its $n$'th component as $m^2_n$. Let $\pi : T^c(A) \rightarrow A$ denote the projection onto $A$. Observe the following:
            \begin{align*}
                & m^2_n = m\circ m_n = m\circ \sum_{p + q + r = n}id^{\otimes (p+1+r)}\circ_{p+1}m_q = \sum_{p + q + r = n}m\circ_{p+1}m_q\tt{,} \\
                & \pi m^2_n = \pi \sum_{p + q + r = n}m\circ_{p+1}m_q = \sum_{p + q + r = n}m_{p+1+r}\circ_{p+1}m_q\tt{.}
            \end{align*}
            By Proposition \ref{prop: free-derivation}, every coderivation is uniquely determined by $\pi$, we get that $m^2 = 0$ if and only if
            \begin{align*}
                \sum_{p+q+r = n}m_{p+1+r}\circ_{p+1}m_q = 0\tt{.}
            \end{align*}
        \end{proof}

        \begin{proof}[Proof of Proposition \ref{prop: A-infinity def}]
            Let $(A,m)$ be an $A_\infty$-algebra. We denote the $n$'th component of $m$ as $\widetilde{m}_n$. The $n$'th components thus define maps $m_n:A^{\otimes n}\rightarrow A$ as $m_n = s\widetilde{m}_n \omega^{\otimes n}$.

            By the above lemma, we know that the $n$'th component of $m^2$ is,
            \begin{multline*}
                \sum_{p + q + r = n}\widetilde{m}_{p+1+r}\circ_{p+1}\widetilde{m}_q \\
                = \sum_{p + q + r = n}\omega m_{p+1+r}s^{\otimes (p+1+r)}\circ_{p+1}\omega m_qs^{\otimes q} = \sum_{p + q + r = n}(-1)^{pq+r}\omega m_{p+1+r}\circ_{p+1}m_q s^{\otimes n}\tt{.}
            \end{multline*}

            The last equation is given by applying Proposition \ref{prop: multi-koszul-sign} twice. In other words, we want to find a parity $p = p_1 + p_2$, which determines the sign above. To get $p_1$ we start with moving the $s$ on the left,
            \begin{align*}
                s^{\otimes p + 1 + r} \circ (id^{\otimes p} \otimes \omega m_q s^{\otimes q} \otimes id^{\otimes r}) = (-1)^{p_1}(s^{\otimes q} \otimes m_q s^{\otimes q} \otimes s^{\otimes r})\tt{.}
            \end{align*}
            By Proposition \ref{prop: multi-koszul-sign},
            \begin{align*}
                p_1 = \sum_{i=1}^n\sum_{1\leq j < i}(\tt{if }j = p+1\tt{ then }1\tt{ otherwise }0) = r\tt{.}
            \end{align*}

            In the next step, we separate the $s$ on the right,
            \begin{align*}
                (id^{\otimes p}\otimes m_q \otimes id^{\otimes r}) \circ s^{\otimes n} = (-1)^{p_2}(s^{\otimes q}\otimes m_qs^{\otimes q}\otimes s^{\otimes r})\tt{.}
            \end{align*}
            We calculate $p_2$ to be,
            \begin{align*}
                p_2 = (2-q)\sum_{1\leq j < p+1}1 = 2p - qp\tt{.}
            \end{align*}
            Thus the parity of $p$ is $p = 2p - qp + r = pq + r$ modulo $2$.

            Since suspension and loop are isomorphisms, we get that $m^2 = 0$ if and only if $(\text{rel}_n)$ are $0$ for every $n\geq 1$, i.e.
            \begin{align*}
                \sum_{p+q+r = n} (-1)^{pq+r} m_{p+1+r} \circ_{p+1} m_q = 0\tt{.}
            \end{align*}
        \end{proof}

        Given an $A_\infty$ algebra $A$, we may either think of it as a differential tensor coalgebra $\overline{T}^c(A[1])$ with differential $m: \overline{T}^c(A[1])\rightarrow \overline{T}^c(A[1])$, or as a graded vector space with morphisms $m_n: A^{\otimes n} \rightarrow A$ satisfying $(\text{rel}_n)$. We will calculate $(\text{rel}_n)$ for $n=1,2,3$:
        \begin{align*}
            (\text{rel}_1)\qquad & m_1\circ m_1 = 0 \\
            (\text{rel}_2)\qquad & m_1\circ m_2 - m_2\circ_{1}m_1 - m_2\circ_2m_1 = 0 \\
            (\text{rel}_3)\qquad & m_1\circ m_3 - m_2\circ_1 m_2 + m_2\circ_2m_2 + m_3\circ_1m_1 + m_3\circ_2m_1 + m_3\circ_3m_1 = 0 
        \end{align*}

        We see that $(\text{rel}_1)$ states that $m_1$ should be a differential. Thus we may think of $(A, m_1)$ as a chain complex. Furthermore, $(\text{rel}_2)$ says that $m_2 : (A^{\otimes 2}, m_1\otimes id_A + id_A\otimes m_1) \rightarrow (A, m_1)$ is a morphism of chain complexes. Lastly, $(\text{rel}_3)$ gives us a homotopy for the associator of $m_2$, namely $m_3$. Thus we may regard $(A, m_1, m_2)$ as an algebra that is associative up to the homotopy $m_3$. Regarding $A$ as a chain complex, instead, we obtain our final equivalent definition of an $A_\infty$-algebra.

        \begin{proposition}
            Suppose that $(A, d)$ is a chain complex and that there exist morphisms $m_n: A^{\otimes n} \rightarrow A$ of degree $2-n$ for any $n\geq 2$. A is an $A_\infty$-algebra if and only it satisfies the following relations:
            \begin{align*}
                (\text{rel'}_n)\qquad & \partial(m_n) = -\sum_{\substack{n = p + q + r \\ k = p + 1 + r \\ k > 1, q > 1}}(-1)^{pq + r}m_k\circ_{p+1}m_q
            \end{align*}
        \end{proposition}

        We define the homotopy of an $A_\infty$-algebra to be the homology of the chain complex $(A, m_1)$. Since $\partial(m_3) = m_2\circ_1m_2 - m_2\circ_2m_2$, we get that $m_2$ is associative in homology. Thus for any $A_\infty$-algebra $A$, the homotopy $HA$ is an associative algebra. The operadic homology of $A$ is defined as the homology of $(T^c(A[1]), m)$, which is the non-unital augmented Hochschild homology of $A$.

        \begin{example}
            Suppose that $V$ is a cochain complex with differential $d$. Then $V$ is an $A_\infty$-algebra with trivial multiplication. In other words $m^1 = d$ and $m^i = 0$ for any $i>1$.
        \end{example}

        \begin{example}
            Suppose that $A$ is a dg-algebra. Then $A$ is an $A_\infty$-algebra where $m^1 = d$, $m^2 = (\cdot)$ and $m^i=0$ for any $i>2$.
        \end{example}

        % \begin{example}
        %     Let $A$ be a connected weight-graded algebra over $\mathbb{K}$. We may then think of $\mathbb{K}$ as an $A$-module with trivial action, i.e. as the quotient $\sfrac{A}{A^{(i)}}_{(i\geq 1)}$, then $Ext^*_A(\mathbb{K},\mathbb{K})$ is an $A_\infty$-algebra. Lu, Palmari, Wu, and Zhang showed this \cite{Lu06}.
        % \end{example}

        Next, we want to understand the category of $A_\infty$-algebras. A morphism between $A_\infty$-algebras is called an $\infty$-morphism. We define such an $\infty$-morphism $f:A\rightsquigarrow B$ between $A_\infty$-algebras as associated dg-coalgebra homomorphism $Bf:(\overline{T}^c(A[1]), m^A)\rightarrow (\overline{T}^c(B[1]), m^B)$. Here $Bf$ is purely formal, and we will make sense of this soon.

        \begin{proposition}
            Let $A,B$ be two $A_\infty$-algebras. A collection of morphisms $f_n : A^{\otimes n} \rightarrow B$ of degree $1-n$ for any $n \geq 1$ defines an $\infty$-morphism $f : A \rightsquigarrow B$ if and only if $f_1$ is a morphism of chain complexes and for any $n\geq 2$ the following relations are satisfied:
            \begin{align*}
                & (\text{rel}_n)\qquad \partial(f_n) = \sum_{\substack{p + 1 + r = k \\ p + q + r = n}}(-1)^{pq+r}f_k\circ_{p+1}m^A_q - \sum_{\substack{k\geq 2 \\ i_1 + ... + i_k = n}}(-1)^{e}m^B_k \circ (f_{i_1}\otimes f_{i_2}\otimes ... \otimes f_{i_k})\tt{,}
            \end{align*}
            where $e$ is
            \begin{align*}
                e = \sum_{l = 1}^k(1-i_l)\sum_{1\leq m < l}i_m\tt{.}
            \end{align*}
        \end{proposition}
        
        \begin{proof}
            Establishing the shape of this equation is immediate by the universal property of cofree coalgebras. We obtain the parity $e$ by factoring the $s$ to the right.
            \begin{align*}
                (f_{i_1} \otimes \cdots \otimes f_{i_k}) \circ s^{\otimes n} = (-1)^e(f_{i_1}s^{\otimes i_1} \otimes \cdots \otimes f_{i_k}s^{\otimes i_k})\tt{.}
            \end{align*}
            By Proposition \ref{prop: multi-koszul-sign}, we arrive at the conclusion,
            \begin{align*}
                e = \sum_{l = 1}^k|f_{i_l}|\sum_{1\leq m < l}|s^{\otimes i_m}|= \sum_{l = 1}^k(1- i_l)\sum_{1 \leq m < l}i_m
            \end{align*}
        \end{proof}

        Since the composition of two dg-coalgebra homomorphisms is again a dg-coalgebra homomorphism, we get that the composition of two $\infty$-morphisms is again an $\infty$-morphism. More explicitly if $f:A\rightsquigarrow B$ and $g: B\rightsquigarrow C$ are two $\infty$-morphisms, then their composition is defined as
        \begin{align*}
            (fg)_n = \sum_r\sum_{i_1 + ... + i_r = n} (-1)^eg_r(f_{i_1}\otimes ... \otimes f_{i_r})\text{.}
        \end{align*}

        Here $e$ denotes the same parity as above.

        \begin{definition}
            An $\infty$-morphism $f: A\rightsquigarrow B$ is called strict if $f_n = 0$ for any $n\geq 2$. 
        \end{definition}

        \begin{definition}
            $\tt{Alg}_\infty$ denotes the category of $A_\infty$-algebras, and the morphisms in this category are the $\infty$-morphisms.
        \end{definition}

        Observe that we may extend the bar construction to $B: \tt{Alg}_\infty \rightarrow \tt{Co\tt{Alg}}_{\mathbb{K},\tt{conil}}^\bullet$ to a fully faithful functor. This construction may be done explicitly by using Proposition \ref{prop: tensor-derivation}. The subcategory of the essential image is the full subcategory of every quasi-cofree dg-coalgebra. Notice that the bar construction on the category of dg-algebras is a non-full injection into the category of $A_\infty$-algebras. This inclusion gives us a recontextualization of a dg-algebra as an $A_\infty$-algebra.

        A quasi-isomorphism between $A_\infty$-algebras is called an $\infty$-quasi-isomorphism. Given an $\infty$-morphism $f: A\rightsquigarrow B$, we say that it is an $\infty$-quasi-isomorphism if $f_1$ is a quasi-isomorphism. If we wanted to be more stringent with this definition, we would define an $\infty$-quasi-isomorphism to be an $\infty$-morphism which is a quasi-isomorphism of dg-coalgebras. We will later see that these definitions are equivalent.

        A homotopy between two $A_\infty$-algebras is a homotopy between the dg-coalgebras they define. We may trace this definition back along the quasi-inverse of the bar construction to get a new definition in terms of many morphisms. Let $f,g: A\rightsquigarrow B$ be two $\infty$-morphisms, we say that $f-g$ is null-homotopic if there is a collection of morphisms $h_n : A^{\otimes n} \rightarrow B$ of degree $n$ such that the following relations are satisfied for any $n \geq 1$:
        \begin{multline*}
            f_n - g_n = \sum (-1)^s m^B_{r+1+t}\circ (f_{i_1}\otimes ... \otimes f_{i_r} \otimes h_k \otimes g_{j_1} \otimes ... \otimes g_{j_t}) +  \sum (-1)^{j+kl}h_i \circ_{j+1} m_k^A\tt{.}
        \end{multline*}
        $s$ is some constant depending on $t,r$, and $k$, and more details may be found in \cite{LefevreHasegawa03}. One may observe that this definition of homotopy is exactly the same as requiring that the morphisms $Bf$ and $Bg$ are homotopic by a $(Bf, Bg)$-coderivation $Bh$.

        As in the same case for algebras, there is also a notion of unital $A_\infty$-algebras and augmented $A_\infty$-algebras. For this discussion, it is essential to observe that the field $\mathbb{K}$ is also an $A_\infty$-algebra. This algebra will be the initial algebra like it does for ordinary algebras.

        \begin{definition}\label{def: strict-unit}
            A strictly unital $A_\infty$-algebra is an $A_\infty$-algebra $A$ together with a unit morphism $\upsilon_A : \mathbb{K} \rightarrow A$ of degree $0$ such that the following are satisfied:
            \begin{itemize}
                \item $m_1\circ \upsilon_A = 0$.
                \item $m_2(id_A \otimes \upsilon_A) = id_A = m_2(\upsilon_A\otimes id_A)$.
                \item $m_i \circ_k \upsilon_A = 0$ for any $i\geq 3$ and $1 \leq k < i$.
            \end{itemize}
        \end{definition}

        A strictly unital $\infty$-morphism $f: A \rightsquigarrow B$ between strictly unital $A_\infty$-algebras is a morphism that preserves the unit. This means that $f_1\upsilon_A = \upsilon_B$ and $f_i \circ_k \upsilon_A = 0$ for any $i \geq 2$ and $1 \leq k < i$. The collection of strictly unital $A_\infty$-algebras and strictly unital $\infty$-morphisms form a non-full subcategory of $A_\infty$-algbras. A strict $\infty$-morphism which is unital at the level of chain complexes is automatically strictly unital. Strict unital will then mean strict and strictly unital. Note that $\mathbb{K}$ is strictly unital where the unit is $id_\mathbb{K}$.

        \begin{definition}\label{def: augmented-sha}
            An augmented $A_\infty$-algebra is a strictly unital $A_\infty$-algebra $A$ together with a strict unital morphism $\varepsilon_A: A \rightarrow \mathbb{K}$. The $\infty$-morphism $\varepsilon_A$ is called the augmentation of $A$.
        \end{definition}

        The collection of augmented $A_\infty$-algebras and strictly unital morphism is the category of augmented $A_\infty$-algebras, denoted as $\tt{Alg}_{\infty,+}$. As in the same way for algebras, there is an equivalence of categories $\tt{Alg}_\infty \simeq \tt{Alg}_{\infty,+}$. The augmentation ideal, or the reduced $A_\infty$-algebra, is the kernel of the augmentation $\varepsilon_A$. It does not make sense to talk about this limit a priori, as we do not know if it exists. However, we will see in section 2.3.3 that such morphisms have kernels. This defines a functor, $\overline{\underline{\phantom{A}}} : \tt{Alg}_{\infty ,+} \rightarrow \tt{Alg}_\infty$, where $\tt{Ker}\varepsilon_A = \overline{A}$. Free augmentations give the quasi-inverse to this functor. Given an $A_\infty$-algebra $A$, we may construct the $A_\infty$-algebra $A\oplus \mathbb{K}$. The structure morphisms are given by $m_i^A$, but there is now a unit $\upsilon_{A\oplus\mathbb{K}}$. Thus we get that $m_1(1) = 0$, $m_2 (a\otimes 1) = a$ and $m_i \circ_k 1 = 0$ in the same manner. We obtain a functor $\argument^+ : \tt{Alg}_\infty \rightarrow \tt{Alg}_{\infty, +}$, where $A\oplus \mathbb{K} = A^+$.

    \subsection{$A_\infty$-Coalgebras}    
        Dual to $A_\infty$-algebras, we got conilpotent $A_\infty$-coalgebras. Here we ask ourselves if the cobar construction has some converse, i.e., if $C$ is a graded vector space such that $T(C[-1])$ together with a derivation $m$ is a dg-algebra, is then $C$ a coalgebra? Again, the answer to this is no, but we obtain a definition for conilpotent $A_\infty$-coalgebras.

        \begin{definition}
            A graded vector space $C$ is called a conilpotent $A_\infty$-coalgebra if it is a dg-algebra of the form $(\overline{T}(C[-1]), d)$ where $d$ is a derivation of degree $1$.
        \end{definition}

        \begin{remark}
            For the rest of this thesis, an $A_\infty$-coalgebra should be understood as a conilpotent $A_\infty$-coalgebra unless otherwise specified.
        \end{remark}

        \begin{corollary}
            $C$ is an $A_\infty$-coalgebra with differential $d$ then there is a chain complex $(C, d^1)$, where $d^1$ is of degree $1$, and together with morphisms $d^n : C \rightarrow C^{\otimes n}$ such that $d$ uniquely determines each $d^i$ for any $i>0$. Conversely, if the morphisms $d^i$ satisfy $\text{(rel)}_n$, then they uniquely determine a $d$ such that $C$ is an $A_\infty$-coalgebra,
            \begin{align*}
                (\text{rel}_n)\tt{ is}\qquad & \sum_{p+q+r = n}(-1)^{pq+r}d^{p+1+q}\circ^{op}_{p+1}d^q = 0
            \end{align*}
        \end{corollary}

        A morphism of $A_\infty$-coalgebras is defined in the same manner as for $A_\infty$-morphisms. An $\infty$-morphism $f: C \rightsquigarrow D$ is then either a morphism $\widetilde{f}: (T(C[-1]), m^C) \rightarrow (T(D[-1]),m^D)$ of dg-algebras; or equivalently it is a collection of morphisms $f_n : C \rightarrow D^{\otimes n}$ of degree $1-n$ such that $f_1$ is a morphism of chain complexes, and for any $n\geq 2$ the following relations are satisfied:
        \begin{align*}
            & (\text{rel}_n)\qquad \partial(f_n) = \sum_{\substack{p + 1 + r = k \\ p + q + r = n}}(-1)^{pq+r}f_k\circ^{op}_{p+1}m^D_q - \sum_{\substack{k\geq 2 \\ i_1 + ... + i_k = n}}(-1)^{e}m^C_k \circ^{op} (f_{i_1}\otimes f_{i_2}\otimes ... \otimes f_{i_k})\tt{,}
        \end{align*}
        where $e$ is
        \begin{align*}
            e = \sum_{l = 1}^k(1-i_l)\sum_{1\leq m < l}i_m\tt{.}
        \end{align*}

        We denote $\tt{coAlg}_\infty$ as the category of $A_\infty$-coalgebras. Similarly, the cobar construction extends to this category and identifies $A_\infty$-coalgebras and a subcategory of dg-algebras. This subcategory consists of every dg-algebra that is isomorphic, as an algebra, to a free tensor algebra. Lastly, every dg-coalgebra is an $A_\infty$-coalgebra by letting every morphism $m^i = 0$ where $i>2$, and this gives a non-full inclusion. 
\end{document}