\documentclass[../thesis.tex]{subfiles}

\begin{document}

        In Stasheffs papers \todo{Se over disse kildene igjen, slik at jeg ikke dummer meg ut}\cite{Stasheff63I} and \cite{Stasheff63II}, a strongly homotopy associative algebra, or $A_\infty$-algebra, over a field is a graded vector space together with homogenous linear maps $m_n:A^{\otimes n}\rightarrow A$ of degree $n-2$ satisfying some homotopical relations. This will be made precise later. We may regard $m_2$ to be a multiplication of $A$, it is however not a priori associative. The associator of $m_2$ is taken to be the homotopical relation of $m_3$. Thus, we know that the homotopy of $A$ is an associative algebra. The maps $m_n$ corresponds uniquely to a map $m^c:BA\rightarrow \bar{A}[1]$, which extends to a coderivation $m^c : BA\rightarrow BA$ of the bar construction of $A$. So we could instead define an $A_\infty$-algebra to be a coalgebra on the form $BA$.


        In order to understand the bar construction we will first study it on associative algebras. Given a differential graded coassociative coalgebra $C$ and a differential graded associative algebra $A$, we say that a homogenous linear transformation $\alpha: C\rightarrow A$ is twisting if it satisfies the Maurer-Cartan equation:
            \begin{equation*}
                \partial\alpha + \alpha\star\alpha = 0.
            \end{equation*}
        Let $Tw(C,A)$ be the set of twisting morphisms, then considering it as a functor $Tw : CoAlg_{\mathbb{K}}^{op}\times Alg_{\mathbb{K}} \rightarrow Ab$ we want to show that it is represented in both arguments. Moreover, these representations give rise to an adjoint pair of functors, called the bar and cobar construction.

        \begin{center}
            \begin{tikzcd}
                Alg_{\mathbb{K}, +}^\bullet \ar[bend left]{r}[pos=0.55]{B} \ar[phantom]{r}{\top} & Coalg_{\mathbb{K}, conil}^\bullet \ar[bend left]{l}[pos=0.45]{\Omega}
            \end{tikzcd}
        \end{center}

        The bar and cobar construction will be the basis for our discussion of $A_\infty$-algebras. As the bar construction can be used to define $A_\infty$-algebras, we may easily dualize this to define $A_\infty$-coalgebras in terms of the cobar construction. This chapter will follow the notations and progression presented in Loday and Vallete \cite{Loday12} to develop the theory for the bar-cobar adjunction.

    \section{Prelimaries}    
    \subsection{Algebras}

            This section is a review of associative algebras. We will define unital associative algebras and possibly non-unital associative algebras, which we will call algebras and non-unital algebras respectively. The collection of algebras together with homomorphisms between them form the category $Alg_{\mathbb{K}}$ of algebras. Other types of algebras such as augmented and tensor algebras will be defined as well.

            \begin{definition}[Algebra]
                Let $\mathbb{K}$ be a field with unit $1$. An algebra $A$ over $\mathbb{K}$ is a vector space with structure morphisms called multiplication and unit,
                \begin{align*}
                    (\nabla_A) & : A\otimes_{\mathbb{K}}A \rightarrow A \\
                    \upsilon_A & : \mathbb{K} \rightarrow A,
                \end{align*}
                satisfying the associativity and identity laws. 
                \begin{align*}
                    \text{(associativity)}\quad & (a \nabla_A b) \nabla_A c = a \nabla_A (b \nabla_A c) \\
                    \text{(unitality)}\quad & \upsilon_A(1) \nabla_A a = a = a \nabla_A \upsilon_A(1)
                \end{align*}
                Whenever $A$ does not posess a unit morphism, we will call $A$ a non-unital algebra. Only the associativity law must hold.
            \end{definition}

            \begin{definition}[Algebra homomorphisms]
                Let $A$ and $B$ be algebras. Then $f: A\rightarrow B$ is an algebra homomorphism if
                \begin{enumerate}
                    \item $f$ is $\mathbb{K}$-linear
                    \item $f(ab)=f(a)f(b)$
                    \item $f\circ\upsilon_A = \upsilon_B$
                \end{enumerate}
                Whenever $A$ and $B$ are non-unital, we only require 1 and 2 for a homomorphism of non-unital algebras.
            \end{definition}

            \begin{definition}[Category of algebras]
                \begin{itemize}
                    \item Let $Alg_{\mathbb{K}}$ denote the category of algebras. It's objects consists of every algebra $A$, and the morphisms are algebra homomorphisms. The sets of morphisms between $A$ and $B$ are denoted as $Alg_{\mathbb{K}}(A,B)$.
                    \item Let $\widehat{}{Alg}_{\mathbb{K}}$ denote the category of non-unital algebras. It's objects consists of every non-unital algebra $A$, and the morphisms are non-unital algebra homomorphisms. The sets of morphisms between $A$ and $B$ are denoted as $\widehat{Alg}_{\mathbb{K}}(A,B)$.
                \end{itemize}
            \end{definition}

            Observe that for an algebra $A$, the triple $(A,\nabla_A,\upsilon_A)$ is a monoid in $mod_{\mathbb{K}}$. Thus, we may say that an algebra is a triple where the following diagrams commute. 
            \begin{center}
                \begin{tikzcd}
                    A \otimes_{\mathbb{K}} A \otimes_{\mathbb{K}} A \ar[]{r}{(\nabla_A)\otimes id_{\mathbb{K}}} \ar[]{d}[]{id_{\mathbb{K}}\otimes (\nabla_A)} & A \otimes_{\mathbb{K}} A \ar[]{d}{(\nabla_A)} \\
                    A \otimes_{\mathbb{K}} A \ar[]{r}[]{(\nabla_A)} & A
                \end{tikzcd} \quad
                \begin{tikzcd}
                    A \otimes_{\mathbb{K}} \mathbb{K} \ar[]{r}[]{id_A \otimes \upsilon_A} \ar[]{rd}[below]{\simeq} & A \otimes_{\mathbb{K}} A \ar[]{d}[]{(\nabla_A)} & \mathbb{K} \otimes_{\mathbb{K}} A \ar[]{l}[above]{\upsilon_A \otimes id_A} \ar[]{ld}[]{\simeq}\\
                    & A
                \end{tikzcd}
            \end{center}
            The final method we will use to represent an algebra are electric circuits. An electric circuit is a diagram read from top to bottom, where each column represent a different vector space in a tensor. Morphisms in such diagrams are figures, conjunctions, twistings and etc. E.g. The multiplication operator may be represented as a converging fork, and the unit as a source.
            \begin{center}
                \begin{tikzpicture}[line cap=round,line join=round,>=triangle 45,x=1cm,y=1cm, thick, op/.style={circle, draw, scale=0.75}, scale=0.7]
                    \node at (-3, 0) {(Multiplication)};
                    
                    \node[op, scale = 0.75] (1) at (0,0) {$\nabla_A$};

                    \draw [line width=1pt] (-0.5, 1) -- (-0.5, 0.5) -- (1) -- (0, -1);
                    \draw [line width=1pt] (0.5, 1) -- (0.5, 0.5) -- (1);


                    \node at (1.5,0) {$=$};

                    \draw [line width=1pt] (2.5,1) -- (2.5, 0.5) -- (3,0) -- (3,-1);
                    \draw [line width=1pt] (3.5,1) -- (3.5, 0.5) -- (3,0);
                \end{tikzpicture} \qquad
                \begin{tikzpicture}[line cap=round,line join=round,>=triangle 45,x=1cm,y=1cm, thick, op/.style={circle, draw, scale=0.75}, scale=0.7]
                    \node at (-1.5,0.25) {(Unit)};
                    
                    \node[op, scale = 0.75] (1) at (0,1) {$\upsilon_A$};
                    \draw [line width=1pt] (1) -- (0,-0.75);

                    \node at (1, 0.25) {$=$};

                    \node[op, scale=1] (2) at (1.75,1) {};
                    \draw [line width=1pt] (2) -- (1.75,-0.75);

                    \node at (0,-0.5) {};
                \end{tikzpicture}
            \end{center}
            Using these operations we can now reformulate the algebra laws. These are the electric laws for an algebra:
            \begin{center}
                \begin{tikzpicture}[line cap=round,line join=round,>=triangle 45,x=1cm,y=1cm, thick, op/.style={circle, draw, scale=0.75}, scale=0.7]
                    \node at (-3,0.5) {(Associativity)};

                    \draw [line width=1pt] (-0.75, 1.25) -- (-0.75, 1) -- (-0.5, 0.75) -- (-0.5, 0.5) -- (0,0) -- (0, -0.5);
                    \draw [line width=1pt] (-0.25, 1.25) -- (-0.25, 1) -- (-0.5, 0.75);
                    \draw (0.5, 1.25) -- (0.5, 0.5) -- (0,0);

                    \node at (1.25,0.5) {$=$};

                    \draw [line width=1pt] (2,1.25) -- (2, 0.5) -- (2.5, 0) -- (2.5, -0.5);
                    \draw [line width=1pt] (2.75, 1.25) -- (2.75, 1) -- (3, 0.75) -- (3, 0.5) -- (2.5, 0);
                    \draw [line width=1pt] (3.25, 1.25) -- (3.25, 1) -- (3, 0.75);
                \end{tikzpicture}
            \end{center}
            \begin{center}
                \begin{tikzpicture}[line cap=round,line join=round,>=triangle 45,x=1cm,y=1cm, thick, op/.style={circle, draw, scale=0.75}, scale=0.7]
                    \node at (-2,0) {(unitality)};

                    \node[op, scale=0.75] (1) at (-0.5, 1) {};

                    \draw [line width=1pt] (1) -- (-0.5, 0.5) -- (0,0) -- (0, -0.5);
                    \draw [line width=1pt] (0.5, 1) -- (0.5, 0.5) -- (0,0);

                    \node at (1.25,0.) {$=$};

                    \draw [line width=1pt] (2, 1) -- (2, -0.5);

                    \node at (2.75,0) {$=$};

                    \node[op, scale=0.75] (2) at (4.5, 1) {};
                    \draw [line width=1pt] (3.5, 1) -- (3.5, 0.5) -- (4, 0) -- (4, -0.5);
                    \draw [line width=1pt] (2) -- (4.5, 0.5) -- (4, 0);

                \end{tikzpicture}
            \end{center}

            \begin{definition}[Augmented algebras]
                Let $A$ be an algebra. It is called augmented if there is an algebra homomorphism $\varepsilon : A \rightarrow \mathbb{K}$.
            \end{definition}

            If $A$ is an augmented algebra, then it decomposes into $\mathbb{K}\oplus Ker\varepsilon$ as a module. The splitting is given by unitality of the morphism $\varepsilon: A \rightarrow \mathbb{K}$, as we know that $\varepsilon(\upsilon_A) = id_{\mathbb{K}}$. The kernel of $\varepsilon$ is called the augmentation ideal or redecued algebra and we will denote it as $\bar{A}$. Taking kernels gives an equivalence of categories between augmented algebras and non-unital algebras, with unitization as the quasi-inverse. The category of augmented algebras is denoted as $Alg_{\mathbb{K},+}$ or $Alg_{\mathbb{K},+}$.

            \begin{definition}[Tensor algebra]
                Let $V$ be a $\mathbb{K}$-module. We define the tensor algebra $T(V)$ of $V$ as the module
                \begin{align*}
                    T(V) = \mathbb{K}\oplus V\oplus V^{\otimes 2} \oplus V^{\otimes 3} \oplus ...
                \end{align*}
                Given two strings $v^1..v^i$ and $w^1...w^j$ in $T(V)$ we define the multiplication by the concatenation operation.
                \begin{align*}
                    \nabla_{T(V)} : T(V)\otimes_{\mathbb{K}} T(V) & \rightarrow T(V) \\
                    (v^1...v^i)\otimes(w^1...w^j) & \mapsto v^1...v^iw^1...w^j
                \end{align*}
                The unit is given by including $\mathbb{K}$ into $T(V)$.
                \begin{align*}
                    \upsilon_{T(V)} : \mathbb{K} & \rightarrow T(V) \\
                    1 & \mapsto 1
                \end{align*}
            \end{definition}

            Observe that the tensor algebra is augmented. The projection from $T(V)$ into $\mathbb{K}$ is an algebra homomorphism, so we may split the tensor algebra into its unit and its augmentation ideal $T(V) \simeq \mathbb{K}\oplus\bar{T}(V)$. We call $\bar{T}(V)$ the reduced tensor algebra.

            \begin{proposition}[Tensor algebra is free]
                The tensor algebra is the free algebra over the category of $\mathbb{K}$-modules, i.e. for any $\mathbb{K}$-module $V$ there is a natural isomorphism $Hom_{\mathbb{K}}(V,A)\simeq Alg_{\mathbb{K}}(T(V),A)$.

                The reduced tensor algebra is the fre non-unital algebra over the category of $\mathbb{K}$-modules, i.e. for any $\mathbb{K}$-module $V$ there is a natural isomorphism $Hom_{\mathbb{K}}(V,A)\simeq \widehat{Alg}_{\mathbb{K}}(\bar{T}(V),A)$.
            \end{proposition}

            \begin{proof}
                This proposition should be evident from the description of an algebra homomorphism from a tensor algebra. If $f: T(V) \rightarrow A$ is an algebra homomorphism, then $f$ must satisfy the following conditions:
                \begin{itemize}
                    \item Unitality:\quad $f(1) = 1$
                    \item Homomorphism property:\quad Given $v,w\in V$, then $f(vw) = f(v)\nabla_Af(w)$
                \end{itemize}
                By induction, we see that $f$ is completely determined by where it sends the elements of $V$. Thus restriction by the inclusion of $V$ into $T(V)$ induces a bijection.
            \end{proof}

            \begin{definition}[Modules]
                Let $A$ be an algebra. A $\mathbb{K}$-module $M$ is said to be a left (right) $A$-module if there exists a structure morphism $\mu_M : A\otimes_{\mathbb{K}}M \rightarrow A$ ($\mu_M : M\otimes_{\mathbb{K}}A \rightarrow A$) called multiplication. We require that $\mu_M$ is associative with respect to the multiplication, and it preserves the unit of $A$. In other words, the electric laws below are satisfied.
                \begin{center}
                    \begin{tikzpicture}[line cap=round,line join=round,>=triangle 45,x=1cm,y=1cm, thick, op/.style={circle, draw, scale=0.75}, scale=0.7]
                        \node at (-2.4,0) {(Associativity)};
                        
                        \node at (0,1.75) {A};
                        \node at (1,1.75) {A};
                        \node at (2,1.75) {M};

                        \node[op, scale = 0.5] (a) at (1,-0.25) {$\mu_M$};

                        \draw [line width=1pt] (0,1.25) -- (0, 1) -- ((0.5, 0.5) -- (0.5, 0.25) -- (a);
                        \draw [line width=1pt] (a) -- (2,0.5) -- (2,1.25);
                        \draw [line width=1pt] (0.5,0.5) -- (1,1) -- (1, 1.25);
                        \draw [line width=1pt] (a) -- (1,-1);
                        
                        \node at (2.5,0) {$=$};
                        
                        \node at (3,1.75) {A};
                        \node at (4,1.75) {A};
                        \node at (5,1.75) {M};

                        \node[op, scale = 0.5] (b) at (4.5,0.5) {$\mu_M$};
                        \node[op, scale = 0.5] (c) at (4,-0.25) {$\mu_M$};

                        \draw [line width=1pt] (3,1.25) -- (3, 0.5) -- (c);
                        \draw [line width=1pt] (c) -- (b) -- (5,1) -- (5,1.25);
                        \draw [line width=1pt] (4, 1.25) -- (4,1) -- (b);
                        \draw [line width=1pt] (c) -- (4,-1);
                    \end{tikzpicture}\\

                    \begin{tikzpicture}[line cap=round,line join=round,>=triangle 45,x=1cm,y=1cm, thick, op/.style={circle, draw, scale=0.75}, scale=0.7]
                        \node at (-2,0) {(unitality)};
    
                        \node at (1.75,1.5) {M};

                        \node[op, scale=0.75] (1) at (0.25, 1) {};
                        \draw [line width=1pt] (1) -- (0.25, 0.75) -- (1,0);
                        \draw [line width=1pt] (1,0) -- (1.75,0.75) -- (1.75, 1);
                        \draw [line width=1pt] (1,0) -- (1,-0.75);
    
                        \node at (2.25,0) {$=$};
    
                        \node at (3,1.5) {M};

                        \draw [line width=1pt] (3,1) -- (3,-0.75);
                    \end{tikzpicture}
                \end{center}
            \end{definition}

            \begin{definition}[A-linear homomorphisms]
                Let $M,N$ be two left $A$-modules. A morphism $f:M\rightarrow N$ is called $A$-linear if it is $\mathbb{K}$-linear and for any $a$ in $A$, i.e. $f(am) = af(m)$.
            \end{definition}

            The category of left $A$-modules is denoted as $Mod_A$, where the morphisms $Hom_A(\_,\_)$ are $A$-linear. Likewise, the category of right $A$-modules is denoted as $Mod^A$. One may check that there is a free functor from $Mod_\mathbb{K}$ to $Mod_A$.

            \begin{proposition}
                Let $M$ be a $\mathbb{K}$-module. The module $A\otimes_{\mathbb{K}}M$ is a left $A$-module. Moreover, it is the free left module over $\mathbb{K}$-modules, i.e. there is an isomorphism $Hom_{\mathbb{K}}(M,N)\simeq Hom_{A}(A\otimes_{\mathbb{K}}M,N)$.
            \end{proposition}
            
    \subsection{Coalgebras}
            This section aims to dualize the definitions from last section. To this end we will define counital coassociative coalgebras and non-counital coassociative coalgebras, which will be called coalgebras and non-counital coalgebras respectively. The collection of coalgebras together with coalgebra homomorphisms is the category $CoAlg_{\mathbb{K}}$. Due to some ill-behavior, this dualization is only a true dualization under some finiteness conditions for the coalgebras. Thus we will see that the proper dual concept will be thath of conilpotent coalgebras.

            \begin{definition}[Coalgebra]
                Let $\mathbb{K}$ be a field. A coalgebra $C$ over $\mathbb{K}$ is a $\mathbb{K}$-module with structure morphisms called comultiplication and counit,
                \begin{align*}
                    (\Delta_C) & : C \rightarrow C\otimes_{\mathbb{K}}C \\
                    \varepsilon_C & : C \rightarrow \mathbb{K},
                \end{align*}
                satisfying the coassociativity and coidentity laws. 
                \begin{align*}
                    \text{(coassociativity)} \quad & (\Delta_C\otimes id_C)\circ\Delta_C(c) = (id_C\otimes\Delta_C)\circ\Delta_C(c) \\
                    \text{(counitality)} \quad & (id_C\otimes\varepsilon_C)\circ\Delta_C(c) = c = (\varepsilon_C\otimes id_C)\circ\Delta_C(c)
                \end{align*}
            \end{definition}

            We define repeated application of comultiplication as $\Delta_C^n = (\Delta_C\otimes id_C\otimes ...)\circ\Delta_C^{n-1}$. Notice that the choice of where we put comultiplication in the tensor does not matter, as coassociativity require all of the choices to be equal.
            
            We may dualize the electric circuits of an algebra to coalgebras. In this manner our structure morphisms would be upside down relative to the algebra morphisms. Thus comultiplication becomes a diverging fork and counit becomes a sink. 
            \begin{center}
                \begin{tikzpicture}[line cap=round,line join=round,>=triangle 45,x=1cm,y=1cm, thick, op/.style={circle, draw, scale = 0.75}, scale = 0.7]
                    \node at (-3.5, 0) {(Comultiplication)};
                    
                    \node[op, scale=0.75] (1) at (0,0) {$\Delta_C$};

                    \draw [line width=1pt] (0,1) -- (1) -- (-0.5, -0.5) -- (-0.5, -1);
                    \draw [line width=1pt] (1) -- (0.5, -0.5) -- (0.5, -1);

                    \node at (1.5,0) {$=$};

                    \draw [line width=1pt] (3, 1) -- (3, 0) -- (2.5, -0.5) -- (2.5, -1);
                    \draw [line width=1pt] (3, 0) -- (3.5, -0.5) -- (3.5, -1);

                \end{tikzpicture} \qquad
                \begin{tikzpicture}[line cap=round,line join=round,>=triangle 45,x=1cm,y=1cm, thick, op/.style={circle, draw, scale=0.75}, scale=0.7]
                    \node at (-2.5, 0) {(Counit)};
                    
                    \node[op, scale=0.75] (1) at (0,-1) {$\varepsilon_C$};
                    \draw [line width=1pt] (0, 1) -- (1);

                    \node at (1, 0) {$=$};

                    \node[op, scale=1] (2) at (2, -1) {};
                    \draw [line width=1pt] (2, 1) -- (2);

                    \node at (0,0.5) {};
                \end{tikzpicture}
            \end{center}
            We then obtain the electric laws for a coalgebra by flipping the circuits around.
            \begin{center}
                \begin{tikzpicture}[line cap=round,line join=round,>=triangle 45,x=1cm,y=1cm, thick, op/.style={circle, draw, scale=0.75}, scale=0.7]
                    \node at (-3.5,0) {(Coassociativity)};

                    \draw [line width=1pt] (0, 1) -- (0, 0) -- (-0.5, -0.5) -- (-0.5, -0.75) -- (-0.75, -1) -- (-0.75, -1.25);
                    \draw [line width=1pt] (-0.5, -0.75) -- (-0.25, -1) -- (-0.25, -1.25);
                    \draw [line width=1pt] (0, 0) -- (0.5, -0.5) -- (0.5, -1.25);

                    \node at (1.5,0) {$=$};

                    \draw [line width=1pt] (3, 1) -- (3, 0) -- (3.5, -0.5) -- (3.5, -0.75) -- (3.25, -1) -- (3.25, -1.25);
                    \draw [line width=1pt] (3.5, -0.75) -- (3.75, -1) -- (3.75, -1.25);
                    \draw [line width=1pt] (3, 0) -- (2.5, -0.5) -- (2.5, -1.25);
                \end{tikzpicture}
            \end{center}
            \begin{center}
                \begin{tikzpicture}[line cap=round,line join=round,>=triangle 45,x=1cm,y=1cm, thick, op/.style={circle, draw, scale=0.75}, scale=0.7]
                    \node at (-3,0) {(Counitality)};

                    \node[op, scale=0.75] (1) at (-0.5, -1) {};

                    \draw [line width=1pt] (0, 1) -- (0, 0) -- (-0.5, -0.5) -- (1);
                    \draw [line width=1pt] (0, 0) -- (0.5, -0.5) -- (0.5, -1);

                    \node at (1,0) {$=$};

                    \draw [line width=1pt] (2,1) -- (2,-1);

                    \node at (3,0) {$=$};
                    
                    \node[op, scale=0.75] (2) at (4.5,-1) {};

                    \draw [line width=1pt] (4, 1) -- (4, 0) -- (3.5, -0.5) -- (3.5, -1);
                    \draw [line width=1pt] (4, 0) -- (4.5, -0.5) -- (2);
                \end{tikzpicture}
            \end{center}

            \begin{definition}[Coalgebra homomorphism]
                Let $C$ and $D$ be coalgebras. Then $f:C\rightarrow D$ is a coalgebra morphism if
                \begin{enumerate}
                    \item $f$ is $\mathbb{K}$-linear
                    \item $(f\otimes f)\circ\Delta_C(c) = \Delta_D(f(c))$
                    \item $\varepsilon_D(f) = \varepsilon_C$
                \end{enumerate}
                Whenever $C$ and $D$ are non-counital, we only require 1 and 2 for a homomorphism of non-counital coalgebras.
            \end{definition}

            \begin{definition}[Category of Coalgebras]
                \begin{itemize}
                    \item Let $CoAlg_{\mathbb{K}}$ denote the category of coalgebras. It's objects consists of every coalgebra $C$, and the morphisms are coalgebra homomorphisms. The sets of morphisms between $C$ and $D$ are denoted as $CoAlg_{\mathbb{K}}(C,D)$.
                    \item Let $\widehat{CoAlg}_{\mathbb{K}}$ denote the category of non-unital algebras. It's objects consists of every non-unital algebra $C$, and the morphisms are non-unital algebra homomorphisms. The sets of morphisms between $C$ and $D$ are denoted as $\widehat{CoAlg}_{\mathbb{K}}(C,D)$.
                \end{itemize}
            \end{definition}

            \begin{example}[The coalgebra $\mathbb{K}$]
                The field $\mathbb{K}$ can be given a coalgebra structure over itself. Since $\{1\}$ is a basis for $\mathbb{K}$ we define the structure morphisms as
                \begin{align*}
                    \Delta_{\mathbb{K}}(1) & = 1\otimes 1 \\
                    \varepsilon(1) & = 1.
                \end{align*}
                One may check that these morphisms are indeed coassociative and counital. Thus we may regard our field as either an algebra or coalgebra over itself.
            \end{example}

            \begin{definition}[Coaugmented coalgebras]
                Let $C$ be a coalgebra. $C$ is coagumented if there is a coalgebra homomorphism $\upsilon:\mathbb{K}\rightarrow C$.
            \end{definition}

            If $C$ is a coaugmented coalgebra, then it splits as $C\simeq \mathbb{K}\oplus Cok\upsilon$. The splitting is given by counitality of $\upsilon$, as $\varepsilon_C(\upsilon) = id_{\mathbb{K}}$. We call the cokernel $Cok\upsilon = \bar{C}$ for the coaugmentation quotient or reduced coalgebra, and its reduced coproduct may be explicitly given as
            \begin{align*}
                \bar{\Delta}_C(c) = \Delta_C(c) - 1\otimes c - c\otimes 1. 
            \end{align*}

            \begin{definition}[Tensor Coalgebras]
                Let $V$ be a $\mathbb{K}$-module. We define the tensor coalgebra $T^c(V)$ of $V$ as the module
                \begin{align*}
                    T^c(V) = \mathbb{K}\oplus V\oplus V^{\otimes 2}\oplus V^{\otimes 3}\oplus ...
                \end{align*}
                Given a string $v^1...v^i$ in $T(V)$ we define the comultiplication by the deconcatenation operation.
                \begin{align*}
                    \Delta_{T^c(V)}:T^c(V) & \rightarrow T^c(V)\otimes_{\mathbb{K}}T^c(V) \\
                    v^1...v^i & \mapsto 1\otimes(v^1...v^i) + (\sum_{j=1}^{n-1} (v^1...v^{j})\otimes(v^{j+1}...v^i)) + (v^1...v^i)\otimes 1
                \end{align*}
                The counit is given by projecting $T^c(V)$ onto $\mathbb{K}$.
                \begin{align*}
                    \varepsilon_{T^c(V)} : T^c(V) & \rightarrow \mathbb{K} \\
                    1 & \mapsto 1 \\
                    v^1...v^i & \mapsto 0
                \end{align*}
            \end{definition}

            Notice that the tensor coalgebra is coaugmented. Its coaugmentation is given by the inclusion of $\mathbb{K}$ into $T^c(V)$. We may split $T^c(V) \simeq \mathbb{K}\oplus \bar{T^c}(V)$, where $\bar{T^c}(V)$ is the reduced tensor coalgebra.

            In order to get cofreeness for the tensor coalgebra we need some finiteness conditions. This is one of the properties which is ill-behaved when we are dualizing the tensor algebra. The extra assumption which we will need is to assume that the coalgebras are conilpotent. Let $C \simeq \mathbb{K} \oplus \bar{C}$ be a coaugmented coalgebra, we define the coradical filtration of $C$ as a filtration $Fr_0C \subseteq Fr_1C \subseteq ... \subseteq Fr_rC \subseteq ...$ by the submodules:
            \begin{align*}
                Fr_0C & = \mathbb{K} \\
                Fr_rC & = \mathbb{K} \oplus \{c\in\bar{C}\mid \forall n\geq r \bar{\Delta}_C(c) = 0\}.
            \end{align*}

            \begin{definition}[Conilpotent coalgebras]
                Let $C$ be a coaugmented coalgebra. We say that $C$ is conilpotent if its coradical filtration is exhaustive, i.e. $\substack{\varinjlim \\ r}Fr_rC \simeq C$. The subcategory of conilpotent coalgebras will be denoted as $CoAlg_{\mathbb{K},conil}$.
            \end{definition}
            
            \begin{proposition}[Conilpotent tensor coalgebra]\label{prop: conilpotent-tensor}
                Let $V$ be a $\mathbb{K}$-module. The tensor coalgebra $T^c(V)$ is conilpotent.
            \end{proposition}

            \begin{proof}
                Let $v\in V$, then $\Delta_{T^c(V)}(v)=1\otimes v + v\otimes 1$ and $\bar{\Delta}_{T^c(V)}(v)=0$. We then observe the following:
                \begin{align*}
                    Fr_0T^c(V) & = \mathbb{K} \\
                    Fr_1T^c(V) & = \mathbb{K} \oplus V \\
                    Fr_rT^c(V) & = \bigoplus_{i\leq r} V^{\otimes i}
                \end{align*}
                This shows that the coradical filtration is exhaustive.
            \end{proof}

            \begin{proposition}[Cofree tensor coalgebra]\label{prop: cofree-tensor}
                The tensor coalgebra is the cofree conilpotent coalgebra over the category of $\mathbb{K}$-modules, i.e. for any $\mathbb{K}$-module $V$ and any conilpotent coalgebra $C$ there is a natural isomorphism $Hom_{\mathbb{K}}(\bar{C}, V)\simeq CoAlg_{\mathbb{K}, conil}(C, T^c(V))$.
            \end{proposition}

            \begin{proof}
                This proposition should be evident from the description of a coalgebra homomorphism into the a tensor coalgebra. If $g:C\rightarrow T^c(V)$ is a coalgebra homomorphism, then $g$ must satisfy the following conditions:
                \begin{enumerate}
                    \item (Coaugmentation)\quad $g(1)=1$
                    \item (Counitality)\quad Given $c\in \bar{C}$ then $\varepsilon_{T^c(V)}\circ g(c)=0$
                    \item (Homomorphism property)\quad Given $c\in C$ then $\Delta_{T^c(V)}(g(c))=(g\otimes g)\circ\Delta_C(c)$
                \end{enumerate}

                We will construct the maps for the isomorphism explicitly. If $g:C\rightarrow T^c(V)$ is a coalgebra homomorphism, then composing with projection gives a map $\pi\circ g:C\rightarrow V$. Note that $\pi\circ g(1)=0$, so this is essentially a map $\pi\circ g:\bar{C}\rightarrow V$. For the other direction, let $\bar{g}:\bar{C}\rightarrow V$. We will then define $g$ as
                \begin{align*}
                    g = id_{\mathbb{K}} \oplus \sum_{i=1}^{\infty}(\otimes^{i}\bar{g})\bar{\Delta}_C^{i-1}.
                \end{align*}
                Observe that $g$ is well defined, since convergence of the sum follows from conilpotency of $C$. One may then check that $g$ is a coalgebra homomorphism, which yields the result.
            \end{proof}

            \begin{definition}[Comodules]
                Let $C$ be a coalgebra. A $\mathbb{K}$-module $M$ is said to ba left (right) $C$-comodule if there exist a structure morphism $\omega_M: M \rightarrow C\otimes_{\mathbb{K}}M$ ($\omega_M: M \rightarrow M\otimes_{\mathbb{K}}C$) called comultiplication. We require that $\omega_M$ is coassociative with respect to the comultiplication of $C$ and preserves the counit of $C$, i.e. the electric laws are satisfied.
                \begin{center}
                    \begin{tikzpicture}[line cap=round,line join=round,>=triangle 45,x=1cm,y=1cm, thick, op/.style={circle, draw, scale=0.75}, scale=0.7]
                        \node at (-3.5,0) {(Coassociativity)};

                        \node at (0, 1.5) {M};
                        \node[op, scale=0.5] (1) at (0, 0) {$\omega_M$};
    
                        \draw [line width=1pt] (0, 1) -- (1) -- (-0.5, -0.5) -- (-0.5, -0.75) -- (-0.75, -1) -- (-0.75, -1.25);
                        \draw [line width=1pt] (-0.5, -0.75) -- (-0.25, -1) -- (-0.25, -1.25);
                        \draw [line width=1pt] (1) -- (0.5, -0.5) -- (0.5, -1.25);
    
                        \node at (1.5,0) {$=$};

                        \node at (3, 1.5) {M};
                        \node[op, scale=0.5] (2) at (3, 0) {$\omega_M$};
                        \node[op, scale=0.5] (3) at (3.5, -0.75) {$\omega_M$};
    
                        \draw [line width=1pt] (3, 1) -- (2) -- (3) -- (3.25, -1) -- (3.25, -1.25);
                        \draw [line width=1pt] (3) -- (3.75, -1) -- (3.75, -1.25);
                        \draw [line width=1pt] (2) -- (2.5, -0.5) -- (2.5, -1.25);
                    \end{tikzpicture}
                \end{center}
                \begin{center}
                    \begin{tikzpicture}[line cap=round,line join=round,>=triangle 45,x=1cm,y=1cm, thick, op/.style={circle, draw, scale=0.75}, scale=0.7]
                        \node at (-3,0) {(Counitality)};
    
                        \node[op, scale=0.75] (1) at (-0.5, -1) {};
    
                        \draw [line width=1pt] (0, 1) -- (0, 0) -- (-0.5, -0.5) -- (1);
                        \draw [line width=1pt] (0, 0) -- (0.5, -0.5) -- (0.5, -1);
    
                        \node at (1,0) {$=$};
    
                        \draw [line width=1pt] (2,1) -- (2,-1);
    
                        \node at (3,0) {$=$};
                        
                        \node[op, scale=0.75] (2) at (4.5,-1) {};
    
                        \draw [line width=1pt] (4, 1) -- (4, 0) -- (3.5, -0.5) -- (3.5, -1);
                        \draw [line width=1pt] (4, 0) -- (4.5, -0.5) -- (2);
                    \end{tikzpicture}
                \end{center}
            \end{definition}

            \begin{definition}[C-colinear homomorphism]
                Let $M,N$ be two left $C$-comodules. A morphism $g:M\rightarrow N$ is called $C$-colinear if it is $\mathbb{K}$-linear and for any $m$ in $M$, $\omega_N(g(m)) = (id_C\otimes g)\omega_M(m)$.
            \end{definition}

            The category of left $C$-comodules is denoted as $CoMod_C$, where the morphisms $Hom_C(\_,\_)$ are $C$-colinear. We would also like to restrict our attention to those $C$-comodules which are conilpotent, i.e. those comodules which have an exhaustive coradical filtration. Notice that for conilpotent coalgebras this requirement is automatic. Likewise, the category of right $C$-comodules is denoted as $CoMod^C$.

            \begin{proposition}
                Let $M$ be a $\mathbb{K}$-module. The module $C\otimes_{\mathbb{K}}M$ is a left $C$-comodule. Moreover, it is the cofree left comodule over $\mathbb{K}$-modules, i.e. there is an isomorphism $Hom_{\mathbb{K}}(N,M)\simeq Hom_C(N,C\otimes_{\mathbb{K}}M)$. 
            \end{proposition}

    \subsection{Derivations and DG-Algebras}
            In this section we will look at differential graded objects and convolution products. We will define derivations and coderivations to obtain differential graded algebras and coalgebras. Moreover we will see that the set of homogenous homomorphisms between differential graded objects is itself differential graded. Moreover, whenever we look at morphisms between dg coalgebras and dg algebras, we can give this object the convolution operator, making the set a dg algebra.

            \begin{definition}[Derivations and Coderivations]
                Let $M$ be an $A$-bimodule. A $\mathbb{K}$-linear morphism $d:A\rightarrow M$ is called a derivation if $d(ab)=d(a)b+ad(b)$, i.e. electrically:

                \begin{center}
                    \begin{tikzpicture}[line cap=round,line join=round,>=triangle 45,x=1cm,y=1cm, thick, op/.style={circle, draw, scale=0.75}, scale=0.7]
                        
                        \node at (-0.5, 1) {a};
                        \node at (0.5, 1) {b};
                        \node[op, scale = 0.75] (d) at (0,-0.5) {d};
                        
                        \draw [line width=1pt] (-0.5, 0.75) -- (-0.5, 0.5) -- (0,0) -- (0.5,0.5) -- (0.5, 0.75);
                        \draw [line width=1pt] (0,0) -- (d) -- (0,-1);
                        
                        \node at (1,0) {$=$};

                        \node at (1.5, 1) {a};
                        \node at (2.5, 1) {b};

                        \node[op, scale = 0.75] (e) at (1.5, 0.25) {d};
                        \node[op, scale = 0.5] (r) at (2,-0.5) {$\mu_M^r$};

                        
                        \draw [line width=1pt] (1.5, 0.75) -- (e) -- (1.5, 0) -- (r) -- (2.5,0) -- (2.5,0.75);
                        \draw [line width=1pt] (r) -- (2,-1);

                        \node at (3,0) {$+$};

                        \node at (3.5, 1) {a};
                        \node at (4.5, 1) {b};

                        \node[op, scale = 0.75] (f) at (4.5, 0.25) {d};
                        \node[op, scale = 0.5] (l) at (4,-0.5) {$\mu_M^l$};

                        
                        \draw [line width=1pt] (3.5, 0.75) --  (3.5, 0) -- (l) -- (4.5,0) -- (f) -- (4.5,0.75);
                        \draw [line width=1pt] (l) -- (4,-1);
                    \end{tikzpicture}
                \end{center}

                Let $N$ be a $C$-bicomodule. A $\mathbb{K}$-linear morphism $d:N\rightarrow C$ is called a coderivation if $\Delta_C\circ d = (d\otimes id_C)\circ\omega_N^r + (id_C\otimes d)\circ\omega_N^l$, i.e. electrically:
                \begin{center}
                    \begin{tikzpicture}[line cap=round,line join=round,>=triangle 45,x=1cm,y=1cm, thick, op/.style={circle, draw, scale=0.75}, scale=0.7]
                        
                        \node[op, scale = 0.75] (d) at (0,0.5) {d};
                        
                        \draw [line width=1pt] (-0.5, -0.75) -- (-0.5, -0.5) -- (0,0) -- (0.5,-0.5) -- (0.5, -0.75);
                        \draw [line width=1pt] (0,0) -- (d) -- (0,1);
                        
                        \node at (1,0) {$=$};

                        \node[op, scale = 0.75] (e) at (1.5, -0.25) {d};
                        \node[op, scale = 0.5] (r) at (2,0.5) {$\omega_N^r$};

                        
                        \draw [line width=1pt] (1.5, -0.75) -- (e) -- (1.5, 0) -- (r) -- (2.5,0) -- (2.5,-0.75);
                        \draw [line width=1pt] (r) -- (2,1);

                        \node at (3,0) {$+$};

                        \node[op, scale = 0.75] (f) at (4.5, -0.25) {d};
                        \node[op, scale = 0.5] (l) at (4,0.5) {$\omega_N^l$};

                        
                        \draw [line width=1pt] (3.5, -0.75) --  (3.5, 0) -- (l) -- (4.5,0) -- (f) -- (4.5,-0.75);
                        \draw [line width=1pt] (l) -- (4,1);
                    \end{tikzpicture}
                \end{center}
            \end{definition}

            \begin{proposition}\label{prop: tensor-derivation}
                Let $V$ be a $\mathbb{K}$-module and $M$ be a $T(V)$-bimodule. A $\mathbb{K}$-linear morphism $f:V\rightarrow M$ uniquely determines a derivation $d_f:T(V)\rightarrow M$, i.e. there is an isomorphism $Hom_{\mathbb{K}}(V,M)\simeq Der(T(V),M)$.


                Let $N$ be a $T^c(V)$-cobimodule. A $\mathbb{K}$-linear morphism $g:M\rightarrow V$ uniquely determines a coderivation $d_g^c:N\rightarrow T^c(V)$, i.e. there is an isomorphism $Hom_{\mathbb{K}}(N,V)\simeq Coder(N,T^c(V))$.
            \end{proposition}

            \begin{proof}
                Let $a_1\otimes ... \otimes a_n$ be an elementary tensor of $T(V)$. We define $d_f(a_1\otimes ... \otimes a_n) = \sum_{i=1}^n a_1...f(a_i)...a_n$ and $d_f(1) = 0$. Notice that $d_f$ is by definition a derivation.
                
                Restriction to $V$ gives the natural isomorphism. Let $i : V\rightarrow T(V)$, then $i^*d_f = f$. Let $d : T(V) \rightarrow M$ be a derivation, then $d_{i^*d}=d$. Suppose that $g: M \rightarrow N$ is a morphism between $T(V)$-bimodules, then naturality follows from bi-linearity.

                In the dual case, $d_g^c$ is a bit tricky to define. Let $\omega^l_N:N\rightarrow N\otimes T^c(V)$ and $\omega^r_N : N\rightarrow T^c(V) \otimes N$ denote the coactions on $N$. Since $T^c(V)$ is conilpotent we get the same kind of finiteness restrictions on $N$. We define the reduced coactions as $\bar{\omega}^l_N = \omega^l_N - \_\otimes 1$ and $\bar{\omega}^r_N = \omega^r_N - 1\otimes\_\ $, this is well-defined by coassociativity. Observe that for any $n\in N$ there are $k, k'>0$ such that ${\bar{\omega}^{l^k}_N}(n) = 0$ and ${\bar{\omega}^{r^{k'}}_N}(n)=0$.

                Let $n_{(k)}^{(i)}$ denote the extension of $n$ by $k$ coactions at position $i$, i.e. $n_{(k)}^{(i)} = \bar{\omega}^{r^i}_N\bar{\omega}^{l^{k-i}}_N(n)$. The extension of $n$ by $k$ coactions is then the sum over every position $i$, $n_{(k)} = \sum_{i=0}^kn_{(k)}^{(i)}$. Observe that $n_{(0)} = n$. The grade of $n$ may be thought of as the smallest $k$ such that $n_{(k)}$ is zero. This grading gives us the coradical filtration of $N$, and it is exhaustive by the finiteness restrictions given above. So every element of $N$ may be given a finite grade.

                If $g: N \rightarrow V$ is a linear map, we may think of it as a map sending every element of $N$ to an element of $T^c(V)$ of grade $1$. To get a map which sends element of grade $k$ to grade $k$, we must extend the morphism. Let $\pi : T^c(V) \rightarrow V$ be the linear projection and define $g_{(k)}^{(i)} = \pi\otimes ... \otimes \pi \circ g \otimes \pi$ as a morphism which is $g$ at the i-th argument, but the projection otherwise. $d_g^c$ is then defined as the sum over each coaction and coordinate.
                \begin{align*}
                    d_g^c(n) = \sum_{k=0}^\infty \sum_{i=0}^k g_{(k)}^{(i)}(n_{(k)}^{(i)})
                \end{align*}
                
                Upon closer inspection we may observe that this is the dual construction of the derivation morphism. It is well-defined as the sum is finite by the finiteness restrictions. The map is a coderivation by duality, and the natural isomorphism is given by composition with the projection map $\pi$.
            \end{proof}

            \begin{definition}[Differential algebra]
                Let $A$ be an algebra. We say that $A$ is a differential algebra if it is equipped with at least one derivation $d:A\rightarrow A$. Dually, a coalgebra $C$ is called differential if it is equipped with at least one coderivation $d:C\rightarrow C$.
            \end{definition}

            \begin{definition}[A-derivation]
                Let $(A,d_A)$ be a differential algebra and $M$ a left $A$-module. A $\mathbb{K}$-linear morphism $d_M:M\rightarrow M$ is called an $A$-derivation if $d_M(am)=d_A(a)m + ad_M(m)$, or electrically:
                \begin{center}
                    \begin{tikzpicture}[line cap=round,line join=round,>=triangle 45,x=1cm,y=1cm, thick, op/.style={circle, draw, scale=0.75}, scale=0.7]
                        
                        \node at (-0.5, 1) {a};
                        \node at (0.5, 1) {m};
                        \node[op, scale = 0.5] (d) at (0,-0.5) {$d_M$};
                        
                        \draw [line width=1pt] (-0.5, 0.75) -- (-0.5, 0.5) -- (0,0) -- (0.5,0.5) -- (0.5, 0.75);
                        \draw [line width=1pt] (0,0) -- (d) -- (0,-1);
                        
                        \node at (1,0) {$=$};

                        \node at (1.5, 1) {a};
                        \node at (2.5, 1) {m};

                        \node[op, scale = 0.5] (e) at (1.5, 0.25) {$d_A$};

                        
                        \draw [line width=1pt] (1.5, 0.75) -- (e) -- (1.5, 0) -- (2,-0.5) -- (2.5,0) -- (2.5,0.75);
                        \draw [line width=1pt] (2,-0.5) -- (2,-1);

                        \node at (3,0) {$+$};

                        \node at (3.5, 1) {a};
                        \node at (4.5, 1) {m};

                        \node[op, scale = 0.5] (f) at (4.5, 0.25) {$d_M$};
                        
                        \draw [line width=1pt] (3.5, 0.75) --  (3.5, 0) -- (4,-0.5) -- (4.5,0) -- (f) -- (4.5,0.75);
                        \draw [line width=1pt] (4,-0.5) -- (4,-1);
                    \end{tikzpicture}
                \end{center}
                Dually, given a differential coalgebra $(C,d_C)$ and $N$ a left $C$-comodule, a $\mathbb{K}$-linear morphism $d_N:N\rightarrow N$ is a coderivation if $\omega_N\circ d_N = (d_C\otimes id_N + id_C\otimes d_N)\circ \omega_N$, or electrically:
                \begin{center}
                    \begin{tikzpicture}[line cap=round,line join=round,>=triangle 45,x=1cm,y=1cm, thick, op/.style={circle, draw, scale=0.75}, scale=0.7]
                        
                        \node[op, scale = 0.5] (d) at (0,0.5) {$d_N$};
                        
                        \draw [line width=1pt] (-0.5, -0.75) -- (-0.5, -0.5) -- (0,0) -- (0.5,-0.5) -- (0.5, -0.75);
                        \draw [line width=1pt] (0,0) -- (d) -- (0,1);
                        
                        \node at (1,0) {$=$};

                        \node[op, scale = 0.5] (e) at (1.5, -0.25) {$d_C$};
                        
                        \draw [line width=1pt] (1.5, -0.75) -- (e) -- (1.5, 0) -- (2,0.5) -- (2.5,0) -- (2.5,-0.75);
                        \draw [line width=1pt] (2,0.5) -- (2,1);

                        \node at (3,0) {$+$};

                        \node[op, scale = 0.5] (f) at (4.5, -0.25) {$d_N$};
                        
                        \draw [line width=1pt] (3.5, -0.75) --  (3.5, 0) -- (4,0.5) -- (4.5,0) -- (f) -- (4.5,-0.75);
                        \draw [line width=1pt] (4,0.5) -- (4,1);
                    \end{tikzpicture}
                \end{center}
            \end{definition}

            \begin{proposition}\label{prop: free-derivation}
                Let $A$ be a differential algebra and $M$ a $\mathbb{K}$-module. A $\mathbb{K}$-linear morphism $f:M\rightarrow A\otimes_{\mathbb{K}} M$ uniquely determines a derivation $d_f:A\otimes M\rightarrow A\otimes M$, i.e. there is an isomorphism $Hom_{\mathbb{K}}(M,A\otimes_{\mathbb{K}}M)\simeq Der(A\otimes_{\mathbb{K}}M)$. Moreover, $d_f$ is given as $(\nabla_A\otimes id_M)\circ (id_A\otimes f) + d_A\otimes id_M$.

                Dually, if $C$ is a differential coalgebra and $N$ is a $\mathbb{K}$-module, then a $\mathbb{K}$-linear morphism $g:C\otimes N\rightarrow N$ uniquely determines a coderivation $d_g:C\otimes_{\mathbb{K}}N\rightarrow C\otimes_{\mathbb{K}}N$. There is an isomorphism $Hom_{\mathbb{K}}(C\otimes_{\mathbb{K}}N,N)\simeq Coder(C\otimes_{\mathbb{K}}N)$, and $d_g$ is given as $(id_C\otimes g)\circ (\Delta_C\otimes id_N) + d_C\otimes id_N$.
            \end{proposition}

            \begin{proof}
                We must check that the given $d$ in facts defines an isomorphism. This is checked by calculation. 
            \end{proof}

            Recall that a module $M^*$ is $\mathbb{Z}$ graded if it decomposes as a sum $M^* = \substack{\bigoplus \\ z:\mathbb{Z}}M^z$. Let $M^*,N^*$ be graded modules and $f:M^*\rightarrow N^*$ is a homogenous $\mathbb{K}$-linear morphism of degree $n$ if it preserves the grading, that is $f(M^i) \subseteq N^{n+i}$. We denote the degree of $f$ as $|f|$. The category of graded modules will be denoted as $GrMod_{\mathbb{K}}$ or $Mod_\mathbb{K}^*$. Generally $\mathcal{C}^*$ is the category of graded objects whenever it makes sense, and the graded $\mathbb{K}$-module of morphisms between two graded objects is denoted as $Hom_{\mathbb{K}}^*(M^*,N^*)$.
            

            $M^{\bullet}$ is called a chain complex if it comes equipped with a homogenous morphism of degree $1$, like $d_M^{\bullet}:M^{\bullet}\rightarrow M^{\bullet}$, such that ${d_M^{\bullet}}^2=0$. This morphism is called differential. A chain morphism $f: M^{\bullet}\rightarrow N^{\bullet}$ is a homogenous $\mathbb{K}$-linear morphism of degree $0$, such that $f\circ d_M^{\bullet} = d_N^{\bullet}\circ f$. The category of chain complexes will be denoted as $ChMod_{\mathbb{K}}$ or $Mod_\mathbb{K}^\bullet$. Generally $\mathcal{C}^\bullet$ is the category of chain complexes whenever it makes sense, and the $\mathbb{K}$-module of morphisms between two chain complexes is denoted as $Hom_{\mathbb{K}}^{\bullet}(M^{\bullet},N^{\bullet})$.


            The functor $\_[n]:Mod_\mathbb{K}^\bullet\rightarrow Mod_\mathbb{K}^\bullet$ shifts the degree on each object by adding $n$ to each grade, it is called the shift functor. Let $\otimes$ denote the total tensor product in $Mod_\mathbb{K}^\bullet$. There is an isomorphism between the identity shift functor and total tensor of the stalk of $\mathbb{K}$, $\_[0] \simeq \bar{\mathbb{K}}\otimes\_$. In the same manner, shifting $n$-fold becomes isomorphic to tensoring with the shifted stalk of $\mathbb{K}$, $\_[n] \simeq \bar{\mathbb{K}}[n]\otimes\_$. For our purposes we will let $(A^\bullet, d_A^\bullet)[n] = (A^{\bullet + n}, -d_A^{\bullet + n})$. The koszul sign rule gives us a switching map for the tensor product. Thus, if $f^*:A^\bullet\rightarrow B^\bullet$ is a morphism of degree $k$, then $f^*[n] = (-1)^{k\cdot n}f^{*+n}$.


            In electric diagrams we will write triangles for the differential if there are no ambiguity.

            \begin{center}
                \begin{tikzpicture}[line cap=round,line join=round,>=triangle 45,x=1cm,y=1cm, thick, op/.style={circle, draw, scale=0.75}, diff/.style={regular polygon, draw, regular polygon sides = 3, scale=0.75, rotate = 180}, scale=0.7]

                    \node[op, scale = 0.75] (1) at (0,0) {$d_M^\bullet$};

                    \draw [line width = 1pt] (0, 1) -- (1) -- (0, -1);

                    \node at (1, 0) {$=$};

                    \node[diff, scale=0.75] (2) at (2, 0) {};

                    \draw [line width = 1pt] (2, 1) -- (2) -- (2, -1);
                    
                \end{tikzpicture}
            \end{center}

            \begin{proposition}
                Let $M^{\bullet}$ and $N^{\bullet}$ be two chain complexes. The graded module of morphisms $Hom_{\mathbb{K}}^*(M^{\bullet},N^{\bullet})$ is a chain complex, given by the differential $\partial(f) = d_N^{\bullet}\circ f - (-1)^{|f|}f\circ d_M^{\bullet}$.
            \end{proposition}

            \begin{proof}
                We observe that $\partial : Hom_{\mathbb{K}}^*(M^{\bullet},N^{\bullet}) \rightarrow Hom_{\mathbb{K}}^*(M^{\bullet},N^{\bullet})$ is a morphism of degree $1$. It remains to check that $\partial^2 = 0$. Pick any homogenous morphism $f : M^{\bullet}\rightarrow N^{\bullet}$.
                \begin{multline*}
                    \partial^2(f) = \partial(d_N^{\bullet}\circ f - (-1)^{|f|}f\circ d_M^{\bullet}) = \partial(d_N^{\bullet}\circ f) - (-1)^{|f|}\partial(f\circ d_M^{\bullet}) \\ = - (-1)^{|d_N^{\bullet}\circ f|}d_N^{\bullet}\circ f\circ d_M^{\bullet} - (-1)^{|f|}d_N^{\bullet}\circ f\circ d_M^{\bullet} = 0
                \end{multline*}
            \end{proof}

            In an electric diagram we write $\partial f$ as a sum of circuits.
            \begin{center}
                \begin{tikzpicture}[line cap=round,line join=round,>=triangle 45,x=1cm,y=1cm, thick, op/.style={circle, draw, scale=0.75}, diff/.style={regular polygon, draw, regular polygon sides = 3, scale=0.75, rotate = 180}, scale=0.7]

                    \node at (0,0) {$\partial f =$};

                    \node[op, scale=0.75] (f1) at (1, 0.5) {$f$};
                    \node[diff, scale=0.75] (d1) at (1, -0.5) {};

                    \draw [line width = 1pt] (1, 1) -- (f1) -- (d1) -- (1, -1);

                    \node at (2.5, 0) {$+(-1)^{|f|}$};

                    \node[diff, scale=0.75] (d2) at (4, 0.5) {};
                    \node[op, scale=0.75] (f2) at (4, -0.5) {$f$};

                    \draw [line width = 1pt] (4, 1) -- (d2) -- (f2) -- (4, -1);

                \end{tikzpicture}
            \end{center}

            Observe that $f:M^{\bullet}\rightarrow N^{\bullet}$ of degree $0$ is a chain morphism if and only if $\partial(f) = 0$. We then observe that $Hom_{\mathbb{K}}^{\bullet}(M^{\bullet},N^{\bullet})\simeq Z^0Hom_{\mathbb{K}}^*(M^{\bullet})$. Moreover, if $f$ is a boundary, i.e. there is some $h$ such that $f = \partial h$, then $f$ null-homotopic. Thus the $0$'th homology $H^0Hom^*_\mathbb{K}(M^\bullet, N^\bullet)$ is the set of every chain map which is not null-homotopic up to null homotopy. We obtain this lemma immediate from this discussion.

            \begin{lemma}
                Suppose that $N^\bullet$ is a contractible chain complex. Then for any chain complex $M^\bullet$, the $0$'th homology of $Hom^*_\mathbb{K}(M^\bullet, N^\bullet)$ is $H^0Hom^*_\mathbb{K}(M^\bullet, N^\bullet)\simeq 0$
            \end{lemma}

            To complete the definitions of graded modules and chain complexes to algebras we would like the structure morphisms to respect the given structure. E.g. if $a$ and $b$ are homogenous elements, we would like that the degree of $ab$ is the sum of its parts, i.e. $|ab| = |a| + |b|$. Since multiplication by identity doesn't do anything, we want that the identity lives in the $0$'th degree, and so forth.

            \begin{definition}[Graded algebra]
                Let $A^*$ be a graded $\mathbb{K}$-module. We say that $A^*$ is a graded algebra if $A^*$ is an algebra such that $\nabla_A$ and $\upsilon_A$ are homogenous and of degree $0$.
                Dually, $C^*$ is a graded coalgebra if $\Delta_C$ and $\varepsilon_C$ are homogenous and of degree $0$.
            \end{definition}

            \begin{definition}[Differential graded algebra]
                Let $A^{\bullet}$ be a chain complex over $\mathbb{K}$. We say that $A^{\bullet}$ is a differential graded algebra, or dg algebra, if it is a graded algebra and the differential is a graded derivation, i.e. $d_A(ab) = d_A(a)b + (-1)^{|a|}ad_A(b)$.

                Dually, $C^{\bullet}$ is a differential graded coalgebra if $C^{\bullet}$ is a graded coalgebra and the differential is a graded coderivation.
            \end{definition}
    \section{Cobar-Bar Adjunction}
    \subsection{Convolution Algebras}

            Let $C$ be a coalgebra and $A$ an algebra, then if $f,g:C\rightarrow A$ are $\mathbb{K}$-linear morphism we may define $f\star g = \nabla_A(f\otimes g)\Delta_C$. We call the operation $\star$ for convolution.

            \begin{center}
                \begin{tikzpicture}[line cap=round,line join=round,>=triangle 45,x=1cm,y=1cm, thick, op/.style={circle, draw, scale=0.75}, diff/.style={triangle, draw, scale=0.75}, scale=0.7]

                    \node at (-2.5, 0) {$f\star g$};

                    \node at (-1.5, 0) {$=$};

                    \node[op, scale=0.75] (a) at (-0.5, 0) {$f$};
                    \node[op, scale=0.75] (b) at (0.5, 0) {$g$};
                    
                    \draw [line width=1pt] (0, 1.25) -- (0, 1) -- (-0.5, 0.5) -- (a) -- (-0.5, -0.5) -- (0, -1) -- (0, -1.25);
                    \draw [line width=1pt] (0, 1) -- (0.5, 0.5) -- (b) -- (0.5, -0.5) -- (0, -1);

                \end{tikzpicture}
            \end{center}

            \begin{proposition}[Convolution algebra]
                The $\mathbb{K}$-module $Hom_{\mathbb{K}}(C,A)$ is an associative algebra when equipped with convolution $\star:Hom_{\mathbb{K}}(C,A)\rightarrow Hom_{\mathbb{K}}(C,A)$. The unit is given by $1 \mapsto \upsilon_A\circ\varepsilon_C$.
            \end{proposition}

            \begin{proof}
                This proposition follows from (co)associativity and (co)unitality of (C) A.

                \begin{center}
                    \begin{tikzpicture}[line cap=round,line join=round,>=triangle 45,x=1cm,y=1cm, thick, op/.style={circle, draw, scale=0.75}, scale=0.7]
                        % \node at (-3, 0) {Associativity};
                        
                        \node at (-3,0) {$(f\star g) \star h$};

                        \node at (-1, 0) {$=$};

                        \node[op, scale = 0.75] (f1) at (0,0) {f};
                        \node[op, scale = 0.75] (g1) at (1,0) {g};
                        \node[op, scale = 0.75] (h1) at (2,0) {h};

                        \draw (1, 1.5) -- (1,1.25) -- (0.5,1) -- (0.5, 0.75) -- (0,0.5) -- (f1) -- (0,-0.5) -- (0.5, -0.75) -- (0.5,-1) -- (1,-1.25) -- (1, -1.5);
                        \draw (0.5, 0.75) -- (1,0.5) -- (g1) -- (1, -0.5) -- (0.5, -0.75);
                        \draw (1,1.25) -- (2,0.75) -- (h1) -- (2, -0.75) -- (1, -1.25);

                        \node at (3,0) {$=$};

                        \node[op, scale = 0.75] (f2) at (4,0) {f};
                        \node[op, scale = 0.75] (g2) at (5,0) {g};
                        \node[op, scale = 0.75] (h2) at (6,0) {h};

                        \draw (5, 1.5) -- (5,1.25) -- (4, 0.75) -- (f2) -- (4,-0.5) -- (4.5, -0.75) -- (4.5,-1) -- (5,-1.25) -- (5, -1.5);
                        \draw (5, 1.25) -- (5.5, 1) -- (5.5, 0.75) -- (5,0.5) -- (g2) -- (5, -0.5) -- (4.5, -0.75);
                        \draw (5.5, 0.75) -- (6, 0.5) -- (h2) -- (6, -0.75) -- (5, -1.25);
                        
                        \node at (7,0) {$=$};

                        \node[op, scale = 0.75] (f3) at (8,0) {f};
                        \node[op, scale = 0.75] (g3) at (9,0) {g};
                        \node[op, scale = 0.75] (h3) at (10,0) {h};

                        \draw (9, 1.5) -- (9,1.25) -- (8, 0.75) -- (f3) -- (8,-0.75) -- (9,-1.25) -- (9, -1.5);
                        \draw (9, 1.25) -- (9.5, 1) -- (9.5, 0.75) -- (9,0.5) -- (g3) -- (9, -0.5) -- (9.5, -0.75);
                        \draw (9.5, 0.75) -- (10, 0.5) -- (h3) -- (10, -0.5) -- (9.5, -0.75) -- (9.5, -1) -- (9,-1.25);

                        \node at (11, 0) {$=$};

                        \node at (13, 0) {$f\star (g\star h)$};
                    \end{tikzpicture}

                    \begin{tikzpicture}[line cap=round,line join=round,>=triangle 45,x=1cm,y=1cm, thick, op/.style={circle, draw, scale=0.75}, scale=0.7]
                        \node at (-6, 0) {$(\upsilon_A\circ\varepsilon_C)\star f$};

                        \node at (-4, 0) {$=$};

                        \node[op, scale = 0.75] (f3) at (-2,0) {f};
                        \node[op, scale = 0.5] (c') at (-3, 0.25) {};
                        \node[op, scale = 0.5] (u') at (-3, -0.25) {};

                        \draw (-2.5, 1.25) -- (-2.5, 1) -- (-2, 0.75) -- (f3) -- (-2,-0.75) -- (-2.5,-1) -- (-2.5,-1.25);
                        \draw (-2.5, 1) -- (-3, 0.75) -- (c');
                        \draw (u') -- (-3, -0.75) -- (-2.5, -1);

                        \node at (-1,0) {$=$};

                        \node[op, scale = 0.75] (f1) at (0,0) {f};

                        \draw (0,1) -- (f1) -- (0,-1);

                        \node at (1,0) {$=$};

                        \node[op, scale = 0.75] (f2) at (2,0) {f};
                        \node[op, scale = 0.5] (c) at (3, 0.25) {};
                        \node[op, scale = 0.5] (u) at (3, -0.25) {};

                        \draw (2.5, 1.25) -- (2.5, 1) -- (2, 0.75) -- (f2) -- (2,-0.75) -- (2.5,-1) -- (2.5,-1.25);
                        \draw (2.5, 1) -- (3, 0.75) -- (c);
                        \draw (u) -- (3, -0.75) -- (2.5, -1);

                        \node at (4,0) {$=$};

                        \node at (6, 0) {$f\star (\upsilon_A\circ\varepsilon_C)$};
                    \end{tikzpicture}
                \end{center}
            \end{proof}

            If $A$ is an algebra and $C$ is a coalgebra, then they may be given the structure of a differential algebra by attaching the $0$ morphism to each algebra as the (co)derivation. In this case proposition \ref{prop: free-derivation} says that a morphism $f : M \rightarrow A \otimes_\mathbb{K} M$ determines the derivation given as $d_f = (\nabla_A \otimes id_M) \circ (id_A\otimes f)$. Dually, a morphism $g : C \otimes_\mathbb{K} M \rightarrow M$ determines the coderivation $d_g = (id_C \otimes g) \circ (\Delta_C \otimes id_N)$.

            If $\alpha : C \rightarrow A$ is a $\mathbb{K}$-linear morphism, then there are two ways to extend $\alpha$ to obtain a (co)derivation. Precomposing with $C$s comultiplication gives us a morphism from $C$ to the free $A$-module $A\otimes_\mathbb{K} C$.  

            \begin{align*}
                (\alpha \otimes id_C) \circ \Delta_C : C \rightarrow A \otimes_\mathbb{K} C
            \end{align*}

            Postcomposing with the multiplication of $A$ gives us a morphism from to the cofree $C$-comodule $C\otimes_\mathbb{K}A$ to $A$.
            \begin{align*}
                \nabla_A \circ (\alpha \otimes id_A) : C \otimes_\mathbb{K} A \rightarrow A
            \end{align*}

            Notice that when applying proposition \ref{prop: free-derivation} to both morphisms yields the same map, and it is thus both a derivation and a coderivation.
            \begin{align*}
                d_\alpha = (\nabla_A\otimes id_C) \circ (id_A \otimes \alpha \otimes id_C) \circ (id_A \otimes \Delta_C)
            \end{align*}

            \begin{center}
                \begin{tikzpicture}[line cap=round,line join=round,>=triangle 45,x=1cm,y=1cm, thick, op/.style={circle, draw, scale=0.75}, scale=0.7]

                    \node at (-3, 0) {$d_\alpha$};

                    \node at (-2, 0) {$=$};

                    \node[op, scale=0.75] (a) at (0,0) {$\alpha$};

                    \draw [line width=1pt] (-0.5, 1.25) -- (-0.5, 1) -- (0, 0.5) -- (a) -- (0, -0.5) -- (0.5, -1) -- (0.5, -1.25);
                    \draw [line width=1pt] (-0.5, 1) -- (-1, 0.5) -- (-1, -1.25);
                    \draw [line width=1pt] (1, 1.25) -- (1, -0.5) -- (0.5, -1);
                    
                \end{tikzpicture}
            \end{center} 

            \begin{proposition}\label{prop: convolution to endomorphism}
                $d_{(\_)}:Hom_\mathbb{K}(C,A)\rightarrow End(C\otimes_\mathbb{K}A)$ is a morphism of algebras. Moreover, if $\alpha\star \alpha = 0$, then $d_\alpha^2 = 0$.
            \end{proposition}

            \begin{proof}
                The proof quickly follows from (co)associativity and (co)unitality.
                \begin{center}
                    \begin{tikzpicture}[line cap=round,line join=round,>=triangle 45,x=1cm,y=1cm, thick, op/.style={circle, draw, scale=0.75}, scale=0.7]
    
                        \node at (-3, 0) {$d_{\alpha\star \beta}$};
    
                        \node at (-2, 0) {$=$};
    
                        \node[op, scale=0.75] (a) at (-0.5, 0) {$\alpha$};
                        \node[op, scale=0.75] (b) at (0.5, 0) {$\beta$};
                        
                        \draw [line width=1pt] (-0.5, 2) -- (-0.5, 1.75) -- (0, 1.25) -- (0, 1) -- (-0.5, 0.5) -- (a) -- (-0.5, -0.5) -- (0, -1) -- (0, -1.25) -- (0.5, -1.75) -- (0.5, -2);
                        \draw [line width=1pt] (0, 1) -- (0.5, 0.5) -- (b) -- (0.5, -0.5) -- (0, -1);
                        \draw [line width=1pt] (-0.5, 1.75) -- (-1, 1.25) -- (-1, -2);
                        \draw (1, 2) -- (1, -1.25) -- (0.5, -1.75);
                        
                        \node at (2, 0) {$=$};

                        \node[op, scale=0.75] (a') at (4, 0) {$\alpha$};
                        \node[op, scale=0.75] (b') at (5, 0) {$\beta$};

                        \draw [line width=1pt] (4, 2) -- (4, 1.75) -- (3.5, 1.25) -- (3.5, 1) -- (3, 0.5) -- (3, -2);
                        \draw [line width=1pt] (3.5, 1) -- (4, 0.5) -- (a') -- (4, -0.75) -- (5, -1.75);
                        \draw [line width=1pt] (4, 1.75) -- (5, 0.75) -- (b') -- (5, -0.5) -- (5.5, -1);
                        \draw [line width=1pt] (6, 2) -- (6, -0.5) -- (5.5, -1) -- (5.5, -1.25) -- (5, -1.75) -- (5, -2);

                        \node at (7, 0) {$=$};

                        \node at (8.5, 0) {$d_\alpha \circ d_\beta$};

                    \end{tikzpicture}
                \end{center} 

                \begin{center}
                    \begin{tikzpicture}[line cap=round,line join=round,>=triangle 45,x=1cm,y=1cm, thick, op/.style={circle, draw, scale=0.75}, scale=0.7]

                        \node at (-3.5, 0) {$d_{\upsilon_A \circ\varepsilon_C}$};

                        \node at (-2, 0) {$=$};

                        \node[op, scale=0.5] (a) at (0, 0.25) {};
                        \node[op, scale=0.5] (b) at (0, -0.25) {};

                        \draw [line width=1pt] (-0.5, 1.25) -- (-0.5, 1) -- (0, 0.5) -- (a);
                        \draw [line width=1pt] (b) -- (0, -0.5) -- (0.5, -1) -- (0.5, -1.25);
                        \draw [line width=1pt] (-0.5, 1) -- (-1, 0.5) -- (-1, -1.25);
                        \draw [line width=1pt] (1, 1.25) -- (1, -0.5) -- (0.5, -1);
                        
                        \node at (2, 0) {$=$};

                        \draw [line width=1pt] (3, 1.25) -- (3, -1.25);
                        \draw [line width=1pt] (4, 1.25) -- (4, -1.25);

                        \node at (5, 0) {$=$};
                        \node at (6.5, 0) {$id_{C \otimes_\mathbb{K} A}$};

                    \end{tikzpicture}
                \end{center}
            \end{proof}

            Suppose that $C$ and $A$ are differential graded (co)algebras. We want to expect that the differential $\partial$ makes $Hom_\mathbb{K}^*(C,A)$ into a dg-algebra.
            
            \begin{proposition}
                The convolution algebra $(Hom_\mathbb{K}^*(C,A),\star)$ is a dg-algebra with differential $\partial$.
            \end{proposition}

            \begin{proof}
                We know that $(Hom_\mathbb{K}^*(C,A),\star)$ is a convolution algebra and that $(Hom_\mathbb{K}^*(C,A),\partial)$ is a chain complex. It remains to verify that the differential is compatible with the multiplication, i.e. $\partial(f\star g) = \partial{f}\star g + (-1)^{|f|}f\star\partial{g}$.

                Let $f,g\in Hom_\mathbb{K}^*(C,A)$ be two homogenous morphisms. The key property to arrive at the result is that the differential in a dg-(co)algebra is a (co)derivation. We denote the degree of $f\star g$ as $|f\star g| = |f| + |g| = d$

                \begin{center}
                    \begin{tikzpicture}[line cap=round,line join=round,>=triangle 45,x=1cm,y=1cm, thick, op/.style={circle, draw, scale=0.75}, diff/.style={regular polygon, draw, regular polygon sides = 3, scale=0.75, rotate = 180}, scale=0.7]

                        \node at (0,0) {$\partial (f\star g) =$};

                        \node at (1.35,0) {$\partial$};
                        \node[op, scale = 0.75] (f1) at (2, 0) {$f$};
                        \node[op, scale = 0.75] (g1) at (3, 0) {$g$};

                        \draw [line width = 1pt] (2.5, 1.25) -- (2.5, 1) -- (2, 0.5) -- (f1) -- (2, -0.5) -- (2.5, -1) -- (2.5, -1.25);
                        \draw [line width = 1pt] (2.5, 1) -- (3, 0.5) -- (g1) -- (3, -0.5) -- (2.5, -1);


                        \node at (4, 0) {$=$};


                        \node[diff, scale=0.75] (d1) at (5.5, -1) {};
                        \node[op, scale=0.75] (f2) at (5, 0.5) {$f$};
                        \node[op, scale=0.75] (g2) at (6, 0.5) {$g$};

                        \draw [line width = 1pt] (5.5, 1.75) -- (5.5, 1.5) -- (5, 1) -- (f2) -- (5, 0) -- (5.5, -0.5) -- (d1) -- (5.5, -1.5);
                        \draw [line width = 1pt] (5.5, 1.5) -- (6, 1) -- (g2) -- (6, 0) -- (5.5, -0.5);

                        \node at (7.5, 0) {$-(-1)^{d}$};

                        \node[op, scale = 0.75] (f3) at (9, -0.5) {$f$};
                        \node[op, scale = 0.75] (g3) at (10, -0.5) {$g$};
                        \node[diff, scale = 0.75] (d2) at (9.5, 1) {};

                        \draw [line width = 1pt] (9.5, 1.5) -- (d2) -- (9.5, 0.5) -- (9, 0) -- (f3) -- (9, -1) -- (9.5, -1.5) -- (9.5, -1.75);
                        \draw [line width = 1pt] (9.5, 0.5) -- (10, 0) -- (g3) -- (10, -1) -- (9.5, -1.5);

                    \end{tikzpicture}
                \end{center}

                \begin{center}
                    \begin{tikzpicture}[line cap=round,line join=round,>=triangle 45,x=1cm,y=1cm, thick, op/.style={circle, draw, scale=0.75}, diff/.style={regular polygon, draw, regular polygon sides = 3, scale=0.75, rotate = 180}, scale=0.7]

                        \node at (0,0) {$=$};

                        \node[op, scale = 0.75] (f1) at (1, 0.5) {$f$};
                        \node[op, scale = 0.75] (g1) at (2, 0) {$g$};
                        \node[diff, scale = 0.75] (d1) at (1, -0.25) {};

                        \draw [line width = 1pt] (1.5, 1.75) -- (1.5, 1.5) -- (1, 1) -- (f1) -- (d1) -- (1, -0.75) -- (1.5, -1.25) -- (1.5, -1.5); 
                        \draw [line width = 1pt] (1.5, 1.5) -- (2, 1) -- (g1) -- (2, -0.75) -- (1.5, -1.25);

                        \node at (3.5, 0) {$+(-1)^{|f|}$};

                        \node[op, scale = 0.75] (f2) at (5, 0) {$f$};
                        \node[op, scale = 0.75] (g2) at (6, 0.5) {$g$};
                        \node[diff, scale = 0.75] (d2) at (6, -0.25) {};

                        \draw [line width = 1pt] (5.5, 1.75) -- (5.5, 1.5) -- (5, 1) -- (f2) -- (5, -0.75) -- (5.5, -1.25) -- (5.5, -1.5);
                        \draw [line width = 1pt] (5.5, 1.5) -- (6, 1) -- (g2) -- (d2) -- (6, -0.75) -- (5.5, -1.25) -- (5.5, -1.5);

                        \node at (8.25, 0) {$-(-1)^{d}((-1)^{|g|}$};

                        \node[op, scale = 0.75] (f3) at (10.5, -0.25) {$f$};
                        \node[op, scale = 0.75] (g3) at (11.5, 0) {$g$};
                        \node[diff, scale = 0.75] (d3) at (10.5, 0.5) {};

                        \draw [line width = 1pt] (11, 1.75) -- (11, 1.5) -- (10.5, 1) -- (d3) -- (f3) -- (10.5, -0.75) -- (11, -1.25) -- (11, -1.5);
                        \draw [line width = 1pt] (11, 1.5) -- (11.5, 1) -- (g3) -- (11.5, -0.75) -- (11, -1.25);

                        \node at (12.25, 0) {$+$};

                        \node[op, scale = 0.75] (f4) at (13, 0) {$f$};
                        \node[op, scale = 0.75] (g4) at (14, -0.25) {$g$};
                        \node[diff, scale = 0.75] (d4) at (14, 0.5) {};

                        \draw [line width = 1pt] (13.5, 1.75) -- (13.5, 1.5) -- (13, 1) -- (f4) -- (13, -0.75) -- (13.5, -1.25) -- (13.5, -1.5);
                        \draw [line width = 1pt] (13.5, 1.5) -- (14, 1) -- (d4) -- (g4) -- (14, -0.75) -- (13.5, -1.25);
                        \node at (14.5, 0) {$)$};
                    \end{tikzpicture}
                \end{center}

                \begin{center}
                    \begin{tikzpicture}[line cap=round,line join=round,>=triangle 45,x=1cm,y=1cm, thick, op/.style={circle, draw, scale=0.75}, diff/.style={regular polygon, draw, regular polygon sides = 3, scale=0.75, rotate = 180}, scale=0.7]

                        \node at (0,0) {$=$};

                        \node[op, scale = 0.75] (f1) at (1, 0.5) {$f$};
                        \node[op, scale = 0.75] (g1) at (2, 0) {$g$};
                        \node[diff, scale = 0.75] (d1) at (1, -0.25) {};

                        \draw [line width = 1pt] (1.5, 1.75) -- (1.5, 1.5) -- (1, 1) -- (f1) -- (d1) -- (1, -0.75) -- (1.5, -1.25) -- (1.5, -1.5); 
                        \draw [line width = 1pt] (1.5, 1.5) -- (2, 1) -- (g1) -- (2, -0.75) -- (1.5, -1.25);
                        
                        \node at (3.5, 0) {$-(-1)^{|f|}$};

                        \node[op, scale = 0.75] (f3) at (5, -0.25) {$f$};
                        \node[op, scale = 0.75] (g3) at (6, 0) {$g$};
                        \node[diff, scale = 0.75] (d3) at (5, 0.5) {};

                        \draw [line width = 1pt] (5.5, 1.75) -- (5.5, 1.5) -- (5, 1) -- (d3) -- (f3) -- (5, -0.75) -- (5.5, -1.25) -- (5.5, -1.5);
                        \draw [line width = 1pt] (5.5, 1.5) -- (6, 1) -- (g3) -- (6, -0.75) -- (5.5, -1.25) -- (5.5, -1.5);

                        \node at (7.5, 0) {$+(-1)^{|f|}($};

                        \node[op, scale = 0.75] (f2) at (9, 0) {$f$};
                        \node[op, scale = 0.75] (g2) at (10, 0.5) {$g$};
                        \node[diff, scale = 0.75] (d2) at (10, -0.25) {};

                        \draw [line width = 1pt] (9.5, 1.75) -- (9.5, 1.5) -- (9, 1) -- (f2) -- (9, -0.75) -- (9.5, -1.25) -- (9.5, -1.5);
                        \draw [line width = 1pt] (9.5, 1.5) -- (10, 1) -- (g2) -- (d2) -- (10, -0.75) -- (9.5, -1.25) -- (9.5, -1.5);


                        \node at (11.5, 0) {$-(-1)^{|g|}$};

                        \node[op, scale = 0.75] (f4) at (13, 0) {$f$};
                        \node[op, scale = 0.75] (g4) at (14, -0.25) {$g$};
                        \node[diff, scale = 0.75] (d4) at (14, 0.5) {};

                        \draw [line width = 1pt] (13.5, 1.75) -- (13.5, 1.5) -- (13, 1) -- (f4) -- (13, -0.75) -- (13.5, -1.25) -- (13.5, -1.5);
                        \draw [line width = 1pt] (13.5, 1.5) -- (14, 1) -- (d4) -- (g4) -- (14, -0.75) -- (13.5, -1.25);
                        \node at (14.5, 0) {$)$};
                    \end{tikzpicture}
                \end{center}

                \begin{center}
                    \begin{tikzpicture}[line cap=round,line join=round,>=triangle 45,x=1cm,y=1cm, thick, op/.style={circle, draw, scale=0.75}, diff/.style={regular polygon, draw, regular polygon sides = 3, scale=0.75, rotate = 180}, scale=0.7]

                        \node at (0,0) {$=$};

                        \node[op, scale = 0.75] (df) at (1, 0) {$\partial f$};
                        \node[op, scale = 0.75] (g) at (2, 0) {$g$};

                        \draw [line width = 1pt] (1.5, 1.25) -- (1.5, 1) -- (1, 0.5) -- (df) -- (1, -0.5) -- (1.5, -1) -- (1.5, -1.25);
                        \draw [line width = 1pt] (1.5, 1) -- (2, 0.5) -- (g) -- (2, -0.5) -- (1.5, -1);

                        \node at (3.5, 0) {$+(-1)^{|f|}$};

                        \node[op, scale = 0.75] (f) at (5, 0) {$f$};
                        \node[op, scale = 0.75] (dg) at (6, 0) {$\partial g$};
                        
                        \draw [line width = 1pt] (5.5, 1.25) -- (5.5, 1) -- (5, 0.5) -- (f) -- (5, -0.5) -- (5.5, -1) -- (5.5, -1.25);
                        \draw [line width = 1pt] (5.5, 1) -- (6, 0.5) -- (dg) -- (6, -0.5) -- (5.5, -1); 

                        \node at (7, 0) {$=$};

                        \node at (10.5, 0) {$\partial (f) \star g + (-1) ^{|f|}f \star \partial (g)$};

                    \end{tikzpicture}
                \end{center}

            \end{proof}

    \subsection{Twisting Morphisms}

            In this section we will define twisting morphisms from coalgebras to algebras. They are of importance as the bifunctor $Tw(C,A)$ is represented in both arguments. To understand the elements of $Tw$ we start this section be reviewing the Maurer-Cartan equation.

            Suppose that $C$ is a dg-coalgebra and $A$ is a dg-algebra. We say that a morphism $\alpha\in Hom_\mathbb{K}^*(C,A)$ is twisting if it is of degree $-1$ and satisfies the Maurer-Cartan equation:
            \begin{align*}
                \partial\alpha + \alpha\star\alpha = 0\text{.}
            \end{align*}
            We say that $\alpha$ is an element of $Tw(C,A)\subset Hom_\mathbb{K}^{-1}(C,A)\subset Hom_\mathbb{K}^*(C,A)$. In light of proposition \ref{prop: convolution to endomorphism}, every morphism between coalgebras and algebras extend to a unique (co)derivation on the tensor product $C\otimes_\mathbb{K}A$. Let $d_\alpha^r$ denote this unique morphism. In the case of dg-coalgebras and dg-algebras we perturbate the total differential on the tensor with $d_\alpha^r$, as in proposition \ref{prop: free-derivation}. We call this derivation for the perturbated derivative.
            \begin{align*}
                d_\alpha^\bullet = d_{C\otimes_\mathbb{K}A}^\bullet + d_\alpha^r = d_C^\bullet\otimes id_A + id_C\otimes d_A^\bullet + d_\alpha^r
            \end{align*}
            \begin{proposition}\label{prop: twisted-differential}
                Suppose that $C$ is a dg-coalgebra and $A$ is a dg-algebra, and $\alpha\in Hom_\mathbb{K}^*(C,A)$. The perturbated derivation satisfies the following relation.
                \begin{align*}
                    {d^\bullet_\alpha}^2 = d^r_{\partial \alpha + \alpha\star\alpha}
                \end{align*}
                Moreover, a morphism is twisting if and only if the perturbated derivative is a differential.
            \end{proposition}

            \begin{proof}
                ${d^\bullet_\alpha}^2 = d_{C\otimes_\mathbb{K}A}^\bullet \circ d_\alpha^r + d_\alpha^r \circ d_{C\otimes_\mathbb{K}A}^\bullet + {d_\alpha^r}^2$. By proposition \ref{prop: convolution to endomorphism} $d_?^r$ is an algebra homomorphism from the convolution algebra to the endomorphism algebra, thus ${d_\alpha^r}^2 = d_{\alpha\star\alpha}^r$.
                \begin{center}
                    \begin{tikzpicture}[line cap=round,line join=round,>=triangle 45,x=1cm,y=1cm, thick, op/.style={circle, draw, scale=0.75}, diff/.style={regular polygon, draw, regular polygon sides = 3, scale=0.75, rotate = 180}, scale=0.7]

                        \node at (0,0) {$d_{C\otimes_\mathbb{K} A}^\bullet\circ d_\alpha^r=$};
                        
                        \node[diff, scale = 0.75] (dc1) at (2.25,0) {};
                        \node[op, scale = 0.75] (f1) at (3.25, 0) {$f$};

                        \draw [line width = 1pt] (2.75, 1.75) -- (2.75, 1.5) -- (2.25, 1) -- (dc1) -- (2.25, -1.75);
                        \draw [line width = 1pt] (2.75, 1.5) -- (3.25, 1) -- (f1) -- (3.25, -1) -- (3.75, -1.5) -- (3.75, -1.75);
                        \draw [line width = 1pt] (4.25, 1.75) -- (4.25, -1) -- (3.75, -1.5);

                        \node at (4.75, 0) {$+$};

                        \node[op, scale = 0.75] (f2) at (6.25, 0.5) {$f$};
                        \node[diff, scale = 0.75] (da1) at (6.25, -0.5) {};

                        \draw [line width = 1pt] (5.75, 1.75) -- (5.75, 1.5) -- (5.25, 1) -- (5.25, -1.75);
                        \draw [line width = 1pt] (5.75, 1.5) -- (6.25, 1) -- (f2) -- (da1) -- (6.25, -1) -- (6.75, -1.5) -- (6.75, -1.75);
                        \draw [line width = 1pt] (7.25, 1.75) -- (7.25, -1) -- (6.75, -1.5);

                        \node at (7.75, 0) {$-$};

                        \node[op, scale = 0.75] (f3) at (9.25, 0) {$f$};
                        \node[diff, scale = 0.75] (da2) at (10.25, 0) {};

                        \draw [line width = 1pt] (8.75, 1.75) -- (8.75, 1.5) -- (8.25, 1) -- (8.25, -1.75);
                        \draw [line width = 1pt] (8.75, 1.5) -- (9.25, 1) -- (f3) -- (9.25, -1) -- (9.75, -1.5) -- (9.75, -1.75);
                        \draw [line width = 1pt] (10.25, 1.75) -- (da2) -- (10.25, -1) -- (9.75, -1.5);

                    \end{tikzpicture}
                \end{center}

                \begin{center}
                    \begin{tikzpicture}[line cap=round,line join=round,>=triangle 45,x=1cm,y=1cm, thick, op/.style={circle, draw, scale=0.75}, diff/.style={regular polygon, draw, regular polygon sides = 3, scale=0.75, rotate = 180}, scale=0.7]

                        \node at (0,0) {$d_\alpha^r\circ d_{C\otimes_\mathbb{K} A}^\bullet=-$};
                        
                        \node[diff, scale = 0.75] (dc1) at (2.25,0) {};
                        \node[op, scale = 0.75] (f1) at (3.25, 0) {$f$};

                        \draw [line width = 1pt] (2.75, 1.75) -- (2.75, 1.5) -- (2.25, 1) -- (dc1) -- (2.25, -1.75);
                        \draw [line width = 1pt] (2.75, 1.5) -- (3.25, 1) -- (f1) -- (3.25, -1) -- (3.75, -1.5) -- (3.75, -1.75);
                        \draw [line width = 1pt] (4.25, 1.75) -- (4.25, -1) -- (3.75, -1.5);

                        \node at (4.75, 0) {$+$};

                        \node[op, scale = 0.75] (f2) at (6.25, -0.5) {$f$};
                        \node[diff, scale = 0.75] (da1) at (6.25, 0.5) {};

                        \draw [line width = 1pt] (5.75, 1.75) -- (5.75, 1.5) -- (5.25, 1) -- (5.25, -1.75);
                        \draw [line width = 1pt] (5.75, 1.5) -- (6.25, 1) -- (da1) -- (f2) -- (6.25, -1) -- (6.75, -1.5) -- (6.75, -1.75);
                        \draw [line width = 1pt] (7.25, 1.75) -- (7.25, -1) -- (6.75, -1.5);

                        \node at (7.75, 0) {$+$};

                        \node[op, scale = 0.75] (f3) at (9.25, 0) {$f$};
                        \node[diff, scale = 0.75] (da2) at (10.25, 0) {};

                        \draw [line width = 1pt] (8.75, 1.75) -- (8.75, 1.5) -- (8.25, 1) -- (8.25, -1.75);
                        \draw [line width = 1pt] (8.75, 1.5) -- (9.25, 1) -- (f3) -- (9.25, -1) -- (9.75, -1.5) -- (9.75, -1.75);
                        \draw [line width = 1pt] (10.25, 1.75) -- (da2) -- (10.25, -1) -- (9.75, -1.5);

                    \end{tikzpicture}
                \end{center}

                By summing the above terms we get 
                \begin{align*}
                    d_{C\otimes_\mathbb{K}A}^\bullet\circ d_\alpha^r + d_\alpha^r\circ d_{C\otimes_\mathbb{K} A}^\bullet = d_{d_C^\bullet\circ\alpha + \alpha\circ d_A^\bullet}^r = d_{\partial\alpha}^r\text{,}
                \end{align*}
                to obtain the result.
                \begin{align*}
                    {d_\alpha^\bullet}^2 = d_{C\otimes_\mathbb{K}A}^\bullet \circ d_\alpha^r + d_\alpha^r \circ d_{C\otimes_\mathbb{Kwe}A}^\bullet + {d\alpha^r}^2 = d_{\partial \alpha}^r + d_{\alpha\star\alpha}^r = d_{\partial \alpha + \alpha\star\alpha}
                \end{align*}
            \end{proof}

            \begin{corollary}
                If $\alpha:C\rightarrow A$ is a twisting morphism, then $(C\otimes_\mathbb{K}A, d_\alpha^\bullet)$ is a chain complex. It is called the right twisted tensor product and is denoted as $C\otimes_\alpha A$.
            \end{corollary}

            Normally $A\otimes C$ and $C\otimes A$ are isomorphic as modules. In general, it is not true that $C\otimes_\alpha A$ and $A\otimes\alpha C$ are isomorphic, since we choose a particular side to perform the twisting. However, if $A$ is commutative and $C$ is cocommutative then they are isomorphic. To illustrate we realize the unique derivation above as a right derivative. The left derivative $d_\alpha^l$ is then defined analogously.
            \begin{center}
                \begin{tikzpicture}[line cap=round,line join=round,>=triangle 45,x=1cm,y=1cm, thick, op/.style={circle, draw, scale=0.75}, diff/.style={regular polygon, draw, regular polygon sides = 3, scale=0.75, rotate = 180}, scale=0.7]

                    \node at (0, 0) {$d_\alpha^l =$};

                    \node[op, scale = 0.75] (a) at (2, 0) {$\alpha$};

                    \draw [line width = 1pt] (1, 1.25) -- (1, -0.5) -- (1.5, -1) -- (1.5, -1.25);
                    \draw [line width = 1pt] (2.5, 1.25) -- (2.5, 1) -- (2, 0.5) -- (a) -- (2, -0.5) -- (1.5, -1) -- (1.5, -1.25);
                    \draw [line width = 1pt] (2.5, 1) -- (3, 0.5) -- (3, -1.25);
                    
                \end{tikzpicture}
            \end{center}

            \begin{remark}
                Functoriality of $\otimes_\alpha$ is obtained from the category of elements. I propose that there is an equivalence of categories, that is:
                \begin{align*}
                    \int_{(C,A)} Tw(C,A) \simeq \text{right twisted tensors.}
                \end{align*}
            \end{remark}

    \subsection{Bar and Cobar Construction}

            The bar construction was first formalized for augmented skew-commutative dg-rings by Eilenberg and Mac Lane \cite{Eilenberg53}. The bar construction then served as a tool to calculate the homology of the Eilenberg-Mac Lane spaces. This construction was later dualized by Adams \cite{Adams56} to obtain the cobar construction. It's first purpose was to serve as a method for constructing an injective resolution in order to calculate the cotor resolution \cite{Eilenberg65}. With time, the bar-cobar construction have been subjected to much generalization, such as a fattened tensor product on simplicially enriched, tensored and cotensored categories \cite{Riehl14}. We will mainly follow the work of \cite{Loday12} to obtain the bar and cobar construction. The approach which we are going to take is also slightly inspired by MacLanes\cite{MacLane71} canonical resolutions of comonads.

            For our purposes, the bar construction of an augmented algebra is a simplicial resoulution with the cofree coalgebra structure. For a dg-algebra, we will realize this resoultion as the total complex of its resoultion. Dually, the cobar construction of a conilpotent coalgebra is a cosimplicial resolution with the free algebra structure. We will see that these constructions defines an adjoint pair of functors.

            \begin{definition}
                The simplex category $\Delta$ consists of ordered sets $[n] = \startset{0,...,n}$ for any $n\in\mathbb{N}$. A morphism in $\Delta$ is a monotone function between the sets.

                $\Delta_+$ is the augmented simplex category, where we add the object $[-1] = \emptyset$. $\Delta^+$ is the non-full subcategory of $\Delta$, where all morphisms are injective functions.
            \end{definition} 

            The simplex category comes equipped with coface and codegeneracy morphisms. The coface maps are the injective morphisms $\delta_i : [n] \rightarrow [n+1]$, and the codegeneracy maps are the surjective morphisms $\sigma_i: [n] \rightarrow [n-1]$.
            \begin{align*}
                \delta_i(k) = \biggl\{\substack{k\text{, if }k<i \\ k+1\text{, otherwise}} \qquad
                \sigma_i(k) = \biggl\{\substack{k\text{, if }k\leq i \\ k-1\text{, otherwise}}
            \end{align*}

            Every morphism in $\Delta$ may be realized as a composition of coface and codegeneracy maps, see \cite{MacLane71}. Furthermore, these maps are characterized by some identites, called the cosimplicial identites. 
            \begin{align*}
                1.&\ \delta_j\delta_i = \delta_i\delta_{j-1} \text{, if }i<j \\
                2.&\ \sigma_j\delta_i = \delta_i\sigma_{j-1} \text{, if }i<j \\
                3.&\ \sigma_j\delta_i = id \text{, if }i=j\text{ or }i=j+1 \\
                4.&\ \sigma_j\delta_i = \delta_{i-1}\sigma_j \text{, if }i>j+1 \\
                5.&\ \sigma_j\sigma_i = \sigma_i\sigma_{j+1} \text{, if }i\leq j
            \end{align*}
            We may arrange the arrows of the augmented simplex category in the following way:
            \begin{center}
                \begin{tikzcd}
                    {[}-1{]} \ar[]{r}[]{} & {[}0{]} \ar[yshift = 0.5ex]{r}[]{\delta_i} \ar[yshift = -0.5ex]{r}[]{} & {[}1{]} \ar[yshift = 0.75ex]{r}[]{\delta_i} \ar[]{r}[]{} \ar[yshift = -0.75ex]{r}[]{} & {[}2{]} \ar[yshift = 1ex]{r}[]{\delta_i} \ar[yshift = 0.33ex]{r}[]{} \ar[yshift = -0.33ex]{r}[]{} \ar[yshift = -1ex]{r}[]{} & ...
                \end{tikzcd}

                \begin{tikzcd}
                    {[}-1{]} & {[}0{]} & {[}1{]} \ar[]{l}[above]{\sigma_1} & {[}2{]} \ar[yshift = 0.5ex]{l}[above]{\sigma_i} \ar[yshift = -0.5ex]{l}[]{} & ... \ar[yshift = 0.75ex]{l}[above]{\sigma_i} \ar[]{l}[]{} \ar[yshift = -0.75ex]{l}[]{}
                \end{tikzcd}
            \end{center}

            Let $\mathcal{C}$ be a category. A simplicial object in $\mathcal{C}$ is a functor $S:\Delta^{op}\rightarrow \mathcal{C}$. It may be viewed as a collection of objects $\startset{S_n}_{n\in\mathbb{N}}$ together with face maps $d^i:S_n\rightarrow S_{n-1}$ and degeneracy maps $s^i:S_n \rightarrow S_{n+1}$ satisfying the simplicial identities. An augmented simplicial object is a functor $S:\Delta_+^{op}\rightarrow \mathcal{C}$. The restricted functor $\bar{S}:\Delta^{op}\rightarrow \mathcal{C}$ is the augmentation ideal of $S$. An augmented semi-simplicial object is a functor $S:(\Delta_+^+)^{op}\rightarrow \mathcal{C}$. Dually, a cosimplicial object is a functor $S:\Delta\rightarrow \mathcal{C}$, it may be regarded as a sequence of objects with coface and codegeneracy maps satisfying the cosimplicial identities.
            
            Let $\mathcal{A}$ be an abelian category. To each semi-simplical object $M:(\Delta^+)^{op}\rightarrow \mathcal{A}$ there is an associated chain complex $M^\bullet$. Let $M^\bullet = \bigoplus_{i=1}^\infty M[i]$ with differential $d_M^n = \sum_{i=1}^n (-1)^{i-1}d^i$. This differential is well-defined by simplicial identity $1$.
            \begin{center}
                \begin{tikzcd}
                    ... \ar[]{r}[]{} & M_2 \ar[]{r}[]{d^1-d^2+d^3} & M_1 \ar[]{r}[]{d^1-d^2} & M_0 \ar[]{r}[]{0} & 0 \ar[]{r}[]{} & ...
                \end{tikzcd}
            \end{center}
            As face maps and degeneracy maps have the same identites, but flipped around, we could also have defined a chain complex by using the degeneracies instead.

            The augmented simplex category has a universal monoid. Let $+:\Delta_+ \times \Delta_+ \rightarrow \Delta_+$ be a functor acting on objects and morphisms as:
            \begin{align*}
                [m]+[n] = [m+n+1] \\
                (f+g)(k) = \biggl\{\substack{f(k)\text{, if }k\leq m \\ g(k)+m\text{, otherwise}}
            \end{align*}
            Notice that $[-1]+\_\simeq Id_\Delta$, so $(\Delta, +, [-1])$ is a monoidal category. Since $[0]$ is terminal in $\Delta$ it becomes a monoid with $\delta_{-1}: [-1]\rightarrow [0]$ as unit and $\sigma_0:[1]\rightarrow [0]$ as multiplication. Associativity and unitality is satisfied by uniqueness of morphisms $f:[n]\rightarrow [0]$.
            \begin{proposition}\label{prop: universal-monoid}
                Let $(\mathcal{C}, \otimes, Z)$ be a monoidal category. If $(C, \eta, \mu)$ is a monoid in $\mathcal{C}$, then there is a strong monoidal functor $:\Delta_+\rightarrow \mathcal{C}$, such that $F[0] \simeq C$, $F\delta_{-1} \simeq \eta$ and $F\sigma_0 \simeq \mu$.
            \end{proposition} 

            \begin{proof}
                This is proved in Mac Lanes book \cite{MacLane71}.
            \end{proof}

            % A monad is a monoid in the monoidal category of endofunctors. This is a functor $M:\mathcal{C}\rightarrow\mathcal{C}$, with natural transformations $\mu:M^2\implies M$ and $\eta:Id_\mathcal{C}\implies M$ called multiplication and unit. The triple $(M, \mu, \eta)$ is a monad whenever it is a monoid, i.e. multiplication satisfies associativity and unit is the unit of the multiplication. Dually a comonad $W:\mathcal{C}\rightarrow\mathcal{C}$ is a triple $(W, \nu, \varepsilon)$ such that it is a comonoid. 

            % \begin{proposition}
            %     Suppose that $W:\mathcal{C}\rightarrow\mathcal{C}$ is comonad. The sequence of functors $(W^{n})_\mathbb{N}$ is an augmented simplicial functor with face and degeneracy maps
            %     \begin{align*}
            %         d_n^i & = W^i\varepsilon_{W^{n-i}} \\
            %         s_n^i & = W^i\nu_{W^{n-i}}\text{.}
            %     \end{align*}

            %     Suppose that $M:\mathcal{C}\rightarrow\mathcal{C}$ is a monad. The sequence of functors $(M^{n})_\mathbb{N}$ is a coaugmented cosimplicial functor with coface and codegeneracy maps
            %     \begin{align*}
            %         d^n_i & = M^i\eta_{M^{n-i}} \\
            %         s^n_i & = M^i\mu_{M^{n-i}}\text{.}
            %     \end{align*}
            % \end{proposition}

            % \begin{proof}
            %     Follows from the universal property of the simplex category, see \cite{MacLane71}.
            % \end{proof}

            % Let $W^?:\Delta^{op}\rightarrow\mathcal{C}$ denote the simplical functor. If $C\in\mathcal{C}$ is an object, then there is a simplical object, denoted $W_C^?$. The face and degeneracy maps are obtained by applying $C$, i.e. $(d_n^i)_C$ and $(s_n^i)_C$. Dually, if $M$ is a monad we obtain a cosimplicial object $M_C^?$.

            % \begin{definition}
            %     Suppose that $\mathcal{A}$ is an abelian category. Let $W:\mathcal{A}\rightarrow\mathcal{A}$ be a comonad and $A$ and object of $\mathcal{A}$. The canonical $W$-projective resolution of $A$ is the chain complex $W^\bullet_A$ together with an augmentation $\varepsilon_A:W^\bullet_A \rightarrow A$.

            %     Dually, suppose that $M:\mathcal{A}\rightarrow\mathcal{A}$ is a monad. The canonical $M$-injective resolution of A is the chain complex $M_A^\bullet$ together with an augmentation $\eta_A:A\rightarrow MA$.
            % \end{definition}

            % \begin{remark}
            %     It makes sense to call these canonical resolutions for projective or injective. Whenever the object $A$ is $W$-projective we get that the augmentation is a quasi-isomorphism. See Weibel for more information.
            % \end{remark}

            % \begin{lemma}\label{lem: adjoints-make-monads}
            %     Suppose that there is an adjoint pair of functors $F: \mathcal{C}\rightleftharpoons\mathcal{D}:G$, where $F$ is left adjoint to $G$. The composite $GF$ is a monad, and $FG$ is a comonad.
            % \end{lemma}

            % We will use this lemma to construct the canonical resolution for algebras. One may observe that this is the same as the Hochschild complex.

            % \begin{example}
            %     Let $A$ be an algebra over the ring $\mathbb{K}$. The functor $A\otimes_\mathbb{K}\_:Mod_\mathbb{K}\rightarrow Mod_A$ is the free $A$-module over a $\mathbb{K}$-module. Let $U:Mod_A\rightarrow Mod_\mathbb{K}$ denote the forgetfull functor. 

            %     \begin{minipage}[c]{0.3\textwidth}
            %         \begin{center}
            %             \begin{tikzcd}
            %                 Mod_\mathbb{K} \ar[phantom]{r}{\top} \ar[bend right]{r}[below]{A\otimes_\mathbb{K}\_} & Mod_A \ar[bend right]{l}[above]{U}
            %             \end{tikzcd}
            %         \end{center}
            %     \end{minipage}
            %     \begin{minipage}[c]{0.7\textwidth}
            %         The unit of the adjunction is given by adjoining the unit of $A$, $\eta_M(m) = 1_A \otimes m$. The counit is given by the structure morphism of each $A$-module, i.e. $\varepsilon_M:A\otimes_\mathbb{K}M\rightarrow M$ is the algebra action. These morphisms are by definition natural and satisfy the triangle identities.
            %     \end{minipage}

            %     By lemma \ref{lem: adjoints-make-monads} $A\otimes_\mathbb{K}U : Mod_A \rightarrow Mod_A$ is a comonad. $\varepsilon$ is the counit and the comultiplication is given by $A\otimes_\mathbb{K}\eta_U$. If $M$ is an $A$-module we get the canonical free-resolution of $M$ as $A\otimes_\mathbb{K}U_M^\bullet$:
            %     \begin{center}
            %         \begin{tikzcd}
            %             ... \ar[]{r}[]{} & A\otimes_\mathbb{K}A\otimes_\mathbb{K}M \ar[]{r}[]{\substack{\varepsilon_{A\otimes\mathbb{K}M} \\ -A\otimes_\mathbb{K}\varepsilon_M}} & A\otimes_\mathbb{K}M \ar[]{r}[]{0} & 0 \ar[]{r}[]{} & ...
            %         \end{tikzcd}
            %     \end{center}
            % \end{example}

            An algebra $A$ is a monoid in the monoidal category $(Mod_\mathbb{K}, \otimes_\mathbb{K}, \mathbb{K})$. By proposition \ref{prop: universal-monoid} we may think of $A$ as an augmented cosimplicial object $A:\Delta_+ \rightarrow Mod_\mathbb{K}$. Notice that all of the cosimplical identities follow from associativity and unitality. If $A$ is an augmented algebra, we may instead give it the structure of an augmented simplicial set. Let $d^0_0 = \varepsilon_A$ be the augmentation. We define $d^n_n = A^{\otimes n-1}\otimes\varepsilon_A$ and set $d^i_n = A^{i-1}\otimes \nabla_A \otimes A^{\otimes n-i-1}$. All the degeneracies are set to be the units, i.e. $s^i_n = A^{\otimes i}\otimes \upsilon_A \otimes A^{\otimes n-i-1}$. One may check that this structure defines an augmented simplical object $A:\Delta_+^{op}\rightarrow Mod_\mathbb{K}$. Observe that the associated chain complex $A^\bullet$ is exactly the Hochschild complex of $A$. We depict the simplicial object as the following diagram:
            \begin{center}
                \begin{tikzcd}
                    \mathbb{K} & A \ar[]{l}[above]{\varepsilon_A} & A^{\otimes 2} \ar[yshift = 0.5ex]{l}[above]{\nabla_A} \ar[yshift = -0.5ex]{l}[]{A\otimes \varepsilon_A} & A^{\otimes 3} \ar[yshift = 0.75ex]{l}[above]{\nabla_A} \ar[]{l}[]{} \ar[yshift = -0.75ex]{l}[]{A^{\otimes 2}\otimes \varepsilon_A} & ... \ar[yshift = 1ex]{l}[above]{\nabla_A} \ar[yshift = 0.33ex]{l}[]{} \ar[yshift = -0.33ex]{l}[]{} \ar[yshift = -1ex]{l}[]{A^{\otimes 4}\otimes\varepsilon_A} 
                \end{tikzcd}

                \begin{tikzcd}
                    \mathbb{K} & A \ar[]{r}[]{s^1} & A^{\otimes 2} \ar[yshift = 0.5ex]{r}[]{s^i} \ar[yshift = -0.5ex]{r}[]{} & A^{\otimes 3} \ar[yshift = 0.75ex]{r}[]{s^i} \ar[]{r}[]{} \ar[yshift = -0.75ex]{r}[]{} & ...
                \end{tikzcd}
            \end{center}

            The augmentation ideal $\bar{A}$ carries a natural semi-simplical structure induced by $A$. By restricting each of the face maps ${\bar{d}}^i = d^i {\mid}_{\bar{A}}:\bar{A}^{\otimes n} \rightarrow \bar{A}^{\otimes n-1}$ we obtain the maps together with the simplical identity 1. This is the non-unital Hochschild complex of $A$. We may depict the semi-simplical object as the following diagram:
            \begin{center}
                \begin{tikzcd}
                    \mathbb{K} & \bar{A} \ar[]{l}[above]{0} & \bar{A}^{\otimes 2} \ar[yshift = 0.5ex]{l}[above]{\nabla_A} \ar[yshift = -0.5ex]{l}[]{0} & \bar{A}^{\otimes 3} \ar[yshift = 0.75ex]{l}[above]{\nabla_A} \ar[]{l}[]{} \ar[yshift = -0.75ex]{l}[]{0} & ... \ar[yshift = 1ex]{l}[above]{\nabla_A} \ar[yshift = 0.33ex]{l}[]{} \ar[yshift = -0.33ex]{l}[]{} \ar[yshift = -1ex]{l}[]{0} 
                \end{tikzcd}
            \end{center}

            Notice that as graded modules, the chain complex $\bar{A}^\bullet$ is isomorphic to $T^c(\bar{A})$. We will now instead consider the suspended non-unital algebra $\bar{A}[1]$. Every algebra may be considered as a graded algebra concentrated in degree $0$, the shift functor then recontextualize the degree the algebra is concentrated in. With Koszul sign rule, we may define the suspended multiplication as $\nabla_{A[1]}(a_1 \otimes a_2) = (-1)^{|a_1|}a_1a_2$. Notice that $\nabla_{A[1]}$ is a morphism of degree $-1$. Repeating Koszul sign rule, we may se that associativity does not longer hold, as multiplying the multiplication on the right first introduces a sign, contrary to first multiplying on the left side.

            \begin{proposition}
                The suspended augmentation ideal $\bar{A}[1]$ is a semi-simplical set with face maps:
                \begin{align*}
                    \bar{d}^i = (-1)^{i-1}d^i = (-1)^{i-1}(\nabla_{A[1]})_{(i-1)}^{(n-1)}\text{.}
                \end{align*}
            \end{proposition}

            \begin{corollary}\label{cor: a-to-dgc}
                The differential $d_{\bar{A}[1]}^\bullet$ is a coderivation for the cofree coalgebra $T^c(\bar{A}[1])$. Thus $(\bar{A}[1]^\bullet, d_{\bar{A}[1]}^\bullet)$ is a dg-coalgebra.
            \end{corollary}

            \begin{proof}
                The differential is given by the alternating sum of face maps.
                \begin{align*}
                    d_{\bar{A}[1]}^n = \sum_{i=1}^n (-1)^{i-1}\bar{d}^i = \sum_{i=1}^n (-1)^{2(i-1)}d^i = \sum_{i=1}^n (\nabla_{A[1]})_{(i-1)}^{(n-1)}
                \end{align*}
                By injecting $\bar{A}[1]$ into $T^c(\bar{A}[1])$ we may think of $\nabla_{\bar{A}[1]} : \bar{A}[1]^{\otimes 2} \rightarrow T^c(\bar{A}[1])$ as a morphism into the tensor coalgebra. By using proposition \ref{prop: tensor-derivation}, $\nabla_{\bar{A}[1]}$ extends uniquely into a coderivation:
                \begin{align*}
                    d_{\bar{A}[1]}^c = \sum_{n=0}^{\infty}\sum_{i=0}^n(\nabla_{\bar{A}[1]})_{(i)}^{(n)} = d_{\bar{A}[1]}^\bullet\text{.}
                \end{align*}
            \end{proof}

            If $(A, d_A^\bullet)$ is an augmented dg-algebra, then $A$ is a simplical object of $Mod_\mathbb{K}^\bullet$. It has an associated chain complex. Taking the alternate sum of face maps gives us a double complex as below. We define the double complex $A^\bullet$ as the associated chain complex to $A$.
            \begin{center}
                \begin{tikzcd}
                    & \vdots \ar[xshift = -1ex]{d}[left]{\nabla_A} \ar[]{d}[]{} \ar[xshift = 1ex]{d}[]{A^{\otimes 2}\otimes\varepsilon_A} & \vdots \ar[xshift = -1ex]{d}[left]{\nabla_A} \ar[]{d}[]{} \ar[xshift = 1ex]{d}[]{A^{\otimes 2}\otimes\varepsilon_A} & \vdots \ar[xshift = -1ex]{d}[left]{\nabla_A} \ar[]{d}[]{} \ar[xshift = 1ex]{d}[]{A^{\otimes 2}\otimes\varepsilon_A} & \\
                    ... \ar[]{r}[below]{d_{A^{\otimes 2}}^\bullet} & (A^{\otimes 2})^{1} \ar[]{r}[below]{d_{A^{\otimes 2}}^\bullet} \ar[xshift = -0.5ex]{d}[left]{\nabla_A} \ar[xshift = 0.5ex]{d}[]{A\otimes\varepsilon_A} & (A^{\otimes 2})^0 \ar[]{r}[below]{d_{A^{\otimes 2}}^\bullet} \ar[xshift = -0.5ex]{d}[left]{\nabla_A} \ar[xshift = 0.5ex]{d}[]{A\otimes\varepsilon_A} & (A^{\otimes 2})^{-1} \ar[]{r}[below]{d_{A^{\otimes 2}}^\bullet} \ar[xshift = -0.5ex]{d}[left]{\nabla_A} \ar[xshift = 0.5ex]{d}[]{A\otimes\varepsilon_A} & ... \\
                    ... \ar[]{r}[below]{d_A^\bullet} & A^{1} \ar[]{r}[below]{d_A^\bullet} \ar[]{d}[]{\varepsilon_A} & A^0 \ar[]{r}[below]{d_A^\bullet} \ar[]{d}[]{\varepsilon_A} & A^{-1} \ar[]{r}[below]{d_A^\bullet} \ar[]{d}[]{\varepsilon_A} & ... \\
                    ... \ar[]{r}[below]{0} & 0 \ar[]{r}[below]{0} & \mathbb{K} \ar[]{r}[below]{0} & 0 \ar[]{r}[below]{0} & ...
                \end{tikzcd}
            \end{center}
        
            For simplicity we write $d_1$ for the horizontal differential and $d_2$ for the vertical differential. The total associated chain complex is the total complex for $Tot(A^\bullet)$, denoted $A^\bullet$ if there are no confusion. In the case of the suspended algebra, the signs mess up commutativity of the squares, thus we change the sign of the horizontal differential to $(-1)^n$. We may also define the differential of the total complex simply as the sum of $d_1$ and $d_2$.

            \begin{proposition}
                Let $A$ an augmented dg-algebra. The bar complex $BA$ is the total associated chain complex $\bar{A}[1]^\bullet$ of the suspended augmentation ideal $\bar{A}$. $(BA, d_{BA}^\bullet)$ is the cofree conilpotent coalgebra equipped with $d_{BA}^\bullet = d_1 + d_2$ as coderivation.
            \end{proposition}

            \begin{proof}
                    It is apparent that $d_1$ and $d_2$ are coderivations with respect to deconcatenation. Since the multiplication $\nabla_A$ is a chain map ${d_{BA}^\bullet}^2 = d_1 \circ d_2 + d_2 \circ d_1= 0$. We will show this for each element in $A^{\otimes 2}$, then the result may be extended to all of $BA$.

                \begin{multline*}
                    d_1 \circ d_2 (a_1\otimes a_2) = (-1)^{|a_1|}d_1 (a_1a_2) = (-1)^{|a_1|}d_A^\bullet[1](a_1a_2) \\ = (-1)^{|a_1|+1}d_A^\bullet(a_1a_2) = (-1)^{|a_1|+1}(d_A^\bullet(a_1)a_2 + (-1)^{|a_1|}a_1d_A^\bullet(a_2)) \\ = (-1)^{|a_1|+1}d_A^\bullet(a_1)a_2 - a_1d_A^\bullet(a_2)
                \end{multline*}
                \begin{multline*}
                    d_2\circ d_1 (a_1\otimes a_2) = d_2\circ (d_A^\bullet[1]\otimes id_{A[1]} + id_{A[1]}\otimes d_A^\bullet[1]) (a_1\otimes a_2) \\ = -d_2 \circ (d_A^\bullet(a_1)\otimes a_2 + (-1)^{|a_1|+1}a_1\otimes d_A^\bullet(a_2)) \\ = (-1)^{|d_A^\bullet(a_1)|+1}d_A^\bullet(a_1)a_2 + (-1)^{2|a_1|+2}a_1d_A^\bullet d_A^\bullet(a_2) \\ = (-1)^{|a_1|}d_A^\bullet(a_1)a_2 + a_1d_A^\bullet (a_2) = -d_1\circ d_2 (a_1\otimes a_2)
                \end{multline*}
            \end{proof}

            \begin{remark}
                For now we don't need to show that $BA$ is a functor. This property follows from $BA$ being the representing object of $Tw(\_,A)$.
            \end{remark}

            On the other hand, a coalgebra $C$ is a comonoid in $Mod_\mathbb{K}$. By the dual of proposition \ref{prop: universal-monoid} we may think of it as an augmented simplical object $C:(\Delta_+)^{op} \rightarrow Mod_\mathbb{K}$. Dually, all of the simplical identities follows from coassociativity and counitality. A coaugmented coalgebra $C$ may be given an augmented cosimplicial structure in the opposite way of algebras. We then get that the coaugmentation quotient $\bar{C}$ is a semi-cosimplical object of $Mod_\mathbb{K}$. Observe that $\bar{C}$ has an associated chain complex like $\bar{A}$, but every arrow goes in the opposite direction.

            \begin{center}
                \begin{tikzcd}
                    \mathbb{K} \ar[]{r}[]{\upsilon_C} & C \ar[yshift = 0.5ex]{r}[]{\Delta_C} \ar[yshift = -0.5ex]{r}[below]{A\otimes \upsilon_C} & C^{\otimes 2} \ar[yshift = 0.75ex]{r}[]{\Delta_C} \ar[]{r}[]{} \ar[yshift = -0.75ex]{r}[below]{C^{\otimes 2}\otimes \upsilon_C} & C^{\otimes 3} \ar[yshift = 1ex]{r}[]{\Delta_C} \ar[yshift = 0.33ex]{r}[]{} \ar[yshift = -0.33ex]{r}[]{} \ar[yshift = -1ex]{r}[below]{C^{\otimes 4}\otimes\upsilon_C} & ... 
                \end{tikzcd}

                \begin{tikzcd}
                    \mathbb{K} & C & C^{\otimes 2} \ar[]{l}[above]{s_1} & C^{\otimes 3} \ar[yshift = 0.5ex]{l}[above]{s_i} \ar[yshift = -0.5ex]{l}[]{} & ... \ar[yshift = 0.75ex]{l}[above]{s_i} \ar[]{l}[]{} \ar[yshift = -0.75ex]{l}[]{}
                \end{tikzcd}
            \end{center}
            
            The cobar construction is made from the inverse shifted, or desuspended coalgebra $C[-1]$. We realize it as the free tensor algebra $T(\bar{C}[-1])$, where the comultiplication $\Delta_{\bar{C}[-1]}$ induces a derivation $d_{\bar{C}[-1]}$ by proposition \ref{prop: tensor-derivation}.

            \begin{remark}
                As we have chosen to define $\nabla_{A[1]}(a_1\otimes a_2)=(-1)^{|a_1|}a_1a_2$, we are forced by the linear dual to define $\Delta_{C[-1]}(c)=-(-1)^{|c_{(1)}|}c_{(1)}\otimes c_{(2)}$.
            \end{remark}

            \begin{proposition}
                Let $C$ be a coaugmented dg-coalgebra. The cobar complex $\Omega C$ is the total associated chain complex $\bar{C}[-1]^\bullet$ of the desuspended coaugmentation quotient $\bar{C}$. $(\Omega C, D_{\Omega C}^\bullet)$ is the free algebra equipped with $d_{\Omega C}^\bullet = d_1 + d_2$ as derivation.
            \end{proposition}

            We will now see that the bar and cobar construction defines an adjoint pair of functors. Note that since for any conilpotent dg-coalgebra $C$, the object $\Omega C$ represents the functor in the category of augmented algebras. By Yoneda's lemma, the data of morphisms are then defined, so $\Omega$ does truly define a functor.

            \begin{thm}
                Let $C$ be a conilpotent dg-coalgebra and $A$ an augmented dg-algebra. The functor $Tw(C,A)$ is represented in both arguments, i.e.
                \begin{align*}
                    Alg_{\mathbb{K},+}^\bullet(\Omega C, A)\simeq Tw(C, A) \simeq CoAlg_{\mathbb{K},Conil}^\bullet(C, BA)\text{.}
                \end{align*}
            \end{thm}

            \begin{proof}
                We will show that $\Omega C$ represents the set of twisting morphisms in the first argument. Showing that $BA$ represents the second argument uses every dual proposition. Thus, it is necessary that $C$ is conilpotent, in order to dualize the arguments.

                Suppose that $f:\Omega C \rightarrow A$ is an augmented dg-algebra homomorphism. $f$ is then a morphism of degree $0$. By freeness, $f$ is uniquely determined by a morphism $f\mid_{\bar{C}[-1]}:\bar{C}\rightarrow \bar{A}$ of degree $0$, which corresponds to a morphism $f':C\rightarrow A$ of degree $-1$. 

                Since $f$ is a morphism of chain complexes it commutes with the differential, i.e. 
                \begin{align*}
                    f\circ d_{\Omega C}^\bullet = d_A^\bullet\circ f \\
                    f\circ (d_1 + d_2) = d_A^\bullet\circ f 
                \end{align*}
                This is equivalent to say that $-f'\circ d_C^\bullet - f'\star f' = d_A^\bullet\circ f'$. Thus $f'$ is a twisting morphism.
            \end{proof}

            For our convenience, we will give these isomorphisms some names. Whenever $\tau : C \rightarrow A$ is a twisting morphism, the induced morphism of algebras is denoted $f_\tau : \Omega C \rightarrow A$ and the induced morphism of coalgebras is denoted $g_\tau : C \rightarrow BA$.

            \begin{remark}
                These functors consists of a composition with the augmentation ideal (quotient) and then the (co)free tensor (co)algebra. By reversing these operations we obtain another adjunction which is more or less the same adjunction. By abuse of notation we will call these functors for the bar and cobar construction as well, and they induce and adjoint pair between dg-algebras and reduced conilpotent dg-coalgebras. In other words, given a dg-algebra $A$ and a reduced conilpotent dg-coalgebra $C$, $BA = \bar{T}^c(A[1])$ and $\Omega C = \bar{T}(C[-1])$.
                \begin{center}
                    \begin{tikzcd}
                        Alg_\mathbb{K}^\bullet \ar[bend right]{r}[below]{B} \ar[phantom]{r}[pos = 0.78]{\bot} & CoAlg_{\mathbb{K},conil}^\bullet \ar[bend right]{l}[above]{\Omega}
                    \end{tikzcd}
                \end{center}
            \end{remark}

            Associated to this adjunction, we obtain universal elements, together with universal properties. Let $A$ be an augmented dg-algebra, then the identity of the coalgebras $id_{BA} : BA \rightarrow BA$, the counit $\varepsilon_A : \Omega BA \rightarrow A$ and a twisting morphism $\pi_A : BA \rightarrow A$ are equivalent by the adjunction and representation. Dually, the identity of algebras $id_{\Omega C} : \Omega C\rightarrow\Omega C$, the unit $\eta_C : C \rightarrow B\Omega C$ and the twisting morphism $\iota_C : C\rightarrow \Omega C$ are equivalent. The morphisms $\pi_A$ and $\iota_C$ are called the universal elements. We summarize their universal property in the following corollary.

            \begin{corollary}\label{cor: universal-twisting}
                Let $A$ be an augmented dg-algebra, and $C$ a conilpotent dg-coalgebra. Any twisting morphism $\alpha : C \rightarrow A$ factors uniquely through either $\pi_A$ or $\iota_C$.
                
                \begin{center}
                    \begin{tikzcd}
                        & \Omega C \ar[dashed]{rd}[]{g_\alpha}\\
                        C \ar[]{rr}[]{\alpha} \ar[]{ru}[]{\iota_C} \ar[dashed]{rd}[]{f_\alpha} & & A \\
                        & BA \ar[]{ru}[]{\pi_A}
                    \end{tikzcd}
                \end{center}
                Moreover, the morphism $f_\alpha$ is a morphism of dg-coalgebras, and $g_\alpha$ is a morphism of dg-algebras.
            \end{corollary}

            \begin{definition}[Augmented Bar-Cobar construction]
                Let $A$ be an augmented dg-algebra. The (right) augmented bar construction is the right twisted tensor product $BA \otimes_{\pi_A} A$, where $\pi_A$ is the universal twisting morphism.

                Let $C$ be a conilpotent dg-coalgebra. The (right) augmented cobar construction is the right twisted tensor product $C \otimes_{\iota_C} \Omega C$, where $\iota_C$ is the universal twisting morphism.
            \end{definition}

            \begin{remark}
                We could have defined the augmented bar-cobar construction as the left twisted tensor product. There is really no preference of handedness. Whenever we wish to be precise which handedness we will use it will be specified, e.g. the left augmented bar construction of $A$.
            \end{remark}

            \begin{proposition}\label{prop: aug-bar-ac}
                The augmentation ideal of the augmented bar (cobar) construction is acyclic, i.e. $BA \bar{\otimes}_{\pi_A} A$ ($A \bar{\otimes}_{\pi_A} BA$) and $C \bar{\otimes}_{\iota_C}\Omega C$ ($\Omega C \bar{\otimes}_{\iota_C} C$) are acyclic.
            \end{proposition}

            \begin{proof}
                calculate kernel of the augmentation map, and find a homotopy of the identity.
            \end{proof}

    \subsection{Comparison Lemma}

        In this section we wish to deduce the fundamental theorem of twisting morphisms. In order to this we will need a result by Henri Cartan \cite{Cartan55}, called the comparison lemma. This section follows Loday \cite{Loday12}, and takes inspiration from Lefevre-Hasagawa \cite{LefevreHasegawa03}.
        
        In order to state the comparison lemma we need to understand what it means for a dg-algebra to be connected. We define weight, which is a second grading for the objects.

        \begin{definition}[Weight graded cochain complexes]
            Let $N = \startset{0, 1, ...}$ be an indexing set, which is possibly finite. A cochain complex $M$ is weight graded if it splits as a direct sum $M\simeq \bigoplus_{n\in N} M_{(n)}$ indexed over $N$. Let $m\in M$, then $m$ has weight $n$ if $m\in M_{(n)}$, has homological degree $n'$ if $m\in M^{n'}$ and (total) degree $|m| = n + n'$.

            A dg-(co)algebra is weight graded if the weight on the cochain complex respects the (co)multiplication. A weight graded dg-(co)algebra will be called a wdg-(co)algebra.
        \end{definition}

        \begin{definition}[Connected cochain complexes]
            A weight graded cochain complex $M$ is called connected if $M_(0)\simeq \mathbb{K}$ and is concentrated in homological degree $0$. 
        \end{definition}

        \begin{lemma}[Comparison Lemma]\label{lem: comparison}
            Let $g:A\rightarrow A'$ be a morphism of connected wdg-algebras, and $f:C\rightarrow C'$ be a morphism of connected wdg-coalgebras. Suppose there are twisting morphisms $\alpha: C \rightarrow A$ and $\alpha ':C'\rightarrow A'$ such that $f$ and $g$ becomes a morphism of twisted tensors. If two out of $f$, $g$ and $f\otimes g$ are quasi-isomorphisms, then so is the third.
        \end{lemma}

        \begin{proof}
            A proof may be found in either Cartans paper \cite{Cartan55} or Loday and Valletes book \cite{Loday12}.
        \end{proof}

        \begin{thm}[Fundamental Theorem of Twisting Morphisms I]\label{thm: fundamental-thm-of-twisting}
            Let $A$ be a connected wdg-algebra, $C$ be a connected wdg-coalgebra and $\alpha : C\rightarrow A$ a twisting morphism. The following are equivalent:
            \begin{enumerate}
                \item The augmentation ideal of the right twisted tensor product $C \bar{\otimes}_\alpha A$ is acyclic
                \item The augmentation ideal of the left twisted tensor product $A \bar{\otimes}_\alpha C$ is acyclic
                \item The morphism $f_\alpha : C \rightarrow BA$ is a quasi-isomorphism
                \item The morphism $g_\alpha : \Omega C \rightarrow A$ is a quasi-isomorphism
            \end{enumerate}
        \end{thm}

        \begin{proof}
            In order to do this proof we must first observe that the bar and cobar construction preserve the weight grading, and therefore the connectedness.
            
            We prove 1. $\iff$ 3., the other equivalences are analogous. By corollary \ref{cor: universal-twisting}, the morphism $id_C \otimes g_\alpha : C \otimes_{\iota_C} \Omega C \rightarrow C \otimes_{\alpha} A$ is a morphism of twisting tensor products. Since $id_C$ is a quasi-isomorphism, we get by \ref{lem: comparison} that $id_C \otimes g_\alpha$ is a quasi-isomorphism if and only if $g_\alpha$ is a quasi-isomorphism if and only if $\bar{g}_\alpha$ is a quasi-isomorphism. By \ref{prop: aug-bar-ac} $C \bar{\otimes}_{\iota_C} \Omega C$ is acyclic, so $\bar{g}_\alpha$ is a quasi-isomorphism if and only if $C\bar{\otimes}_\alpha A$ is acyclic.
        \end{proof}

        \begin{corollary}
            Let $A$ be a connected wdg-algebra, $C$ be a connected wdg-coalgebra. The counit $\varepsilon_A : \Omega BA \rightarrow A$, and the unit $\eta_C : C \rightarrow B\Omega C$ are quasi-isomorphisms. 
        \end{corollary}

        We now know some cases where the unit and the counit are quasi-isomorphisms. However, we would like to promote this result to every (conilpotent) augmented dg-(co)algebra. To this end we will find suitable filtrations to realize the associated graded as a connected wdg-(co)algebra and then get isomorphisms on the associated graded. The problem would then be to lift such quasi-isomorphisms back to our original objects.
        
        Recall that a filtration on a chain complex $M$ is a sequence of inclusions $M_0 \subseteq M_1 \subseteq ...$. The filtration is called exhaustive if $\varinjlim M_i \simeq M$, and admissable if $Fr_0M \simeq \mathbb{K}$ as well. Since each inclusion respect the differential, we may find the cokernel of each inclusion. The associated graded $grM$ is then the graded complex given by $gr_0M = M_0$ and $gr_iM = M_i/M_{i-1}$. A dg-(co)algebra is filtered if the filtration respects the (co)multiplication. If $f : M \rightarrow N$ is a morphism of filtered cochain complexes, then it defines a morphism $grf : grM \rightarrow grN$ on the associated graded. We call $f$ a graded quasi-isomorphism if $grf$ is a quasi-isomorphism.

        Let $C$ be a conilpotent dg-coalgebra. The coradical filtration $Fr_0C \subseteq Fr_1C \subseteq ...$ is an exhaustive filtration by assumption and $Fr_0C\simeq \mathbb{K}$, so it is admissable. Moreover, since $\Delta Fr_{i+1}C \subseteq Fr_iC\otimes Fr_iC$ we may define the comultiplication of the associated graded to be trivial, and this we will do. To obtain quasi-isomorphisms of chain complexes we do not need the extra structure of the coalgebra. If not specified otherwise, whenever $C$ is assumed to be a conilpotent dg-coalgebra then $grC$ should be the associated graded of the coradical filtration.

        \begin{lemma}\label{lem: graded-qif-are-w}
            Let $f : C\rightarrow C'$ be a graded quasi-isomorphism between conilpotent dg-coalgebras, then $\Omega f : \Omega C \rightarrow \Omega C'$ is a quasi-isomorphism. 
        \end{lemma}

        \begin{proof}
            We do this by considering a spectral sequence. Endow $C$ with a grading (as a vector space) induced by the coradical filtration, i.e. $c\in C$ has degree $|c|=n$ if $n$ is the smallest number such that $\bar{\Delta}^nc = 0$. We define a filtration on $\Omega C$ by
            \begin{align*}
                F_p\Omega C = \startset{c_1[-1]\otimes ... \otimes c_n[-1]\mid |c_1|+...+|c_n|\leq p}
            \end{align*}

            Since $C$ is a dg-coalgebra, the coradical filtration respects the differential. In other words, $F_p\Omega C$ is still a chain complex, which is a subcomplex of $\Omega C$. This filtration is clearly bounded below and exhaustive. Thus by the classical convergence theorem of spectral sequences, theorem 5.5.1 \cite{Weibel94}, the spectral sequence converges to the homology $E\Omega C \Rightarrow H^*\Omega C$.

            By definition, the $0$'th page is defined as $E^0_{p,q}\Omega C = \sfrac{(F_p\Omega C)_{p+q}}{(F_{p-1}\Omega grC)_{p+q}}$. Furthermore, notice that at this page we have the following isomorphism $E^0_{p,q}\Omega C \simeq (\Omega grC)^{(p)}_{p+q}$, where $(\Omega grC)^{(p)} = \startset{c_1[-1]\otimes ... \otimes c_n[-1]\mid |c_1|+...+|c_n| = p}$.

            Evaluating $f$ at the $0$'th page would the look like $E^0\Omega f \simeq \Omega grf$. By the mapping lemma, exercise 5.2.3 \cite{Weibel94}, it is enough to check that $\Omega grf$ is a quasi-isomorphism, to see that $\Omega f$ is a quasi-isomorphism. We show that $\Omega grf$ is a quasi-isomorphism by inspecting every $E^0_{p,\bullet}\Omega C$.

            We define a filtration $G_k$ on $E^0_{p,\bullet}\Omega C$ as
            \begin{align*}
                G_k = \startset{c_1[-1]\otimes ... \otimes c_n[-1] \mid n \geq -k}.
            \end{align*}
            We see that $G_0 = E^0_{p, \bullet}\Omega C$ by definition and $G_{-p-1} \simeq 0$ on the augmentation ideal $\bar{C}$. Again, by the classical convergence theorem of spectral sequences, this defines a spectral sequence such that $EG \Rightarrow H^*E^0_{p, \bullet}\Omega C$.

            To see that $\Omega grf$ is a quasi-isomorphism, it is now enough to see that $E^0Gf$ is a quasi-isomorphism for any $p$. Notice that $E^0_{l,\bullet}G \subseteq (grC[-1])^{\otimes l}$ where the total grading is $p$. Since $f$ is a graded quasi-isomorphisms and by the Kunneth-formual, theorem 3.6.3 \cite{Weibel94}, it follows that $E^Gf$ is a quasi-isomorphism.
        \end{proof}

        For completeness we include the following statement.

        \begin{lemma}
            Let $f : A \rightarrow A'$ be a quasi-isomorphism between dg-algebras, then $Bf : BA \rightarrow BA'$ is a quasi-isomorphism.
        \end{lemma}

        \begin{proof}
            Notice that the homology of $BA$ may calculated from the double complex used to define $BA$. In fact, at the $0$'th page we have $E^0_{p, \bullet}f \simeq f^{\otimes p}$. It follows that $f$ is a quasi-isomorphism on the $0$'th page from the Kunnet formula, theorem 3.6.3 \cite{Weibel94}.            
        \end{proof}

        \begin{proposition}\label{prop: unit-counit-qif}
            Let $A$ be an augmented dg-algebra and $C$ a conilpotent dg-coalgebra. The counit $\varepsilon_A : \Omega BA \rightarrow A$ a quasi-isomorphism. The unit $\eta_C : C \rightarrow B\Omega C$ is a filtered quasi-isomorphism, moreover $B\eta_C$ is a quasi-isomorphism.
        \end{proposition}

        \begin{proof}
            We pick the filtration $\mathbb{K} \subseteq A \subseteq A \subseteq ...$ for $A$. The associated graded $grA$ is $A$ but with almost trivial multiplication $grA \simeq \mathbb{K}\oplus \bar{A}$. Given a dg-coalgebra $M$, we may find a filtration on $\Omega M$ as $\Omega M_i = \bigoplus_{k \leq i} \Omega M^k$. Notice that the counit $\varepsilon_A : \Omega BA \rightarrow A$ becomes a filtered morphism together with these filtrations. thus $gr\varepsilon_A : gr\Omega BA \rightarrow grA$ is a morphism of augmented dg-algebras. By \ref{thm: fundamental-thm-of-twisting} $gr\varepsilon_A$ is a quasi-isomorphism if and only if the augmentation ideal of $gr\Omega BA \otimes grA$ is acyclic.

            Considered as cochain complexes, $A\simeq grA$ and $gr\Omega BA \simeq \Omega BA$, so there is a quasi-isomorphism $gr\varepsilon_A : \Omega BA \rightarrow A$.

            The other direction is analogously, but we use the coradical filtration for $C$ to see that $gr\eta_C : grC \rightarrow grB\Omega C$ is a quasi-isomorphism.
        \end{proof}

    \section{Strongly Homotopy Associative Algebras and Coalgebras}
    \subsection{$A_\infty$-Algebras}
        We have seen from corollary \ref{cor: a-to-dgc} that any dg-algebra $A$ defines a dg-coalgebra $T^c(A[1])$, the bar construction, with a coderivation $m^c$ of degree $-1$. Does this however work in reverse? I.e. if $A$ is a vector space such that the coalgebra $T^c(A[1])$ together with a coderivation $m^c$ is a dg-coalgebra, is then $A$ an algebra? The answer to this is no, but it leads to the definition of a strongly homotopy associative algebra.

        \begin{definition}
            An $A_\infty$-algebra is a graded vector space $A$ together with a differential $m:\bar{T}^c(A[1])\rightarrow\bar{T}^c(A[1])$ that is a coderivation of degree $-1$.
        \end{definition}

        The differential $m$ induces structure morphisms on $A[1]$. By proposition \ref{prop: tensor-derivation} there is a natural bijection $Hom_\mathbb{K}(\bar{T}^c(A[1]),A[1])\simeq Coder(\bar{T}^c(A[1]),\bar{T}^c(A[1]))$ given by the projection onto $A[1]$. Thus $m:\bar{T}^c(A[1])\rightarrow\bar{T}^c(A[1])$ corresponds to maps $\widetilde{m}_n:A[1]^{\otimes n}\rightarrow A[1]$ of degree $-1$ for any $n\geq 1$. We define maps $m_n : A^{\otimes n}\rightarrow A$ by the composite $s^{-1}\widetilde{m}_ns^{\otimes n}$. Since $s^{\otimes n}$ is of degree $n$, $\widetilde{m}_n$ and $s^{-1}$ is of degree $-1$, we get that $m_n$ is of degree $n-2$.
        \begin{center}
            \begin{tikzcd}
                A^{\otimes n} \ar[]{r}[]{m_n} \ar[]{d}[]{\simeq}[left]{s^{\otimes n}} & A \\
                A[1]^{\otimes n} \ar[]{r}[]{\widetilde{m}_n} & A[1] \ar[]{u}[right]{\simeq}[]{s^{-1}}
            \end{tikzcd}
        \end{center}

        \begin{proposition}\label{prop: A-infinity def}
            An $A_\infty$-algebra is equivalent to a graded vector space $A$ together with homogenous morphisms $m_n:A^{\otimes n}\rightarrow A$ of degree $n-2$. Moreover, the morphism must satisfy the following relations for any $n\geq 1$:
            \begin{align*}
                (\text{rel}_n)\qquad \sum_{p+q+r = n}(-1)^{p+qr}m_{p+1+r}\circ (id^{\otimes p}\otimes m_q \otimes id^{\otimes r}) = 0
            \end{align*}
        \end{proposition}

        \begin{remark}
            We make a more convenient notation for $(\text{rel}_n)$, called partial composition $\circ_i$.
            \begin{align*}
                & m_{p+1+r} \circ_{p+1} m_q = m_k \circ (id^{\otimes p}\otimes m_q \otimes id^{\otimes r}) \\
                (\text{rel}_n)\qquad & \sum_{p+q+r = n} (-1)^{p+qr} m_{p+1+r} \circ_{p+1} m_q = 0
            \end{align*}
        \end{remark}

        Before starting with the proof we will use a lemma for checking whether a coderivation $m: T^c(A) \rightarrow T^c(A)$ is a differential.

        \begin{lemma}\label{lem: coderivation-is-diff?}
            Let $m: T^c(A) \rightarrow T^c(A)$ be a coderivation, and denote $m_n = m|_{A^{\otimes n}}$. $m$ is a differential if and only if the following relations are satisfied:
            \begin{align*}
                & \sum_{p+q+r = n}m_{p+1+r}\circ_{p+1}m_q = 0
            \end{align*}
        \end{lemma}

        \begin{proof}
            By proposition \ref{prop: tensor-derivation} we may write $m = \sum_{n = 0}^\infty \sum_{i = 0}^n m_{(n)}^{(i)}$. By using partial composition, we rewrite its n'th component as:
            \begin{align*}
                m_n = \sum_{q=1}^n\sum_{p = 1}^n id^{\otimes (n-q)}\circ_{p} m_q = \sum_{p + q + r = n}id^{\otimes (p+1+r)}\circ_{p+1}m_q
            \end{align*}

            For $m^2$ we denote it's n'th component as $m^2_n$. Observe the following:
            \begin{align*}
                & m^2_n = m\circ m_n = m\circ \sum_{p + q + r = n}id^{\otimes (p+1+r)}\circ_{p+1}m_q = \sum_{p + q + r = n}m\circ_{p+1}m_q \\
                & \pi m^2_n = \pi \sum_{p + q + r = n}m\circ_{p+1}m_q = \sum_{p + q + r = n}m_{p+1+r}\circ_{p+1}m_q
            \end{align*}
            Since every coderivation are uniquely determined by $\pi$, its projection onto $A$ we get that $m^2 = 0$ if and only if
            \begin{align*}
                \sum_{p+q+r = n}m_{p+1+r}\circ_{p+1}m_q = 0\text{.}
            \end{align*}
        \end{proof}

        \begin{proof}[Proof of proposition \ref{prop: A-infinity def}]
            Let $(A,m)$ be an $A_\infty$-algebra. We denote the n'th component of $m$ as $\widetilde{m}_n$. The n'th components thus define maps $m_n:A^{\otimes n}\rightarrow A$ as $m_n = s^{-1}\widetilde{m}_ns^{\otimes n}$.

            By the above lemma we know that the n'th component of $m^2$ is:
            \begin{multline*}
                \sum_{p + q + r = n}\widetilde{m}_{p+1+r}\circ_{p+1}\widetilde{m}_q \\
                = \sum_{p + q + r = n}sm_{p+1+r}s^{-\otimes (p+1+r)}\circ_{p+1}sm_qs^{-\otimes q} = \sum_{p + q + r = n}(-1)^{pq+r}sm_{p+1+r}\circ_{p+1}m_q s^{-\otimes n}
            \end{multline*}

            Since suspension and desuspension are isomorphism we get that $m^2 = 0$ if and only if $(\text{rel}_n)$ are $0$ for every $n\geq 1$, i.e.
            \begin{align*}
                \sum_{p+q+r = n} (-1)^{p+qr} m_{p+1+r} \circ_{p+1} m_q = 0
            \end{align*}
        \end{proof}

        Given an $A_\infty$ algebra $A$ we may either think of it as a differential tensor coalgebra $\bar{T}^c(A[1])$ with differential $m: \bar{T}^c(A[1])\rightarrow \bar{T}^c(A[1])$ or as a graded vector space with morphisms $m_n:A^{\otimes n} \rightarrow A$ satisfying $(\text{rel}_n)$. We will calculate $(\text{rel}_n)$ for $1,2,3$:
        \begin{align*}
            (\text{rel}_1)\qquad & m_1\circ m_1 = 0 \\
            (\text{rel}_2)\qquad & m_1\circ m_2 - m_2\circ_{1}m_1 - m_2\circ_2m_1 = 0 \\
            (\text{rel}_3)\qquad & m_1\circ m_3 + m_2\circ_1 m_2 - m_2\circ_2m_2 + m_3\circ_1m_1 + m_3\circ_2m_1 + m_3\circ_3m_1 = 0 
        \end{align*}

        We see that $(\text{rel}_1)$ states that $m_1$ should be a differential, we may thus think of $(A, m_1)$ as a chain complex. Furthermore, $(\text{rel}_2)$ says that $m_2 : (A^{\otimes 2}, m_1\otimes id_A + id_A\otimes m_1) \rightarrow (A, m_1)$ is a morphism of chain complexes. Lastly, $(\text{rel}_3)$ gives us a homotopy for the associator of $m_2$, namely $m_3$. Thus we may regard $(A, m_1, m_2)$ as an algebra which is associative up to homotopy. Regarding $A$ as a chain complex instead we obtain our final definition of an $A_\infty$-algebra.

        \begin{proposition}
            Suppose that $(A, d)$ is a chain complex, and that there exists morphisms $m_n: A^{\otimes n} \rightarrow A$ for any $n\geq 2$. A is an $A_\infty$-algebra if and only it satisfies the following relations:
            \begin{align*}
                (\text{rel'}_n)\qquad & \partial(m_n) = -\sum_{\substack{n = p + q + r \\ k = p + 1 + r \\ k > 1, q > 1}}(-1)^{p + qr}m_k\circ_p+1m_q
            \end{align*}
        \end{proposition}

        We define the homotopy of an $A_\infty$-algebra to be the homology of the chain complex $(A, m_1)$. Since $\partial(m_3) = m_2\circ_1m_2 - m_2\circ_2m_2$, we get that $m_2$ is associative in homology. Thus for any $A_\infty$-algebra $A$, the homotopy $HA$ is an associative algebra. The operadic homology of $A$ is defined as the homology of $(T^c(A[1]), m)$, which is the non-unital augmented Hochschild homology of $A$.

        \begin{example}
            Suppose that $V$ is a cochain complex with differential $d$. Then $V$ is an $A_\infty$-algebra with trivial multiplication. In other words $m^1 = d$ and $m^i = 0$ for any $i>1$.
        \end{example}

        \begin{example}
            Suppose that $A$ is a dg-algebra. Then $A$ is an $A_\infty$-algebra where $m^1 = d$, $m^2 = \cdot$ and $m^i=0$ for any $i>2$.
        \end{example}

        \begin{example}
            Let $A$ be a connected weight-graded algebra over $\mathbb{K}$. We may then think of $\mathbb{K}$ as an $A$-module with trivial action, i.e. as the quotient $\sfrac{A}{A^{(i)}}_{(i\geq 1)}$, then $Ext^*_A(\mathbb{K},\mathbb{K})$ is an $A_\infty$-algebra. This was shown by Lu, Palmari, Wu and Zhang \cite{Lu06}.
        \end{example}

        Next we want to understand the category of $A_\infty$-algebras. A morphism between $A_\infty$-algebras is called an $\infty$-morphism. We define such an $\infty$-morphism $f:A\rightsquigarrow B$ between $A_\infty$-algebras as associated dg-coalgebra homomorphism $Bf:(\bar{T}^c(A[1]), m^A)\rightarrow (\bar{T}^c(B[1]), m^B)$. Here $Bf$ is purely formal, we will make sense of this soon.

        \begin{proposition}
            Let $A,B$ be two $A_\infty$-algebras. A collection of morphisms $f_n : A^\otimes n \rightarrow B$ of degree $n-1$ for any $n \geq 1$ defines an $\infty$-morphism $f : A \rightsquigarrow B$ if and only if $f_1$ is a morphism of chain complexes and for any $n\geq 2$ the following relations are satisfied:
            \begin{align*}
                & (\text{rel}_n)\qquad \partial(f_n) = \sum_{\substack{p + 1 + r = k \\ p + q + r = n}}(-1)^{pq+r}f_k\circ_{p+1}m^A_q - \sum_{\substack{k\geq 2 \\ i_1 + ... + i_k = n}}(-1)^{e}m^B_k \circ (f_{i_1}\otimes f_{i_2}\otimes ... \otimes f_{i_k}) \\
                & \text{where } e \text{ is given as: } e = (k-1)(i_1-1) + (k-2)(i_2-1) + ... + 2(i_{k-2}-1) + (i_{k-1} - 1)
            \end{align*}
        \end{proposition}
        
        \begin{proof}
            This is immediate by the universal property of cofree coalgebras.
        \end{proof}

        Since the composition of two dg-coalgebra homomorphism is again a dg-algebra homomorphism, we get that the composition of two $\infty$-morphisms is again an $\infty$-morphism. More explicitly if $f:A\rightsquigarrow B$ and $g: B\rightsquigarrow C$ are two $\infty$-morphisms, then their composition is defined as:
        \begin{align*}
            (fg)_n = \sum_r\sum_{i_1 + ... + i_r = n} (-1)^eg_r(f_{i_1}\otimes ... \otimes f_{i_r})\text{.}
        \end{align*}

        \begin{definition}
            An $\infty$-morphism $f: A\rightsquigarrow B$ is called strict if $f_n = 0$ for any $n\geq 2$. 
        \end{definition}

        \begin{definition}
            $Alg_\infty$ denotes the category of $A_\infty$-algebras, and the morphisms in this category are the $\infty$-morphisms.
        \end{definition}

        Observe that we may extend the bar construction to $B : Alg_\infty \rightarrow CoAlg_{\mathbb{K},conil}^\bullet$ to a fully fatihful functor. This may be done explicitly by using proposition \ref{prop: tensor-derivation}. The subcategory of the essential image is the full subcategory of every dg-coalgebra that is isomorphic, as a graded coalgebra, to a cofree tensor coalgebra. Notice that the bar construction on the category of dg-algebras is a non-full injection into the category of $A_\infty$-algebras. Observe that this inclusion is the recontextualization of a dg-algebra into an $A_\infty$-algebra.

        We call a quasi-isomorphism between $A_\infty$-algebras for an $\infty$-quasi-isomorphism. Given an $\infty$-morphism $f: A\rightsquigarrow B$, then we say that it is an $\infty$-quasi-isomorphism if $f_1$ is a quasi-isomorphism. If we wanted to be more stringent with this definition we would want an $\infty$-quasi-isomorphism to be an $\infty$-morphism which is a quasi-isomorphism of dg-coalgebras. We will see that these definitions are equivalent later.

        A homotopy between two $A_\infty$-algebras is a homotopy between the dg-coalgebras they define. We may however trace this definition back along the quasi-inverse of the bar construction to get a new definition in terms of many morphisms. Let $f,g: A\rightsquigarrow B$ be two $\infty$-morphisms, we say that $f-g$ is null-homotopic if there is a collection of morphisms $h_n : A^{\otimes n} \rightarrow B$ of degree $n$ such that the following relations are satisfied for any $n \geq 1$:
        \begin{multline*}
            f_n - g_n = \sum (-1)^s m^B_{r+1+t}\circ (f_{i_1}\otimes ... \otimes f_{i_r} \otimes h_k \otimes g_{j_1} \otimes ... \otimes g_{j_t}) +  \sum (-1)^{j+kl}h_i \circ_{j+1} m_k^A
        \end{multline*}
        Where $s$ is some constant depending on $t,r$ and $k$. More details may be found in \cite{LefevreHasegawa03}.

        As in the same case for algebras, there is also a notion of unital $A_\infty$-algebras and augmented $A_\infty$-algebras. For this discussion it is important to note that the field $\mathbb{K}$ is also an $A_\infty$-algebra. This algebra will serve as the initial algebra, in the same way as it did for ordinary algebras.

        \begin{definition}
            A strictly unital $A_\infty$-algebra is an $A_\infty$-algebra $A$ together with a unit morphism $\upsilon_A : \mathbb{K} \rightarrow A$ of degree $0$ such that the following are satisfied:
            \begin{itemize}
                \item $m_1\circ \upsilon_A = 0$.
                \item $m_2(id_A \otimes \upsilon_A) = id_A = m_2(\upsilon_A\otimes id_A)$.
                \item $m_i \circ_k \upsilon_A = 0$ for any $i\geq 3$ and $0 \leq k < i$.
            \end{itemize}
        \end{definition}

        A strictly unital $\infty$-morphism $f: A \rightsquigarrow B$ between strictly unital $A_\infty$-algebras is a morphism which preserves the unit. This means that $f_1\upsilon_A = \upsilon_B$ and $f_i \circ_k \upsilon_A = 0$ for any $i \geq 2$ and $0 \leq k < i$. The collection of strictly unital $A_\infty$-algebras and strictly unital $\infty$-morphisms form a non-full subcategory of $A_\infty$-algbras. A strict $\infty$-morphism which is unital at the level of chain complexes is automatically stricly unital. Strict unital will then mean strict and strictly unital. Note that $\mathbb{K}$ is strictly unital where the unit is $id_\mathbb{K}$.

        \begin{definition}
            An augmented $A_\infty$-algebra is a strictly unital $A_\infty$-algebra $A$ together with a strict unital morphism $\varepsilon_A : A \rightarrow \mathbb{K}$. The $\infty$-morphism $\varepsilon_A$ is called the augmentation of $A$.
        \end{definition}

        The collection of augmented $A_\infty$-algebras and strictly unital morphism is the category of augmented $A_\infty$-algebras, denoted as $Alg_{\infty,+}$. As in the same way for algebras, there is an equivalence of categories $Alg_\infty \simeq Alg_{\infty,+}$. The augmentation ideal, or the reduced $A_\infty$-algebra is the kernel of the augmentation $\varepsilon_A$. A priori, it does not make sense to talk about this limit as we do not know if it exists. However, we will see in section 2.3.3 that such morphisms does in fact have a kernel. This defines a functor, $\bar{\_} : Alg_{\infty ,+} \rightarrow Alg_\infty$, where $Ker\varepsilon_A = \bar{A}$. The quasi-inverse to this functor is free augmentations. Given an $A_\infty$-algebra $A$, we may construct the $A_\infty$-algebra $A\oplus \mathbb{K}$. The structure morphisms are given by $m_i^A$, but there is now a unit $\upsilon_{A\oplus\mathbb{K}}$. Thus we get that $m_1(1) = 0$, $m_2 (a\otimes 1) = a$ and $m_i \circ_k 1 = 0$ in the same manner. We obtain a functor $\_^+ : Alg_\infty \rightarrow Alg_{\infty, +}$, where $A\oplus \mathbb{K} = A^+$.

    \subsection{$A_\infty$-Coalgebras}    
        Dual to $A_\infty$-algebras we got conilpotent $A_\infty$-coalgebras. Here we instead ask ourselves if the cobar construction has some converse. I.e. if $C$ is a graded vector space such that $T(C[-1])$ together with a derivation $m$ is a dg-algebra, is then $C$ a coalgebra? Again, the answer to this is no, but we do obtain a definition for conilpotent $A_\infty$-coalgebras.

        \begin{definition}
            A graded vector space $C$ is called a conilpotent $A_\infty$-coalgebra if it is a dg-algebra of the form $(\bar{T}(C[-1]), d)$ where $d$ is a derivation of degree $-1$.
        \end{definition}

        \begin{remark}
            For the rest of this thesis, an $A_\infty$-coalgebra should be understood as a conilpotent $A_\infty$-coalgebra unless otherwise specified.
        \end{remark}

        \begin{corollary}
            $C$ is an $A_\infty$-coalgebra with differential $d$ then there is a chain complex $(C, d^1)$, where $d^1$ is of degree $1$, and together with morphisms $d^n : C \rightarrow C^{\otimes n}$ such that $d$ uniquely determines each $d^i$ for any $i>0$. Conversely, if the morphisms $d^i$ satisfy $\text{(rel)}_n$, then they uniquely determine a $d$ such that $C$ is an $A_\infty$-coalgebra.
            \begin{align*}
                (\text{rel}_n)\qquad & \sum_{p+q+r = n}(-1)^{pq+r}d^{p+1+q}\circ^{op}_{p+1}d^q = 0
            \end{align*}
        \end{corollary}

        A morphism of $A_\infty$-coalgebras would be defined in the same manner, but opposite. So an $\infty$-comorphism $f: C \rightsquigarrow D$ is either a morphism $\widetilde{f}: (T(C[-1]), m^C) \rightarrow (T(D[-1]),m^D)$ of dg-algebras. Equivalently such an $\infty$-comorphism is a collection of morphisms $f_n : C \rightarrow D^{\otimes n}$ of degree $n-2$ such that $f_1$ is a morphism of chain complexes and for any $n\geq 2$ the following relations are satisfied:
        \begin{align*}
            & (\text{rel}_n)\qquad \partial(f_n) = \sum_{\substack{p + 1 + r = k \\ p + q + r = n}}(-1)^{pq+r}f_k\circ^{op}_{p+1}m^D_q - \sum_{\substack{k\geq 2 \\ i_1 + ... + i_k = n}}(-1)^{e}m^C_k \circ^{op} (f_{i_1}\otimes f_{i_2}\otimes ... \otimes f_{i_k}) \\
            & \text{where } e \text{ is given as: } e = (k-1)(i_1-1) + (k-2)(i_2-1) + ... + 2(i_{k-2}-1) + (i_{k-1} - 1)
        \end{align*}

        We denote $Coalg_\infty$ as the category of $A_\infty$-coalgebras. In the same manner, the cobar construction extends to this category and gives us an identification of $A_\infty$-coalgebras and a subcategory of dg-algebras. This subcategory consists of every dg-algebra that is isomorphic, as an algebra, to a free tensor algebra. Lastly, every dg-coalgebra is an $A_\infty$-coalgebra by letting every morphism $m^i = 0$ where $i>2$. This gives a non full inclusion. 
\end{document}