\documentclass[../thesis.tex]{subfiles}

\begin{document}

    \chapter*{Abstract}
        In this thesis, we study the homotopy theory of associative dg-algebras, conilpotent coassociative dg-coalgebras, and strongly homotopy associative algebras. We employ twisting morphisms to show that the cobar-bar construction defines a Quillen equivalence between conilpotent dg-coalgebras and dg-algebras. Every $A_\infty$-algebra is a bifibrant object of the category of conilpotent dg-coalgebras, and the three associated homotopy categories are all equivalent.

        Similarly, there are Quillen equivalences between comodule categories associated to conilpotent dg-coalgebras and module categories associated to dg-algebras. Every polydule of an $A_\infty$-algebra is considered to be a bifibrant object of a comodule category, and the derived module category, homotopy category of the comodule category, and the derived polydule category are all equivalent.

    \chapter*{Sammendrag}
        I denne avhandlingen studerer vi homotopiteorien til assosiative dg-algebraer, konilpotente koassosiative dg-koalgebraer og sterkt homotopi-assosiative algebraer. Vi bruker vridde morfier for å vise at kobar-bar konstruksjonen definerer en Quillen-ekvivalens mellom konilpotente dg-koalgebraer og dg-algebraer. Enhver $A_\infty$-algebra er et bifibrant objekt i kategorien av konilpotente dg-koalgebraer, og de tre assosierte homotopikategoriene er ekvivalente.

        På samme måte, er det Quillen-ekvivalenser mellom komodulkategorier assosiert til konilpotente dg-koalgebraer og modulkategorier assosiert til dg-algebraer. Enhver polydul til en $A_\infty$-algebra kan ansees som et bifibrant objekt i en komodulkategori, og den deriverte modulkategorien, homotopikategorien til komodulkategorien og den deriverte polydulkategorien er alle ekvivalente.

    \newpage

    

    \chapter*{Acknowledgements}
        This thesis marks the conclusion of my studies at NTNU.

        I want to express my deepest gratitude to Professor Steffen Opperman for his guidance and feedback, his encouragement to explore homological and homotopical algebra, and for showing me the wonder of this subject. He has always been very supportive and has provided good guidance when needed.

        I would like to thank my friends and "Matteland" for making each day different and worthwhile. My time at NTNU would not have been the same without them. A special thanks go to Elias Klakken Angelsen, Tallak Manum, and Markus Valås Hagen for their proofreading, editing support, and much-appreciated opinions.
        
        Lastly, I would like to thank my family for their support during my time as a student.


\end{document}